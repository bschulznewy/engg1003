\documentclass[14pt]{beamer}
\usetheme{EastLansing}
\usecolortheme{spruce}

\usepackage{xcolor}
\usepackage{listings}
\usepackage{courier}
\usepackage{graphicx}
\usepackage{amsmath}
\usepackage{algorithm2e}
\usepackage[font=small]{caption}

\usefonttheme[onlymath]{serif}


\definecolor{mGreen}{rgb}{0,0.6,0}
\definecolor{mGray}{rgb}{0.5,0.5,0.5}
\definecolor{mPurple}{rgb}{0.8,0,0.82}
\definecolor{backgroundColour}{rgb}{0.95,0.95,0.92}
\definecolor{lightBlue}{rgb}{0.1, 0.1, 0.8}

\lstdefinestyle{CStyle}{
    backgroundcolor=\color{backgroundColour},   
    commentstyle=\color{mGreen},
    keywordstyle=\color{magenta},
    numberstyle=\tiny\color{mGray},
    stringstyle=\color{mPurple},
    basicstyle=\ttfamily\footnotesize,
    breakatwhitespace=false,         
    breaklines=true,                 
    captionpos=b,                    
    keepspaces=true,                 
    numbers=left,                    
    numbersep=5pt,                  
    showspaces=false,                
    showstringspaces=false,
    showtabs=false,                  
    tabsize=2,
    language=C
}

\lstdefinestyle{pseudo}{
        basicstyle=\ttfamily\footnotesize,
        keywordstyle=\color{lightBlue},
        morekeywords={BEGIN,END,IF,ELSE,ENDIF,PRINT,WHILE,RETURN},
        morecomment=[l]{//},
        commentstyle=\color{mGreen},
        tabsize=2
}

\lstset{basicstyle=\footnotesize\ttfamily,breaklines=true}
\lstset{framextopmargin=50pt,tabsize=4}

\title{ELEC3850 - Embedded Systems 1}
\subtitle{Real World Development}
\author{Brenton Schulz}
\institute{University of Newcastle}
\date{\today}

\begin{document}
\titlepage

\begin{frame}
\frametitle{Summary}
\begin{itemize}
\item Sample development timeline
	\begin{itemize}
	\item Pick a project requirement
	\item Identify hardware module which meets requirement
	\item Read datasheets - identify microcontroller interface
	\item Work out how to drive microcontroller interface with your system
		\begin{itemize}
		\item Bit-banging software interface?
		\item Existing microcontroller peripheral (I2C, SPI, etc)?
		\item Hardware configuration?
		\item Do you need to write a driver library?
		\item Are there complex interrupt / DMA timing constraints?
		\end{itemize}
	\end{itemize}
\end{itemize}
\end{frame}

\begin{frame}[fragile]
\frametitle{Pick An Imaginary Project Requirement}
\begin{itemize}
\item Display information to the user with an 8x8 pixel LED matrix
\pause
\item Lets search eBay for parts...
\pause
\item Great, we pick the MAX7219
\item Why? Is this good design?
	\begin{itemize}
	\item It meets requirements
	\item It is available
	\item (Presumably) it is within budget
	\end{itemize}
\end{itemize}
\end{frame}

\begin{frame}[fragile]
\frametitle{MAX2719}
\begin{itemize}
\item Note: There are other factors we have not considered
\begin{itemize}
\item Update rate
\item Size
\item Power requirements
\item Heat output
\item ...etc
\end{itemize}
\item It might not make it to the final design!
\item Be preapred to re-evaluate after prototyping.
\end{itemize}
\end{frame}

\begin{frame}[fragile]
\frametitle{MAX7219}
\begin{itemize}
\item Lets read the datasheet
\begin{itemize}
\item How do we find it?
\item What information are we looking for?
\pause
	\begin{itemize}
		\item Digital interface specifications
	\end{itemize}
\item In practice: flick over everything as there might be surprises
\end{itemize}
\pause
\item Note that (annoyingly) the timing diagram is on page 6 but the timing data is back on page 3
\item What other crucial information does the datasheet have?
\pause
\begin{itemize}
\item Serial data format
\item Register map
\end{itemize}
\end{itemize}
\end{frame}

\begin{frame}[fragile]
\frametitle{MAX7219}
\begin{itemize}
\item Ok, we have data. What now?
\pause
\item Observe the register map - what needs to happen before we display data?
\pause
\item Device initialisation!
\begin{itemize}
\item Power-on settings might (or might not) be random
\item Initialisation sets the configuration registers to a known or wanted state
\item This is needed for almost all peripheral modules
\end{itemize}
\end{itemize}
\end{frame}

\begin{frame}[fragile]
\frametitle{MAX7219 - Initialisation}
\begin{itemize}
\item Pen and paper time
\item Re-read datasheet
\item Note all configuration registers and calculate initial values for your application
\end{itemize}
\end{frame}

\begin{frame}[fragile]
\frametitle{MAX7219 - Initialisation}
\begin{itemize}
\small
\item To save time, have a summary:
\begin{itemize}
	\item NB: \texttt{X} indicates ``don't care'' in the address data.
	\item Table below assumes these bits are zero for clarity
\end{itemize}
\end{itemize}
\vspace{-0.5cm}
\begin{center}
\small
\begin{tabular}{ |l|c|c|l| } 
 \hline
Register & Address & Set Value & Notes \\
  \hline
Digit Data & 0x01 & Don't care & Data will be filled in later \\
Decode Mode & 0x09 & 0x00 & No decode - driving matrix \\
Intensity & 0x0A &  0x0F & Full brightness \\
Scan Limit & 0x0B & 0x07 & Drive all pixels \\
Shutdown & 0x0C & 0x00 & Normal Operation \\
Display Test & 0x0F & ?? & 0x01 to test, 0x00 otherwise\\
\hline
\end{tabular}
\end{center}
\begin{itemize}
\small
\item Does the initilisation order matter? Does shutdown have to be turned off first? Sometimes this matters, sometimes it doesn't.
\end{itemize}
\end{frame}

\begin{frame}[fragile]
\frametitle{MAX7219 - Software}
\begin{itemize}
\item Programming time - What to do?
\item Brainstorm some useful functions:
\begin{itemize}
\item \texttt{MAX7219\_WriteData();}
\item \texttt{MAX7219\_Init();}
\item \texttt{MAX7219\_DrawAll();}
\item \texttt{MAX7219\_WriteRow();}
\end{itemize}
\item This is otherwise known as writing a \textit{driver}
\end{itemize}
\end{frame}

\begin{frame}[fragile]
\frametitle{Bit Banging}
\begin{itemize}
\item Slang term - see Wikipedia for more details\\
\small
\url{https://en.wikipedia.org/wiki/Bit_banging}
\normalsize
\item Can refer to many bit manipulations
\item Most commonly refers to a software-driven serial data interface
\item Compromises:
	\begin{itemize}
		\item Useful when hardware interface is not available
		\item Can be debugged at low speed using instruction stepping
		\item A good exercise for students
		\item Much slower and resource intensive than using dedicated hardware
	\end{itemize}
\end{itemize}
\end{frame}

\begin{frame}[fragile]
\frametitle{\texttt{MAX7219\_WriteData()}}
\begin{itemize}
\item Implementation options:
	\begin{itemize}
		\item Bit bang'ed
		\item SPI peripheral
			\begin{itemize}
				\item Polled
				\item Interrrupt driven
				\item DMA'ed
			\end{itemize}
	\end{itemize}
\pause
\item Start with bit banging
	\begin{itemize}
		\item Sometimes you need it
		\item It isn't taught anywhere
		\item Low-level bit manipulation is tricky but sometimes necessary
			\begin{itemize}
				\item eg: WS2812 RGB LEDs and DS18B20 temperature sensors use proprietary protocols
			\end{itemize}
	\end{itemize}
\item Use SPI later - once everything else it working
\end{itemize}
\end{frame}


\begin{frame}[fragile]
\frametitle{Bit Banging}
\begin{itemize}
\item Example, assuming a synchronous (clocked) serial interface
\item General algorithm:
\begin{lstlisting}[style=CStyle]
BEGIN
	uint8_t data;

	FOR each bit in `data`
		Place bit on an output pin
		Cycle a clock pin
	ENDFOR
END
\end{lstlisting}
\item Typically needs \texttt{>>} or \texttt{<<} shift operators and bitwise logic operators
\end{itemize}
\end{frame}

\begin{frame}[fragile]
\frametitle{Bit Banging}
\begin{itemize}
\item With more detail, assuming MSB first:
\begin{lstlisting}[style=CStyle]
uint8_t data, i;

for(i = 0; i < 8; i++)
{
	OUTPUT_PIN = (data & 0x80) >> 8;
	data = data << 1;
	CLK_PIN = 1;
	CLK_PIN = 0;
}
\end{lstlisting}
\item Assumes data is latched on the \textit{rising} edge
\item \texttt{(data \& 0x80)} is either zero or 128 (\texttt{0x80}). Shift right by 8 bits to make this zero or 1.
\end{itemize}
\end{frame}

\begin{frame}[fragile]
\frametitle{MAX7219 Interface}
\begin{itemize}
\item Check the MAX7219 datasheet again
\item We need to know two things:
	\begin{itemize}
		\item The state the clock idles at
		\item The clock edge which latches data
	\end{itemize}
\item Make a note of these facts somewhere
\end{itemize}
\end{frame}

\begin{frame}[fragile]
\frametitle{STM32CubeIDE}
\begin{itemize}
\item Next step: STM32 configuration
\item Initial coding will be done with bit banging
\item Later we plan to use SPI
\item Therefore: choose pins that also connect to SPI
	\begin{itemize}
		\item CubeMX will tell you which pins these are
		\item Check dev board datasheets to work out where they are physically
		\item Configure them as GPIO outputs for now
	\end{itemize}
\end{itemize}
\end{frame}

\begin{frame}[fragile]
\frametitle{STM32CubeIDE}
\begin{itemize}
\item What else needs checking or modifying?
\pause
\item Probably clocks - by default the CPU clock may not be optimal
	\begin{itemize}
		\item Note: Maximum clock isn't always best!
		\item There is a small power/heat compromise when increasing clock rate
	\end{itemize}
\pause
\item Generate the project
\item Observe HAL files which were included
\end{itemize}
\end{frame}

\begin{frame}[fragile]
\frametitle{STM32CubeIDE - HAL}
\begin{itemize}
\item How do we learn about the HAL?
\item Documentation for STM32F4: \url{https://www.st.com/resource/en/user_manual/dm00105879-description-of-stm32f4-hal-and-ll-drivers-stmicroelectronics.pdf}
\item HAL requires solid C programming knowledge
	\begin{itemize}
		\item Lots of pointers...everywhere
		\item Lots of structures
	\end{itemize}
\end{itemize}
\end{frame}

\begin{frame}[fragile]
\frametitle{STM32CubeIDE - HAL}
\begin{itemize}
\item Aside: why does it use so many structures?
\pause
\item eg:
\begin{lstlisting}[style=CStyle]
typedef struct
{
  __IO uint32_t CR1;
  __IO uint32_t CR2;
  __IO uint32_t SR;
  __IO uint32_t DR;
  __IO uint32_t CRCPR;
  __IO uint32_t RXCRCR;
  __IO uint32_t TXCRCR;
  __IO uint32_t I2SCFGR;
} SPI_TypeDef;
\end{lstlisting}
\end{itemize}
\end{frame}

\begin{frame}[fragile]
\frametitle{STM32CubeIDE - HAL}
\begin{itemize}
\item Check the SPI memory map in the reference manual: \url{https://www.st.com/resource/en/reference_manual/dm00031020-stm32f405415-stm32f407417-stm32f427437-and-stm32f429439-advanced-armbased-32bit-mcus-stmicroelectronics.pdf}
\item This is on Page 916
\item Compare register map to the \texttt{SPI\_TypeDef} structure...
\end{itemize}
\end{frame}

\begin{frame}[fragile]
\frametitle{STM32CubeIDE - HAL}
\begin{itemize}
\item Why is this structure:register mapping useful?
\pause
\item Structure pointers!
\item A \texttt{SPI\_TypeDef*} pointer can access a register with \texttt{->}
\item eg:
\begin{lstlisting}[style=CStyle]
SPI_TypeDef *spi1 = 0x40013000; // SPI1 base address
spi1->DR = dataByte; // Write dataByte to tx reg
\end{lstlisting}
\item The SPI1 base address is built into the HAL - no explicit need to do things this way
\end{itemize}
\end{frame}

\begin{frame}[fragile]
\frametitle{STM32CubeIDE - writing \texttt{MAX7219\_WriteData()}}
\begin{itemize}
\item Finally some coding!
\item Lets write the low-level function
\item Then we will \textit{test} the function
	\begin{itemize}
		\item ALWAYS TEST!
		\item Use a logic analyser - Saleae units are in the lab
		\item Have a hand-written output to compare against
	\end{itemize}
\pause
\item Next step: write the initialisation
\item Test init works with the lamp test register
\item Write the other functions after
\end{itemize}
\end{frame}

\begin{frame}[fragile]
\frametitle{MAX7219 - converting to SPI}
\begin{itemize}
\item Are the other functions working now? Can we draw to the LED matrix?
\item Next step: replace bit-banging with SPI
\item Development steps:
	\begin{itemize}
	\item Re-configure device for SPI in CubeMX
	\item Replace bit-banging code with a call to HAL\_SPI\_Transmit();
	\item NB: the \texttt{CS} line needs to be manually controlled if cascading multiple modules
	\item Try to use the HAL files as documentation
	\begin{itemize}
		\item What is in the .h files?
		\item What is in the .c files?
		\item Make use of ``Open Declaration'' feature
	\end{itemize}
	\end{itemize}
\end{itemize}
\end{frame}

\end{document}
