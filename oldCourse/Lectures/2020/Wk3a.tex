\documentclass[14pt]{beamer}
\usetheme{Dresden}
\usecolortheme{orchid}

\usepackage{xcolor}
\usepackage{listings}
\usepackage{courier}
\usepackage{graphicx}
\usepackage{amsmath}
\usepackage{algorithm2e}
\usepackage{multicol}

\usefonttheme[onlymath]{serif}

\definecolor{mGreen}{rgb}{0,0.6,0}
\definecolor{mGray}{rgb}{0.5,0.5,0.5}
\definecolor{mPurple}{rgb}{0.8,0,0.82}
\definecolor{backgroundColour}{rgb}{0.95,0.95,0.92}
\definecolor{lightBlue}{rgb}{0.1, 0.1, 0.8}

\lstdefinestyle{CStyle}{
    backgroundcolor=\color{backgroundColour},   
    commentstyle=\color{mGreen},
    keywordstyle=\color{magenta},
    numberstyle=\tiny\color{mGray},
    stringstyle=\color{mPurple},
    basicstyle=\footnotesize\ttfamily,
    breakatwhitespace=false,         
    breaklines=true,                 
    captionpos=b,                    
    keepspaces=true,                 
    numbers=left,                    
    numbersep=5pt,                  
    showspaces=false,                
    showstringspaces=false,
    showtabs=false,                  
    tabsize=2,
    language=C
}

\lstdefinestyle{Ctable}{
    backgroundcolor=\color{backgroundColour},   
    commentstyle=\color{mGreen},
    keywordstyle=\color{magenta},
    numberstyle=\tiny\color{mGray},
    stringstyle=\color{mPurple},
    basicstyle=\footnotesize\ttfamily,
    breakatwhitespace=false,         
    breaklines=true,                 
    captionpos=b,                    
    keepspaces=true,                                  
    showspaces=false,                
    showstringspaces=false,
    showtabs=false,                  
    tabsize=2,
    language=C
}

\lstdefinestyle{pseudo}{
        basicstyle=\ttfamily\footnotesize,
        keywordstyle=\color{lightBlue},
        morekeywords={BEGIN,END,IF,ELSE,ENDIF,ELSEIF,PRINT,WHILE,RETURN,ENDWHILE,DO,FOR,TO,IN,ENDFOR,BREAK,INPUT,READ},
        morecomment=[l]{//},
        commentstyle=\color{mGreen}
}

\lstset{basicstyle=\footnotesize\ttfamily,breaklines=true}
\lstset{framextopmargin=50pt,tabsize=2}

\title{ENGG1003 - Monday Week 3}
\subtitle{What does \texttt{=} \textit{Really} do?\\More Flow Control}
\author{Brenton Schulz}
\institute{University of Newcastle}
\date{\today}

\begin{document}
\titlepage

\begin{frame}
\frametitle{Who Should Pass This Subject?}
\begin{itemize}
\item My \textit{intention} is that if:
	\begin{itemize}
		\item You attend (or watch) all lectures
		\item You attend all labs
		\item You read all course material
		\item You attempt extra problems \textit{in your own time}
		\item You follow all instructions in lab notes
		\item You chase up with demonstrators, me, textbooks, YouTube, etc any concepts you don't understand
	\end{itemize}
\item ...then you should confidently pass.
\item ie: You should have to work hard to earn a fail
\end{itemize}
\end{frame}

\begin{frame}
\frametitle{What Even \textit{is} a Pass?}
From the course outline: \\
~\\
Satisfactory standard indicating an adequate knowledge and understanding of the relevant materials; demonstration of an adequate level of academic achievement; satisfactory development of skills; and achivement of all learning outcomes.
\end{frame}

\begin{frame}
\frametitle{What is a High Distinction?}

Outstanding standard indicating comprehensive knowledge and understanding of the relevant materials; demonstration of an outstanding level of academic achievement; mastery of skills; and achivement of all assessment objectives.
\end{frame}

\begin{frame}
\frametitle{Grade Expectations}
\begin{itemize}
\item You get a ``pass'' if you learn enough that you won't \textit{obviously fail} later courses
\item You get a HD if you thoroughly understood \textit{everything} and can recall and apply any relevant piece of course content to a problem, showing a sound level of engineering judgement in doing so
\item If ``too many'' students earn Ds or HDs it is cause for change
\end{itemize}
\end{frame}

\begin{frame}
\frametitle{Revision}
\begin{itemize}
\item In mathematics, an \textit{iterative} (or \textit{recursive}) equation is written:
\begin{equation}
x_n = x_{n-1} + 1
\end{equation}
\item In programming, the change with time is implicit with program execution when we write:\\
\begin{equation}
\texttt{x = x + 1;}
\end{equation}
\item The \texttt{=} operator is \texttt{assignment} and overwrites (destroys) the variable's previous value
\end{itemize}
\end{frame}

\begin{frame}
\frametitle{Fibonacci Sequence}
\begin{itemize}
\item The \textit{Fibonacci Sequence} is the list of numbers, starting with 0 and 1, where each number is the sum of the two which came before it

\item ie: 0, 1, 1, 2, 3, 5, 8, 13, 21, ...

\item This sequence has interesting properties, eg:
	\begin{itemize}
		\item Its members appear in nature \textit{for some reason}
		\item The ratio of successive numbers converges towards the golden ratio: $\phi \approx 1.618$
		\item This has applications in art. eg: 16:10 screen ratio is 1.6 (16:9 is inferior and I can prove it mathematically)
	\end{itemize}
\end{itemize}
\end{frame}

\begin{frame}
\frametitle{Fibonacci Sequence}
\begin{itemize}
\item Mathematically, we can write this as a list of numbers, $x_0,x_1, x_2, x_3, ... , x_n$, where:
\begin{equation}
x_n = x_{n-1} + x_{n-2}
\end{equation}
and:
\begin{align}
x_0 &= 0\\
x_1 &= 1
\end{align}
\end{itemize}
\end{frame}

\begin{frame}
\frametitle{Arithmetic Sequences in General}
\begin{itemize}
\item Sequences seen in HSC mathematics are a subset of \textit{constant-recursive sequences}
\item \textit{Linear} sequences have the form:
\begin{equation}\label{eq:sequences}
x_n = b_1x_{n-1} + b_2x_{n-2} + ... + b_Nx_{n-N}
\end{equation}
\item Where $b_1$, $b_2$, etc are constant real numbers
\item ie: Each number, $x_n$, is a \textit{linear combination} of the $N$ numbers before it
\item The Fibonacci sequence is Equation \eqref{eq:sequences} with $b_1 = 1$, $b_2 = 1$, $N = 2$, $x_0 = 0$, and $x_1 = 1$.
\end{itemize}
\end{frame}

\begin{frame}
\frametitle{Aside: Digital Signal Processing}
\begin{itemize}
\item The previous equation is very similar to equations which underpin digital signal processing (DSP)
\item Digial \textit{filters} are of the form:
\begin{equation}\label{eq:dsp}
y_n = b_0x_n + ... + b_Nx_{n-N} - (a_0y_{n} + ... + a_Ny_{n-N})
\end{equation}
where $b_n$ and $a_n$ are numbers which influence the filter's behaviour, $y_n$ is the output signal, and $x_n$ is an input signal.
\end{itemize}
\end{frame}

\begin{frame}
\frametitle{Fibonacci Sequence}
\begin{itemize}
\item \textbf{Task:} Write a C program which outputs the Fibonacci Sequence for all integers small enough to fit into an \texttt{int}.
\item Lets break this into two problems:
	\begin{enumerate}
		\item Calculate the Fibonacci Sequence
		\item Worry about the stop condition
	\end{enumerate}
\item Always try to break programming problems down into small chunks
\item Real-world problems are too difficult to complete ``all in one go''
\end{itemize}
\end{frame}


\begin{frame}[fragile]
\frametitle{Fibonacci Sequence}
\begin{itemize}
	\item How do we calculate the Fibonacci Sequence?
	\item Note that we need to keep track of \textit{three} numbers:
		\begin{itemize}
			\item The next number, $x_n$
			\item The previous two numbers, $x_{n-1}$ and $x_{n-2}$
		\end{itemize}
	\item Lets also remember $n$
	\item I will use these variable names:
		\begin{lstlisting}[style=CStyle]
int n;
int xN;		// x N
int xNm1; // N minus 1
int xNm2; // N minus 2
\end{lstlisting}
\end{itemize}
\end{frame}

\begin{frame}
\frametitle{Fibonacci Sequence}
\begin{itemize}
\item Aside: We might want to \textit{remember} the whole sequence
\item This would require all numbers to be stored as unique variables
\item Declaring hundreds (or millions) of variables is impractical
\item The concept of \textit{arrays} will be introduced later to deal with this
\end{itemize}
\end{frame}

\begin{frame}[fragile]
\frametitle{Fibonacci Sequence}
\begin{itemize}
\item Lets sketch out what happens to these variables \textit{by hand}
\item Start at \texttt{n=2}, as that is the first unknown
\texttt{
\begin{table}[H]
\begin{tabular}{cccc}
n & xNm2 & xNm1 & xN \\
\hline
2 & 0    & 1    & 1  \\
3 & 1    & 1    & 2  \\
4 & 1    & 2    & 3  \\
5 & 2    & 3    & 5 \\
\end{tabular}
\end{table}
}
\item See the pattern? Numbers shift diagonally down to the left.
\end{itemize}
\end{frame}

\begin{frame}[fragile]
\frametitle{Fibonacci Sequence}
\begin{itemize}
\item Each time a new number is calculated, what happens to the variables?
\item All 3 variables change, the order in which they change is \underline{\textbf{crucial}}:
	\begin{lstlisting}[style=CStyle]
xN = xNm1 + xNm2; // Calculate next value
xNm2 = xNm1;//Move old values "down the chain"
xNm1 = xN;
\end{lstlisting}
\item With \texttt{=} the old \texttt{\textit{lvalue}} is lost
\item Note that the \textit{oldest} value is overwritten first
	\begin{itemize}
		\item It is the one which is no longer needed
	\end{itemize}
\end{itemize}
\end{frame}

\begin{frame}
\frametitle{L and R-Values}
\begin{itemize}
\item Oops, I used new jargon...
\item An \textit{lvalue} is something that goes to the left of an \texttt{=} operator
\item Likewise, the right side is an \textit{rvalue}
\item Not everything is a valid L- or R-value
\item Lvalues have to be variables
\item The following generate errors:
	\begin{itemize}
		\item \texttt{4 = x;}
		\item \texttt{rand() = 2;}
	\end{itemize}
\end{itemize}
\end{frame}

\begin{frame}[fragile]
\frametitle{Fibonacci Sequence}
\begin{itemize}
	\item Lets sketch some pseudocode to calculate the first 20 or so values:
	\begin{lstlisting}[style=pseudo]
BEGIN
	int xNm2 = 0
	int xNm1 = 1
	int xN
	int n = 2
	WHILE n < 20
		xN = xNm1 + xNm2
		PRINT xN
		n = n + 1
		xNm2 = xNm1
		xNm1 = xN
	ENDWHILE
END
	\end{lstlisting}
\end{itemize}
\end{frame}

\begin{frame}[fragile]
\frametitle{Fibonacci Sequence}
...and convert it to C:
\begin{table}[H]
\centering

\begin{tabular}{ll}
\hline

\begin{lstlisting}[style=pseudo,basicstyle=\ttfamily\footnotesize]
BEGIN
	int xNm2 = 0
	int xNm1 = 1
	int xN
	int n = 2
	WHILE n < 20
		xN = xNm1 + xNm2
		PRINT xN
		n = n + 1
		xNm2 = xNm1
		xNm1 = xN
	ENDWHILE
END
\end{lstlisting} &

\begin{lstlisting}[style=Ctable,basicstyle=\ttfamily\footnotesize]
int main() {
	int xNm2 = 0;
	int xNm1 = 1;
	int xN;
	int n = 2;
	while(n < 20) {
		xN = xNm1 + xNm2;
		printf("%d\n", xN);
		n++;
		xNm2 = xNm1;
		xNm1 = xN;
	}
}
\end{lstlisting}
\\

\hline
\end{tabular}
\end{table}
\end{frame}

\begin{frame}
\frametitle{Fibonacci Sequence}
\begin{itemize}
\item Does the code work?
\item Compare with: \url{https://www.wolframalpha.com/input/?i=first+20+fibonacci+sequence}
\item What about the 2nd requirement in the original problem?
\item How to tell if next value \textit{exceeds} an \texttt{int}?
\end{itemize}
\end{frame}

\begin{frame}[fragile]
\frametitle{Fibonacci Sequence}
\begin{itemize}
\item There are a few solutions:
	\begin{itemize}
		\item Calculate \texttt{xN}, see if result \textit{overflowed}
		\item Do calculation as \texttt{unsigned int} (or \texttt{long}) and compare with \texttt{INT\_MAX}
		\begin{itemize}
			\item \texttt{INT\_MAX} is defined in \texttt{limits.h}
			\item \texttt{\#include $<$limits.h$>$} if you want to use it
			\item It includes the line:
			\begin{lstlisting}[style=CStyle]
#define INT_MAX	2147483647
\end{lstlisting}
			\item We will learn about \texttt{\#define} later
		\end{itemize}
	\end{itemize}
\item In this context we can't run with a fixed iteration limit as that is ``unknown'' until the program is executed
\end{itemize}
\end{frame}

\begin{frame}[fragile]
\frametitle{Fibonacci Sequence}
\begin{itemize}
\item If overflow occurs when using \texttt{int} then the result of a calculation which \textit{should} be positive will be negative
\item Lets test for overflow with:\\
\begin{lstlisting}[style=CStyle]
if(xNm1 + xNm2 > 0) {
	// Do an iteration
}
\end{lstlisting}

\end{itemize}
\end{frame}

\begin{frame}[fragile]
\frametitle{Fibonacci Sequence}
\begin{lstlisting}[style=Ctable,basicstyle=\ttfamily\footnotesize]
int main() {
	int xNm2 = 0, xNm1 = 1;
	int xN;
	int n = 1;
	while(xNm1 + xNm2 > 0) {
		xN = xNm1 + xNm2;
		printf("%d\n", xN);
		n++;
		xNm2 = xNm1;
		xNm1 = xN;
	}
}
\end{lstlisting}
\begin{itemize}
\item \textbf{NB:} If \textit{optimisation} is enabled the calculation of \texttt{xNm1 + xNm2} will only occur once
\end{itemize}
\end{frame}

\begin{frame}[fragile]
\frametitle{Fibonacci Sequence}
Or, pre-testing overflow with \texttt{unsigned int}:
\begin{lstlisting}[style=Ctable,basicstyle=\ttfamily\footnotesize]
(unsigned int)xNm1 + (unsigned int)xNm2 < 2147483647u)
\end{lstlisting}
Lets try this one in Che...
\end{frame}

\begin{frame}[fragile]
\frametitle{DO ... WHILE}
\begin{itemize}
\item Same as WHILE except executes \textit{at least once}
\item The condition is tested at the end
\item Loops repeats if condition is TRUE
\item Pseudocode syntax:
\begin{lstlisting}[style=pseudo]
DO
	stuff
WHILE condition
\end{lstlisting}
\item C syntax:
\begin{lstlisting}[style=CStyle]
do {
	// do stuff
} while(condition);
\end{lstlisting}
\end{itemize}
\end{frame}

\begin{frame}[fragile]
\frametitle{DO ... WHILE}
\begin{itemize}
\item A toy example in C:
\begin{lstlisting}[style=CStyle]
int main() {
	int x = 0;
	do {
		x = x - 1;
	} while(x > 0);
	return 0;
}
\end{lstlisting}
\end{itemize}
\end{frame}

\begin{frame}[fragile]
\frametitle{DO ... WHILE}
\begin{itemize}
\item A slightly less toy example:
\begin{lstlisting}[style=CStyle]
#include <stdio.h>
int main() {
	int x;
	do {
		printf("Enter an integer: ");
		scanf("%d", &x);
		if(x%2==0)
			printf("%d is even\n", x);
		else
			printf("%d is odd\n", x);
	} while(x >= 0);
	return 0;
}
\end{lstlisting}
\end{itemize}
\end{frame}

\begin{frame}[fragile]
\frametitle{DO ... WHILE}
\begin{itemize}
\item \textbf{NB:} The previous example had:
\begin{lstlisting}[style=CStyle]
		if(x%2==0)
			printf("%d is even\n", x);
		else
			printf("%d is odd\n", x);
\end{lstlisting}
\item The \{ \} block is optional if \textit{only one statement} is after an \texttt{if()}, \texttt{while()}, etc
\item I omitted it to reduce line count so that the code would fit on the slide
\end{itemize}
\end{frame}

\begin{frame}[fragile]
\frametitle{DO ... WHILE is Optional}
\begin{itemize}
	\item It is never \textit{absolutely necessary}
	\item But sometimes it is easier or neater
\end{itemize}
\vspace{-5mm}
\begin{table}[H]
\centering

\begin{tabular}{ll}
\texttt{while()} & \texttt{do while();} \\
\hline

\begin{lstlisting}[style=Ctable,basicstyle=\ttfamily\scriptsize]
int x = 1;
while(x >= 0) {
	printf("Enter an integer: ");
	scanf("%d", &x);
	if(x%2==0)
		printf("%d is even\n", x);
	else
		printf("%d is odd\n", x);
}
\end{lstlisting} &

\begin{lstlisting}[style=Ctable,basicstyle=\ttfamily\scriptsize]
int x; // Uninitialised
do {
	printf("Enter an integer: ");
	scanf("%d", &x);
	if(x%2==0)
		printf("%d is even\n", x);
	else
		printf("%d is odd\n", x);
} while(x >= 0);
\end{lstlisting}
\\

\hline
\end{tabular}
\end{table}

\end{frame}


\begin{frame}[fragile]
\frametitle{FOR Loops}
\begin{itemize}
\item A FOR loop loops a given number of times
\item Typically used when the number of loop repeats is known \textit{before} entering the loop
	\begin{itemize}
		\item Repeat count could be ``hard coded'' as a number
		\item Could also be a variable
	\end{itemize}
\item Can be easier to read than WHILE
\item Example pseudocode syntax:
\begin{lstlisting}[style=pseudo]
FOR x = 1 to 10
	Do something ten times
ENDFOR
\end{lstlisting}
\item The \textit{loop variable} is automatically incremented
\end{itemize}
\end{frame}

\begin{frame}[fragile]
\frametitle{FOR Loops in C}
\begin{itemize}
\item The C FOR loop syntax is:
\begin{lstlisting}[style=CStyle]
for( initial ; condition ; increment ) {
	// Loop block
}
\end{lstlisting}
\item Where:
	\begin{itemize}
		\item \texttt{initial} is a statement executed \textit{once}
		\item \texttt{condition} is a statement executed and tested \textit{before} every loop iteration
		\item \texttt{increment} is a statement executed \textit{after} every loop iteration, but \textit{before} the \texttt{condition} is tested
	\end{itemize}
\end{itemize}
\end{frame}

\begin{frame}[fragile]
\frametitle{FOR Loops in C}
\begin{lstlisting}[style=CStyle]
for( x = 0 ; x < 10 ; x++ ) {
	printf("%d\n", x);
}
\end{lstlisting}
\begin{itemize}
\item Run this code
\item Observe that:
	\begin{itemize}
		\item 0 is printed
		\item 10 is \textbf{not} printed
		\item \texttt{x} increments automatically
	\end{itemize}

\end{itemize}
\end{frame}


\begin{frame}[fragile]
\frametitle{FOR Example 1 - Factorials}
\begin{itemize}
\item Use FOR to count from 2 to our input number
\item Keep a running product as we go
\begin{lstlisting}[style=pseudo]
BEGIN
	INPUT x
	result = 1
	FOR k = 2 TO x
		result = result * k
	ENDFOR
END
\end{lstlisting}
\item Is this algorithm robust? What happens if:
	\begin{itemize}
		\item x = -1
		\item x = 1
		\item x = 0 (\textbf{NB:} 0! = 1 because \textit{maths})
	\end{itemize}
\end{itemize}
\end{frame}

\begin{frame}
\frametitle{BREAK Statements}
\begin{itemize}
\item Sometimes you want to exit a loop \textit{before} the condition is re-tested
\item The flow-control mechanism for this is a BREAK statement
\item If executed, the loop quits
\item BREAKs typically go inside an IF
\item It adds an extra condition on loop exit placed at any point in the loop
\end{itemize}
\end{frame}

\begin{frame}[fragile]
\frametitle{FOR Example 2}
\begin{itemize}
\item Two equivalent ways to implement the $\sqrt{a}$ algorithm from last week:
\end{itemize}
{\small\textbf{NB:} $|$\texttt{tmp}$|$ means ``absolute value of tmp''.}
\begin{multicols}{2}
\begin{lstlisting}[style=pseudo,mathescape=true,basicstyle=\ttfamily\scriptsize]
BEGIN
	INPUT a
	x = a
	FOR n = 0 to 10
		xOld = x
		x = 0.2*(x + a/x)
		tmp = xOld - x
		IF |tmp| < 1e-6
			BREAK
		ENDIF
	ENDWHILE 
END
\end{lstlisting}
\columnbreak
\begin{lstlisting}[style=pseudo,mathescape=true,basicstyle=\ttfamily\scriptsize]
BEGIN
	INPUT a
	n = 0
	x = a
	WHILE (n<10)AND(|tmp|>1e-6)
		xOld = x
		x = 0.2*(x + a/x)
		tmp = xOld - x
		n = n + 1
	ENDWHILE 
END
\end{lstlisting}

\end{multicols}
\end{frame}



\begin{frame}[fragile]
\frametitle{FOR Loops in C (Advanced)}
\begin{itemize}
\item \texttt{for()} syntax allows multiple expressions in the \texttt{inital} / \texttt{condition} /\texttt{increment} sections
\item Separate expressions with commas
\item eg:
\begin{lstlisting}[style=CStyle]
int x, y=10;
for( x = 0 ; x < 10 ; x++, y++ ) {
	printf("x: %d y: %d\n", x, y);
}
\end{lstlisting}
\item This increments both \texttt{x} and \texttt{y} but only \texttt{x} is used in the condition
\end{itemize}
\end{frame}

\begin{frame}[fragile]
\frametitle{Loop \texttt{continue} Statements}
\begin{itemize}
	\item A \texttt{continue} causes execution to jump back to the loop start
	\item The \textit{condition} is tested before reentry	
	\item eg, run this with the debugger:
	\begin{lstlisting}[style=CStyle]
int x;
for(x = 0; x < 10; x++) {
	if(x%2 == 0)
		continue;
	printf("%d is odd\n");
}
\end{lstlisting}
\item {\small(Not the best example but gets the point across)}
\end{itemize}
\end{frame}

\begin{frame}
\frametitle{\texttt{break} and \texttt{continue}}
\begin{itemize}
\item Some programmers claim that \texttt{break} and \texttt{continue} are ``naughty''
\item Well, yes, but actually no
\item They can make your code needlessly complicated
\item They might make it simpler
\item It is up to you to judge
\item As engineers you shouldn't follow strict rules
\item Always try to choose the best tool for the job
\end{itemize}
\end{frame}

\begin{frame}
\frametitle{GOTO}
\begin{itemize}
\item There exists a GOTO flow control mechanism
	\begin{itemize}
		\item Sometimes also called a \textit{branch}
	\end{itemize}
\item It ``jumps'' from one line to a different line
	\begin{itemize}
		\item An ability some consider to be \textit{unnatural}
	\end{itemize}
\item It exists for a purpose
\item That purpose does not (typically) exist when writing C code
	\begin{itemize}
		\item C \textit{supports} a \texttt{goto} statement
		\item It results in ``spaghetti code'' which is hard to read
		\item Don't use it in ENGG1003
	\end{itemize}
\item You \textit{must} use branch instructions in ELEC1710
\end{itemize}
\end{frame}

\begin{frame}[fragile]
\frametitle{Loose End Increment Example}
\begin{lstlisting}[style=CStyle]
#include <stdio.h>
int main() {
	int x = 0;
	int y = 0;
	int z = 0;
	y = ++x + 10;
	printf("Pre-increment: %d\n", y);
	y = z++ + 10;
	printf("Post-increment: %d\n", y);
	return 0;
}
\end{lstlisting}
Pre/post-inc/decrements have many applications, more details in coming weeks.
\end{frame}

\begin{frame}
\frametitle{Binary Nomenclature}
\begin{itemize}
\item The value range of datatypes is a result of the underlying binary storage mechanism
\item A single binary digit is called a \textit{bit}
\item There are 8 bits in a \textit{byte}
\item In programming we use the ``power of two'' definitions of kB, MB, etc:
	\begin{itemize}
		\item 1 kilobyte is $2^{10} = 1024$ bytes
		\item 1 Megabyte is $2^{20} = 1048576$ bytes
		\item 1 Gigabyte is $2^{30} = 1073741824$ bytes
		\item (Advanced) These numbers look better in hex: \texttt{$2^{10}$=0x400}, \texttt{$2^{20}$=0x100000}, etc.
	\end{itemize}
\end{itemize}
\end{frame}

\begin{frame}
\frametitle{Binary Nomenclature}
\begin{itemize}
\item Observe that kilobyte, Megabyte, Gigabyte, etc use scientific prefixes
\item These \textit{normally} mean a power of 10:
	\begin{itemize}
		\item kilo- = $10^3$
		\item Mega- = $10^6$
		\item Giga- = $10^9$
		\item ...etc (see the inside cover of a physics text)
	\end{itemize}
\item Computer science stole these terms and re-defined them

\end{itemize}
\end{frame}

\begin{frame}
\frametitle{Binary Nomenclature}
\begin{itemize}
\item This has made some people \textit{illogically angry}
\item Instead, we can use a more modern standard:
	\begin{itemize}
		\item $2^{10}$ bytes = 1 kibiByte (KiB)
		\item $2^{20}$ bytes = 1 Mebibyte (MiB)
		\item $2^{30}$ bytes = 1 Gibibyte (GiB)
		\item ...etc
	\end{itemize}
\item Generally speaking, KB (etc) implies:
	\begin{itemize}
		\item powers of two to \textit{engineers}
		\item powers of ten to \textit{marketing}
			\begin{itemize}
				\item The number is smaller
				\item Hard drive manufacturers, ISPs, etc like this
			\end{itemize}
	\end{itemize}
\end{itemize}
\end{frame}

\begin{frame}
\frametitle{Unambiguous Integer Data Types}
\begin{itemize}
\item Because the standard \texttt{int} and \texttt{long} data types don't have fixed size unambiguous types exist
\item Under \texttt{gcc} these are defined in \texttt{stdint.h} (\texttt{\#include} it)
\item You will see them used commonly in embedded systems programming (eg: Arduino code)
\item The types are:
	\begin{itemize}
		\item \texttt{int8\_t}
		\item \texttt{uint8\_t}
		\item \texttt{int16\_t}
		\item ...etc
	\end{itemize}
\end{itemize}
\end{frame}

\begin{frame}[fragile]
\frametitle{Code Blocks in C}
\begin{itemize}
\item Semi-revision:
\item The curly braces \{ \} encompass a \textit{block}
\item You have used these with \texttt{if()} and \texttt{while()}
\item They define the set of lines executed inside the \texttt{if()} or \texttt{while()}

\end{itemize}
\end{frame}

\begin{frame}[fragile]
\frametitle{Code Blocks in C}
\begin{itemize}
\item You can place blocks anywhere you like
\item Nothing wrong with:
\begin{lstlisting}[style=CStyle]
int main() {
	int x;
	{
		printf("%d\n", x);
	}
	return 0;
}
\end{lstlisting}
\item This just places the \texttt{printf();} inside a block
\item It doesn't do anything useful, but...
\end{itemize}
\end{frame}

\begin{frame}[fragile]
\frametitle{Variable Scope}
\begin{itemize}
\item A variable's ``existence'' is limited to the block where it is declared
	\begin{itemize}
		\item Plus any blocks within that one
	\end{itemize}
\item Example this code won't compile:
\begin{lstlisting}[style=CStyle]
#include <stdio.h>
int main() {
	int x = 2;
	if(x == 2) {
		int k;
		k = 2*x;
	}
	printf("%d\n", k);
	return 0;
}
\end{lstlisting}
\end{itemize}
\end{frame}

\begin{frame}
\frametitle{Variable Scope}
\begin{itemize}
\item Note that \texttt{k} was declared inside the \texttt{if()}
\item That means that it no longer exists when the \texttt{if()} has finished
\item This generates a compiler error
\item It frees up some RAM
\item It also lets the variable's name be reused elsewhere
	\begin{itemize}
		\item This can be \textit{really} confusing. Be careful.
	\end{itemize}
\end{itemize}
\end{frame}

\begin{frame}
\begin{center}
Oh, end of the lecture already?
Lets go read the lab notes...
\end{center}
\end{frame}



\end{document}
