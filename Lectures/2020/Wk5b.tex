\documentclass[14pt]{beamer}
\usetheme{Dresden}
\usecolortheme{orchid}

\usepackage{xcolor}
\usepackage{listings}
\usepackage{courier}
\usepackage{graphicx}
\usepackage{amsmath}
\usepackage{algorithm2e}
\usepackage{multicol}
\usepackage{amssymb}

\usefonttheme[onlymath]{serif}

\definecolor{mGreen}{rgb}{0,0.6,0}
\definecolor{mGray}{rgb}{0.5,0.5,0.5}
\definecolor{mPurple}{rgb}{0.8,0,0.82}
\definecolor{backgroundColour}{rgb}{0.95,0.95,0.92}
\definecolor{lightBlue}{rgb}{0.1, 0.1, 0.8}

\lstdefinestyle{CStyle}{
    backgroundcolor=\color{backgroundColour},   
    commentstyle=\color{mGreen},
    keywordstyle=\color{magenta},
    numberstyle=\tiny\color{mGray},
    stringstyle=\color{mPurple},
    basicstyle=\footnotesize\ttfamily,
    breakatwhitespace=false,         
    breaklines=true,                 
    captionpos=b,                    
    keepspaces=true,                 
    numbers=left,                    
    numbersep=5pt,                  
    showspaces=false,                
    showstringspaces=false,
    showtabs=false,                  
    tabsize=2,
    language=C
}

\lstdefinestyle{Ctable}{
    backgroundcolor=\color{backgroundColour},   
    commentstyle=\color{mGreen},
    keywordstyle=\color{magenta},
    numberstyle=\tiny\color{mGray},
    stringstyle=\color{mPurple},
    basicstyle=\footnotesize\ttfamily,
    breakatwhitespace=false,         
    breaklines=true,                 
    captionpos=b,                    
    keepspaces=true,                                  
    showspaces=false,                
    showstringspaces=false,
    showtabs=false,                  
    tabsize=2,
    language=C
}

\lstdefinestyle{pseudo}{
        basicstyle=\ttfamily\footnotesize,
        keywordstyle=\color{lightBlue},
        morekeywords={BEGIN,END,IF,ELSE,ENDIF,ELSEIF,PRINT,WHILE,RETURN,ENDWHILE,DO,FOR,TO,IN,ENDFOR,BREAK,INPUT,READ},
        morecomment=[l]{//},
        commentstyle=\color{mGreen}
}

\lstset{basicstyle=\footnotesize\ttfamily,breaklines=true}
\lstset{framextopmargin=50pt,tabsize=2}

\title{ENGG1003 - Tuesday Week 5}
\subtitle{Arrays and Functions:\\Together at Last!\\{\tiny{Does anyone even read the title page?}}\\Also: Maybe Strings \& ASCII Codes}
\author{Brenton Schulz}
\institute{University of Newcastle}
\date{\today}


\begin{document}
\titlepage

\begin{frame}
\frametitle{The Story So Far}
\begin{itemize}
\item Course summary:
	\begin{itemize}
		\item Flow control
			\begin{itemize}
				\item \texttt{if()}
				\item \texttt{while()}
				\item \texttt{for()}
				\item \texttt{switch()}
			\end{itemize}
		\item Variables and data types
		\item Functions
		\item Arrays
	\end{itemize}
\item Today: Arrays and functions together
	\begin{itemize}
		\item Subtext: Pointers
	\end{itemize}
\item Today (maybe): Strings
\item Later: File input-output (I/O)
\end{itemize}
\end{frame}

\begin{frame}
\frametitle{Programming Assignment And Quiz}
\begin{itemize}
\item The programming assignment will use everything from the previous slide
\item The quiz can include everything up to, and including, the Week 5 Tuesday lecture
	\begin{itemize}
		\item Details TBA because the old plan of holding it in a lecture theater can't be done anymore
		\item I will be held during the lecture time (6pm Tues March 31st)
	\end{itemize}
\end{itemize}
\end{frame}

\begin{frame}[fragile]
\frametitle{Arrays and Functions}
\begin{itemize}
\item Last lecture:
	\begin{itemize}
		\item Studied arrays
		\item Studied functions
		\item Didn't mix them
	\end{itemize}
\pause
\item There are two ways to pass arrays to functions:
\vspace{-5mm}
	\begin{itemize}
	\pause
		\item Pass an array element, eg:
		\begin{lstlisting}[style=CStyle]
int function(int x);
// ...
int array[12];
// ...
function(array[6]);
\end{lstlisting}
\pause
		\item Give a function a \textit{pointer} to an array
			\begin{itemize}
				\item Ok, lets break this one down a bit...
			\end{itemize}
	\end{itemize}
\end{itemize}
\end{frame}

\begin{frame}
\frametitle{Arrays and Functions}
\begin{itemize}
\item Firstly: why don't we pass a whole array?
\pause
	\begin{itemize}
		\item Arrays can be \textit{huge}
		\item Passing a whole array \textit{copies} everything
		\item This is a bad idea so C doesn't support it
		\item (Advanced) Arguments are put to the \textit{stack}
			\begin{itemize}
				\item Google stack Vs heap memory allocation for more information. This is beyond ENGG1003.
			\end{itemize}
	\end{itemize}
\pause
\item Instead, C passes a \textit{pointer}
	\begin{itemize}
		\item This is the \textit{memory address} of the array's start
		\pause
		\item In C, \texttt{name} is equivalent to \texttt{\&name[0]}
	\end{itemize}
\end{itemize}
\end{frame}

\begin{frame}[fragile]
\frametitle{Arrays in Memory}
\begin{itemize}
\item Review: When we declare an array, eg,
\begin{lstlisting}[style=CStyle]
int x[20];
\end{lstlisting}
the compiler allocates 20*sizeof(int) = 80 bytes to store it
\item The \textit{memory address} of \texttt{x[0]} is some seemingly random number, \texttt{p}
\item \texttt{p} is a \textit{byte} address
\item Other elements are stored in sequential memory addresses:
	\begin{itemize}
		\item The address of \texttt{x[1]} is \texttt{p + 4}
		\item The address of \texttt{x[i]} is \texttt{p + i*4}
	\end{itemize}
\end{itemize}
\end{frame}

\begin{frame}[fragile]
\frametitle{Arrays in Memory}
\begin{itemize}
\item Therefore, to access a given element, \texttt{i}, of an array all we need is:
	\begin{itemize}
		\item A pointer, \texttt{p} to the first element
		\item Knowledge of the arrays \textit{data type}
			\begin{itemize}
				\item Specifically, the type's \textit{size}
			\end{itemize}
		\item The calculation result of \texttt{p + i*size}
	\end{itemize}
\pause
\item So that's what we do with functions:
	\begin{itemize}
		\item The function argument is a \textit{pointer} to a \textit{data type}
	\end{itemize}
\pause
\item C syntax:
\begin{lstlisting}[style=CStyle]
return_type function_name(data_type *varName);
\end{lstlisting}
\item Key syntax element: the \texttt{*} character
\pause
\item Inside the function use \texttt{var[i]} syntax
\end{itemize}
\end{frame}

\begin{frame}
\frametitle{Key Points}
\begin{itemize}
\item Because arrays are passed via a pointer the function gets \textit{the actual array}
\item Modifying the array in the function modifies the original variable
\item You don't \textit{need} a return value
	\begin{itemize}
		\item In a technically incorrect way: all the array's elements are ``returned''
	\end{itemize}
\end{itemize}
\end{frame}

\begin{frame}
\frametitle{Example}
\begin{itemize}
\item Write a function which zeros the first N elements of an array of \texttt{int}s
\begin{itemize}
	\item Function prototype:
	\pause
		\begin{itemize}
			\item \texttt{void zero(int *x, int N);}
		\end{itemize}
	\pause
	\item The value of \texttt{N} is needed because C won't tell you how long an array is \textit{within the context of the function}
		\begin{itemize}
			\item (Advanced) \texttt{sizeof(x)} will just be the size of the pointer -  4, or 8 bytes
		\end{itemize}
	\end{itemize}
\end{itemize}
\end{frame}

\begin{frame}[fragile]
\frametitle{Example}
\begin{itemize}
\item Function definition:
\begin{lstlisting}[style=CStyle]
// Zeros first N elements of x
void zero(int *x, int N) {
	int i; // Array index loop counter
	for(i = 0; i < N; i++)
		x[i] = 0; // Use array syntax
	return; // Optional
}
\end{lstlisting}
\end{itemize}
\end{frame}

\begin{frame}
\frametitle{Other Examples}
\begin{itemize}
\item Lets write and test these live...
\item Write a function which:
	\begin{itemize}	
		\item Returns the sum of an array of length N
		\item Returns the maximum value in an array of length N
		\item Fills an array with integers between two given numbers \texttt{min} and \texttt{max}
			\begin{itemize}
				\item Prototype:\\\texttt{void countArray(int *x,\\\quad \quad \quad \quad int min, int max);}
				\item eg: countArray(x, 10, 15) sets:\\ \texttt{x[] = \{10, 11, 12, 13, 14, 15\}}
			\end{itemize}
	\end{itemize}
\end{itemize}
\end{frame}

\end{document}
