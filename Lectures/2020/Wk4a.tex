\documentclass[14pt]{beamer}
\usetheme{Dresden}
\usecolortheme{orchid}

\usepackage{xcolor}
\usepackage{listings}
\usepackage{courier}
\usepackage{graphicx}
\usepackage{amsmath}
\usepackage{algorithm2e}
\usepackage{multicol}
\usepackage{amssymb}

\usefonttheme[onlymath]{serif}

\definecolor{mGreen}{rgb}{0,0.6,0}
\definecolor{mGray}{rgb}{0.5,0.5,0.5}
\definecolor{mPurple}{rgb}{0.8,0,0.82}
\definecolor{backgroundColour}{rgb}{0.95,0.95,0.92}
\definecolor{lightBlue}{rgb}{0.1, 0.1, 0.8}

\lstdefinestyle{CStyle}{
    backgroundcolor=\color{backgroundColour},   
    commentstyle=\color{mGreen},
    keywordstyle=\color{magenta},
    numberstyle=\tiny\color{mGray},
    stringstyle=\color{mPurple},
    basicstyle=\footnotesize\ttfamily,
    breakatwhitespace=false,         
    breaklines=true,                 
    captionpos=b,                    
    keepspaces=true,                 
    numbers=left,                    
    numbersep=5pt,                  
    showspaces=false,                
    showstringspaces=false,
    showtabs=false,                  
    tabsize=2,
    language=C
}

\lstdefinestyle{Ctable}{
    backgroundcolor=\color{backgroundColour},   
    commentstyle=\color{mGreen},
    keywordstyle=\color{magenta},
    numberstyle=\tiny\color{mGray},
    stringstyle=\color{mPurple},
    basicstyle=\footnotesize\ttfamily,
    breakatwhitespace=false,         
    breaklines=true,                 
    captionpos=b,                    
    keepspaces=true,                                  
    showspaces=false,                
    showstringspaces=false,
    showtabs=false,                  
    tabsize=2,
    language=C
}

\lstdefinestyle{pseudo}{
        basicstyle=\ttfamily\footnotesize,
        keywordstyle=\color{lightBlue},
        morekeywords={BEGIN,END,IF,ELSE,ENDIF,ELSEIF,PRINT,WHILE,RETURN,ENDWHILE,DO,FOR,TO,IN,ENDFOR,BREAK,INPUT,READ},
        morecomment=[l]{//},
        commentstyle=\color{mGreen}
}

\lstset{basicstyle=\footnotesize\ttfamily,breaklines=true}
\lstset{framextopmargin=50pt,tabsize=2}

\title{ENGG1003 - Monday Week 4}
\subtitle{\texttt{switch() \{ case: \}} \\ Functions}
\author{Brenton Schulz}
\institute{University of Newcastle}
\date{\today}


\begin{document}
\titlepage

\begin{frame}
\frametitle{Subscript Notation}
\begin{itemize}
\item Last chance to learn that we use:
\begin{equation}
x_1, x_2, x_3, ... , x_n
\end{equation}
and
\begin{equation}
x_n = x_{n-1} + x_{n-2}
\end{equation}
notation because it is the simplest method that gets the point across.
\end{itemize}
\end{frame}

\begin{frame}[fragile]
\frametitle{Subscript Notation}
\begin{itemize}
\item $x_n$ means that $x$ is ``some number'' and $n$ is an \textit{integer} value
\item $n$ implies \textit{uniqueness} (ie: $x_1$ and $x_2$ can differ)
\item $n$ implies an \textit{order} to the $x$'s
\item A formal mathematical statement of the above would be something like:
\begin{equation}
x_n :~x \in \mathbb{R}~\textrm{and} ~ n \in \mathbb{Z}
\end{equation}
\item $\mathbb{R}$ is the set of real numbers
\item $\mathbb{Z}$ is the set of all integers
\end{itemize}
\end{frame}

\begin{frame}[fragile]
\frametitle{Subscript Notation}
\begin{itemize}
\item Without this notation it is \textit{really} hard to write things like:
\begin{equation}
x_n = x_{n-1} + x_{n-2}
\end{equation}
\pause
\item If you instead wrote:\\
``Calculate a sequence of numbers, $a,b,c,d,...$''\\
how would you write the equation?
\end{itemize}
\end{frame}

\begin{frame}
\frametitle{Considering Dropping?}
\begin{itemize}
\item HECS census is Fri 22nd
\item Before you drop:
	\begin{itemize}
		\item Talk to me
		\item Are you \textit{legitimately} unprepared or experiencing ``imposter syndrome''?
			\begin{itemize}
				\item It is surprisingly common
			\end{itemize}
		\item Most of you have to pass eventually
		\item There are some legitimate reasons
	\end{itemize}
\item Ignore unsolicited advice from demonstrators
	\begin{itemize}
		\item Seriously, this isn't their job
	\end{itemize}
\end{itemize}
\end{frame}

\begin{frame}[fragile]
\frametitle{\texttt{switch()} - \texttt{case:}}
\begin{itemize}
\item Sometimes you want to code something like:
\begin{lstlisting}[style=CStyle]
if(x == 0) {
	// stuff
} else if(x == 1) {
	// stuff
} else if(x == 2) {
	// stuff
} ...etc
\end{lstlisting}
\item This is difficult to read and gets unwieldy. Fast.
\end{itemize}
\end{frame}

\begin{frame}[fragile]
\frametitle{\texttt{switch()} - \texttt{case:}}
\begin{itemize}
\item Instead, C has:
\begin{lstlisting}[style=CStyle]
switch(expression) {
	case constant:
			break;
	case constant:
			break;
	default:
}
\end{lstlisting}
\item The \textit{expression} is anything which evaluates to a number
\item The \textit{constant}s are either literals or variables declared as \texttt{const} (covered later)
\end{itemize}
\end{frame}

\begin{frame}[fragile]
\frametitle{\texttt{switch()} - \texttt{case:} Example}
\begin{lstlisting}[style=CStyle]
int x=1, y=2;

switch(x==y) { // Evaluates to 0 or 1
	case 0:
			printf("x and y differ\n");
			break;
	case 1:
			printf("x and y are equal\n");
			break;
	default:
			printf("Something went very wrong\n");
}
\end{lstlisting}
\begin{itemize}
\item The \texttt{default:} case happens if the expression doesn't match any \texttt{case} statements
\end{itemize}
\end{frame}

\begin{frame}[fragile]
\frametitle{\texttt{switch()} - \texttt{case:} Example}
\begin{itemize}
\item If the \texttt{break;} is omitted execution continues line by line - example:
\end{itemize}
\begin{lstlisting}[style=CStyle]
#include<stdio.h>
int main() {
	int x = 2;
	switch(x) {
		case 1: printf("x is 1\n");
		case 2: printf("x is 2\n");
		case 3: printf("x is 3\n");
		default: printf("x is not 1, 2, or 3\n");
	}
	return 0;
}
\end{lstlisting}
\end{frame}

\begin{frame}[fragile]
\frametitle{\texttt{switch()} - \texttt{case:} Limits}
\begin{itemize}
\item Because the \texttt{case} statements only accept \textit{constants} there are some limitations
\item Example, this doesn't translate well:
\begin{lstlisting}[style=CStyle]
if(x < 0) {
	// stuff
} else if (x == 0) {
	// stuff
} else if (x > 0) {
	// stuff
}
\end{lstlisting}
\item \texttt{(x<0)}, \texttt{(x==0)}, and \texttt{(x>0)} are all 0 or 1
\item Can't \textit{easily} translate this into three unique constants
\end{itemize}
\end{frame}

\begin{frame}
\frametitle{Functions}
\begin{itemize}
\item A \textit{function} is a block of code which can be \textit{called} multiple times, from multiple places
\item They are used when you want the same block of code to execute in many places throughout your code
\item A function requires:
	\begin{itemize}
		\item A name
		\item (optional) A \textit{return value}
		\item (optional) One or more \textit{arguments}
	\end{itemize}
\end{itemize}
\end{frame}

\begin{frame}
\frametitle{Functions in Mathematics}
\begin{itemize}
\item In mathematics you saw functions written as:
\begin{equation*}
y = f(x)
\end{equation*}
\item Here, the function is called $f$, takes an argument of $x$ and returns a value which is assigned to $y$
\item C and pure mathematics have these general ideas in common
\pause
\item The similarities stop there
\end{itemize}
\end{frame}

\begin{frame}[fragile]
\frametitle{Functions in Programming}
\begin{itemize}
\item When a function is called:
	\begin{enumerate}
		\item Program execution jumps to the function
		\item The function's code is executed
		\item Program execution jumps back to where it left off
			\begin{itemize}
				\item In C, the function will jump back when it hits a \texttt{return} statement or the end of the function's code
				\item Functions which return a value \textit{must} have a \texttt{return} statement
				\item Functions which return \texttt{void} (ie: nothing) don't
			\end{itemize}
	\end{enumerate}
\item The code inside the function can be any valid C code, not just mathematics
\end{itemize}
\end{frame}

\begin{frame}[fragile]
\frametitle{\texttt{return} Statements}
\begin{itemize}
\item A function stops executing when it reaches a \texttt{return} statement
\item The \texttt{return} statement has the following syntax:
\begin{lstlisting}
return [variable or literal];
\end{lstlisting}
\item The variable or literal is optional
	\begin{itemize}
		\item Omitting it requires the function's return type to be \texttt{void}
	\end{itemize}
\item Examples:
\begin{lstlisting}[style=CStyle]
return x; // The value of x is passed back
return 1; // The constant "1" is passed back
return; // Nothing (void) is returned
\end{lstlisting}
\end{itemize}
\end{frame}

\begin{frame}
\frametitle{Function Examples}
\begin{itemize}
\item So far, some of you have used \textit{library functions}
\item These are functions which are pre-existing within the compiler (and its libraries)
\item I have shown you:
	\begin{itemize}
		\item \texttt{scanf();}
		\item \texttt{printf();}
		\item \texttt{rand();}
	\end{itemize}
\end{itemize}
\end{frame}

\begin{frame}[fragile]
\frametitle{Function Syntax}
\begin{itemize}
\item Function call syntax is:\\
{\small \texttt{name([arguments])} }
\item Not all functions take arguments
\item The function ``turns into'' its \textit{return value}
	\begin{itemize}
		\item This value can be assigned to a variable, used in an expression, or ignored
	\end{itemize}
\pause
\item Writing \texttt{rand()} in you code is \textit{calling} the function
\item The program execution ``jumps'' into the function's code, executes it, then jumps back
\end{itemize}
\end{frame}

\begin{frame}[fragile]
\frametitle{Function Examples}
\begin{itemize}
\item Example 1:
\begin{lstlisting}[style=CStyle]
x = rand();
\end{lstlisting}
	\begin{itemize}
		\item \texttt{rand} is the function name
		\item It returns a ``random'' integer
		\item The return value is assigned to \texttt{x}
		\item It doesn't take an argument 	
	\end{itemize}
\pause
\item Example 2:
\begin{lstlisting}[style=CStyle]
y = sqrtf(x);
\end{lstlisting}
	\begin{itemize}
		\item \texttt{sqrtf} is the function name
		\item \texttt{x} is the argument
		\item It returns the square root of \texttt{x}
		\item The return value is assigned to \texttt{y}
	\end{itemize}
\end{itemize}
\end{frame}

\begin{frame}
\frametitle{Functions}
\begin{itemize}
\item Function arguments and return values have pre-defined data types
\pause
\item Example from documentation
\begin{itemize}
\item \texttt{int~rand(void);}
	\begin{itemize}
		\item The return value is an \texttt{int}
		\item The argument is type \texttt{void}
		\item This just means ``there are no arguments''
	\end{itemize}
\pause
\item \texttt{float sqrtf(float x);}
	\begin{itemize}
		\item The return value is a \texttt{float}
		\item The argument is a \texttt{float}
		\item Argument is called \texttt{x} in documentation but you can pass it any \texttt{float} variable or literal
	\end{itemize}
\end{itemize}
\end{itemize}
\end{frame}

\begin{frame}
\frametitle{Return Values (an Engineer's View)}
\begin{itemize}
\item The function's \textit{return value} is the number a function gets ``replaced with'' in a line of code
\pause
\item Function return values, variables, and literals can all be used in the same places:
	\begin{itemize}
		\item In arithmetic
		\item In conditions
		\item As arguments to other functions
	\end{itemize}	 
\pause
\item The C standard is \textit{very} specific about what return values are but I will be informal for now
	\begin{itemize}
		\item Technically, for example, an expression like \texttt{x=y+5.0;} also has a ``return value`` equal to the value assigned to \texttt{x}
	\end{itemize}
\end{itemize}
\end{frame}

\begin{frame}
\frametitle{Return Values}
\begin{itemize}
\item I use functions from \texttt{math.h} in these examples, we'll cover them in a few slides
	\begin{itemize}
		\item I should also discuss \texttt{.h} ``header'' files later, too
	\end{itemize}
\pause
\item The following are all valid:
	\begin{itemize}
		\item \texttt{x = rand();}
		\item \texttt{printf("\%f\textbackslash n", sin(y));}
		\item \texttt{if( (rand()\%6) < 2)}
		\item \texttt{while( sin(x) < 0 )}
		\pause
		\item This next one is complicated...
		\pause
		\item \texttt{x = sin((double)rand());}
			\begin{itemize}
			\pause
				\item Generates a random integer, casts to \texttt{double}, uses that number as an argument to the \texttt{sin()} function
			\end{itemize}
	\end{itemize}
\end{itemize}
\end{frame}

\begin{frame}
\frametitle{Using Functions}
\begin{itemize}
\item Before you use a function you must:
	\begin{itemize}
		\item Read the documentation
		\item \texttt{\#include} the correct header file
		\item Add the correct library to the compiler options
			\begin{itemize}
				\item CodeBlocks \textit{links} to the math library when \textit{linking} with \texttt{g++}
				\item \texttt{stdio} and \texttt{stdlib} are always included
			\end{itemize}
		\item Be aware of the data types
			\begin{itemize}
				\item Do you need any type casting?
				\item Are you using the correct function?
			\end{itemize}
	\end{itemize}
\end{itemize}
\end{frame}

\begin{frame}
\frametitle{Maths Functions}
\begin{itemize}
\item Since some of you have already used them, lets learn about the maths library...
\item It includes functions for:
	\begin{itemize}
		\item Trigonometry
		\item Exponentials (base e) \& logarithms (base e, 10, 2)
		\item Exponents (\texttt{pow();})
		\item Rounding (\texttt{floor();} \& \texttt{ceil();})
		\item Floating point modulus (\texttt{fmod();})
			\begin{itemize}
				\item Modulus and modulo are poorly defined in common language. This function is a ``floating point remainder'' and not ``absolute value''
			\end{itemize}
		\item Square roots
		\item ...etc
	\end{itemize}
\end{itemize}
\end{frame}

\begin{frame}
\frametitle{Maths Functions}
\begin{itemize}
\item There are typically \textit{different} functions for \texttt{float} and \texttt{double}
\item This can have a huge speed impact
\item Use the right ones!
\item \texttt{float} maths functions typically end in '\texttt{f}'
	\begin{itemize}
		\item \texttt{cosf();}
		\item \texttt{sqrtf();}
		\item \texttt{atanf();}
		\item ...etc
	\end{itemize}
\item \texttt{double} maths functions \textbf{don't}
	\begin{itemize}
		\item \texttt{cos();}
		\item \texttt{log();}
	\end{itemize}
\end{itemize}
\end{frame}

\begin{frame}
\frametitle{Maths Functions}
\begin{itemize}
\item Math functions are written by mathematicians
	\begin{itemize}
		\item All angles are in radians
		\pause
		\item \texttt{log();} is $\log_e$
			\begin{itemize}
				\item \texttt{log10();} is $\log_{10}$
				\item \texttt{log2();} is $\log_2$
			\end{itemize}
		\pause
		\item Inverse trig functions are called ``arcus functions''
			\begin{itemize}
				\item $\sin^{-1}$ is \texttt{asin();}
				\item $\cos^{-1}$ is \texttt{acos();}
				\item $\tan^{-1}$ is \texttt{atan();}
			\end{itemize}
		\pause
		\item The ``4 quadrant'' arctan function is \texttt{atan2();}
			\begin{itemize}
				\item \texttt{atan(x);} returns $[-\pi/2,\pi/2]$
				\item \texttt{atan2(x,y);} returns $[-\pi, \pi]$ depending on the quadrant of the point \texttt{x,y}
				\item Very useful for polar to Cartesian coordinate transforms (probably beyond 1st semester 1st year)
			\end{itemize}
	\end{itemize}
\end{itemize}
\end{frame}

\begin{frame}
\frametitle{Example - Quadratic Equation}
Write a C program which uses the standard library function \texttt{sqrtf();} as part of the calculations required to produce solutions to a quadratic equation:
\begin{equation}
ax^2 + bx + c = 0
\end{equation}
using
\begin{equation}
x = \frac{-b \pm \sqrt{b^2 - 4ac}}{2a}
\end{equation}
\end{frame}

\begin{frame}
\begin{center}
...do it live
\end{center}
\end{frame}

\begin{frame}
\frametitle{Example - \texttt{scanf();}'s Return Value}
Read the \texttt{scanf();} \underline{\href{http://man7.org/linux/man-pages/man3/scanf.3.html}{documentation}} and observe that it returns an \texttt{int}. What does that \texttt{int} represent? Write some test code and experiment with its behaviour.
\\~\\
Demonstrate it live...
\end{frame}

\begin{frame}
\frametitle{Writing Functions}
\begin{itemize}
\item What about writing your own functions?
\pause
\item Do the following:
	\begin{enumerate}
		\item Define (for yourself) what the function needs to do
		\pause
		\item Choose a name
		\pause
		\item Decide on the function arguments
		\pause
		\item Decide on the return value
		\pause
		\item Write a \textit{function prototype}
			\begin{itemize}
				\item Write it at the top of your code [or in a header file]
			\end{itemize}
		\pause
		\item Somewhere below \texttt{main()} (or in another \texttt{.c} file) write the function \textit{definition}
	\end{enumerate}
\pause
\item For now just keep everything in one file
	\begin{itemize}
		\item Unless you study ahead. I won't stop you.
	\end{itemize}
\end{itemize}
\end{frame}

\begin{frame}
\frametitle{Function Prototypes}
\begin{itemize}
\item Huh? What's a function prototype?
\pause
\item Before a function is \textit{called} the compiler needs to know:
	\begin{itemize}
		\item Its name
		\item Its argument's data type(s)
		\item Its return data type 
	\end{itemize}
\pause
\item A function prototype documents these things for the compiler
\end{itemize}
\end{frame}

\begin{frame}[fragile]
\frametitle{Function Prototypes}
\begin{itemize}
\item The function prototype syntax is:
\begin{lstlisting}[style=CStyle]
<returned data type> function_name(arguments);
\end{lstlisting}
\pause
\item The \texttt{arguments} section is a comma separated list with the following syntax:
\begin{lstlisting}[style=CStyle]
(datatype name, datatype name, ...)
\end{lstlisting}
\item Examples:
	\begin{itemize}
		\item \texttt{float sqrtf(float x);}
		\item \texttt{int rand(void);}
		\item \texttt{double log(double x);}
		\item \texttt{double atan2(double x, double y);}
	\end{itemize}
\end{itemize}
\end{frame}

\begin{frame}
\frametitle{Void}
\begin{itemize}
\item If either the arguments or return value aren't required declare them as \texttt{void}
\item This is an explicit way of saying ``this item doesn't exist'' 
\end{itemize}
\end{frame}

\begin{frame}
\frametitle{Function Prototypes}
\begin{itemize}
\item The function prototype must be \textit{before} the function's first use
\item For ``small'' projects: above \texttt{main()}
\item For ``big'' projects: in their own \textit{header file}
	\begin{itemize}
		\item We'll cover this later
	\end{itemize}
\item Don't leave the prototype's arguments blank
	\begin{itemize}
		\item The compiler won't complain but it is a deprecated language feature
	\end{itemize}
\end{itemize}
\end{frame}

\begin{frame}[fragile]
\frametitle{Function Definitions}
\begin{itemize}
\item The function prototype tells the compiler how the function interacts with other code
\item The function definition is the \textit{actual code} that gets executed when the function is called
\begin{lstlisting}[style=CStyle]
int add(int a, int b); // Prototype

main() {
	// do stuff
}

int add(int a, int b) { // Definition
	return a + b;
}
\end{lstlisting}
\end{itemize}
\end{frame}

\begin{frame}
\frametitle{Function Prototypes Vs Definitions}
\begin{itemize}
\item For the time being:
\begin{itemize}
	\item The prototype goes \textit{above} \texttt{main()}
		\begin{itemize}
			\item It is 1 line and ends with a semicolon ;
		\end{itemize}
	\item The definition goes \textit{below} \texttt{main()}
		\begin{itemize}
			\item It is the prototype repeated followed by a \texttt{\{ \}} block
			\item The code within the \{ \} block is known as the function \textit{body}
		\end{itemize}
\end{itemize}
\end{itemize}
\end{frame}

\begin{frame}[fragile]
\frametitle{Writing Functions - Example 1}
\begin{itemize}
\item Write a C function, \texttt{isEven()}, which takes a single \texttt{int} as an argument and returns 1 if the argument is even and zero otherwise.
\pause
	\begin{itemize}
		\item Name: \texttt{isEven}
		\item Argument: \texttt{int x}
		\item Return Value: \texttt{int}
	\end{itemize}
\pause
\item \textbf{NB:} The variable names given to each argument are the names you use when writing the function body
\end{itemize}
\end{frame}

\begin{frame}[fragile]
\frametitle{Writing Functions - Example 1}
\begin{itemize}
\item The function prototype is therefore:
\begin{lstlisting}[style=CStyle]
int isEven(int x); // Put before main()
\end{lstlisting}
\item The function definition template is:
\begin{lstlisting}[style=CStyle]
// Put after main()
int isEven(int x)
{
// Fill this in
}
\end{lstlisting}
\end{itemize}
\end{frame}

\begin{frame}[fragile]
\frametitle{Writing Functions - Example 1}
\begin{itemize}
\item The function's definition can then be written
\begin{lstlisting}[style=CStyle]
int isEven(int x)
{
	if(x%2 == 0)
		return 1;
	else
		return 0;
}
\end{lstlisting}
\item In this example there is more than one \texttt{return} statement
\end{itemize}
\end{frame}

\begin{frame}[fragile]
\frametitle{Writing Functions - Example 1}
\begin{itemize}
\item We can now write some test code around the function:
\begin{lstlisting}[style=CStyle]
#include <stdio.h>
int isEven(int x);
int main() {
	printf("%d\n", isEven(1));
	printf("%d\n", isEven(2));
}
int isEven(int x)
{
	if(x%2 == 0)
		return 1;
	else
		return 0;
}
\end{lstlisting}
\end{itemize}
\end{frame}

\begin{frame}[fragile]
\frametitle{Writing Functions - Example 2}
\begin{itemize}
\item Lets implement the Week 2 \texttt{sqrt} algorithm as a function
\item ...Then compare with \texttt{sqrtf();}
\item Keep it simple: fixed iteration count \texttt{n=10}

\end{itemize}
\end{frame}

\begin{frame}[fragile]
\frametitle{Writing Functions - Example 2}
\begin{itemize}
\item In mathematics, calculate $\sqrt{k}$ by iterating:
\begin{align*}
x_n &= \frac{1}{2}\left(x_{n-1} + \frac{k}{x_{n-1}}\right)\\
x_0 &\neq 0
\end{align*}
\item In a code snippet:
\begin{lstlisting}[style=CStyle]
// Calculate sqrt(k)
float k = 26; // Test value, sqrt(26)=5.0990
float xn = x/2.0; // x0 = x/2 because why not?
int n;
for(n = 0; n < 10; n++) {
	xn = 0.5*(xn + k/xn);
}
\end{lstlisting}
\end{itemize}
\end{frame}

\begin{frame}[fragile]
\frametitle{Writing Functions - Example 2}
\begin{itemize}
\item Lets make some design decisions:
	\begin{itemize}
		\item Name: \texttt{mySqrt();}
		\item Argument: \texttt{float k}
		\item Return Value: \texttt{float}
	\end{itemize}
\item The function prototype is therefore:
\begin{lstlisting}[style=CStyle]
float mySqrt(float k);
\end{lstlisting}
\end{itemize}
\end{frame}

\begin{frame}[fragile]
\frametitle{Writing Functions - Example 2}
\begin{itemize}
\item Place the function prototype before \texttt{main()}:
\begin{lstlisting}[style=CStyle]
#include <stdio.h>

float mySqrt(float k);

int main() {
	// Do stuff
}
\end{lstlisting}
\end{itemize}
\end{frame}

\begin{frame}[fragile]
\frametitle{Writing Functions - Example 2}
\begin{itemize}
\item Write the function definition below \texttt{main()}
\begin{lstlisting}[style=CStyle]
#include <stdio.h>
float mySqrt(float k);
int main() {
	printf("sqrt(26) = %f\n", mySqrt(26.0));
}

float mySqrt(float k) {
	int n;
	float xn = k/2.0;
	for(n = 0; n < 10; n++)
		xn = 0.5*(xn + k/xn);
	return xn;
}
\end{lstlisting}
\end{itemize}
\end{frame}

\end{document}
