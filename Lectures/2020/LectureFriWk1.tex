\documentclass[14pt]{beamer}
\usetheme{Dresden}
\usecolortheme{orchid}

\usepackage{xcolor}
\usepackage{listings}
\usepackage{courier}
\usepackage{graphicx}
\usepackage{amsmath}
\usepackage{algorithm2e}
\usepackage{multicol}

\usefonttheme[onlymath]{serif}

\definecolor{mGreen}{rgb}{0,0.6,0}
\definecolor{mGray}{rgb}{0.5,0.5,0.5}
\definecolor{mPurple}{rgb}{0.8,0,0.82}
\definecolor{backgroundColour}{rgb}{0.95,0.95,0.92}
\definecolor{lightBlue}{rgb}{0.1, 0.1, 0.8}


\lstdefinestyle{CStyle}{
    backgroundcolor=\color{backgroundColour},   
    commentstyle=\color{mGreen},
    keywordstyle=\color{magenta},
    numberstyle=\tiny\color{mGray},
    stringstyle=\color{mPurple},
    basicstyle=\footnotesize,
    breakatwhitespace=false,         
    breaklines=true,                 
    captionpos=b,                    
    keepspaces=true,                 
    numbers=left,                    
    numbersep=5pt,                  
    showspaces=false,                
    showstringspaces=false,
    showtabs=false,                  
    tabsize=2,
    language=C
}
\lstdefinestyle{pseudo}{
        basicstyle=\ttfamily\footnotesize,
        keywordstyle=\color{lightBlue},
        morekeywords={BEGIN,END,IF,ELSE,ENDIF,ELSEIF,PRINT,WHILE,RETURN,ENDWHILE,DO,FOR,TO,IN,ENDFOR,BREAK,INPUT},
        morecomment=[l]{//},
        commentstyle=\color{mGreen}
}

\lstset{basicstyle=\footnotesize\ttfamily,breaklines=true}
\lstset{framextopmargin=50pt,tabsize=2}

\title{ENGG1003 - Tuesday Week 1}
\subtitle{Algorithms and Pseudocode}
\author{Brenton Schulz}
\institute{University of Newcastle}
\date{\today}

\begin{document}
\titlepage

\begin{frame} % Algo defn
\frametitle{Algorithms}
\begin{itemize}
\item Informally, an \textit{algorithm} is a series of steps which accomplishes a task
\item More accurately, the steps (instructions) must:
	\begin{itemize}
		\item Have a strict order
		\item Be unambiguous
		\item Be executable
	\end{itemize}
\item ``Executable" means that the \textit{target platform} is capable of performing that task.
	\begin{itemize}
		\item eg: An industrial welding robot can execute ``move welding tip 1~cm left". A mobile phone can't.
	\end{itemize}
\end{itemize}
\end{frame}

\begin{frame} % Algo communication
\frametitle{Algorithms}
\begin{itemize}
\item An algorithm exists purely as an abstract concept until it is communicated
\item We will use:
	\begin{itemize}
	\item \textit{Pseudocode} to communicate algorithms to ourselves and other people
	\item The languages C and MATLAB to communicate algorithms to computers
	\end{itemize}
\item Pseudocode \textit{can} be very formal, but as engineers we will only use formal rules if required
	\begin{itemize}
		\item eg: When documenting algorithms for other people
		\item Your own ``working out'' can be anything that helps \textit{you}
	\end{itemize}
\end{itemize}
\end{frame}

\begin{frame}[fragile] % car starting example

\frametitle{Algorithm Example 1}
{\footnotesize
\textbf{Name:} Algorithm given to mum to start my car (2015 Tarago) \\
\textbf{Result:} The vehicle's engine is idling \\
\textbf{Initialisation:} stand next to the vehicle, key fob in hand 
\begin{enumerate}
\setlength{\itemsep}{1pt}
  \setlength{\parskip}{0pt}
  \setlength{\parsep}{0pt}
\item Depress the unlock button on the key fob, car will beep twice
\item Place key fob in your pocket
\item Enter the vehicle, sit in the driver's seat
\item Ensure that the gear selector has P engaged
\item Depress the brake pedal
\item Press the engine start button
\item Wait 5 seconds
\item If engine is not idling
	\begin{itemize}
		\footnotesize{
		\item Call me}
	\end{itemize}
\end{enumerate}
}
\end{frame}

\begin{frame} % Car algo discussion
\frametitle{Example Discussion}
\begin{itemize}
\item Algorithms typically need to feel over-explained
	\begin{itemize}
		\item Computers are \textit{really stupid}; get in the habit of over-thinking everything
	\end{itemize}
\item The algorithm contained \textit{flow control} in the form of an ``if'' statement
	\begin{itemize}
		\item The final step (``call me") was \textit{conditional} on the car not starting
	\end{itemize}
\item We will discuss conditional logical statements later, but first...
\end{itemize}
\end{frame}

\begin{frame}
\frametitle{Algorithm Example 2}
A wife asks her husband, a programmer, ``Could you please go shopping for me and buy one carton of milk, and if they have eggs, get 6?”
\linebreak \linebreak
A short time later the husband comes back with 6 cartons of milk and his wife asks, ``Why did you buy 6 cartons of milk?”
\linebreak \linebreak
He replies, “They had eggs.”
\end{frame}

\begin{frame}
\frametitle{Algorithm Example 2a}
\textit{Lets make this more realistic.}
\linebreak \linebreak
A wife asks her robot helper, ``Could you please go shopping for me and buy one carton of milk, and if they have eggs, get 6?”
\linebreak \linebreak
The robot replies: ``Unknown instruction: `get 6'. ''
\end{frame}

\begin{frame}
\frametitle{Flow Control}
\begin{itemize}
\item Instructions in an algorithm execute in an ordered list
	\begin{itemize}
		\item ie: top to bottom
	\end{itemize}
\item Flow Control is any algorithmic mechanism which changes the default ``top to bottom'' execution behaviour
\item We will discuss IF statements and \textit{loops}
\item Flow control (almost) always requires a \textit{condition}
\end{itemize}
\end{frame}

\begin{frame}
\frametitle{Conditions}
\begin{itemize}
\item Computers don't understand ``maybe''
\item A \textit{condition} must be absolutely \textbf{true} or \textbf{false}
\item Human examples:
	\begin{itemize}
		\item I am within the boundary of the Callaghan campus
		\item I am alive
		\item My net worth is below AU\$100M
	\end{itemize}
\item Computer examples:
	\begin{itemize}
		\item \texttt{i} is less than 184
		\item \texttt{x} plus \texttt{y} is not equal to zero
		\item Input data has been given to the program
		\item A division by zero has occurred
	\end{itemize}
\end{itemize}
\end{frame}

\begin{frame}[fragile]
\frametitle{Code Blocks}
\begin{itemize}
\item A \textit{block} is a set of instructions which are grouped together
\item If a single condition controls multiple instructions they can go together in a block
\item A block is typically indicated via indentation
\item Eg:
\begin{lstlisting}[style=pseudo]
IF it is raining
	Pack an umbrella
	Drive to campus instead of walking
	Leave home 40mins early to find parking
ENDIF
\end{lstlisting}
\end{itemize}
\end{frame}

\begin{frame}
\frametitle{IF Variants}
\begin{itemize}
\item There are several versions of IF flow control:
	\begin{itemize}
		\item IF ... ENDIF
		\item IF ... ELSE ... ENDIF
		\item IF ... ELSEIF ... ENDIF
	\end{itemize}
\item The IF and ELSEIF keywords indicate conditions
\item The ELSE keyword is \textit{unconditional}
\item Which one you choose depends on need
	\begin{itemize}
		\item Is there one thing which is conditional?
		\item Do I need to make a choice between two or more options?
		\item Could nothing be executed?
	\end{itemize}
\end{itemize}
\end{frame}

\begin{frame}[fragile]
\frametitle{IF Statement Syntax}
\begin{multicols}{2}
\begin{itemize}
\footnotesize{
\item The IF ... ENDIF syntax is:
\begin{lstlisting}[style=pseudo]
IF condition
	do some things
ENDIF
\end{lstlisting}

\item Likewise: IF ... ELSEIF ... ENDIF syntax is:
\begin{lstlisting}[style=pseudo]
IF condition1
	do some things
ELSEIF condition2
	do other things
ENDIF
\end{lstlisting}
\columnbreak
\item And finally:
\begin{lstlisting}[style=pseudo]
IF condition
	do some things
ELSE
	do some things
ENDIF
\end{lstlisting}
}
\end{itemize}
\end{multicols}
\end{frame}

\begin{frame}[fragile]
\frametitle{IF ... ELSEIF}
\begin{itemize}
\item The IF ... ELSEIF construct can have multiple ELSEIF sections
\item A \textit{crucial} point:
	\begin{itemize}
		\item Conditions are only tested \textit{if the previous ones fail}
		\item Once a condition is TRUE the others are ignored
		\item ie: IF - ELSE implements a choice priority
	\end{itemize}
\end{itemize}
\end{frame}


\begin{frame} % Quadratic root finding intro
\frametitle{Algorithm Example 3 - Quadratic Root Finding}
From high school you should know that the equation
\begin{equation}
a x^2 + b x + c = 0
\end{equation}
has solutions given by
\begin{equation}
x = \frac{-b \pm \sqrt{b^2 - 4 a c}}{2a}
\end{equation}
lets write an algorithm which provides real valued solutions to a quadratic equation.
\end{frame}

\begin{frame} % Quadratic root finding 
\frametitle{Algorithm Example 3 - Quadratic Root Finding}
{\footnotesize
\textbf{Input:} Real numbers $a$, $b$, and $c$ \\
\textbf{Output:} Three numbers:
\begin{enumerate}
\item The number of solutions, N
\item One of the roots, $x_1$
\item The other root, $x_2$
\end{enumerate}
\textbf{Behaviour:}
\begin{itemize}
\item If N is 2 then $x_1$ and $x_2$ are different real numbers
\item If N is 1 then $x_1$ is the unique solution and $x_2$ is undefined
\item If N is 0 then $x_1$ and $x_2$ are undefined
\end{itemize} 
} 
\end{frame}

\begin{frame}[fragile] % Quadratic root finding algorithm
\frametitle{Algorithm Example 3 - Quadratic Root Finding}
\begin{columns}
\column{2in}
\begin{lstlisting}[style=pseudo,basicstyle=\ttfamily\scriptsize]
BEGIN
	INPUT: a, b, c
	D = b^2 - 4ac
	IF D < 0
		N = 0
	ELSEIF D == 0
		N = 1
		x1 = -b/(2a)
	ELSEIF D > 0
		N = 2
		x1 = (-b + sqrt(D))/(2a)
		x2 = (-b - sqrt(D))/(2a)
	ENDIF
END
\end{lstlisting}
\column{2.5in}
\footnotesize{
\begin{itemize}
\item Reasonably formal pseudocode
\item The IF ... ELSE IF flow control construct forces exclusive execution of only \textit{one} block
\item The first condition that is true causes execution of that block
\item Subsequent blocks ignored
\item Contains 3 \textit{conditions}
\end{itemize}
}
\end{columns}
\end{frame}

\begin{frame} % Boolean Logic
\frametitle{Boolean Algebra Basics}
\begin{itemize}
\item What if we want more complicated conditions? Boolean algebra is needed!
\item Boolean algebra (or Boolean logic) is a field of mathematics which evaluates combinations of \textit{logical variables} as either true or false
\item Boolean \textit{variables} can only take the values \textbf{true} (or 1) or \textbf{false} (or 0)
\item Boolean algebra defines three \textit{operators}:
	\begin{itemize}
		\item OR
		\item AND
		\item NOT
	\end{itemize}
\end{itemize}
\end{frame}

\begin{frame} % Boolean Algebra Basics
\frametitle{Boolean Algebra Basics}
\begin{itemize}
\item Boolean variables can be allocated any symbols (just like in ``normal'' algebra)
	\begin{itemize}
		\item Typically get upper-case letters
		\item eg: X = A OR B
	\end{itemize}
\item Various symbols can be used for OR/AND/NOT, we will only use the words here
	\begin{itemize}
		\item Write them in capitals to remove ambiguity
		\item C and MATLAB have their own symbols for Boolean algebra
		\item Other courses (eg: ELE17100) will use different symbols again
	\end{itemize}
\end{itemize}
\end{frame}

\begin{frame} % Boolean Operators
\frametitle{Boolean Operators}
\begin{itemize}
\item An \textit{operand} is a value on which a mathematical operation takes place
	\begin{itemize}
		\item eg: In ``1 + 2'' the 1 and 2 are operands and + is the operator
	\end{itemize}
\item OR - Evaluates true if either operand is true
	\begin{itemize}
		\item X = A OR B
		\item X is true if either one of A or B is true
	\end{itemize}
\item AND- Evaluates true only when \textit{both} operands are true
	\begin{itemize}
		\item X = A AND B
		\item X is true only if both A and B are true
	\end{itemize}
\end{itemize}
\end{frame}

\begin{frame} % Boolean Operators
\frametitle{Boolean Operators}
\begin{itemize}
\item OR and AND are \textit{binary} operators
	\begin{itemize}
		\item They operate on two operands
		\item From Latin ``bini'' meaning ``two together''
	\end{itemize}
\item The NOT operator is \textit{unary}
	\begin{itemize}
		\item It only operates on \textit{one} operand
		\item NB: The operand could be a single variable or complex expression
	\end{itemize}
\item NOT performs a logical inversion
	\begin{itemize}
		\item NOT true = false
		\item NOT false = true
	\end{itemize}
\end{itemize}
\end{frame}

\begin{frame} % Boolean condition examples
\frametitle{Boolean Condition Examples}

\begin{itemize}
	\item My car needs a service if, since the last service, (more than 6 months has past) OR (more than 15000km have been travelled)
	\item You will pass this course if (you score 40\% or more in the final exam) AND (the weighted sum of all assessments is more than 50\%)
	\item A computer program repeats an algorithm if (there is still data to process) AND (errors have not occurred) AND ( NOT (the user has terminated the program) )
\end{itemize}
\end{frame}

\begin{frame}
\frametitle{Algorithm Example 4 - Boolean Conditions}
{\small
\textbf{Problem:} How can square roots be calculated by a computer?
\linebreak \linebreak
\textbf{One Solution:} The \textit{Babylonian Method}.
\linebreak \linebreak
The square root of $a$, $\sqrt{a}$, can be found by \textit{iterating}:
\begin{equation}
x_{n+1} = \frac{1}{2}\left(x_n + \frac{a}{x_n}\right)
\end{equation}
until $x_n$ is ``close enough'' to the true value of $\sqrt{a}$ for our liking.
Execution of this algorithm can use two things:
\begin{enumerate}
\item The \textit{loop} flow control concept
\item Some kind of \textit{stop condition}
\end{enumerate}
}
\end{frame}

\begin{frame}[fragile]
\frametitle{Iteration}
\begin{itemize}
\item In this context iteration is the process of repeatedly applying a formula to the same variables
\item Iteration typically creates a sequence of numbers:\\$x_0$, $x_1$, ..., $x_n$
\begin{itemize}
\item Eg: The equation $x_n = x_{n-1} + 1$ with a choice of $x_0 = 0$ just counts $1,2,3,4...$
\end{itemize}
\item We will study a lot of equations like this in the coming weeks
\end{itemize}
\end{frame}

\begin{frame}
\frametitle{Square Root By Hand}
\begin{itemize}

\item Lets find $\sqrt{2}$ ``manually''
\item In our notation, $a=2$
\item The choice of $x_n$ doesn't \textit{really} matter, lets go with $x_0=2$
\item Applying the formula $x_{n+1} = \frac{1}{2}\left(x_n + \frac{a}{x_n}\right)$:
\footnotesize
\begin{align*}
x_1 = \frac{1}{2}\left( 2 + \frac{2}{2} \right) &= 1.5 \\
x_2 = \frac{1}{2}\left( 1.5 + \frac{2}{1.5} \right) &= 1.4167 \\
x_3 = \frac{1}{2}\left( 1.4167 + \frac{2}{1.4167} \right) &= 1.4142 \\
\end{align*}

\end{itemize}
\end{frame}

\begin{frame}
\frametitle{Square Root By Spreadsheet}
\begin{center}
Well that's tedious. Lets try it on a spreadsheet
~\\
~\\
Questions: When do we stop calculating? How would be write a \textit{stop condition} in computer language terms?\\
~\\
Note that the ``difference'' is always negative.
\end{center}
\end{frame}

\begin{frame}
\frametitle{Algorithm Example 4 - Boolean Conditions}
\begin{itemize}
\item For this example we will choose two exit conditions:
	\begin{enumerate}
		\item An acceptable precision is reached
		\item An iteration limit is reached
	\end{enumerate}
\item The resulting Boolean expression is something like:\\
~\\
``If the change between $x_n$ and $x_{n+1}$ is smaller than some precision value and the number of iterations is greater than maximum then stop iterating''
\end{itemize}
\end{frame}

\begin{frame}
\frametitle{Loops}
\begin{itemize}
\item A \textit{loop} causes an algorithm to execute a given block of instructions multiple times
\item Loops typically require an \textit{exit condition}
	\begin{itemize}
		\item Without an exit condition they are called \textit{infinite loops}
		\begin{itemize}
			\item Yes, these have a purpose
		\end{itemize}
	\end{itemize}
\item Multiple types of loops
	\begin{itemize}
		\item WHILE \textit{condition}...ENDWHILE
		\item DO...WHILE \textit{condition}
		\item FOR \textit{counter} FROM 1 TO \textit{something}
	\end{itemize}
\end{itemize}
\end{frame}

\begin{frame}[fragile]
\frametitle{Algorithm Example 4 - Boolean Conditions}
\begin{itemize}
\item Implementing the square root algorithm:
	\begin{itemize}
		\item Choose max iterations as 10 and precision as 0.001
	\end{itemize}

\begin{lstlisting}[style=pseudo,mathescape=true,basicstyle=\ttfamily\scriptsize]
BEGIN
	INPUT a
	x = a
	xOld = 0 // Why do we do this?
	n = 0
	WHILE (n<10) AND ( (x-xOld) < -0.0001 )
		xOld = x
		x = 1/2*(x + a/x)
		n = n + 1
	ENDWHILE 
END
\end{lstlisting}

\item Here it loops until 10 \textit{iterations} have occurred OR a precision limit is reached
	\begin{itemize}
		\item \textbf{NB:} This is the reverse logic to previous slides.
	\end{itemize}
\end{itemize}
\end{frame}

\begin{frame}
\frametitle{Loop Details}
\begin{itemize}
\item WHILE conditions are tested before "entering"
\item The condition is tested before every repeat
\item Variables in the condition should change inside the loop
	\begin{itemize}
		\item Try to avoid infinite loops \textit{unless you want one}
	\end{itemize}
\item What if we want to force the loop to execute \textit{at least once}?
\end{itemize}
\end{frame}

\if false
\begin{frame}[fragile] % C lstlisting examples
\frametitle{C listing template}
\begin{lstlisting}[style=CStyle]
#include <stdio.h>
int main() {
	printf("Custom lstlisting template\n");
}
\end{lstlisting}

\begin{lstlisting}[language=c]
#include <stdio.h>
int main() {
	printf("default C style\n");
}
\end{lstlisting}
\end{frame}

\begin{frame}
\frametitle{Columns Template}
\begin{columns}
\column{1.5in}
left side
\column{1.5in}
right side
\begin{figure}
\includegraphics[scale=0.2]{test}
\end{figure}
\end{columns}
\end{frame}
\fi

\end{document}
