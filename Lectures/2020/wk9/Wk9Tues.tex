\documentclass[14pt]{beamer}
\usetheme{Dresden}
\usecolortheme{orchid}

\usepackage{xcolor}
\usepackage{listings}
\usepackage{courier}
\usepackage{graphicx}
\usepackage{amsmath}
\usepackage{algorithm2e}
\usepackage{multicol}
\usepackage{amssymb}

\usefonttheme[onlymath]{serif}

\definecolor{mGreen}{rgb}{0,0.6,0}
\definecolor{mGray}{rgb}{0.5,0.5,0.5}
\definecolor{mPurple}{rgb}{0.8,0,0.82}
\definecolor{backgroundColour}{rgb}{0.95,0.95,0.92}
\definecolor{lightBlue}{rgb}{0.1, 0.1, 0.8}

\lstdefinestyle{CStyle}{
    backgroundcolor=\color{backgroundColour},   
    commentstyle=\color{mGreen},
    keywordstyle=\color{magenta},
    numberstyle=\tiny\color{mGray},
    stringstyle=\color{mPurple},
    basicstyle=\footnotesize\ttfamily,
    breakatwhitespace=false,         
    breaklines=true,                 
    captionpos=b,                    
    keepspaces=true,                 
    numbers=left,                    
    numbersep=5pt,                  
    showspaces=false,                
    showstringspaces=false,
    showtabs=false,                  
    tabsize=2,
    language=C
}

\lstdefinestyle{Ctable}{
    backgroundcolor=\color{backgroundColour},   
    commentstyle=\color{mGreen},
    keywordstyle=\color{magenta},
    numberstyle=\tiny\color{mGray},
    stringstyle=\color{mPurple},
    basicstyle=\footnotesize\ttfamily,
    breakatwhitespace=false,         
    breaklines=true,                 
    captionpos=b,                    
    keepspaces=true,                                  
    showspaces=false,                
    showstringspaces=false,
    showtabs=false,                  
    tabsize=2,
    language=C
}

\lstdefinestyle{pseudo}{
        basicstyle=\ttfamily\footnotesize,
        keywordstyle=\color{lightBlue},
        morekeywords={BEGIN,END,IF,ELSE,ENDIF,ELSEIF,PRINT,WHILE,RETURN,ENDWHILE,DO,FOR,TO,IN,ENDFOR,BREAK,INPUT,READ},
        morecomment=[l]{//},
        commentstyle=\color{mGreen}
}

\lstset{basicstyle=\footnotesize\ttfamily,breaklines=true}
\lstset{framextopmargin=50pt,tabsize=2}

\title{ENGG1003 - Monday Week 9}
\subtitle{Introduction to MATLAB\\Variables \& Arithmetic\\Vectorisation}
\author{Brenton Schulz}
\institute{University of Newcastle}
\date{\today}

\begin{document}
\titlepage

\begin{frame}
\frametitle{Assumed Knowledge}
\begin{itemize}
\item These notes are written for ENGG1003 and assume C was taught first
\item In particular, they require knowledge of:
	\begin{itemize}
		\item Program top-to-bottom sequential execution
		\item Flow control (IF / WHILE / FOR / etc)
		\item Variables
	\end{itemize}
\end{itemize}
\end{frame}

\begin{frame}
\frametitle{What is MATLAB?}
\begin{itemize}
\item MATLAB is an \textit{interpreted} programming language designed for quickly performing numerical analysis
\item It is sometimes criticised for not being a ``legitimate'' programming language.
	\begin{itemize}
		\item Engineers use it to solve a complex numerical problem quickly, then throw the code away
		\item NB: The code is \textit{written} quickly. It doesn't necessarily \textit{execute} quickly (compared to a compiled language like C)
	\end{itemize}
\item Arithmetic is fast, flow control is \textit{very} slow
\begin{itemize}
	\item Problems need \textit{vectorisation}
\end{itemize}
\end{itemize}
\end{frame}

\begin{frame}
\frametitle{...Interpreted?}
\begin{itemize}
\item A language is \textit{compiled} when the entire source code listing gets converted to a binary executable in one step
\item An \textit{interpreted} language is read and executed line-by-line
	\begin{itemize}
		\item The language interpreter is running the whole time your code is running
	\end{itemize}

\end{itemize}
\end{frame}

\begin{frame}
\frametitle{...Interpreted?}
\begin{itemize}
\item Interpreted languages are slower, but have the advantages of:
	\begin{itemize}
		\item Being more forgiving of mistakes
		\item Having more advanced memory management, eg:
			\begin{itemize}
				\item Variables don't need to be declared
				\item Arrays automatically grow and shrink as needed
			\end{itemize}
		\item Allowing code snippets to easily be executed in isolation
			\begin{itemize}
				\item You can run single lines instantly by typing them into a command prompt
				\item Code doesn't \textit{need} to be written to a source file, although it is a good idea
			\end{itemize}
	\end{itemize}
\end{itemize}
\end{frame}

\begin{frame}
\frametitle{MATLAB Vs C}
\begin{itemize}
\item Some big contrasts:
	\begin{itemize}
		\item MATLAB is ``weakly typed''
			\begin{itemize}
				\item There are no strict data types
				\item By default, (almost) everything is a complex valued array of type \texttt{double}
			\end{itemize}
		\item Arithmetic (mostly) follows rules of linear algebra
			\begin{itemize}
				\item Somewhat beyond this course. We won't cover matrix multiplication.
				\item Many language behaviours will make more sense after you've studied linear algebra
				\item The fact that ``everything is an array'' makes for some possibly confusing rules
			\end{itemize}
		\item MATLAB has ``high level'' features like plotting
			\begin{itemize}
				\item It is more of a ``calculator engine'' than a programming language
			\end{itemize}
	\end{itemize}
\end{itemize}
\end{frame}

\begin{frame}
\frametitle{Installing MATLAB}
\begin{itemize}
\item MATLAB is (expensive) commercial software
	\begin{itemize}
		\item Python is more popular in industry (c.f. IEEE survey), partly because it is free
	\end{itemize}
\item The university pays a site licence which allows students to install it for free
	\begin{itemize}
		\item Instructions here (hopefully...): \url{https://uonau.service-now.com/itservices?id=kb_article_view&sysparm_article=KB0023081&sys_kb_id=a7ccc3334f3953c08e8fa90f0310c7f7}
	\end{itemize}
\item The ``standard'' licence (for companies) is \$1260 per year, per computer
\end{itemize}
\end{frame}

\begin{frame}
\frametitle{Installing Octave}
\begin{itemize}
\item Octave is a cost-free (and open source) MATLAB-like interpreter
	\begin{itemize}
		\item Some employers prefer this over MATLAB
	\end{itemize}
\item It will probably execute all of the code for this course without modification
\item Available for Windows / Mac / Linux: \url{https://www.gnu.org/software/octave/download.html}
\item Demonstration of projects in Octave is fine
	\begin{itemize}
		\item It tends to load \textit{much} faster than MATLAB
	\end{itemize}
\end{itemize}
\end{frame}

\begin{frame}
\frametitle{Variable Classification}
\begin{itemize}
\item MATLAB may be ``weakly typed'' but the following classifications are useful:
	\begin{itemize}
		\item A \textit{scalar} is a single number
		\item A \textit{vector} is a row or column of numbers 
		\begin{itemize}
			\item A 1D array in C
		\end{itemize}
		\item A \textit{matrix} is rectangular array of numbers
		\begin{itemize}
			\item A 2D array in C
		\end{itemize}
		\item MATLAB also supports arrays of matricies
		\begin{itemize}
			\item A 3D array in C
		\end{itemize}
	\end{itemize}
\pause
\item Arithmetic operations have different behaviours with different arguments, especially when mixed (eg: what does scalar plus vector do?)
\end{itemize}
\end{frame}

\begin{frame}
\frametitle{Getting Started}
\begin{itemize}
\item Lets load up MATLAB and:
	\begin{itemize}
		\item Learn what the different GUI segments do
		\item Assign values to some random variables
			\begin{itemize}
				\item Observe them appear in the ``workspace''
			\end{itemize}
		\item Do some basic arithmetic on scalar variables
		\item Run a basic script
		\item Observe output suppression
	\end{itemize}
\end{itemize}
\end{frame}

\begin{frame}[fragile]
\frametitle{Variable Assignment Syntax}
\begin{itemize}
\item When allocating a constant to a variable we have a few basic methods:
	\begin{itemize}
		\item Scalar: just like in C
			\begin{lstlisting}[style=pseudo]
x = 5
\end{lstlisting}
		\item Row Vector: space separated list inside \texttt{[ ]}'s
\begin{lstlisting}[style=pseudo]
x = [1 2 3 4]		
\end{lstlisting}
		\item Column Vector: like row vectors, but uses \texttt{;} to separate rows:
\begin{lstlisting}[style=pseudo]
x = [1;2;3;4]
\end{lstlisting}
		\item Matrix: A mix of row and column syntax:
\begin{lstlisting}[style=pseudo]
x = [1 2 3; 4 5 6; 7 8 9]
\end{lstlisting}
	\end{itemize}
\end{itemize}
\end{frame}

\begin{frame}
\frametitle{Observations}
\begin{itemize}
\item First, run the code from the previous slides
\pause
\item Observe that the variables are created automatically
\pause
\item Observe that MATLAB prints the result after each line
\item Add a \texttt{;} to the end of a line to suppress the output
\end{itemize}
\end{frame}

\begin{frame}[fragile]
\frametitle{Other Initialisation Methods}
\begin{itemize}
\item MATLAB supports many other methods for ``creating'' data:
	\begin{itemize}
		\item Create a list of numbers from A to B with increment C with:
\begin{lstlisting}[style=pseudo]
x = A:C:B
\end{lstlisting}
eg:
\begin{lstlisting}[style=pseudo]
x = 0:0.1:2
\end{lstlisting}
\item Use the \texttt{linspace()} function
	\begin{itemize}
		\item View the help page and run an example
	\end{itemize}
	\item We can also read data from files; see this later
	\end{itemize}
\end{itemize}
\end{frame}

\begin{frame}
\frametitle{Arithmetic}
\begin{itemize}
\item For scalar data, MATLAB supports all basic arithmetic operators just like C
	\begin{itemize}
		\item \texttt{+}
		\item \texttt{-}
		\item \texttt{/}
		\item \texttt{*}
	\end{itemize}
\item It also supports exponents with \texttt{\^}
	\begin{itemize}
		\item Shift-6 on a US keyboard
		\item In C, this means a bitwise exclusive-OR
	\end{itemize}
\end{itemize}
\end{frame}

\begin{frame}
\frametitle{Arithmetic}
\begin{itemize}
\item For vector and matrix data addition and subtraction is \textit{element-wise}
	\begin{itemize}
		\item The arguments should be the same size
			\begin{itemize}
				\item Or \textit{at least one} dimension must match but this is beyond ENGG1003 (and only supported on recent versions of MATLAB)
			\end{itemize}
		\item If the argument dimensions differ an error occurs
	\end{itemize}
\item Run some examples...
\end{itemize}
\end{frame}

\begin{frame}
\frametitle{Arithmetic}
\begin{itemize}
\item Vector and matrix multiplication and division are beyond ENGG1003
	\begin{itemize}
		\item You will learn matrix multiplication in MATH..something
		\item Division is advanced: multiplication by \textit{matrix inverse}
	\end{itemize}
\pause
\item However, you can do element-wise multiplication if the arguments are the same size
	\begin{itemize}
		\item This is done with special operators:
			\begin{itemize}
				\item \texttt{.*} Multiplication
				\item \texttt{./} Division
				\item \texttt{.\^} Exponent 
			\end{itemize}
	\end{itemize}
\item Run some examples...
\end{itemize}
\end{frame}

\begin{frame}
\frametitle{Arithmetic}
\begin{itemize}
\item Final examples (for now):
	\begin{itemize}
		\item Scalar plus/minus vector/matrix adds/subtracts that scalar from every element
		\item Scalar multiplication does the same
		\item Scalar divided by vector/matrix causes an error (unless \texttt{./} is used)
		\item Vector/matrix divided by scalar does an element-wise operation
	\end{itemize}
\item Run some examples...
\end{itemize}
\end{frame}

\begin{frame}
\frametitle{Vectorisation}
\begin{itemize}
\item \textit{Vectorisation} is the process of arranging a large numerical computing task so that it can be broken up into sub-tasks which can all be executed \textit{simultaneously} (ie: in parallel)
\pause
\item Element-wise addition of two arrays is an example of a task which is easy to vectorise
	\begin{itemize}
		\item Such tasks are often said to be \textit{embarrassingly parallel}
	\end{itemize}
\end{itemize}
\end{frame}

\begin{frame}
\frametitle{Vectorisation}
\begin{itemize}
\item In MATLAB all vector and matrix arithmetic is said to be vectorised
	\begin{itemize}
		\item A \texttt{for()} loop that processes an array is a \textit{linear} operation
		\item This can be \textit{vectorised} by processing multiple array elements simultaneously
	\end{itemize}
\pause
\item Unlike C, MATLAB is automatically \textit{multi-threaded}
	\begin{itemize}
		\item An arithmetic operation on a large array will be spread over multiple CPU cores to increase execution speed
	\end{itemize}
\item Run an example
\end{itemize}
\end{frame}

\begin{frame}
\frametitle{Maths Functions}
\begin{itemize}
\item MATLAB has hundreds (thousands?) of built-in functions
\item Lets play with trig functions
\item Run an example of \textit{scalar} \texttt{sin()}
\pause
\item What about vectors and matrices?
\pause
\item Mathematics doesn't define what a ``matrix sine'' (etc) is
\item MATLAB simply evaluates them element-wise
\item Run an example
\end{itemize}
\end{frame}

\begin{frame}
\frametitle{Scripts}
\begin{itemize}
\item MATLAB supports running source code from a file
\item There is a built-in editor we can use to write (and debug) said files
\item Long story short:
	\begin{itemize}
		\item Create a file with the editor
		\item Save it with the \texttt{.m} extension
		\item Click the ``run'' button
	\end{itemize}
\item Script files can also be run from the command prompt by typing their name \textit{without the \texttt{.m}}
	\begin{itemize}
		\item Prompt must have the correct ``working directory''
	\end{itemize}
\end{itemize}
\end{frame}

\begin{frame}
\frametitle{Plotting}
\begin{itemize}
\item MATLAB has very powerful built-in plotting tools
	\begin{itemize}
		\item They are commonly used in academic literature
		\item They make Excel plots look like something my 4 year old drew at daycare
		\item An impressive example I found on Google images: \url{http://www.asu.cas.cz/~bezdek/vyzkum/rotating_3d_globe/rotating_3d_globe/manual.html}
	\end{itemize}
\end{itemize}
\end{frame}

\begin{frame}
\frametitle{Plotting}
\begin{itemize}
\item Basic 2D plotting is done with the \texttt{plot()} function
\item It does the Excel equivalent of a scatter plot
\item The first two arguments are vectors (1D arrays) listing x-y pairs of points
\item Run a basic example...
\end{itemize}
\end{frame}

\begin{frame}
\frametitle{Plotting}
\begin{itemize}
\item We can finally bring a few things together to do an interesting plot
\pause
\item Task: plot \texttt{sin(x)} from $0$ to $2\pi$
\item Steps
	\begin{enumerate}
		\item Create a vector, \texttt{x}, that contains ``many'' elements going from $0$ to $2\pi$
		\item Pass this vector to the \texttt{sin()} function
			\begin{itemize}
				\item This is an example of vectorisation
			\end{itemize}
		\item Run the plot command as plot(x, sin(x))
	\end{enumerate}
\item Do the above manually then repeat it from a script
\end{itemize}
\end{frame}

\end{document}
