\documentclass[english,14pt]{beamer}
\usetheme{EastLansing}
\usecolortheme{spruce}

\usepackage{xcolor}
\usepackage{listings}
\usepackage{courier}
\usepackage{graphicx}
\usepackage{amsmath}
\usepackage{algorithm2e}
\usepackage{multicol}
\usepackage{hyperref}
\usepackage{textcomp}

% http://mirrors.ibiblio.org/CTAN/macros/latex/contrib/datetime2/datetime2.pdf
\usepackage{babel}
\usepackage[useregional]{datetime2}

% https://tex.stackexchange.com/questions/42619/x-mark-to-match-checkmark
\usepackage{pifont}% http://ctan.org/pkg/pifont

%% https://stackoverflow.com/questions/1435837/how-to-remove-footers-of-latex-beamer-templates
%%gets rid of bottom navigation bars
%\setbeamertemplate{footline}[page number]
%
%gets rid of navigation symbols
\setbeamertemplate{navigation symbols}{}


\usefonttheme[onlymath]{serif}

\definecolor{mGreen}{rgb}{0,0.6,0}
\definecolor{mGray}{rgb}{0.5,0.5,0.5}
\definecolor{mPurple}{rgb}{0.8,0,0.82}
\definecolor{backgroundColour}{rgb}{0.95,0.95,0.92}
\definecolor{lightBlue}{rgb}{0.1, 0.1, 0.8}
\definecolor{darkGreen}{rgb}{0, 0.39, 0}

\newcommand\red[1]{{\color{red} #1}}
\newcommand\green[1]{{\color{green} #1}}
\newcommand\blue[1]{{\color{blue} #1}}
\newcommand\darkGreen[1]{{\color{darkGreen} #1}}

\newcommand{\cmark}{\ding{51}}%
\newcommand{\xmark}{\ding{55}}%

\lstdefinestyle{CStyle}{
    backgroundcolor=\color{backgroundColour},   
    commentstyle=\color{mGreen},
    keywordstyle=\color{magenta},
    numberstyle=\tiny\color{mGray},
    stringstyle=\color{mPurple},
    basicstyle=\footnotesize,
    breakatwhitespace=false,         
    breaklines=true,                 
    captionpos=b,                    
    keepspaces=true,                 
    numbers=left,                    
    numbersep=5pt,                  
    showspaces=false,                
    showstringspaces=false,
    showtabs=false,                  
    tabsize=2,
    language=Python
}

\lstdefinestyle{MStyle}{
    backgroundcolor=\color{backgroundColour},   
    commentstyle=\color{mGreen},
    keywordstyle=\color{magenta},
    numberstyle=\tiny\color{mGray},
    stringstyle=\color{mPurple},
    basicstyle=\footnotesize,
    breakatwhitespace=false,         
    breaklines=true,                 
    captionpos=b,                    
    keepspaces=true,                 
    numbers=left,                    
    numbersep=5pt,                  
    showspaces=false,                
    showstringspaces=false,
    showtabs=false,                  
    tabsize=2,
    language=MATLAB
}

\lstdefinestyle{pseudo}{
        basicstyle=\ttfamily\footnotesize,
        keywordstyle=\color{lightBlue},
        morekeywords={BEGIN,END,IF,ELSE,ENDIF,ELSEIF,PRINT,WHILE,RETURN,ENDWHILE,DO,FOR,TO,IN,ENDFOR,BREAK,INPUT,CONDITIONS},
        morecomment=[l]{//},
        commentstyle=\color{mGreen}
}

\lstset{basicstyle=\footnotesize\ttfamily,breaklines=true}
\lstset{framextopmargin=50pt,tabsize=2}

\title{ENGG1003 - Thursday Week 12}
\subtitle{Final exam preparation}
\author{Steve Weller}
\institute{University of Newcastle}
%\date{\today}
\date{27 May 2021}

% following is a bit of a hack, but forces page numbers (technically: frame numbers) to run 1,2,3,... 
% with titlepage counting as frame 1

\addtocounter{framenumber}{1}
\titlepage

\begin{document}

\begin{flushleft}
{\scriptsize Last compiled:~\DTMnow}
\vspace*{-5mm}
\end{flushleft}
\framebreak

%==============================================================

\begin{frame}[fragile]

\frametitle{Lecture overview}
\begin{enumerate}
	\item Final exam organisational details

	\item[]
	
	\item Overview of final exam questions
	\begin{itemize}
		\item Q1
		\item Q2
		\item Q3
		\item Q4
	\end{itemize}

	\item[]

	\item Questions \& answers

	\item[]

\end{enumerate}

\end{frame}

%==============================================================

\begin{frame}[fragile]

\frametitle{1) Final exam organisational details}

\begin{itemize}
	\item \textbf{Date:} Tuesday 8 June
	\item \textbf{Time:} 2:00pm AEST
	\item \textbf{Location:} ONLINE exam, via Blackboard (BB)
	\item \textbf{Duration:} 130 minutes
	\begin{itemize}
		\item 2:00pm--4:10pm
	\end{itemize}
	\item[]
	\item counts for 35\% of overall grade
\end{itemize}

\begin{figure}[ht]
	\centering
	\includegraphics[width=\textwidth]{figures/finalexamdatetime}
\end{figure}

\end{frame}

%==============================================================

\begin{frame}[fragile]

\frametitle{Final exam: underlying assumptions}

\begin{itemize}
	\item assumptions: you have access to PyCharm---will need to run code
	\item collusion and plagiarism: SACO
	\item oral examination (viva)
\end{itemize}

\end{frame}

%==============================================================

\begin{frame}[fragile]

\frametitle{2) Overview of final exam questions}

\begin{itemize}
	\item four (4) questions, 10 marks per question, with marks for parts (a),(b),(c) etc indicated
	\item NOTE: exact format may differ to fit BB requirements
	\begin{itemize}
		\item will advise any changes to \# questions on BB/email/discord
		\item BUT will only be a \emph{re-organisation} of Q1--Q4
	\end{itemize}
	\item questions tend to get more difficult: Q1 "easy", Q2 harder, Q3 harder again, Q4 challenging
	\item mc auto-graded, Q2--Q4 manually graded
\end{itemize}
\end{frame}



%==============================================================

\begin{frame}[fragile]

\frametitle{Question 1}

\begin{itemize}
	\item ten multiple choice (1 mark each)
	\item given Python code, asked to:
	\begin{itemize}
		\item identify coding error
		\item what is the output when code runs?
		\item maybe other styles of MC questions
	\end{itemize}
	
\end{itemize}

\end{frame}

%==============================================================

\begin{frame}[fragile]

\frametitle{Question 2}

\begin{itemize}
	\item xxx
\end{itemize}

\end{frame}

%==============================================================

\begin{frame}[fragile]

\frametitle{Question 3}

\begin{itemize}
	\item xxx
\end{itemize}

\end{frame}

%==============================================================

\begin{frame}[fragile]

\frametitle{Question 4}

\begin{itemize}
	\item xxx
\end{itemize}

\end{frame}

%==============================================================

\begin{frame}[fragile]

\frametitle{3) Questions \& answers}

\begin{itemize}
	\item xxx
\end{itemize}

\end{frame}

\end{document}