\documentclass[english,14pt]{beamer}
\usetheme{EastLansing}
\usecolortheme{spruce}

\usepackage{xcolor}
\usepackage{listings}
\usepackage{courier}
\usepackage{graphicx}
\usepackage{amsmath}
\usepackage{algorithm2e}
\usepackage{multicol}
\usepackage{hyperref}

% http://mirrors.ibiblio.org/CTAN/macros/latex/contrib/datetime2/datetime2.pdf
\usepackage{babel}
\usepackage[useregional]{datetime2}

% https://tex.stackexchange.com/questions/42619/x-mark-to-match-checkmark
\usepackage{pifont}% http://ctan.org/pkg/pifont

%% https://stackoverflow.com/questions/1435837/how-to-remove-footers-of-latex-beamer-templates
%%gets rid of bottom navigation bars
%\setbeamertemplate{footline}[page number]
%
%gets rid of navigation symbols
\setbeamertemplate{navigation symbols}{}


\usefonttheme[onlymath]{serif}

\definecolor{mGreen}{rgb}{0,0.6,0}
\definecolor{mGray}{rgb}{0.5,0.5,0.5}
\definecolor{mPurple}{rgb}{0.8,0,0.82}
\definecolor{backgroundColour}{rgb}{0.95,0.95,0.92}
\definecolor{lightBlue}{rgb}{0.1, 0.1, 0.8}

\newcommand\red[1]{{\color{red} #1}}
\newcommand\green[1]{{\color{green} #1}}
\newcommand\blue[1]{{\color{blue} #1}}

\newcommand{\cmark}{\ding{51}}%
\newcommand{\xmark}{\ding{55}}%

\lstdefinestyle{CStyle}{
    backgroundcolor=\color{backgroundColour},   
    commentstyle=\color{mGreen},
    keywordstyle=\color{magenta},
    numberstyle=\tiny\color{mGray},
    stringstyle=\color{mPurple},
    basicstyle=\footnotesize,
    breakatwhitespace=false,         
    breaklines=true,                 
    captionpos=b,                    
    keepspaces=true,                 
    numbers=left,                    
    numbersep=5pt,                  
    showspaces=false,                
    showstringspaces=false,
    showtabs=false,                  
    tabsize=2,
    language=C
}

\lstdefinestyle{pseudo}{
        basicstyle=\ttfamily\footnotesize,
        keywordstyle=\color{lightBlue},
        morekeywords={BEGIN,END,IF,ELSE,ENDIF,ELSEIF,PRINT,WHILE,RETURN,ENDWHILE,DO,FOR,TO,IN,ENDFOR,BREAK,INPUT},
        morecomment=[l]{//},
        commentstyle=\color{mGreen}
}

\lstset{basicstyle=\footnotesize\ttfamily,breaklines=true}
\lstset{framextopmargin=50pt,tabsize=2}

\title{ENGG1003 - Monday Week 6}
\subtitle{Interpolation, Assignment 1 and Mid-term quiz}
\author{Steve Weller}
\institute{University of Newcastle}
%\date{\today}
\date{29 March 2021}

% following is a bit of a hack, but forces page numbers (technically: frame numbers) to run 1,2,3,... 
% with titlepage counting as frame 1

\addtocounter{framenumber}{1}
\titlepage

\begin{document}

\begin{flushleft}
{\scriptsize Last compiled:~\DTMnow}
\vspace*{-5mm}
\end{flushleft}
\framebreak

%==============================================================

\begin{frame}[fragile]

\frametitle{Lecture overview}
\begin{enumerate}
	\item Interpolation
	\item[]
	
	\item Assignment 1
	
	\item[]
	
	\item Mid-term quiz

\end{enumerate}

\end{frame}

%==============================================================

\begin{frame}[fragile]

\frametitle{The story so far}

\begin{itemize}
	\item see SJJ lecture thursday week 5
	\item xxx
	\item xxx
\end{itemize}

\end{frame}

%==============================================================

\begin{frame}[fragile]

\frametitle{$1)$ Interpolation}

Compare and contrast, both special cases of \red{\emph{curve-fitting}}

\begin{itemize}
	\item interpolation---today
	\item regression---later in course
	\item demonstrate both with the same small dataset
\end{itemize}

\end{frame}

%==============================================================

\begin{frame}[fragile]

\frametitle{}

\begin{itemize}
	\item regression: curve-fitting (eg: fitting a straight line)
	\begin{itemize}
		\item obtain a ``functional form'' eg: identify a model, F=kx for Hookes' law
		\item too much data
		\item simplify data down to a straight line (plus noise)
	\end{itemize}
	\item[]
	\item interpolation: joining the dots
	\begin{itemize}
		\item obtain value of $y$ at some intermediate point
	\end{itemize}
	\item[]
	\item both involve creating a function from data
	\begin{itemize}
		\item that requires some explanation! so let's do that
	\end{itemize}
\end{itemize}

\end{frame}

%==============================================================

\begin{frame}[fragile]

\frametitle{Functions}

\begin{itemize}

	\item review mathematical functions: week 5 Monday lecture, page 3
	\item function $f$ takes data point $x$ and returns $y = f(x)$
	\item review in PyCharm
\end{itemize}

\end{frame}

%==============================================================

\begin{frame}[fragile]

\frametitle{}

\begin{itemize}
	\item xxx
\end{itemize}

\end{frame}

%==============================================================

\begin{frame}[fragile]

\frametitle{}

\begin{itemize}
	\item xxx
\end{itemize}

\end{frame}

%==============================================================

\begin{frame}[fragile]

\frametitle{}

\begin{itemize}
	\item xxx
\end{itemize}

\end{frame}

%==============================================================

\begin{frame}[fragile]

\frametitle{}

\begin{itemize}
	\item xxx
\end{itemize}

\end{frame}

%==============================================================

\begin{frame}[fragile]

\frametitle{}

\begin{itemize}
	\item xxx
\end{itemize}

\end{frame}

%==============================================================

\begin{frame}[fragile]

\frametitle{$2)$ Assignment 1}

\begin{itemize}
	\item key dates: out, due date for submission
	\item counts for $20$\% of course grade
	\item how assessed: in lab, week 7 (after recess)
	\item the basic ideas behind the lab
	\item this weeks 2-hr face-face lab:
	\begin{itemize}
		\item get started on the assignment
		\item there isn't a week 6 lab sheet: assignment in place of work sheet
	\end{itemize}
\end{itemize}

\end{frame}

%==============================================================

\begin{frame}[fragile]

\frametitle{$3)$ Mid-term quiz}

\begin{itemize}
	\item Thursday 1 April, 4--5pm
	\begin{itemize}
		\item during scheduled lecture time
		\item but there will not be any Zoom or YouTube livestream on 1 April
	\end{itemize}
	\item 40-minute quiz
	\item open-book

	\item quiz will appear on BB at 4:15pm
	\item counts for $15$\% of course grade
	\item what you'll be asked
\end{itemize}

\end{frame}

%==============================================================

\begin{frame}[fragile]

\frametitle{}

\begin{itemize}
	\item what you can do to prepare for the quiz
	\begin{itemize}
		\item read THIS csv--- can get started now!
		\item you'll be asked to write Python code to do some calculations on a specified column
		\item enter your results into BB
		\item cut-and-paste code into BB
	\end{itemize}
	\item can practice NOW in BB
	\item demo to class in lecture
\end{itemize}

\end{frame}

%==============================================================

\begin{frame}[fragile]

\frametitle{}

\begin{itemize}
	\item xxx
\end{itemize}

\end{frame}

%==============================================================

\begin{frame}[fragile]

\frametitle{}

\begin{itemize}
	\item xxx
\end{itemize}

\end{frame}

%==============================================================

\begin{frame}[fragile]

\frametitle{Lecture summary}
\begin{itemize}
	\item Interpolation
	\begin{itemize}
		\item linear interpolation (straight line ``join the dots'')
		\item cubic spline
	\end{itemize}

	\item[]
	
	\item Assignment 1
	\begin{itemize}
		\item xxx
	\end{itemize}

	\item[]
	
	\item Mid-term quiz
		\begin{itemize}
			\item xxx
		\end{itemize}
		
\end{itemize}
\end{frame}

\end{document}