\documentclass[english,14pt]{beamer}
\usetheme{EastLansing}
\usecolortheme{spruce}

\usepackage{xcolor}
\usepackage{listings}
\usepackage{courier}
\usepackage{graphicx}
\usepackage{amsmath}
\usepackage{algorithm2e}
\usepackage{multicol}
\usepackage{hyperref}
\usepackage{textcomp}

% http://mirrors.ibiblio.org/CTAN/macros/latex/contrib/datetime2/datetime2.pdf
\usepackage{babel}
\usepackage[useregional]{datetime2}

% https://tex.stackexchange.com/questions/42619/x-mark-to-match-checkmark
\usepackage{pifont}% http://ctan.org/pkg/pifont

%% https://stackoverflow.com/questions/1435837/how-to-remove-footers-of-latex-beamer-templates
%%gets rid of bottom navigation bars
%\setbeamertemplate{footline}[page number]
%
%gets rid of navigation symbols
\setbeamertemplate{navigation symbols}{}


\usefonttheme[onlymath]{serif}

\definecolor{mGreen}{rgb}{0,0.6,0}
\definecolor{mGray}{rgb}{0.5,0.5,0.5}
\definecolor{mPurple}{rgb}{0.8,0,0.82}
\definecolor{backgroundColour}{rgb}{0.95,0.95,0.92}
\definecolor{lightBlue}{rgb}{0.1, 0.1, 0.8}

\newcommand\red[1]{{\color{red} #1}}
\newcommand\green[1]{{\color{green} #1}}
\newcommand\blue[1]{{\color{blue} #1}}

\newcommand{\cmark}{\ding{51}}%
\newcommand{\xmark}{\ding{55}}%

\lstdefinestyle{CStyle}{
    backgroundcolor=\color{backgroundColour},   
    commentstyle=\color{mGreen},
    keywordstyle=\color{magenta},
    numberstyle=\tiny\color{mGray},
    stringstyle=\color{mPurple},
    basicstyle=\footnotesize,
    breakatwhitespace=false,         
    breaklines=true,                 
    captionpos=b,                    
    keepspaces=true,                 
    numbers=left,                    
    numbersep=5pt,                  
    showspaces=false,                
    showstringspaces=false,
    showtabs=false,                  
    tabsize=2,
    language=Python
}

\lstdefinestyle{pseudo}{
        basicstyle=\ttfamily\footnotesize,
        keywordstyle=\color{lightBlue},
        morekeywords={BEGIN,END,IF,ELSE,ENDIF,ELSEIF,PRINT,WHILE,RETURN,ENDWHILE,DO,FOR,TO,IN,ENDFOR,BREAK,INPUT},
        morecomment=[l]{//},
        commentstyle=\color{mGreen}
}

\lstset{basicstyle=\footnotesize\ttfamily,breaklines=true}
\lstset{framextopmargin=50pt,tabsize=2}

\title{ENGG1003 - Thursday Week 4}
\subtitle{Using random numbers, and reading from spreadsheets}
\author{Steve Weller \& Sarah Johnson}
\institute{University of Newcastle}
%\date{\today}
\date{18 March, 2021}

% following is a bit of a hack, but forces page numbers (technically: frame numbers) to run 1,2,3,... 
% with titlepage counting as frame 1

\addtocounter{framenumber}{1}
\titlepage

\begin{document}

\begin{flushleft}
{\scriptsize Last compiled:~\DTMnow}
\vspace*{-5mm}
\end{flushleft}
\framebreak

%==============================================================

\begin{frame}[fragile]

\frametitle{Lecture overview}
\begin{enumerate}
	\item Using random numbers


	\item[]
	
	\item Reading from spreadsheets

	\item[]
\end{enumerate}

\begin{itemize}
	\item More practice with arrays, iteration, \\ conditional code execution (\texttt{if}) and plotting
\end{itemize}

\end{frame}

%==============================================================

\begin{frame}[fragile]

\frametitle{$1)$ Using random numbers}

\textbf{Recap:~} generating random numbers

\begin{figure}[ht]
	\centering
	\includegraphics[width=\textwidth]{figures/LLp55b}
\end{figure}
\vspace*{-3mm}
Using \texttt{numpy} library:
\begin{itemize}
	\item random integers
	\item random floats from $[0,1)$
	\item random floats from $[a,b]$
\end{itemize}

\end{frame}

%==============================================================

\begin{frame}[fragile]

\frametitle{Random integers: simulating coin toss}

\textbf{Example 1}\\
\vspace*{5mm}

Simulate the toss of a coin $N$ times as follows:

\begin{itemize}
	\item generate a length-$N$ array of randomly chosen $0$s and $1$s
	\begin{itemize}
		\item $0 = $~heads, $1 = $~tails
		\item equally likely heads and tails ie: fair coin
	\end{itemize}
	\item[]
	\item display \emph{expected} number of heads observed
	\item display \emph{actual} number of heads observed
	\item test/debug with $N=100$, then $N=100,000$
\end{itemize}

\end{frame}

%==============================================================

\begin{frame}[fragile]

\frametitle{Coin toss simulation}
\vspace*{-3mm}
\begin{lstlisting}[style=CStyle]
import numpy as np

# generate random array of 0s and 1s
# 0==heads & 1==tails
# N integers from [0,2) ie: 0 or 1
N = 100000
x = np.random.randint(0, 2, N)
print(x)

headCnt = 0;
for i in range(0,N,1):
    if x[i]==0:
        headCnt += 1

print('Expected number of heads: {}'.format(N/2))
print('Observed number of heads: {}'.format(headCnt))
\end{lstlisting}
\vspace*{-3mm}
\begin{itemize}
	\item Live demo of \texttt{headsTails.py}
\end{itemize}

\end{frame}

%==============================================================

\begin{frame}[fragile]

\frametitle{Random floats: engineering tolerance}

\textbf{Example 2}\\
\vspace*{2mm}
\begin{itemize}
	\item steel bolts manufactured with length uniformly distributed between $17~$mm and $19$~mm
	\item only bolts with length in range $[17.25,18.75]$~mm are within tolerance ie: acceptable
	\item write Python code with random numbers to estimate percentage of bolts which are acceptable
	\begin{itemize}
		\item demonstrate your code for $N=10,000$
		\item compare with expected percentage of acceptable bolts:
		\[
			100 \times \frac{18.75-17.25}{19-17} = 75\%
		\]
	\end{itemize}
\end{itemize}

\end{frame}

%==============================================================

\begin{frame}[fragile]

\frametitle{Engineering tolerance simulation}
\vspace*{-3mm}
\begin{lstlisting}[style=CStyle]
import numpy as np

# generate random array of N floats in range [17,19]
N = 10000
x = np.random.uniform(17,19,N)
tolLow = 17.25
tolHigh = 18.75
#print(x)

goodCnt = 0;
for i in range(0,N,1):
    if tolLow <= x[i] <= tolHigh:
        goodCnt += 1

print('Percentage of parts within tolerance: {}%'.format(100*goodCnt/N))
\end{lstlisting}
\vspace*{-3mm}
\begin{itemize}
	\item Live demo of \texttt{engTolerance.py}
\end{itemize}
\end{frame}

%==============================================================

\begin{frame}[fragile]

\frametitle{Random floats: simulate dartboard}

\textbf{Example 3}\\
\vspace*{1mm}
\begin{itemize}
	\item a circular dartboard radius $1$~m is centred in the middle of a square, side length $2$~m
	\item darts are thrown randomly and land with uniform probability in the square
	\begin{itemize}
		\item most (but not all) darts land inside the circle
	\end{itemize}
	\pause
	\item write a Python program which puts a \red{red} dot if the dart lands inside (or on the perimeter) of the circle, and a \blue{blue} dot otherwise
	\begin{itemize}
		\item Hint: points inside (or on perimeter) of circle satisfy
			\[
			x^2 + y^2 \leq 1
			\]
		\end{itemize}
	\item run your program with $N=100, 1000$ and $10,000$
\end{itemize}

\end{frame}

%==============================================================

\begin{frame}[fragile]

\frametitle{Dartboard simulation: code}
\begin{lstlisting}[style=CStyle,basicstyle=\scriptsize]
import numpy as np
import matplotlib.pyplot as plt

# generate random array of (x,y) pairs covering
# square with edge length 2
N = 10000
x = np.random.uniform(-1, 1, N)   # N floats from [-1,1]
y = np.random.uniform(-1, 1, N)   # N floats from [-1,1]

insideCnt = 0;
for i in range(0,N,1):
    if x[i]**2 + y[i]**2 <= 1:
        plt.plot(x[i],y[i],'r.')
    else:
        plt.plot(x[i],y[i],'b.')

plt.axis('equal')   # plot with aspect ratio 1:1
plt.show()
\end{lstlisting}
\vspace*{-3mm}
\begin{itemize}
	\item Live demo of \texttt{dartboard.py}
\end{itemize}
\end{frame}

%==============================================================

\begin{frame}[fragile]

\frametitle{Dartboard simulation: output}

%Corners of square:
%\begin{itemize}
%	\item NE corner: $[+1,+1]$
%	\item SE corner: $[+1,-1]$
%	\item NW corner: $[-1,+1]$
%	\item SW corner: $[-1,-1]$
%\end{itemize}

\begin{itemize}
	\item two length-$N$ arrays of random numbers
	\begin{itemize}
		\item \texttt{x = [x[0],x[1],...,x[N-1]]}
		\item \texttt{y = [y[0],y[1],...,y[N-1]]}
	\end{itemize}
	\item \texttt{x[i]} $\in [-1,+1]$ for all $i$, similar for \texttt{y[i]}
	\item position $(x_i,y_i)$ of $i$-th dart is (\texttt{x[i],y[i]})
\end{itemize}

\begin{figure}[ht]
	\centering
	\includegraphics[width=0.33\textwidth]{figures/dartboard100}%
	\pause
	\includegraphics[width=0.33\textwidth]{figures/dartboard1000}%
	\pause
	\includegraphics[width=0.33\textwidth]{figures/dartboard10000}
\end{figure}

\end{frame}

%==============================================================

\begin{frame}[fragile]

\frametitle{Random floats: estimate $\pi$}

\textbf{Example 4}\\
\begin{itemize}
	\item modify Example~3 to count number of points inside (or on) circle
	\item use your results to estimate the value of $\pi$
	\pause
	\item strategy:
	\begin{itemize}
		\item area of square: $2 \times 2 = 4$
		\item area of circle: $\pi r^2$, where radius $r=1$, hence area is $\pi$
		\item[]
		\[
			R = \frac{\mathrm{area~of~circle}}{\mathrm{area~of~square}} = \frac{\pi}{4}
		\]
		\pause		
		\begin{eqnarray*}
			\mathrm{estimate~of~} \pi & = & 4\times \mathrm{estimate~of~ratio~} R \\
			& =& 4 \times \frac{\mathrm{number~of~points~in~circle}}{\mathrm{total~number~of~points}}
		\end{eqnarray*}
	\end{itemize}
\end{itemize}

\end{frame}

%==============================================================

\begin{frame}[fragile]

\frametitle{Estimate $\pi$: code}
\vspace*{-3mm}
\begin{lstlisting}[style=CStyle,basicstyle=\scriptsize]
import numpy as np
import matplotlib.pyplot as plt

# generate random array of (x,y) pairs covering
# square with edge length 2
N = 10000
x = np.random.uniform(-1, 1, N)      # N floats from [-1,1]
y = np.random.uniform(-1, 1, N)      # N floats from [-1,1]

insideCnt = 0;
for i in range(0,N,1):
    if x[i]**2 + y[i]**2 <= 1:
        plt.plot(x[i],y[i],'r.')
        insideCnt += 1
    else:
        plt.plot(x[i],y[i],'b.')

R = insideCnt/N
print('Estimate of pi: {}'.format(4*R))

plt.axis('equal')
plt.show()
\end{lstlisting}

\end{frame}

%%==============================================================
%
%\begin{frame}[fragile]
%
%\frametitle{}
%
%\begin{itemize}
%	\item xxx
%\end{itemize}
%
%\end{frame}
%
%%==============================================================
%
%\begin{frame}[fragile]
%
%\frametitle{}
%
%\begin{itemize}
%	\item xxx
%\end{itemize}
%
%\end{frame}
%
%%==============================================================
%
%\begin{frame}[fragile]
%
%\frametitle{}
%
%\begin{itemize}
%	\item xxx
%\end{itemize}
%
%\end{frame}

%==============================================================

\begin{frame}[fragile]

\frametitle{$2)$ Reading from spreadsheets}

\begin{itemize}
    \item so far have been working ``toy problems'' with small data sets
    \item engineers and scientists often work with \emph{large} datasets
        \begin{itemize}
            \item spreadsheets eg: CSV or XML files
            \item databases eg: SQL files 
            \item wide range of other software packages
        \end{itemize}
    \item there's a package for that:~it's called \texttt{pandas}
    \item[] \ldots and we can import it just like \texttt{numpy} and \texttt{matplotlib}
\begin{lstlisting}[style=CStyle]
import pandas as pd
\end{lstlisting}
\end{itemize}

\end{frame}

%==============================================================

\begin{frame}[fragile]
\frametitle{Working with Data}
\begin{itemize}
    \item To import the data is then relatively straightforward
    \item For a CSV file the instruction is
\begin{lstlisting}[style=CStyle]
import pandas as pd
mydata = pd.read_csv('filename.csv')
\end{lstlisting}
    \item If the CSV file is not in the same folder as the python project you can specify its location with a path
\begin{lstlisting}[style=CStyle]
import pandas as pd
mydata = pd.read_csv('c:\Myfolder\filename.csv')
\end{lstlisting}
\end{itemize}
\end{frame}

%==============================================================

\begin{frame}[fragile]
\frametitle{Working with Data}
\begin{itemize}
    \item Pandas will import the contents of the spreadsheet into something called a data structure (which we won't spend any time on today but may get back to later in the course)
    \item To check on the data imported into 'mydata' the following instruction is very useful
\begin{lstlisting}[style=CStyle]
print(mydata.head())
\end{lstlisting}
    \item This prints out the first few lines of the data set
\end{itemize}
\end{frame}

%==============================================================

\begin{frame}[fragile]
\frametitle{Working with Data}
\begin{itemize}
    \item One final very useful instruction will extract a column of the data and save it as a numpy array using the column name in the original spreadsheet
\begin{lstlisting}[style=CStyle]
mycolumn = mydata['column_name'].values   
\end{lstlisting}
    \item We can now process this data using the python skills we have been building in this course 
\end{itemize}
\end{frame}

%==============================================================

\begin{frame}[fragile]
\frametitle{Example}
\begin{itemize}
    \item We have a csv spreadsheet (Rainfall.csv) which records annual rainfall for 20 regions over the last 10 years
    \begin{figure} 
        \centering
        \includegraphics[width=5cm]{figures/excel_rainfall.PNG}
        \label{fig:my_label}
    \end{figure}
    \item We would like to know how many regions have seen a reduction in rainfall between 2010 and 2020
\end{itemize}
\end{frame}

%==============================================================

\begin{frame}[fragile]
\frametitle{Example}
\begin{lstlisting}[style=CStyle]
import pandas as pd

# import the rainfall data 
Rainfalldata = pd.read_csv("Rainfall.csv")
print(Rainfalldata.head())

# extract the columns for 2010 and 2020 
Rainfall2010 = Rainfalldata['Rainfall 2010'].values         
Rainfall2020 = Rainfalldata['Rainfall 2020'].values         

\end{lstlisting}
\end{frame}

%==============================================================

\begin{frame}[fragile]
\frametitle{Example}
\begin{lstlisting}[style=CStyle]
# How many regions have seen a decrease in rainfall 
N = len(Rainfall2010)
count = 0
for i in range(0, N, 1):
    if Rainfall2020[i] < Rainfall2010[i]:
        count = count + 1

percentage_decreased = count/N*100

print('Of the {} regions, {} have reduced rainfall which is {:f6.2}% '.format(N, count, percentage_decreased))
\end{lstlisting}
\end{frame}

%==============================================================

\begin{frame}[fragile]

\frametitle{Lecture summary}
\begin{enumerate}
	\item Using random numbers
	\item[]
	
	\item Reading from spreadsheets
\end{enumerate}

\begin{itemize}
	\item[]
	\item \textbf{Next lecture:~} functions in Python, Chapter 4 \\ of LL textbook
\end{itemize}

\end{frame}

\end{document}