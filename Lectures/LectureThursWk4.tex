\documentclass[english,14pt]{beamer}
\usetheme{EastLansing}
\usecolortheme{spruce}

\usepackage{xcolor}
\usepackage{listings}
\usepackage{courier}
\usepackage{graphicx}
\usepackage{amsmath}
\usepackage{algorithm2e}
\usepackage{multicol}
\usepackage{hyperref}

% http://mirrors.ibiblio.org/CTAN/macros/latex/contrib/datetime2/datetime2.pdf
\usepackage{babel}
\usepackage[useregional]{datetime2}

% https://tex.stackexchange.com/questions/42619/x-mark-to-match-checkmark
\usepackage{pifont}% http://ctan.org/pkg/pifont

%% https://stackoverflow.com/questions/1435837/how-to-remove-footers-of-latex-beamer-templates
%%gets rid of bottom navigation bars
%\setbeamertemplate{footline}[page number]
%
%gets rid of navigation symbols
\setbeamertemplate{navigation symbols}{}


\usefonttheme[onlymath]{serif}

\definecolor{mGreen}{rgb}{0,0.6,0}
\definecolor{mGray}{rgb}{0.5,0.5,0.5}
\definecolor{mPurple}{rgb}{0.8,0,0.82}
\definecolor{backgroundColour}{rgb}{0.95,0.95,0.92}
\definecolor{lightBlue}{rgb}{0.1, 0.1, 0.8}

\newcommand\red[1]{{\color{red} #1}}
\newcommand\green[1]{{\color{green} #1}}
\newcommand\blue[1]{{\color{blue} #1}}

\newcommand{\cmark}{\ding{51}}%
\newcommand{\xmark}{\ding{55}}%

\lstdefinestyle{CStyle}{
    backgroundcolor=\color{backgroundColour},   
    commentstyle=\color{mGreen},
    keywordstyle=\color{magenta},
    numberstyle=\tiny\color{mGray},
    stringstyle=\color{mPurple},
    basicstyle=\footnotesize,
    breakatwhitespace=false,         
    breaklines=true,                 
    captionpos=b,                    
    keepspaces=true,                 
    numbers=left,                    
    numbersep=5pt,                  
    showspaces=false,                
    showstringspaces=false,
    showtabs=false,                  
    tabsize=2,
    language=Python
}

\lstdefinestyle{PStyle}{
    backgroundcolor=\color{backgroundColour},   
    commentstyle=\color{mGreen},
    keywordstyle=\color{magenta},
    numberstyle=\tiny\color{mGray},
    stringstyle=\color{mPurple},
    basicstyle=\scriptsize,
    breakatwhitespace=false,         
    breaklines=true,                 
    captionpos=b,                    
    keepspaces=true,                 
    numbers=left,                    
    numbersep=5pt,                  
    showspaces=false,                
    showstringspaces=false,
    showtabs=false,                  
    tabsize=2,
    language=Python
}

\lstdefinestyle{pseudo}{
        basicstyle=\ttfamily\footnotesize,
        keywordstyle=\color{lightBlue},
        morekeywords={BEGIN,END,IF,ELSE,ENDIF,ELSEIF,PRINT,WHILE,RETURN,ENDWHILE,DO,FOR,TO,IN,ENDFOR,BREAK,INPUT},
        morecomment=[l]{//},
        commentstyle=\color{mGreen}
}

\lstset{basicstyle=\footnotesize\ttfamily,breaklines=true}
\lstset{framextopmargin=50pt,tabsize=2}

\title{ENGG1003 - Thursday Week 7}
\subtitle{Arrays and Images}
\author{Sarah Johnson}
\institute{University of Newcastle}
%\date{\today}
\date{22 April, 2021}

% following is a bit of a hack, but forces page numbers (technically: frame numbers) to run 1,2,3,... 
% with titlepage counting as frame 1

\addtocounter{framenumber}{1}
\titlepage

\begin{document}


%\begin{flushleft}
%{\scriptsize Last compiled:~\DTMnow}
%\vspace*{-5mm}
%\end{flushleft}
\framebreak

\begin{frame}[fragile]

\frametitle{Lecture Overview}
\begin{enumerate}
	\item 2D arrays for black and grey-scale images
	\item []
	
	\item 3D arrays for colour images
	
\end{enumerate}

\end{frame}






\begin{frame}[fragile]
\frametitle{ Creating an Image using 2D Arrays}
\begin{lstlisting}[style=CStyle]
import matplotlib.pyplot as plt
import numpy as np

m = np.zeros([10,10])
m[4:6, :] = 1
m[:, 4:6] = 1

# showing the image in a figure
plt.imshow(m, cmap='Greys')
plt.xticks([])
plt.yticks([])

# saving the newly created figure
plt.savefig('toy_example.png')

plt.show()
\end{lstlisting}
\end{frame}


\begin{frame}[fragile]
\frametitle{Creating an Image using 2D Arrays}
\begin{lstlisting}[style=CStyle]
m = np.zeros([10,10],int)
m[4:6, :] = 1
m[:, 4:6] = 1
\end{lstlisting}
\vspace{-1em}
\[		
\hspace{5em} m = \left[ 
\begin{array}{cccccccccc}
     0 & 0 & 0 & 0 & 1 & 1 & 0 & 0 & 0 & 0    \\
     0 & 0 & 0 & 0 & 1 & 1 & 0 & 0 & 0 & 0    \\
     0 & 0 & 0 & 0 & 1 & 1 & 0 & 0 & 0 & 0    \\   
     0 & 0 & 0 & 0 & 1 & 1 & 0 & 0 & 0 & 0    \\
     1 & 1 & 1 & 1 & 1 & 1 & 1 & 1 & 1 & 1    \\
     1 & 1 & 1 & 1 & 1 & 1 & 1 & 1 & 1 & 1    \\
     0 & 0 & 0 & 0 & 1 & 1 & 0 & 0 & 0 & 0    \\
     0 & 0 & 0 & 0 & 1 & 1 & 0 & 0 & 0 & 0    \\
     0 & 0 & 0 & 0 & 1 & 1 & 0 & 0 & 0 & 0    \\ 
     0 & 0 & 0 & 0 & 1 & 1 & 0 & 0 & 0 & 0    \\     
\end{array} \right]  
\]
\end{frame}

\begin{frame}[fragile]
\frametitle{Creating an Image using 2D Arrays}
    \begin{itemize}
		\item The \texttt{matplotlib.pyplot function} \texttt{imshow(m)} displays an image in a plot
		\begin{itemize}
     		\item each entry in the 2D array \texttt{m} is a pixel in the image
     	\end{itemize}
     		\item The \texttt{cmap} variable defines the mapping of values to colours
		\item If \texttt{cmap = 'Greys'}: 
			\begin{itemize}
			    \item the maximum value in the array is mapped to black
			    \item the minimum value in the array is mapped to white
			    \item values in between are a shade of grey relative to their value on the white-black range
			\end{itemize}
    \end{itemize}
\end{frame}

\begin{frame}[fragile]
\frametitle{Creating an Image using 2D Arrays}
 \vspace{-1em}
    \begin{figure} 
        \centering
        \includegraphics[width = 0.9\textwidth]{figures/toy_example.png}
   \end{figure}
\end{frame}


\begin{frame}[fragile]
\frametitle{Matplotlib.pyplot Figures }
    \begin{itemize}
		\item \texttt{xticks([])} and \texttt{yticks([])} remove the axis ticks
     	\item Another option is \texttt{axis('off')} to remove the axis altogether
     	\item \texttt{savefig('figurename.ext')} saves the figure. If not specified, the format is inferred from the extension (.png, .jpg etc)
        \item Note: in a script \texttt{savefig()} must come before \texttt{show()} - if you close the figure it does not exist anymore to save it 
    \end{itemize}
\end{frame}


\begin{frame}[fragile]
\frametitle{Matplotlib.pyplot Figures }
    \begin{itemize}
		\item You can create multiple figures by using a \texttt{figure()} to create a new empty plot instead of overwriting the current plot
     	\item After all the plots have been created use \texttt{show()} to show them all
        \item Note: all of these functions apply to all \texttt{pyplot} figures (such as plots) not just figures created using \texttt{imshow()} 
    \end{itemize}
\end{frame}



\begin{frame}[fragile]
\frametitle{Creating a Colour Image using 3D Arrays}
\begin{itemize}
    \item A colour image can be stored as a 3D array
    \item Think of it as a 2D array where each entry is itself a length-3 array specifying the colour in RGB (red, green, blue) format
   \vspace{0.5em}
    \begin{figure} 
        \centering
        \left[ \begin{tabular}{ccccc}
        & [R, G, B] & ... & [R, G, B] & \\
        & [R, G, B] & ... & [R, G, B] & \\
        & [R, G, B] & ... & [R, G, B] & \\
        \end{tabular} \right]
   \end{figure}
   \vspace{0.5em}
   \item E.g. red: [255, 0 ,0] green: [0, 255, 0]  \\ blue: [0, 0, 255], a purple: [65, 0, 125]
   \end{itemize}
\end{frame}


\begin{frame}[fragile]
\frametitle{Creating a Colour Image using 3D Arrays}
\begin{lstlisting}[style=CStyle]
# create a 3D array using numpy to make an image of a purple cross on a red background
m = np.zeros([20,20,3],int)
m[:, :]    = [255, 0, 0]     
m[8:12, :] = [65, 0, 125];  m[:, 8:12] = [65, 0, 125]   

plt.imshow(m); plt.axis('off')
plt.savefig('purple_cross.jpeg')
plt.show()
\end{lstlisting}
\vspace{-1em}
    \begin{figure} 
        \centering
        \includegraphics[width = 0.4\textwidth]{figures/purple_Cross.jpeg}
   \end{figure}
\end{frame}

\begin{frame}[fragile]
\frametitle{Creating a Colour Image using 3D Arrays}
	\begin{itemize}
	
     	\item For \texttt{imshow(m)}, if \texttt{m} is a 3D array with size (M,N,3) the values in \texttt{m} are interpreted as RGB 
     		\begin{itemize}
     		    \item the R, B and G values can be a float between 0 and 1 or an integer between 0 and 255
     		    \item integers above 255 (floats above 1) are cropped to 255 (respectively 1)
			\end{itemize}
		 \item There are a multitude of colour charts / colour pickers online which will give the RGB values for you. One is: \\
    \end{itemize}
    
\small \texttt{https://www.w3schools.com/colors/colors\_rgb.asp}
\end{frame}

\begin{frame}[fragile]
\frametitle{Creating a Colour Image using 3D Arrays}
\begin{figure}
\begin{lstlisting}[style=PStyle]:
# use a 3D array to recreate Tasneem's dartboard
import numpy as np
import matplotlib.pyplot as plt

def dartboard_array(N, r_cent, c_cent, radius):
    m = np.zeros([N, N, 3],int)
    for r in range(0, N):
        for c in range(0, N):
            d_to_cent = np.sqrt((r_cent-r)**2+(c_cent-c)**2)
            if d_to_cent < 0.1*radius:
                m[r, c] = [0, 255, 0]   #RGB green
            elif d_to_cent < 0.2*radius:
                m[r, c] = [255, 255, 0] #RGB yellow
            elif d_to_cent < 0.6*radius:
                m[r, c] = [255, 0, 0]   #RGB red
            elif d_to_cent < radius:
                m[r, c] = [0, 0, 255]   #RGB blue
            else:
                m[r, c] = [0, 0, 0]     #RGB black
    return m
\end{lstlisting}
\end{figure}
\end{frame}

\begin{frame}[fragile]
\frametitle{Creating a Colour Image using 3D Arrays}
\begin{lstlisting}[style=PStyle]:
#continued 

N = 500
r_center_point = round(N/2)
c_center_point = round(N/2)
radius = N/2

m = dartboard_array(N, r_center_point, c_center_point, radius)

plt.imshow(m)
plt.title('Dartboard')
plt.xticks([])
plt.yticks([])
plt.show()
\end{lstlisting}
\end{frame}


\begin{frame}[fragile]
\frametitle{Importing Images}
	\begin{itemize}
     	\item There are a number of libraries to import and process figures, one of which is the \texttt{matplotlib} \texttt{image} module 

\begin{lstlisting}[style=PStyle]
import matplotlib.pyplot as plt
import matplotlib.image as mpimg

img = mpimg.imread('stinkbug.png')
plt.imshow(img)
plt.show()
\end{lstlisting}
     	\item A benefit, for us, with \texttt{image}  is that the image is imported into a \texttt{numpy} array
    \end{itemize}
\end{frame}



\begin{frame}[fragile]
\frametitle{Importing Images}
print(img):
\begin{lstlisting}[style=PStyle]

[[[0.40784314 0.40784314 0.40784314]
  [0.40784314 0.40784314 0.40784314]
  [0.40784314 0.40784314 0.40784314]
  ...
  [0.42745098 0.42745098 0.42745098]
  [0.42745098 0.42745098 0.42745098]
  [0.42745098 0.42745098 0.42745098]]
 [[0.4117647  0.4117647  0.4117647 ]
  [0.4117647  0.4117647  0.4117647 ]
  [0.4117647  0.4117647  0.4117647 ]
  ...
  [0.42745098 0.42745098 0.42745098]
  [0.42745098 0.42745098 0.42745098]
  [0.42745098 0.42745098 0.42745098]]

\end{lstlisting}

\end{frame}

\end{document}