\documentclass[14pt]{beamer}
\usetheme{Dresden}
\usecolortheme{orchid}

\usepackage{xcolor}
\usepackage{listings}
\usepackage{courier}
\usepackage{graphicx}
\usepackage{amsmath}
\usepackage{algorithm2e}

\usefonttheme[onlymath]{serif}

\definecolor{mGreen}{rgb}{0,0.6,0}
\definecolor{mGray}{rgb}{0.5,0.5,0.5}
\definecolor{mPurple}{rgb}{0.8,0,0.82}
\definecolor{backgroundColour}{rgb}{0.95,0.95,0.92}

\lstdefinestyle{CStyle}{
    backgroundcolor=\color{backgroundColour},   
    commentstyle=\color{mGreen},
    keywordstyle=\color{magenta},
    numberstyle=\tiny\color{mGray},
    stringstyle=\color{mPurple},
    basicstyle=\footnotesize,
    breakatwhitespace=false,         
    breaklines=true,                 
    captionpos=b,                    
    keepspaces=true,                 
    numbers=left,                    
    numbersep=5pt,                  
    showspaces=false,                
    showstringspaces=false,
    showtabs=false,                  
    tabsize=2,
    language=C
}

\lstset{basicstyle=\footnotesize\ttfamily,breaklines=true}
\lstset{framextopmargin=50pt,frame=bottomline}

\title{ENGG1003 - Tuesday Week 2}
\subtitle{C Arithmetic\\Datatypes\\Standard Input-Output}
\author{Brenton Schulz}
\institute{University of Newcastle}
\date{\today}

\begin{document}
\titlepage

\begin{frame}
\frametitle{C Arithmetic}
\begin{itemize}
\item Basic arithmetic was seen in the lab
\begin{itemize}
\item You all did the lab, right?
\end{itemize}
\begin{table}[H]
\centering
\begin{tabular}{|l|c|}
\hline
Operation      & C Symbol \\
\hline
Addition       & +        \\
Subtraction    & -        \\
Multiplication & *        \\
Division       & /       \\
\hline
\end{tabular}
\caption{Basic arithmetic operators in C}
\end{table}
\item Complex expressions can be built from these operators and parentheses
\end{itemize}
\end{frame}

\begin{frame}
\frametitle{C Arithmetic}
Examples:
\begin{small}
\begin{table}
\centering
\begin{tabular}{|c|c|}
\hline
$z = x^2 + 5(y + b)$ & \texttt{z = x*x + 5*(y + b);} \\
$u = \frac{x + 1}{x - 1}$ & \texttt{u = (x + 1)/(x - 1);} \\
$v = z^3 + \frac{5(y + b)}{2}$ & \texttt{v = z*z*z+(5*(y + b))/2;} \\
\hline
\end{tabular}
\end{table}
\end{small}

\begin{itemize}
\item Multiplication is not assumed. If you write \texttt{5(y+b)} the compiler will generate a syntax error.
\item To be valid C expressions the semicolon is required.
\end{itemize}
\end{frame}

\begin{frame}
\frametitle{C Arithmetic}
\begin{itemize}
\item C supports two time-saving \textit{unary} operators:
\begin{itemize}
\item Very useful in loops.
\end{itemize}

\begin{table}
\centering
\begin{tabular}{|c|c|c|}
\hline
Operation & C Syntax & Replaces\\
\hline
Increment & \texttt{x++;} or \texttt{++x;} & \texttt{x = x + 1;} \\
Decrement & \texttt{x--;} or \texttt{--x;} & \texttt{x = x - 1;} \\
\hline
\end{tabular}
\end{table}

\item It also supports the following shorthand syntax:

\begin{table}
\centering
\begin{tabular}{|c|c|c|}
\hline
\texttt{x = x + y;} & \texttt{x += y;} \\
\texttt{x = x - y;} & \texttt{x -= y;} \\
\texttt{x = x * y;} & \texttt{x *= y;} \\
\texttt{x = x / y;} & \texttt{x /= y;} \\
\hline
\end{tabular}
\end{table}
\end{itemize}
\end{frame}

\begin{frame}
\frametitle{C Arithmetic Example}
TODO: The cos() series from last lecture
\end{frame}

\end{document}
