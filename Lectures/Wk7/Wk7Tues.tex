\documentclass[14pt]{beamer}
\usetheme{Dresden}
\usecolortheme{orchid}

\usepackage{xcolor}
\usepackage{listings}
\usepackage{courier}
\usepackage{graphicx}
\usepackage{amsmath}
\usepackage{algorithm2e}
\usepackage{multicol}
\usepackage{amssymb}

\usefonttheme[onlymath]{serif}

\definecolor{mGreen}{rgb}{0,0.6,0}
\definecolor{mGray}{rgb}{0.5,0.5,0.5}
\definecolor{mPurple}{rgb}{0.8,0,0.82}
\definecolor{backgroundColour}{rgb}{0.95,0.95,0.92}
\definecolor{lightBlue}{rgb}{0.1, 0.1, 0.8}

\lstdefinestyle{CStyle}{
    backgroundcolor=\color{backgroundColour},   
    commentstyle=\color{mGreen},
    keywordstyle=\color{magenta},
    numberstyle=\tiny\color{mGray},
    stringstyle=\color{mPurple},
    basicstyle=\footnotesize\ttfamily,
    breakatwhitespace=false,         
    breaklines=true,                 
    captionpos=b,                    
    keepspaces=true,                 
    numbers=left,                    
    numbersep=5pt,                  
    showspaces=false,                
    showstringspaces=false,
    showtabs=false,                  
    tabsize=2,
    language=C
}

\lstdefinestyle{Ctable}{
    backgroundcolor=\color{backgroundColour},   
    commentstyle=\color{mGreen},
    keywordstyle=\color{magenta},
    numberstyle=\tiny\color{mGray},
    stringstyle=\color{mPurple},
    basicstyle=\footnotesize\ttfamily,
    breakatwhitespace=false,         
    breaklines=true,                 
    captionpos=b,                    
    keepspaces=true,                                  
    showspaces=false,                
    showstringspaces=false,
    showtabs=false,                  
    tabsize=2,
    language=C
}

\lstdefinestyle{pseudo}{
        basicstyle=\ttfamily\footnotesize,
        keywordstyle=\color{lightBlue},
        morekeywords={BEGIN,END,IF,ELSE,ENDIF,ELSEIF,PRINT,WHILE,RETURN,ENDWHILE,DO,FOR,TO,IN,ENDFOR,BREAK,INPUT,READ},
        morecomment=[l]{//},
        commentstyle=\color{mGreen}
}

\lstset{basicstyle=\footnotesize\ttfamily,breaklines=true}
\lstset{framextopmargin=50pt,tabsize=2}

\title{ENGG1003 - Tuesday Week 7}
\subtitle{File I/O\\More Pointers}
\author{Brenton Schulz}
\institute{University of Newcastle}
\date{\today}

\begin{document}
\titlepage

\begin{frame}[fragile]
\frametitle{Che C Documentation}
\begin{itemize}
\item Linux systems have a program called ``man''
	\begin{itemize}
		\item Short for ``manual''
	\end{itemize}
\item It is used to display a wide variety of documentation called ``man pages''
\item To install it type this in the terminal:
\begin{lstlisting}[style=pseudo]
sudo apt update 
sudo apt install man
\end{lstlisting}
and press \texttt{y} (or \texttt{$<$enter$>$}) when prompted to confirm installation
\item Afterwards, C documentation can be accessed by typing \texttt{man $<$topic$>$}
\end{itemize}
\end{frame}

\begin{frame}
\frametitle{Che C Documentation}
\begin{itemize}
\item For example, all library functions have a \texttt{man} page you can read by typing:\\\texttt{man $<$function name$>$}
\item eg, try:
	\begin{itemize}
		\item \texttt{man fopen}
		\item \texttt{man printf}
		\item \texttt{man sin}
		\item \texttt{man string}
		\item etc..
	\end{itemize}
\end{itemize}
\end{frame}

\begin{frame}[fragile]
\frametitle{File I/O}
\begin{itemize}
\item A stream is kept in a variable of type \texttt{FILE~*}
	\begin{itemize}
		\item Read as ``pointer to \texttt{FILE}'' or ``\texttt{FILE}-star''
	\end{itemize}
\item Three already exist in your C programs:
	\begin{itemize}
		\item \texttt{stdin}
		\item \texttt{stdout}
		\item \texttt{stderr}
	\end{itemize}
\item Additional streams are declared like other variables, eg:
\begin{lstlisting}[style=CStyle]
FILE *input, *output;
\end{lstlisting}
\end{itemize}
\end{frame}

\begin{frame}[fragile]
\frametitle{Correction: String Initialisation}
\begin{itemize}
\item This is totally fine:
\begin{lstlisting}[style=CStyle]
char string[] = "initial value";
\end{lstlisting}
\item The compiler copies the string literal into \texttt{string[]}
\item The length is automatically calculated
	\begin{itemize}
		\item You may specify a length \textit{longer} than necessary:
\begin{lstlisting}[style=CStyle]
char string[1024] = "initial value";			
\end{lstlisting}
	\end{itemize}
\pause
\item A constant string is created with:
\begin{lstlisting}[style=CStyle]
char *str = "some string";
\end{lstlisting}
	\begin{itemize}
		\item We will study this \textit{pointer} syntax later
	\end{itemize}
\end{itemize}
\end{frame}

\begin{frame}[fragile]
\frametitle{File I/O - Quick Review}
\begin{itemize}
\item Before a file can be accessed you must \textit{open} it with the \texttt{fopen()} function
\item In order to open files you need two pieces of information:
	\begin{itemize}
		\item The file's name
		\item The data direction (mode)
			\begin{itemize}
				\item Reading
				\item Writing
				\item Both
			\end{itemize}
	\end{itemize}
\end{itemize}
\end{frame}

\begin{frame}[fragile]
\frametitle{File I/O}
\begin{itemize}
\item \texttt{fopen()}'s function prototype is:
\end{itemize}
\begin{lstlisting}[style=CStyle]
FILE *fopen(const char *name, const char *mode);
\end{lstlisting}
\begin{itemize}
\item \texttt{const char *name} is a string holding the file's name
\item \texttt{const char *mode} is a string describing the desired data direction
\item Both of these can be passed as variable strings or hard-coded
\end{itemize}
\end{frame}

\begin{frame}[fragile]
\frametitle{File I/O}
\begin{itemize}
\item The \texttt{*mode} argument can be one of the following:
	\begin{itemize}
		\item \texttt{"r"} (reading)
		\item \texttt{"r+"} (reading and writing)
		\item \texttt{"w"} (writing)
		\item \texttt{"w+"} (reading and writing, file truncated)
		\item \texttt{"a"} (appending)
		\item \texttt{"a+"} (reading and appending)
	\end{itemize}
\item Read \underline{\href{http://man7.org/linux/man-pages/man3/fopen.3.html}{documentation}} for details
\item \texttt{fopen()} example:
\begin{lstlisting}[style=CStyle]
FILE *input;
input = fopen("data.txt", "r");
\end{lstlisting}
\end{itemize}
\end{frame}

\begin{frame}[fragile]
\frametitle{\texttt{fopen()} Errors}
\begin{itemize}
\item The return value of \texttt{fopen()} is \texttt{NULL} on error
\item Check it! Attempting to access a \texttt{NULL} stream will result in a segmentation fault!
\begin{lstlisting}[style=CStyle]
FILE *input;
input = fopen("data", "r");
if(input == NULL) {
	perror("fopen()");
	return;
}
\end{lstlisting}
\item \texttt{perror()} prints a user-friendly error message
\end{itemize}
\end{frame}

\begin{frame}[fragile]
\frametitle{File I/O}
\begin{itemize}
\item Once opened, a file can be accessed with:
	\begin{itemize}
		\item \texttt{fscanf()}
		\item \texttt{fprintf()}
	\end{itemize}
\item These functions behave just like \texttt{scanf()} and \texttt{printf()} except they take an extra argument:
\begin{lstlisting}[style=CStyle]
int fscanf(FILE *stream, const char *format, ...);
\end{lstlisting}
\item The first argument is a \texttt{FILE *}
\item The rest is identical to \texttt{printf()} and \texttt{scanf()}
\end{itemize}
\end{frame}

\begin{frame}
\frametitle{File I/O - Position Indicators}
\begin{itemize}
\item Concept: bytes in files have an address known as a \textit{position indicator}
\item The address is the number of bytes, starting at zero, from the start of the file
\item Unless otherwise controlled, files are only read from and written to \textit{sequentially}
\item The position indicator automatically increments when a byte is read or written
\end{itemize}
\end{frame}

\begin{frame}[fragile]
\frametitle{File I/O - Position Indicators}
\begin{itemize}
\item Some useful functions:
	\begin{itemize}
		\item \texttt{ftell()} - Returns the position indicator
		\item \texttt{fseek()} - Sets the position indicator
		\item \texttt{feof()} - Returns TRUE if the position indicator is at the end of the file
	\end{itemize}
\item For example, to process data until the end of file is reached:
\begin{lstlisting}[style=CStyle]
FILE *stream;
// open file etc
while(!feof(stream)) {
	// Read from file
	// Do stuff
}
\end{lstlisting}
\end{itemize}
\end{frame}

\begin{frame}
\frametitle{File I/O Example}
Write a C program which opens a file, \texttt{test.txt}, and prints its contents to \texttt{stdout}, reading and writing one character at a time.
\begin{itemize}
\pause
\item Declare \texttt{FILE *input;}
\pause
\item Use \texttt{fopen()} to open it for reading
\pause
\item Write a loop which reads and writes characters until the whole file has been read
	\begin{itemize}
		\item Read with: \texttt{fscanf(input, "\%c", \&c);}
		\item Write with: \texttt{fprintf("\%c", c);}
	\end{itemize}
\end{itemize}
\end{frame}

\begin{frame}
\frametitle{File I/O Example 1}
Write a C program which opens a file, \texttt{input.txt}, then reads and prints each character to the console on a new line, indicating the position indicator's value \textit{after} reading each character.
\end{frame}

\begin{frame}
\frametitle{File I/O Example 2}
Write a C program which copies a file, \texttt{input.txt}, into a new file, \texttt{output.txt}. While copying, the program should count how many spaces there are in the input and print the final count to the terminal before exiting.
\end{frame}

\begin{frame}
\frametitle{File I/O Example 3}
Write a C program which opens a file, \texttt{input.txt}, and counts the number of times the string \texttt{"the"} appears.
\\~\\
The program should include a function, \texttt{isThe()}, which tests if a string is equal to \texttt{"the"} or not.
\end{frame}

\begin{frame}
\frametitle{Pointers}
\begin{itemize}

\end{itemize}
\end{frame}

\end{document}
