\documentclass[english,14pt]{beamer}
\usetheme{EastLansing}
\usecolortheme{spruce}

\usepackage{xcolor}
\usepackage{listings}
\usepackage{courier}
\usepackage{graphicx}
\usepackage{amsmath}
\usepackage{algorithm2e}
\usepackage{multicol}
\usepackage{hyperref}
\usepackage{textcomp}

% http://mirrors.ibiblio.org/CTAN/macros/latex/contrib/datetime2/datetime2.pdf
\usepackage{babel}
\usepackage[useregional]{datetime2}

% https://tex.stackexchange.com/questions/42619/x-mark-to-match-checkmark
\usepackage{pifont}% http://ctan.org/pkg/pifont

%% https://stackoverflow.com/questions/1435837/how-to-remove-footers-of-latex-beamer-templates
%%gets rid of bottom navigation bars
%\setbeamertemplate{footline}[page number]
%
%gets rid of navigation symbols
\setbeamertemplate{navigation symbols}{}


\usefonttheme[onlymath]{serif}

\definecolor{mGreen}{rgb}{0,0.6,0}
\definecolor{mGray}{rgb}{0.5,0.5,0.5}
\definecolor{mPurple}{rgb}{0.8,0,0.82}
\definecolor{backgroundColour}{rgb}{0.95,0.95,0.92}
\definecolor{lightBlue}{rgb}{0.1, 0.1, 0.8}
\definecolor{darkGreen}{rgb}{0, 0.39, 0}

\newcommand\red[1]{{\color{red} #1}}
\newcommand\green[1]{{\color{green} #1}}
\newcommand\blue[1]{{\color{blue} #1}}
\newcommand\darkGreen[1]{{\color{darkGreen} #1}}

\newcommand{\cmark}{\ding{51}}%
\newcommand{\xmark}{\ding{55}}%

\lstdefinestyle{CStyle}{
    backgroundcolor=\color{backgroundColour},   
    commentstyle=\color{mGreen},
    keywordstyle=\color{magenta},
    numberstyle=\tiny\color{mGray},
    stringstyle=\color{mPurple},
    basicstyle=\footnotesize,
    breakatwhitespace=false,         
    breaklines=true,                 
    captionpos=b,                    
    keepspaces=true,                 
    numbers=left,                    
    numbersep=5pt,                  
    showspaces=false,                
    showstringspaces=false,
    showtabs=false,                  
    tabsize=2,
    language=Python
}

\lstdefinestyle{pseudo}{
        basicstyle=\ttfamily\footnotesize,
        keywordstyle=\color{lightBlue},
        morekeywords={BEGIN,END,IF,ELSE,ENDIF,ELSEIF,PRINT,WHILE,RETURN,ENDWHILE,DO,FOR,TO,IN,ENDFOR,BREAK,INPUT,CONDITIONS},
        morecomment=[l]{//},
        commentstyle=\color{mGreen}
}

\lstset{basicstyle=\footnotesize\ttfamily,breaklines=true}
\lstset{framextopmargin=50pt,tabsize=2}

\title{ENGG1003 - Thursday Week 10}
\subtitle{Assignment 2: Image processing}
\author{Steve Weller}
\institute{University of Newcastle}
%\date{\today}
\date{13 May 2021}

% following is a bit of a hack, but forces page numbers (technically: frame numbers) to run 1,2,3,... 
% with titlepage counting as frame 1

\addtocounter{framenumber}{1}
\titlepage

\begin{document}

\begin{flushleft}
{\scriptsize Last compiled:~\DTMnow}
\vspace*{-5mm}
\end{flushleft}
\framebreak

%==============================================================

\begin{frame}[fragile]

\frametitle{Lecture overview}
\begin{enumerate}
	\item images as 3D arrays
%	\begin{itemize}
%		\item review Sarah's material
%	\end{itemize}
	
	\item digital image formats
%	\begin{itemize}
%		\item raster image format
%		\item colourspaces: RGB and HSL
%	\end{itemize}

	\item data types
%	\begin{itemize}
%		\item uint8, uint16, float32, float64
%	\end{itemize}	
		
	\item structure of assignment
		\begin{itemize}
			\item five (5) functions to get started; no marks, infinite help
			\item eight (8) functions, write some or all for 15 marks
		\end{itemize}	

	\item how to go about the assignment
		\begin{itemize}
			\item week 10 lab
			\item lab question sheet in BB $>$ Assessment
			\item write a module called imageProcessing, upload to BB
			\item test scripts; getting assessed
		\end{itemize}	
	
\end{enumerate}

\end{frame}

%==============================================================

\begin{frame}[fragile]

\frametitle{$1)$ images as 3D arrays}

\begin{itemize}
% bitmap image is composed of a fixed set of pixels, while the vector image is composed of a fixed set of shapes. In the picture, scaling the bitmap reveals the pixels while scaling the vector image preserves the shapes.

% https://en.wikipedia.org/wiki/Scalable_Vector_Graphics

	\item raster images
	\begin{itemize}
		\item gif, jpeg, png
		\item contrast with vector images: svg
	\end{itemize}

	\item[]
	
	\item review of Sarah's material
\end{itemize}

\end{frame}

%==============================================================

\begin{frame}[fragile]

\frametitle{$2)$ digital image formats}

\begin{itemize}

	\item colourspaces
	\begin{itemize}
		\item RGB
		\item HSL
		\item two different ways of represeting the \emph{same} colour
		\item key theme of assignment: RGB \texttt{<-->} HSL
	\end{itemize}
	\item[]
	\item $[0,1]$ and $[0,255]$
	\item[]
	\item use colour images and links to colour picker
\end{itemize}

\end{frame}

%==============================================================

\begin{frame}[fragile]

\frametitle{$3)$ data types}

\begin{itemize}
	\item uint8
	\item uint16
	\item float32
	\item float64
	\item[]
	\item type conversions
		
\end{itemize}

\end{frame}

%==============================================================

\begin{frame}[fragile]

\frametitle{$4)$ structure of assignment}

 first 5 functions
\begin{itemize}
	\item \texttt{loadImage}
	\begin{itemize}
		\item read image file into 3D numpy array
	\end{itemize}
		
	\item \texttt{saveImage}
	\begin{itemize}
		\item save 3D numpy array as image file
	\end{itemize}
			
	\item \texttt{rgb2hsl}
	\begin{itemize}
		\item convert image in RGB format to HSL format
	\end{itemize}
			
	\item \texttt{rgb2hsl}
	\begin{itemize}
		\item convert image in HSL format to RGB format
	\end{itemize}
			
	\item \texttt{showImage}
	\begin{itemize}
		\item display image in window
	\end{itemize}
\end{itemize}

\end{frame}

%==============================================================

\begin{frame}[fragile]

\frametitle{$5)$ how to go about the assignment}

eight (8) functions to be graded in assignment

\begin{itemize}
	\item \texttt{brightness}
	\begin{itemize}
		\item adjust image brightness
	\end{itemize}
		
	\item \texttt{contrast}
	\begin{itemize}
		\item adjust image contrast
	\end{itemize}
			
	\item \texttt{saturation}
	\begin{itemize}
		\item adjust image saturation
	\end{itemize}
			
	\item \texttt{toneMap}
	\begin{itemize}
		\item adjust image by setting H and S channels of each pixel
	\end{itemize}
\end{itemize}

\end{frame}

%==============================================================

\begin{frame}[fragile]

\frametitle{}

eight (8) functions to be graded in assignment (ctd.)

\begin{itemize}
	\item \texttt{crops}
	\begin{itemize}
		\item crop image
	\end{itemize}
		
	\item \texttt{histogram}
	\begin{itemize}
		\item plot histogram of image
	\end{itemize}
			
	\item \texttt{saturated}
	\begin{itemize}
		\item compute percentage of pixels which have at least one RGB channel value which has undergone clipping saturation
	\end{itemize}
			
	\item \texttt{unsharpMask}
	\begin{itemize}
		\item apply image sharpening technique to image
	\end{itemize}
\end{itemize}

\end{frame}

%==============================================================

\begin{frame}[fragile]

\frametitle{strategies}

\begin{itemize}
	\item start small
	\item Lab sheet week 10 first
	\item test RGB/HSL conversion against colour picker
	\item remember first 5 functions: infinite help from discord, demonstrators, fellow students
	\begin{itemize}
		\item no marks for these questions
	\end{itemize}
	\item while submission to BB will be a single file imageProcessing.py with definitions and code for eight (8) functions, encouraged to develop and test as follows:
	\begin{itemize}
		\item each function own script, test it
		\item define code into function in same file, test it
		\item copy/paste working function into imageProcessing.py
	\end{itemize}
\end{itemize}

\end{frame}

%==============================================================

\begin{frame}[fragile]

\frametitle{Lecture summary}
\begin{itemize}
	\item xxx
\end{itemize}

\end{frame}

\end{document}