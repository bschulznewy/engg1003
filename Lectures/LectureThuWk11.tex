\documentclass[english,14pt]{beamer}
\usetheme{EastLansing}
\usecolortheme{spruce}

\usepackage{xcolor}
\usepackage{listings}
\usepackage{courier}
\usepackage{graphicx}
\usepackage{amsmath}
\usepackage{algorithm2e}
\usepackage{multicol}
\usepackage{hyperref}
\usepackage{textcomp}

% http://mirrors.ibiblio.org/CTAN/macros/latex/contrib/datetime2/datetime2.pdf
\usepackage{babel}
\usepackage[useregional]{datetime2}

% https://tex.stackexchange.com/questions/42619/x-mark-to-match-checkmark
\usepackage{pifont}% http://ctan.org/pkg/pifont

%% https://stackoverflow.com/questions/1435837/how-to-remove-footers-of-latex-beamer-templates
%%gets rid of bottom navigation bars
%\setbeamertemplate{footline}[page number]
%
%gets rid of navigation symbols
\setbeamertemplate{navigation symbols}{}


\usefonttheme[onlymath]{serif}

\definecolor{mGreen}{rgb}{0,0.6,0}
\definecolor{mGray}{rgb}{0.5,0.5,0.5}
\definecolor{mPurple}{rgb}{0.8,0,0.82}
\definecolor{backgroundColour}{rgb}{0.95,0.95,0.92}
\definecolor{lightBlue}{rgb}{0.1, 0.1, 0.8}
\definecolor{darkGreen}{rgb}{0, 0.39, 0}

\newcommand\red[1]{{\color{red} #1}}
\newcommand\green[1]{{\color{green} #1}}
\newcommand\blue[1]{{\color{blue} #1}}
\newcommand\darkGreen[1]{{\color{darkGreen} #1}}

\newcommand{\cmark}{\ding{51}}%
\newcommand{\xmark}{\ding{55}}%

\lstdefinestyle{CStyle}{
    backgroundcolor=\color{backgroundColour},   
    commentstyle=\color{mGreen},
    keywordstyle=\color{magenta},
    numberstyle=\tiny\color{mGray},
    stringstyle=\color{mPurple},
    basicstyle=\footnotesize,
    breakatwhitespace=false,         
    breaklines=true,                 
    captionpos=b,                    
    keepspaces=true,                 
    numbers=left,                    
    numbersep=5pt,                  
    showspaces=false,                
    showstringspaces=false,
    showtabs=false,                  
    tabsize=2,
    language=Python
}

\lstdefinestyle{MStyle}{
    backgroundcolor=\color{backgroundColour},   
    commentstyle=\color{mGreen},
    keywordstyle=\color{magenta},
    numberstyle=\tiny\color{mGray},
    stringstyle=\color{mPurple},
    basicstyle=\footnotesize,
    breakatwhitespace=false,         
    breaklines=true,                 
    captionpos=b,                    
    keepspaces=true,                 
    numbers=left,                    
    numbersep=5pt,                  
    showspaces=false,                
    showstringspaces=false,
    showtabs=false,                  
    tabsize=2,
    language=MATLAB
}

\lstdefinestyle{pseudo}{
        basicstyle=\ttfamily\footnotesize,
        keywordstyle=\color{lightBlue},
        morekeywords={BEGIN,END,IF,ELSE,ENDIF,ELSEIF,PRINT,WHILE,RETURN,ENDWHILE,DO,FOR,TO,IN,ENDFOR,BREAK,INPUT,CONDITIONS},
        morecomment=[l]{//},
        commentstyle=\color{mGreen}
}

\lstset{basicstyle=\footnotesize\ttfamily,breaklines=true}
\lstset{framextopmargin=50pt,tabsize=2}

\title{ENGG1003 - Thursday Week 11}
\subtitle{MATLAB vs.~Python}
\author{Sarah Johnson and Steve Weller}
\institute{University of Newcastle}
%\date{\today}
\date{20 May 2021}

% following is a bit of a hack, but forces page numbers (technically: frame numbers) to run 1,2,3,... 
% with titlepage counting as frame 1

\addtocounter{framenumber}{1}
\titlepage

\begin{document}

\begin{flushleft}
{\scriptsize Last compiled:~\DTMnow}
\vspace*{-5mm}
\end{flushleft}
\framebreak

%==============================================================

\begin{frame}[fragile]

\frametitle{Lecture overview}
\begin{enumerate}
	\item blah

	\item[]

	\item blah

	\item[]
	
	\item blah

	\item[]

\end{enumerate}

\end{frame}

%==============================================================

\begin{frame}[fragile]

\frametitle{MATLAB vs.~Python}

\begin{itemize}
	\item xxx
\end{itemize}

\href{https://realpython.com/matlab-vs-python/}{https://realpython.com/matlab-vs-python/}
\end{frame}

%==============================================================

\begin{frame}[fragile]

\frametitle{MATLAB}

\begin{itemize}
	\item MATLAB is an abbreviation for Matrix Laboratory. It is a programming language perfectly suited for matrix manipulation and linear algebra.
	\item MATLAB offers many additional Toolboxes such as control design, image processing, digital signal processing, fluid dynamics etc. They are all very well documented.
	\item MATLAB includes Simulink which allows visualisation of systems begin simulated which can be very useful in many engineering applications.
\end{itemize}

\end{frame}

%==============================================================

\begin{frame}[fragile]
\frametitle{MATLAB}
\begin{itemize}
\item For this reason you may use MATLAB in some courses later on. A non exhaustive list of courses currently using MATLAB includes: \\

CHEE4945, CHEE4975,  ENGG2440, AERO3600, MCHA3400, MCHA3500, MCHA4100 ELEC2132, ELEC2430, ELEC3400, ELEC3410, ELEC3540, ELEC4100, many FYPs
\end{itemize}
\end{frame}

%==============================================================

\begin{frame}[fragile]

\frametitle{Why are we teaching Python then?}

\begin{itemize}
	\item Unlike Python which is free, MATLAB licences are very expensive. You will be able to keep programming in Python once you graduate.  
	\item Matplotlib and Numpy have functionality very similar to MATLAB (making it easier to move across).
	\item Python libraries offer much of the functionality as the MATLAB toolboxes and are growing much faster. 
	\item Python has become a very popular, in demand language. Many students are finding that employers are requesting Python coding skills for both positions and internships. 
	\item Python is a general-purpose programming language. 
\end{itemize}
\end{frame}

%==============================================================

\begin{frame}[fragile]

\frametitle{Syntax}

A few general differences between MATLAB and Python:

\begin{itemize}
	\item In MATLAB comments start with \%. In Python, comments start with \#.
	\item White-space and indenting are very important in Python. MATLAB does not require the same (but it is highly recommended anyway for readability) 
	\item MATLAB function \texttt{disp( )} replaces \texttt{print( )}
	\item help for a function via \texttt{help fname} rather than \texttt{help(fname)}
\end{itemize}
\end{frame}

%==============================================================

\begin{frame}[fragile]

\frametitle{Maths}

\begin{itemize}

\item Addition, subtraction, multiplication and division are the same. A difference is that
	\begin{itemize}
		\item  MATLAB uses \^ \;\; not ** for exponential. 
		\item I.e.  \texttt{x**2} becomes \texttt{x\^\;2}
	\end{itemize}
\item Relational operators $==$, $>$, $<$, $>=$, $<=$ are the same, except
	\begin{itemize}
		\item MATLAB uses $\sim =$  instead of $!=$ for 'not equal to'
	\end{itemize}
\item In MATLAB the value of a variable is automatically printed to the terminal. You use ';' to suppress     \begin{itemize}
\item I.e. \texttt{a = 3} prints value of a, whereas \texttt{a = 3;} does not
	\end{itemize}
\end{itemize}
\end{frame}

%==============================================================

\begin{frame}[fragile]

\frametitle{Flow control}

\begin{itemize}
	\item If Else statements and For / While loops work exactly the same way in both languages with some small differences in syntax
	\begin{itemize}
		\item MATLAB does not need the ':' used in Python at the end of the loop definition / if condition
		\item MATLAB designates the end of an If statement or loop by 'end' instead of by indenting
		\item Python shortens \texttt{elseif} to \texttt{elif}. MATLAB does not.
	\end{itemize}
\end{itemize}
\end{frame}

%==============================================================

\begin{frame}[fragile]

\frametitle{Flow control}

 E.g. Nested If Else

\begin{lstlisting}[style=CStyle]
num = 10                      # Python
if num == 10:
    print("num is equal to 10")
elif num == 20:
    print("num is equal to 20")
else:
    print("num is neither 10 nor 20")
\end{lstlisting}   

\begin{lstlisting}[style=MStyle]
num = 10                    % MATLAB
if num == 10
    disp("num is equal to 10")
elseif num == 20
    disp("num is equal to 20")
else
    disp("num is neither 10 nor 20")
end
\end{lstlisting}

\end{frame}

%==============================================================

\begin{frame}[fragile]

\frametitle{Flow control}

E.g. Use a For loop to add the integers from 1 to 10

\begin{lstlisting}[style=CStyle]
sum = 0                  # Python
for i in range(1,11):
    sum = sum+i
\end{lstlisting}
\begin{lstlisting}[style=MStyle]
sum = 0                  % MATLAB
for i = 1:10
    sum = sum+i
end
\end{lstlisting} 
\end{frame}

%==============================================================

\begin{frame}[fragile]

\frametitle{Flow control}

E.g. Use a While loop to find the first (smallest) integer which when squared is greater than 100

\begin{lstlisting}[style=CStyle]
n = 1                        # Python
n_squared = 1
while n_squared < 100:
    n = n+1
    n_squared = n**2
\end{lstlisting}
\begin{lstlisting}[style=MStyle]
n = 1                       % MATLAB
n_squared = 1
while n_squared < 100
    n = n+1
    n_squared = n^2
end
\end{lstlisting} 
\end{frame}

%==============================================================

\begin{frame}[fragile]

\frametitle{Arrays}

\begin{itemize}
	\item MATLAB arrays work a lot like \texttt{numpy} arrays. If you can work with Numpy arrays you will find MATLAB arrays are easy. Both do vectorisation in the same way
	\item There are a couple of syntax differences to note:
	\begin{itemize}
		\item In MATLAB, when you want to index an array, you use round brackets \texttt{'()'}. Square brackets \texttt{'[]'} are used to create arrays
\begin{lstlisting}[style=CStyle]
arr = np.array([10, 20, 30])   # Python
s = arr[i]
\end{lstlisting}
\begin{lstlisting}[style=MStyle]
arr = [10, 20, 30]             % MATLAB
s = arr(i)
\end{lstlisting}
    \end{itemize}
\end{itemize}

\end{frame}

%==============================================================

\begin{frame}[fragile]

\frametitle{Arrays}

	\begin{itemize}
		\item 2D arrays use ';' to designate the next row
		
		\begin{lstlisting}[style=CStyle]
a = np.array([[2,3],[4,5]])    # Python
s = arr[i,j]
\end{lstlisting}
\begin{lstlisting}[style=MStyle]
a = [2 3; 4 5]              % MATLAB
s = a(i,j)
\end{lstlisting}

		\item All zero / all one matrices are very similar
		
		\begin{lstlisting}[style=CStyle]
m = np.zeros([5,10,3])  # Python 5x10x3 array
\end{lstlisting}
\begin{lstlisting}[style=MStyle]
m = zeros(5,10,3)       % MATLAB 5x10x3 array
\end{lstlisting}

	      \item In MATLAB a new copy of an array is created by default
	      
\begin{lstlisting}[style=CStyle]
b = a.copy()  # Python
\end{lstlisting}
\begin{lstlisting}[style=MStyle]
b = a         % MATLAB
\end{lstlisting}

	\end{itemize}

\end{frame}

%==============================================================

\begin{frame}[fragile]

\frametitle{Arrays}

\begin{itemize}
	\item Many Numpy functions exist in MATLAB including \texttt{linspace() sqrt() log() exp() round() ceil() floor() max() min() mean()} and many more
	\item A few tips to keep in mind:
	\begin{itemize}
		\item In Python, the index of the first element in an array is '0', in MATLAB it is '1'
		\item In Python, the index of the last element in an array is '-1', in MATLAB it is '\texttt{end}'
		\item The length of an array is \texttt{length} rather than \texttt{len}
		\item Array slicing works similarly in MATLAB to Python
            \item You don't need to call \texttt{linspace()} to create an array \texttt{start:step:end} will do
\begin{lstlisting}[style=MStyle]
a = 5:1:100              % MATLAB
\end{lstlisting}
	\end{itemize}
\end{itemize}
\end{frame}

%==============================================================

\begin{frame}[fragile]

\frametitle{Plotting}

\begin{itemize}       
	\item Plotting in MATLAB is very similar to \texttt{Matplotlib.pyplot}
	\item Including the functions \texttt{plot() imshow() subplot() scatter() title() axis() xlabel() tick() figure()} and many more
	\item Two tips tips be aware of:
	\begin{itemize}
		\item In Python plotting a new curve will add it to the figure. In MATLAB it will replace the first curve unless you use a \texttt{hold on} command first
		\item MATLAB does not need a \texttt{show} command, the figure is shown automatically and it does not need to be closed for the remainder of the program to be run.
    \end{itemize}
\end{itemize}
\end{frame}

%==============================================================

\begin{frame}[fragile]
%\frametitle{Plotting}
\begin{lstlisting}[style=CStyle]
import numpy as np                  # Python
import matplotlib.pyplot as plt
v0 = 5
g = 9.81
t = np.linspace(0,1,1001) 
y = v0*t - 0.5*g*t**2
plt.plot(t, y)
plt.xlabel('t (s)')
plt.title('Velocity over time')
plt.show()
\end{lstlisting}
\begin{lstlisting}[style=MStyle]
v0 = 5                            % MATLAB
g = 9.81
t = linspace(0,1,1001)   
y = v0*t - 0.5*g*t.^2
plot(t, y)
xlabel('t (s)')
title('Velocity over time')
\end{lstlisting}
\end{frame}

%==============================================================

\begin{frame}[fragile]
\frametitle{Functions}
\begin{itemize}
\item Functions work exactly the same way as in Python. 
    \begin{itemize}
        \item You declare them by defining the function name, inputs, outputs and operation in the function declaration and then call the function by name as needed.

        \item Functions start with the keyword \texttt{function} not \texttt{def}
        \item The function output is defined at the start of the function not the end
        \item The end of the function is indicated with the keyword \texttt{end}
    \end{itemize}
\begin{lstlisting}[style=CStyle]
def addition(num_1, num_2):               # Python
    total = num_1 + num_2
    return total
\end{lstlisting}
\begin{lstlisting}[style=MStyle]
function [total] = addition(num_1, num_2)  % MATLAB
    total = num_1 + num_2;
end
\end{lstlisting}
\end{itemize}
\end{frame}

%==============================================================

%==============================================================

%==============================================================

%==============================================================

%==============================================================


\end{document}