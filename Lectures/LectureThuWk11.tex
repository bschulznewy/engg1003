\documentclass[english,14pt]{beamer}
\usetheme{EastLansing}
\usecolortheme{spruce}

\usepackage{xcolor}
\usepackage{listings}
\usepackage{courier}
\usepackage{graphicx}
\usepackage{amsmath}
\usepackage{algorithm2e}
\usepackage{multicol}
\usepackage{hyperref}
\usepackage{textcomp}

% http://mirrors.ibiblio.org/CTAN/macros/latex/contrib/datetime2/datetime2.pdf
\usepackage{babel}
\usepackage[useregional]{datetime2}

% https://tex.stackexchange.com/questions/42619/x-mark-to-match-checkmark
\usepackage{pifont}% http://ctan.org/pkg/pifont

%% https://stackoverflow.com/questions/1435837/how-to-remove-footers-of-latex-beamer-templates
%%gets rid of bottom navigation bars
%\setbeamertemplate{footline}[page number]
%
%gets rid of navigation symbols
\setbeamertemplate{navigation symbols}{}


\usefonttheme[onlymath]{serif}

\definecolor{mGreen}{rgb}{0,0.6,0}
\definecolor{mGray}{rgb}{0.5,0.5,0.5}
\definecolor{mPurple}{rgb}{0.8,0,0.82}
\definecolor{backgroundColour}{rgb}{0.95,0.95,0.92}
\definecolor{lightBlue}{rgb}{0.1, 0.1, 0.8}
\definecolor{darkGreen}{rgb}{0, 0.39, 0}

\newcommand\red[1]{{\color{red} #1}}
\newcommand\green[1]{{\color{green} #1}}
\newcommand\blue[1]{{\color{blue} #1}}
\newcommand\darkGreen[1]{{\color{darkGreen} #1}}

\newcommand{\cmark}{\ding{51}}%
\newcommand{\xmark}{\ding{55}}%

\lstdefinestyle{CStyle}{
    backgroundcolor=\color{backgroundColour},   
    commentstyle=\color{mGreen},
    keywordstyle=\color{magenta},
    numberstyle=\tiny\color{mGray},
    stringstyle=\color{mPurple},
    basicstyle=\footnotesize,
    breakatwhitespace=false,         
    breaklines=true,                 
    captionpos=b,                    
    keepspaces=true,                 
    numbers=left,                    
    numbersep=5pt,                  
    showspaces=false,                
    showstringspaces=false,
    showtabs=false,                  
    tabsize=2,
    language=Python
}

\lstdefinestyle{MStyle}{
    backgroundcolor=\color{backgroundColour},   
    commentstyle=\color{mGreen},
    keywordstyle=\color{magenta},
    numberstyle=\tiny\color{mGray},
    stringstyle=\color{mPurple},
    basicstyle=\footnotesize,
    breakatwhitespace=false,         
    breaklines=true,                 
    captionpos=b,                    
    keepspaces=true,                 
    numbers=left,                    
    numbersep=5pt,                  
    showspaces=false,                
    showstringspaces=false,
    showtabs=false,                  
    tabsize=2,
    language=MATLAB
}

\lstdefinestyle{pseudo}{
        basicstyle=\ttfamily\footnotesize,
        keywordstyle=\color{lightBlue},
        morekeywords={BEGIN,END,IF,ELSE,ENDIF,ELSEIF,PRINT,WHILE,RETURN,ENDWHILE,DO,FOR,TO,IN,ENDFOR,BREAK,INPUT,CONDITIONS},
        morecomment=[l]{//},
        commentstyle=\color{mGreen}
}

\lstset{basicstyle=\footnotesize\ttfamily,breaklines=true}
\lstset{framextopmargin=50pt,tabsize=2}

\title{ENGG1003 - Thursday Week 11}
\subtitle{MATLAB vs.~Python}
\author{Sarah Johnson and Steve Weller}
\institute{University of Newcastle}
%\date{\today}
\date{20 May 2021}

% following is a bit of a hack, but forces page numbers (technically: frame numbers) to run 1,2,3,... 
% with titlepage counting as frame 1

\addtocounter{framenumber}{1}
\titlepage

\begin{document}

\begin{flushleft}
{\scriptsize Last compiled:~\DTMnow}
\vspace*{-5mm}
\end{flushleft}
\framebreak

%==============================================================

\begin{frame}[fragile]

\frametitle{Lecture overview}
\begin{enumerate}
	\item Context
	\begin{itemize}
		\item ENGG1003
		\item what is MATLAB?
		\item do we even need MATLAB?
	\end{itemize}

	\item[]
	
	\item MATLAB vs.~Python
	\begin{itemize}
		\item features and philosophy
		\item key language details in MATLAB
	\end{itemize}
	

	\item[]

	\item Next steps
	\begin{itemize}
		\item getting MATLAB, if you need it
		\item Octave: free \& mostly compataible with MATLAB
	\end{itemize}

	\item[]

\end{enumerate}

\end{frame}

%==============================================================

\begin{frame}[fragile]

\frametitle{$1)$ Context}

\begin{itemize}
	\item $\leq$ 2020, ENGG1003 used \red{\emph{MATLAB}} and \red{\emph{C}}
	\begin{itemize}
%		\item \ldots and ENGG1002 used \red{\emph{Visual Basic/VBA/Excel}}
%		\item ENGG1002 now wrapped into larger ENGG1003 class
		\item \textbf{from 2021, ENGG1003 uses Python only} %now all (Engineering) students learn Python in ENGG1003
		\item \ldots yet some students will use MATLAB \&/or C in later courses
	\end{itemize}
	\item[]
	\item today's lecture: overview of MATLAB
	\item[]
	\item Monday week 12: overview of C

\end{itemize}

\end{frame}

%==============================================================

\begin{frame}[fragile]

\frametitle{What is MATLAB?}

\begin{itemize}
	\item MATLAB is a computing environment and programming language for \emph{matrix manipulation}
	\begin{itemize}
		\item a matrix is a 2D array
		\item MATLAB is an abbreviation for ``matrix laboratory''
	\end{itemize}
	\item[]
	\item MATLAB offers many additional ``Toolboxes'':
	\begin{itemize}
		\item control design
		\item image processing
		\item machine learning
		\item digital signal processing
		\item computational fluid dynamics
		\item etc.
	\end{itemize}
\end{itemize}

\end{frame}

%==============================================================

\begin{frame}[fragile]

\frametitle{Do we even need MATLAB?}

\begin{itemize}
	\item \textbf{MATLAB is \emph{not} assessable in ENGG1003}
	
	\item[]
	
	\item BUT\ldots MATLAB is currently used in a number of later courses in Engineering programs
	\item[]
	\item non-exhaustive list:
	\begin{itemize}
		\item[] CHEE4945, CHEE4975,  ENGG2440, AERO3600, MCHA3400, MCHA3500, MCHA4100, ELEC2132, ELEC2430, ELEC3400, ELEC3410, ELEC3540, ELEC4100, many FYPs, \ldots
	\end{itemize}

\end{itemize}
\end{frame}

%==============================================================

\begin{frame}[fragile]

\frametitle{$2)$ MATLAB vs.~Python}

Features and philosophy:
\vspace*{7mm}

\begin{itemize}
	\item \textbf{MATLAB} is proprietary, closed-source software
	\begin{itemize}
		\item developed by MathWorks \href{https://www.mathworks.com/}{https://www.mathworks.com/}
		\item MATLAB license is free for students \ldots
		\item[] \emph{but very expensive otherwise}
	\end{itemize}
	\item[]
	\item \textbf{Python} is free and open-source software
	\begin{itemize}
		\item you can keep programming in Python once you graduate!
	\end{itemize}
\end{itemize}

\end{frame}

%==============================================================

\begin{frame}[fragile]

\frametitle{Advantages of Python over MATLAB}

\begin{itemize}
	\item Python libraries offer similar functionality as MATLAB toolboxes
	\begin{itemize}
		\item Python libraries growing much faster, are free, \& supported by active online community 
	\end{itemize}
	\item[]
	\item Python is a very popular, in-demand language
	\begin{itemize}
		\item many students are finding that employers are requesting Python coding skills for both positions and internships
		\item MATLAB: 4 million users
		\item Python: $>$ 8 million users (2 million new users in 2018)
		\item Python ranked \#1 most popular language in 2020 (IEEE)
	\end{itemize}
	\item[]
	\item {\small \emph{MATLAB vs Python: Why and How to Make the Switch}}\\
\href{https://realpython.com/matlab-vs-python/}{https://realpython.com/matlab-vs-python/}
\end{itemize}

\end{frame}

%==============================================================

\begin{frame}[fragile]

\frametitle{MATLAB: key language details}

\begin{itemize}
	\item syntax
	\item arithmetic and relational operators
	\item flow control
	\item arrays
	\item plotting
	\item functions
\end{itemize}

\begin{center}
\textbf{If you're familiar with Python at level of ENGG1003, estimate transition to \\ MATLAB in 1--2 weeks}
\end{center}

\end{frame}

%==============================================================

\begin{frame}[fragile]

\frametitle{MATLAB syntax}

\begin{itemize}
	\item MATLAB comments start with \%. Python comments start with \#
	\item[]
	\item white-space and indenting are very important in Python. MATLAB does not require the same (but it is highly recommended anyway for readability) 
	\item[]
	\item MATLAB function \texttt{disp( )} replaces \texttt{print( )}
	\item[]
	\item help for a function via \texttt{help fname} rather than \texttt{help(fname)}
\end{itemize}
\end{frame}



%==============================================================

\begin{frame}[fragile]

\frametitle{Arithmetic and relational operators}

\begin{itemize}

\item addition, subtraction, multiplication and division are the same as Python. A difference is that
	\begin{itemize}
		\item MATLAB uses \^ \;\; not ** for exponential
		\item ie: \texttt{x**2} becomes \texttt{x\^\;2}
	\end{itemize}
	\item[]
\item relational operators $==$, $>$, $<$, $>=$, $<=$ are the same, except
	\begin{itemize}
		\item MATLAB uses $\sim =$  instead of $!=$ for 'not equal to'
	\end{itemize}
	\item[]
\item value of a MATLAB variable is \emph{automatically} printed to the command window (console), unless ';' used to suppress
\begin{itemize}
\item ie: \texttt{a = 3} prints value of \texttt{a}, whereas \texttt{a = 3;} does not
	\end{itemize}
\end{itemize}
\end{frame}

%==============================================================

\begin{frame}[fragile]

\frametitle{Flow control}

\begin{itemize}
	\item \texttt{if}-\texttt{else} statements and \texttt{for} / \texttt{while} loops work exactly the same way in both languages, with some small differences in syntax:
	\item[]
	\begin{itemize}
		\item MATLAB does not need the ':' used in Python at the end of the loop definition / if condition
		\item[]
		\item MATLAB designates the end of an If statement or loop by 'end' instead of by indenting
		\item[]
		\item Python shortens \texttt{elseif} to \texttt{elif}. MATLAB does not
	\end{itemize}
\end{itemize}
\end{frame}

%==============================================================

\begin{frame}[fragile]

\frametitle{Flow control}

Nested If-Else

\begin{lstlisting}[style=CStyle]
num = 10                      # Python
if num == 10:
    print("num is equal to 10")
elif num == 20:
    print("num is equal to 20")
else:
    print("num is neither 10 nor 20")
\end{lstlisting}   

\begin{lstlisting}[style=MStyle]
num = 10;                   % MATLAB
if num == 10
    disp("num is equal to 10")
elseif num == 20
    disp("num is equal to 20")
else
    disp("num is neither 10 nor 20")
end
\end{lstlisting}

\end{frame}

%==============================================================

\begin{frame}[fragile]

\frametitle{Flow control}

\textbf{Example:} use a \texttt{for} loop to add integers from 1 to 10

\begin{lstlisting}[style=CStyle]
sum = 0                  # Python
for i in range(1,11):
    sum = sum+i
\end{lstlisting}
\begin{lstlisting}[style=MStyle]
sum = 0;                  % MATLAB
for i = 1:10
    sum = sum+i;
end
\end{lstlisting} 
\end{frame}

%==============================================================

\begin{frame}[fragile]

\frametitle{Flow control}

\textbf{Example:} use a \texttt{while} loop to find the first (smallest) integer which, when squared, is greater than 100

\begin{lstlisting}[style=CStyle]
n = 1                        # Python
n_squared = 1
while n_squared < 100:
    n = n+1
    n_squared = n**2
\end{lstlisting}
\begin{lstlisting}[style=MStyle]
n = 1                       % MATLAB
n_squared = 1
while n_squared < 100
    n = n+1
    n_squared = n^2
end
\end{lstlisting} 
\end{frame}

%==============================================================

\begin{frame}[fragile]

\frametitle{Arrays}

\begin{itemize}
	\item MATLAB arrays work a lot like \texttt{numpy} arrays. If you can work with Numpy arrays you will find MATLAB arrays are easy. Both do vectorisation in the same way
	\item There are a couple of syntax differences to note:
	\begin{itemize}
		\item In MATLAB, when you want to index an array, you use round brackets \texttt{'()'}. Square brackets \texttt{'[]'} are used to create arrays
\begin{lstlisting}[style=CStyle]
arr = np.array([10, 20, 30])   # Python
s = arr[i]
\end{lstlisting}
\begin{lstlisting}[style=MStyle]
arr = [10, 20, 30]             % MATLAB
s = arr(i)
\end{lstlisting}
    \end{itemize}
\end{itemize}

\end{frame}

%==============================================================

\begin{frame}[fragile]

\frametitle{Arrays}

	\begin{itemize}
		\item 2D arrays use ';' to designate the next row
		
		\begin{lstlisting}[style=CStyle]
a = np.array([[2,3],[4,5]])    # Python
s = arr[i,j]
\end{lstlisting}
\begin{lstlisting}[style=MStyle]
a = [2 3; 4 5]              % MATLAB
s = a(i,j)
\end{lstlisting}

		\item All zero / all one matrices are very similar
		
		\begin{lstlisting}[style=CStyle]
m = np.zeros([5,10,3])  # Python 5x10x3 array
\end{lstlisting}
\begin{lstlisting}[style=MStyle]
m = zeros(5,10,3)       % MATLAB 5x10x3 array
\end{lstlisting}

	      \item In MATLAB a new copy of an array is created by default
	      
\begin{lstlisting}[style=CStyle]
b = a.copy()  # Python
\end{lstlisting}
\begin{lstlisting}[style=MStyle]
b = a         % MATLAB
\end{lstlisting}

	\end{itemize}

\end{frame}

%==============================================================

\begin{frame}[fragile]

\frametitle{Arrays}

%\begin{itemize}
%%	\item Many Numpy functions exist in MATLAB including \texttt{linspace() sqrt() log() exp() round() ceil() floor() max() min() mean()} and many more
%	\item A few tips to keep in mind:
	\begin{itemize}
		\item \textbf{in Python, the index of the first element in an array is '0', in MATLAB it is '1'}
		\item in Python, the index of the last element in an array is '-1', in MATLAB it is '\texttt{end}'
		\item length of an array is \texttt{length} rather than \texttt{len}
		\item array slicing works similarly in MATLAB to Python
            \item no need to call \texttt{linspace()} to create an array in MATLAB,  \texttt{start:step:end} will do
\begin{lstlisting}[style=MStyle]
a = 5:1:100              % MATLAB
\end{lstlisting}
	\end{itemize}
%\end{itemize}
\end{frame}

%==============================================================

\begin{frame}[fragile]

\frametitle{Plotting}

\begin{itemize}       
	\item plotting in MATLAB is very similar to \texttt{Matplotlib.pyplot}
	\item including the functions \texttt{plot() imshow() subplot() scatter() title() axis() xlabel() tick() figure()} and many more
	\item two tips:
	\begin{itemize}
		\item in Python plotting a new curve will add it to the figure. In MATLAB it will replace the first curve unless you use a \texttt{hold on} command first
		\item MATLAB does not need a \texttt{show()} command, the figure is shown automatically and does not need to be closed for the remainder of the program to be run
    \end{itemize}
\end{itemize}
\end{frame}

%==============================================================

\begin{frame}[fragile]

\frametitle{Plotting}

\begin{lstlisting}[style=CStyle]
import numpy as np                  # Python
import matplotlib.pyplot as plt
v0 = 5
g = 9.81
t = np.linspace(0,1,1001) 
y = v0*t - 0.5*g*t**2
plt.plot(t, y)
plt.xlabel('t (s)')
plt.title('Velocity over time')
plt.show()
\end{lstlisting}
\begin{lstlisting}[style=MStyle]
v0 = 5;                            % MATLAB
g = 9.81;
t = 0:.001:1;   
y = v0*t - 0.5*g*t.^2;
plot(t, y)
xlabel('t (s)')
title('Velocity over time')
\end{lstlisting}
\end{frame}

%==============================================================

\begin{frame}[fragile]

\frametitle{Functions}

\begin{itemize}
	\item MATLAB functions work similarly to Python
	\begin{itemize}
		\item declared by defining the function name, inputs, outputs and operation in the function declaration, then call the function by name as needed

	\item functions start with the keyword \texttt{function} not \texttt{def}
	\item function output is defined at the \emph{start} of the function, not the end
	\item end of function is indicated with the keyword \texttt{end}
	\end{itemize}
\begin{lstlisting}[style=CStyle]
def addition(num_1, num_2):               # Python
    total = num_1 + num_2
    return total
\end{lstlisting}
\begin{lstlisting}[style=MStyle]
function [total] = addition(num_1, num_2)  % MATLAB
    total = num_1 + num_2;
end
\end{lstlisting}
\end{itemize}
\end{frame}

%==============================================================

\begin{frame}[fragile]

\frametitle{Next steps}
\begin{itemize}
	\item getting MATLAB, if you need it \emph{for later courses}
	\item[]
	\begin{itemize}
		\item \textbf{MATLAB is \emph{not} assessable in ENGG1003}
		\item \href{https://www.newcastle.edu.au/current-students/support/it/software-and-tools}{https://www.newcastle.edu.au/current-students/support/it/software-and-tools}
	\end{itemize}
	\item[]
	\item Octave: free \& mostly compatible with MATLAB
	\begin{itemize}
		\item \href{https://www.gnu.org/software/octave/index}{https://www.gnu.org/software/octave/index}
	\end{itemize}
\end{itemize}
	
\end{frame}

\end{document}