\documentclass[english,14pt]{beamer}
\usetheme{EastLansing}
\usecolortheme{spruce}

\usepackage{xcolor}
\usepackage{listings}
\usepackage{courier}
\usepackage{graphicx}
\usepackage{amsmath}
\usepackage{algorithm2e}
\usepackage{multicol}
\usepackage{hyperref}
\usepackage{textcomp}

% http://mirrors.ibiblio.org/CTAN/macros/latex/contrib/datetime2/datetime2.pdf
\usepackage{babel}
\usepackage[useregional]{datetime2}

% https://tex.stackexchange.com/questions/42619/x-mark-to-match-checkmark
\usepackage{pifont}% http://ctan.org/pkg/pifont

%% https://stackoverflow.com/questions/1435837/how-to-remove-footers-of-latex-beamer-templates
%%gets rid of bottom navigation bars
%\setbeamertemplate{footline}[page number]
%
%gets rid of navigation symbols
\setbeamertemplate{navigation symbols}{}


\usefonttheme[onlymath]{serif}

\definecolor{mGreen}{rgb}{0,0.6,0}
\definecolor{mGray}{rgb}{0.5,0.5,0.5}
\definecolor{mPurple}{rgb}{0.8,0,0.82}
\definecolor{backgroundColour}{rgb}{0.95,0.95,0.92}
\definecolor{lightBlue}{rgb}{0.1, 0.1, 0.8}
\definecolor{darkGreen}{rgb}{0, 0.39, 0}

\newcommand\red[1]{{\color{red} #1}}
\newcommand\green[1]{{\color{green} #1}}
\newcommand\blue[1]{{\color{blue} #1}}
\newcommand\darkGreen[1]{{\color{darkGreen} #1}}

\newcommand{\cmark}{\ding{51}}%
\newcommand{\xmark}{\ding{55}}%

\lstdefinestyle{CStyle}{
    backgroundcolor=\color{backgroundColour},   
    commentstyle=\color{mGreen},
    keywordstyle=\color{magenta},
    numberstyle=\tiny\color{mGray},
    stringstyle=\color{mPurple},
    basicstyle=\footnotesize,
    breakatwhitespace=false,         
    breaklines=true,                 
    captionpos=b,                    
    keepspaces=true,                 
    numbers=left,                    
    numbersep=5pt,                  
    showspaces=false,                
    showstringspaces=false,
    showtabs=false,                  
    tabsize=2,
    language=Python
}

\lstdefinestyle{pseudo}{
        basicstyle=\ttfamily\footnotesize,
        keywordstyle=\color{lightBlue},
        morekeywords={BEGIN,END,IF,ELSE,ENDIF,ELSEIF,PRINT,WHILE,RETURN,ENDWHILE,DO,FOR,TO,IN,ENDFOR,BREAK,INPUT,CONDITIONS},
        morecomment=[l]{//},
        commentstyle=\color{mGreen}
}

\lstset{basicstyle=\footnotesize\ttfamily,breaklines=true}
\lstset{framextopmargin=50pt,tabsize=2}

\title{ENGG1003 - Thursday Week 8}
\subtitle{Numerical integration }%\\ \& computing integrals}
\author{Steve Weller}
\institute{University of Newcastle}
%\date{\today}
\date{29 April 2021}

% following is a bit of a hack, but forces page numbers (technically: frame numbers) to run 1,2,3,... 
% with titlepage counting as frame 1

\addtocounter{framenumber}{1}
\titlepage

\begin{document}

\begin{flushleft}
{\scriptsize Last compiled:~\DTMnow}
\vspace*{-5mm}
\end{flushleft}
\framebreak

%==============================================================

\begin{frame}[fragile]

\frametitle{Lecture overview}
\begin{enumerate}
	\item Basic ideas of numerical integration \red{\S6.1}
	\begin{itemize}
		\item engineering applications
		\item terminology \& notation
		\item additivity
	\end{itemize}
	
	\item[]
	
	\item Trapezoidal method \red{\S6.2}
	
	\item[]
	
	\item Midpoint method, upper/lower and left/right Riemann sums \red{\S6.3}
	
	\item[]
	
	\item Simpson's rule
	
\end{enumerate}

\end{frame}

%==============================================================

\begin{frame}[fragile]

\frametitle{$1)$ Basic ideas of integration}

% https://en.wikipedia.org/wiki/File:Integral_as_region_under_curve.svg
\vspace*{-5mm}
\begin{figure}[ht]
	\centering
	\includegraphics[width=0.35\textwidth]{figures/integralArea}
\end{figure}
\vspace*{-5mm}
\[
S = \int_a^b f(x) dx
\]
\vspace*{-5mm}
\begin{itemize}
	\item area $S$ is area under function $f(x)$ between lower limit $a$ and upper limit $b$
	\item assume $f(x) \geq 0$
	\item calculus, eg: MATH1002, MATH1110
\end{itemize}

\end{frame}

%==============================================================

\begin{frame}[fragile]

\frametitle{Engineering applications of integration}

% https://www.whitman.edu/mathematics/calculus_online/chapter09.html

\begin{itemize}
	\item 1. Area between curves
2. Distance, Velocity, Acceleration
3. Volume
4. Average value of a function
5. Work
6. Center of Mass
7. Kinetic energy; improper integrals
8. Probability
9. Arc Length
10. Surface Area
\end{itemize}

\end{frame}

%==============================================================

\begin{frame}[fragile]

\frametitle{Distance:~area under speed vs.~time function}

Our specific integral is taken from basic physics. Assume that you speed up your car from rest, on a straight road, and wonder how far you go in T seconds. The displacement is given by the integral
\[
\int_0^T v(t)dt
\]
where $v(t)$ is the velocity (speed) as a function of time

Example:
\[
v(t) = 3t^2e^{t^3}
\]

\end{frame}

%==============================================================

\begin{frame}[fragile]

\frametitle{}

\begin{figure}[ht]
	\centering
	\includegraphics[width=0.5\textwidth]{figures/LLp134}
\end{figure}

distance traveled in first second is cross-hatched area:
\[
\int_0^1 v(t)dt
\]

Start at time $0$, end at time $1$ (these are the lower and upper limits)
\end{frame}

%==============================================================

\begin{frame}[fragile]

\frametitle{$2)$ Trapezoidal method}

Example:
\vspace*{-5mm}
\begin{figure}[ht]
	\centering
	\includegraphics[width=0.5\textwidth]{figures/fourPanel}
\end{figure}
\vspace*{-3mm}
\begin{itemize}
	\item approximate area under curve by total area of four trapezoids
	\begin{itemize}
		\item black + \darkGreen{green} + \red{red} + \blue{blue}
	\end{itemize}
	\item area of trapezoids is easy %: $\frac{a+b}{2} h$
\end{itemize}



\end{frame}

%==============================================================

\begin{frame}[fragile]

\frametitle{Numerical integration}

\begin{itemize}
	\item \red{\emph{trapezoid}} is a ``convex quadrilateral with at least one pair of parallel sides''
	% https://en.wikipedia.org/wiki/Trapezoid
	\item area of trapezoid $\frac{a+b}{2} h$
	\item want ``vertical'' version -- create image in PPT
	
%	\item use \texttt{fill\_between} ??
% https://www.kite.com/python/docs/matplotlib.pyplot.fill_between
\end{itemize}

\begin{figure}[ht]
	\centering
	\includegraphics[width=0.5\textwidth]{figures/Trapezoid}
\end{figure}

%\begin{figure}[ht]
%	\centering
%	\includegraphics[width=0.5\textwidth]{figures/LLp134}
%\end{figure}

%\begin{figure}[ht]
%	\centering
%	\includegraphics[width=0.5\textwidth]{figures/fourPanel}
%\end{figure}
%
%\begin{itemize}
%	\item approximate area under curve by total area of four trapezoids
%	\begin{itemize}
%		\item black + \darkGreen{green} + \red{red} + \blue{blue}
%	\end{itemize}
%	\item area of trapezoids is easy: $\frac{a+b}{2} h$
%\end{itemize}

\end{frame}

%==============================================================

\begin{frame}[fragile]

\frametitle{}

\begin{figure}[ht]
	\centering
	\includegraphics[width=0.5\textwidth]{figures/fourPanel}
\end{figure}
\vspace*{-5mm}
{\small
\begin{eqnarray*}
\int_0^1 v(t)dt & = & \int_0^{0.2} v(t)dt + \int_{0.2}^{0.6} v(t)dt + \int_{0.6}^{0.8} v(t)dt + \int_{0.8}^1 v(t)dt \\
\pause
&\approx & h_1 \frac{v(0) + v(0.2)}{2} + \darkGreen{h_2 \frac{v(0.2) + v(0.6)}{2}} + \\
& & +h_3 \red{\frac{v(0.6) + v(0.8)}{2}} + \blue{h_4 \frac{v(0.8) + v(1)}{2}} 
\end{eqnarray*}
}
\pause
\vspace*{-5mm}
\[
h_1 = 0.2, \quad h_2 = 0.4, \quad h_3 = 0.2, \quad h_4 = 0.2
\]

\end{frame}

%==============================================================

\begin{frame}[fragile]

\frametitle{Python code for trapezoidal method}

\begin{itemize}
	\item write as a function
	\item live demo, experiment with number of panels
\end{itemize}

\end{frame}

%==============================================================

\begin{frame}[fragile]

\frametitle{$3)$ Midpoint method}

\begin{itemize}
	\item skip details, all give quite similar results to trapezoidal method, esp for narrow width panels, some details in \red{\S6.3}
	\item[]
	\item mention/visualise different methods:
	\begin{itemize}
		\item midpoint method
		\item upper/lower Riemann sums 
		\item left/right Riemann sums
	\end{itemize}
\end{itemize}

\end{frame}

%==============================================================

\begin{frame}[fragile]

\frametitle{$4)$ Simpson's rule}

% https://en.wikipedia.org/wiki/Simpson%27s_rule
\begin{figure}[ht]
	\centering
	\includegraphics[width=0.4\textwidth]{figures/SimpsonsRule}
\end{figure}

\begin{itemize}
	\item approximate \blue{$f(x)$} with parabola \red{$P(x)$}
	\item parabola $P(x)$ takes same values as $f(x)$ at end-points $a$ and $b$, and midpoint $m = (a+b)/2$
\end{itemize}

\end{frame}

%==============================================================

\begin{frame}[fragile]

\frametitle{Simpson's rule}

\begin{itemize}
	\item area under parabola $P(x)$ between $a$ and $b$ is:
\[
\int_a^b P(x) dx
\]
\item[] \ldots which can be calculated \emph{exactly} (proof omitted):

\[
\int_a^b f(x)dx \approx \frac{b-a}{6} \left[ f(a) + 4f\left(\frac{a+b}{2}\right) + f(b)\right]
\]
\item as for trapezoidal rule, apply Simpson's rule on each ``panel'' of width $h$, composite method
\end{itemize}

\end{frame}

%==============================================================

\begin{frame}[fragile]

\frametitle{Python code for Simpson's rule}

\begin{itemize}
	\item write as a function
	\item live demo, experiment with number of panels
\end{itemize}

\end{frame}

%==============================================================

\begin{frame}[fragile]

\frametitle{Lecture summary}

\begin{enumerate}
	\item Basic ideas of integration
	
	\item[]
	
	\item Trapezoidal method \red{\S6.2}
	
	\item[]
	
	\item Midpoint method \red{\S6.3}
	
	\item[]
	
	\item Simpson's rule
	
\end{enumerate}

\end{frame}

%%==============================================================
%
%\begin{frame}[fragile]
%
%\frametitle{More information}
%\begin{itemize}
%	\item Newton's method in textbook \red{\S7.2}
%	\begin{itemize}
%		\item needs \emph{differentiation} from calculus (MATH1110)
%		\item in particular: need expression for \emph{tangent lines} to function $f(x)$, written as $f'(x)$
%	\end{itemize}
%
%	\item[]
%	
%	\item ``optimised'' versions of bisection and secant methods in textbook \red{\S7.3} and \red{\S7.4}
%	\begin{itemize}
%		\item maximise speed of computation by minimising number of function evaluations $f(x)$
%	\end{itemize}
%	
%	\item[]
%	
%	\item volume of truncated cone based on volumes of solids of revolution (needs calculus, MATH1110) \href{https://bit.ly/3sOsaj4}{https://bit.ly/3sOsaj4}
%\end{itemize}
%
%\end{frame}

\end{document}