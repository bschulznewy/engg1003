\documentclass[english,14pt]{beamer}
\usetheme{EastLansing}
\usecolortheme{spruce}

\usepackage{xcolor}
\usepackage{listings}
\usepackage{courier}
\usepackage{graphicx}
\usepackage{amsmath}
\usepackage{algorithm2e}
\usepackage{multicol}
\usepackage{hyperref}

% http://mirrors.ibiblio.org/CTAN/macros/latex/contrib/datetime2/datetime2.pdf
\usepackage{babel}
\usepackage[useregional]{datetime2}

% https://tex.stackexchange.com/questions/42619/x-mark-to-match-checkmark
\usepackage{pifont}% http://ctan.org/pkg/pifont

%% https://stackoverflow.com/questions/1435837/how-to-remove-footers-of-latex-beamer-templates
%%gets rid of bottom navigation bars
%\setbeamertemplate{footline}[page number]
%
%gets rid of navigation symbols
\setbeamertemplate{navigation symbols}{}


\usefonttheme[onlymath]{serif}

\definecolor{mGreen}{rgb}{0,0.6,0}
\definecolor{mGray}{rgb}{0.5,0.5,0.5}
\definecolor{mPurple}{rgb}{0.8,0,0.82}
\definecolor{backgroundColour}{rgb}{0.95,0.95,0.92}
\definecolor{lightBlue}{rgb}{0.1, 0.1, 0.8}

\newcommand\red[1]{{\color{red} #1}}
\newcommand\green[1]{{\color{green} #1}}
\newcommand\blue[1]{{\color{blue} #1}}

\newcommand{\cmark}{\ding{51}}%
\newcommand{\xmark}{\ding{55}}%

\lstdefinestyle{CStyle}{
    backgroundcolor=\color{backgroundColour},   
    commentstyle=\color{mGreen},
    keywordstyle=\color{magenta},
    numberstyle=\tiny\color{mGray},
    stringstyle=\color{mPurple},
    basicstyle=\footnotesize,
    breakatwhitespace=false,         
    breaklines=true,                 
    captionpos=b,                    
    keepspaces=true,                 
    numbers=left,                    
    numbersep=5pt,                  
    showspaces=false,                
    showstringspaces=false,
    showtabs=false,                  
    tabsize=2,
    language=Python
}

\lstdefinestyle{pseudo}{
        basicstyle=\ttfamily\footnotesize,
        keywordstyle=\color{lightBlue},
        morekeywords={BEGIN,END,IF,ELSE,ENDIF,ELSEIF,PRINT,WHILE,RETURN,ENDWHILE,DO,FOR,TO,IN,ENDFOR,BREAK,INPUT},
        morecomment=[l]{//},
        commentstyle=\color{mGreen}
}

\lstset{basicstyle=\footnotesize\ttfamily,breaklines=true}
\lstset{framextopmargin=50pt,tabsize=2}

\title{ENGG1003 - Monday Week 5}
\subtitle{Functions}
\author{Sarah Johnson}
\institute{University of Newcastle}
%\date{\today}
\date{22 March, 2021}

% following is a bit of a hack, but forces page numbers (technically: frame numbers) to run 1,2,3,... 
% with titlepage counting as frame 1

\addtocounter{framenumber}{1}
\titlepage

\begin{document}

\begin{flushleft}
{\scriptsize Last compiled:~\DTMnow}
\vspace*{-5mm}
\end{flushleft}
\framebreak



\begin{frame}
\frametitle{Functions}
\begin{itemize}
\item A \textit{function} is a block of code which can be \textit{called} multiple times, from multiple places
\item They are used when you want the same block of code to execute in many places throughout your code
\item A function requires:
	\begin{itemize}
		\item A name
		\item (optional) A \textit{return value}
		\item (optional) One or more \textit{arguments}
	\end{itemize}
\end{itemize}
\end{frame}

\begin{frame}
\frametitle{Functions in Mathematics}
\begin{itemize}
\item In mathematics you saw functions written as:
\begin{equation*}
y = f(x)
\end{equation*}
\item Here, the function is called $f$, takes an argument of $x$ and returns a value which is assigned to $y$
\item Python and pure mathematics have these general ideas in common
\pause
\item The similarities stop there
\end{itemize}
\end{frame}

\begin{frame}[fragile]
\frametitle{Functions in Programming}
\begin{itemize}
\item When a function is called:
	\begin{enumerate}
		\item Program execution jumps to the function
		\item The function's code is executed
		\item Program execution jumps back to where it left off
		\item The function often 'returns' a value required by the program
		(functions can also return many or no values but we will assume one return value for now)
	\end{enumerate}
\item The code inside the function can be any valid Python code, not just mathematics
\end{itemize}
\end{frame}



\begin{frame}
\frametitle{Function Examples}
\begin{itemize}
\item So far, we have already used several functions
\item These are functions are either built in python functions or imported from the libraries we have been using
\item Function call syntax is:\\
{\small \texttt{name(argument1,argument2, ...)} } \\
 though not all functions take arguments 
\item Some examples:
	\begin{itemize}
		\item \texttt{sqrt()  \;\;\;\; -- from numpy library}
		\item \texttt{print() \;\; -- built in function}
		\item \texttt{random()  \;\; -- from random library}
	\end{itemize}
\end{itemize}
\end{frame}



\begin{frame}[fragile]
\frametitle{Function Examples}
\begin{itemize}
\item Example 1:
\begin{lstlisting}[style=CStyle]
x = random.random()
\end{lstlisting}
	\begin{itemize}
		\item \texttt{random} is the function name (from the library \texttt{random})
		\item It returns a random number between 0.0 and 1.0
		\item The return value is assigned to \texttt{x}
		\item It doesn't take an argument 	
	\end{itemize}
\pause
\item Example 2:
\begin{lstlisting}[style=CStyle]
y = numpy.sqrt(x)
\end{lstlisting}
	\begin{itemize}
		\item \texttt{sqrt} is the function name (from the library \texttt{numpy})
		\item \texttt{x} is the argument
		\item It returns the square root of \texttt{x}
		\item The return value is assigned to \texttt{y}
	\end{itemize}
\end{itemize}
\end{frame}

\begin{frame}[fragile]
\frametitle{Function Examples}
\begin{itemize}
\item Example 3:
\begin{lstlisting}[style=CStyle]
print('Hello World')
\end{lstlisting}
	\begin{itemize}
		\item \texttt{print} is the function name (it's a built-in function)
		\item The string \texttt{'Hello World'} is the argument
		\item There is no return value (technically it returns 'None')
		\item Even though it does not return a value it still does something
		\item Another example would be the \texttt{seed()} function
	\end{itemize}
\end{itemize}
\end{frame}


%\begin{frame}
%\frametitle{Functions}
%\begin{itemize}
%\item Function arguments and return values have pre-defined %data types
%\pause
%\item Example from documentation
%\begin{itemize}
%\item \texttt{int~rand(void);}%
%	\begin{itemize}
%		\item The return value is an \texttt{int}
%		\item The argument is type \texttt{void}
%		\item This just means ``there are no arguments''
%	\end{itemize}
%\pause
%\item \texttt{float sqrtf(float x);}
%	\begin{itemize}
%		\item The return value is a \texttt{float}
%		\item The argument is a \texttt{float}
%		\item Argument is called \texttt{x} in documentation %but you can pass it any \texttt{float} variable or literal
%	\end{itemize}
%\end{itemize}
%\end{itemize}
%\end{frame}

\begin{frame}
\frametitle{Return Values (an Engineer's View)}
\begin{itemize}
\item The function's \textit{return value} is the number a function gets ``replaced with'' in a line of code
\pause
\item Function return values, variables, and literals can all be used in the same places:
	\begin{itemize}
		\item In arithmetic
		\item In conditions
		\item As arguments to other functions
	\end{itemize}	 
\pause
%\item The C standard is \textit{very} specific about what return values are but I will be informal for now
%	\begin{itemize}
%		\item Technically, for example, an expression like %\texttt{x=y+5.0;} also has a ``return value`` equal to the %value assigned to \texttt{x}
%	\end{itemize}
\item The following are all valid:
	\begin{itemize}
		\item \texttt{x = rand()}
		\item \texttt{print("Sine \{\} is \{\}".value(x,sin(x)))}
		\item \texttt{if( (rand()\%6) < 2)}
		\item \texttt{while( cos(x) < 0 )}
		%\item This next one is complicated...
		%\pause
		%\item \texttt{x = sin((double)rand());}
		%	\begin{itemize}
		%	\pause
		%		\item Generates a random integer, casts to %\texttt{double}, uses that number as an argument to %the \texttt{sin()} function
		%	\end{itemize}
	\end{itemize}
\end{itemize}
\end{frame}


\begin{frame}
\frametitle{Writing Functions}
Writing your own functions:
	\begin{enumerate}
		\item Decide (for yourself) what the function needs to do
		\pause
		\item Choose a name
		\pause
		\item Decide on the function parameters
		\pause
		\item Decide on the return value(s)
		\pause
		\item Define the function (write the function code)
		\item Call the function in your code where you need to use it	
	\end{enumerate}
\end{frame}

\begin{frame}[fragile]
\frametitle{Writing Functions - Example}
\begin{itemize}
\item Lets see our code to calculate the height of a ball (from week 1) as a function
\begin{lstlisting}[style=CStyle]
# Function Definition
def ball_height(v0, t):           # Function header
    g = 9.81                     # Function body
    y = v0*t - 0.5*g*t**2
    return y                     # Return statement
\end{lstlisting}
\pause
\item Here is how we could use this function in our main program
\begin{lstlisting}[style=CStyle]
v0 = 5
time1 = 0.6
height1 = ball_height(v0,time1)
time2 = 0.9
height2 = ball_height(v0,time2)
\end{lstlisting}
\end{itemize}
\end{frame}


\begin{frame}[fragile]
\frametitle{Function Definition}
 The function definition must be \textit{before} the function's first use
 \begin{itemize}
        \item This could be in a library which must be \textit{included} before the function is called
		\item Or in the same .py file placed before (above) the function call 
	\end{itemize}
\end{frame}

\begin{frame}[fragile]
\frametitle{Function Definition}
 \begin{itemize}
		\item The first line in the function definition is the function \textit{header}
			\begin{lstlisting}[style=CStyle]
			      def ball_height(v0, t): 
			\end{lstlisting}
		\item The header starts with the reserved word \texttt{def} 
		\item Followed by the function name (here \texttt{ball\_height}) and 
		\item Followed by the function \textit{parameters} in brackets (here the initial velocity and the time point to calculate the height) 
		\item Ending with a colon
	\end{itemize}
\end{frame}

\begin{frame}[fragile]
\frametitle{Function Definition}
\begin{itemize}
    \item After the function header is the function body which is all the lines of code inside the function
    \item The block of statements inside the function body must be indented  
    \item An optional docstring describing the function can be added as the first line in the function body
\end{itemize}
\begin{lstlisting}[style=CStyle]
       def ball_height(v0, t):       
        """ Calculates the height of a ball at time t
             given an initial velocity of v0 """
        g = 9.81                    
        y = v0*t - 0.5*g*t**2
        return y                    
\end{lstlisting}
\end{frame}

\begin{frame}[fragile]
\frametitle{\texttt{return} Statements}
\begin{itemize}
    \item The function will jump back when it hits a \texttt{return} statement or the end of the function's code
	\item Functions which return a value \textit{must} have a \texttt{return} statement
	\item Functions which return nothing don't need one
	\item Omitting it will cause the function to return \texttt{None}
    \item Examples:
    \begin{lstlisting}[style=CStyle] 
        return  
        return x          
        return v0*t - 0.5*g*t**2 
        return 1          
    \end{lstlisting} %# Nothing ('None') is returned
\end{itemize}
\end{frame}


\begin{frame}
\frametitle{Function Example}
\begin{center}
... Do it live!
\end{center}
\end{frame}

\begin{frame}[fragile]
\frametitle{Writing Functions}
\begin{itemize}
    \item Let's view a common error
    \begin{lstlisting}[style=pseudo]
    Traceback (most recent call last):
    File "Path/main.py", line 14, in <module>
    y = array_sum(array_x)
    NameError: name 'array_sum' is not defined
    \end{lstlisting}
    \item The function definition is missing. Some possible causes:
    \begin{itemize}
        \item The function definition may be below the first call on line 14
        \item You may have forgotten to include the library which defines it
        \item The function exits but you have a typo in the name 
    \end{itemize}
\end{itemize}
\end{frame}


\begin{frame}[fragile]
\frametitle{Writing Functions - Terminology Reminder}
\begin{lstlisting}[style=CStyle]
# Function Definition 
# Define by function_name(function parameters)  
def ball_height(v0, t):          # Function header
   """ Details .... """          # Docustring
    g = 9.81                     # Function body
    y = v0*t - 0.5*g*t**2        # Function body
    return y                     # Return statement
    
# Main Program
# Function call via function_name(function arguments)
v0 = 5                        
time1 = 0.6                       
height1 = ball_height(v0,time1)  # Function call  
time2 = 0.9
height2 = ball_height(v0,time2)  # Function call  
\end{lstlisting}
\end{frame}

\begin{frame}[fragile]
\frametitle{Function Variables}
\begin{itemize}
    \item The input variables specified in the function definition are called the \textit{parameters} of the function. The function the values provided in the call are called \textit{arguments}
    \item Variable names of the arguments in function calls do not have to have the same name as in the parameters in the function definition 
\end{itemize}
\end{frame}


\begin{frame}
\frametitle{Function Variables}
\begin{itemize}
    \item By default, function arguments are ``passed by value''
    \item The function gets \textit{copies} of the variable's value
    \item Modifying them in a function doesn't change the original variable
	\begin{itemize}
		\item No, not even if they have the same name
		\item The function's copy occupies a different memory address
	\end{itemize}
    \end{itemize}
\end{frame}

\begin{frame}
\frametitle{Function Variables}
\begin{itemize}
    \item Any \textit{local variables}, those defined inside the function, are only known inside the function (e.g. \texttt{g} in the function above)
    \item The argument variables and local variables are discarded when the function finishes (returns)
    \item The return value is the \textit{only thing} that goes back
    \item An exception is to define a variable as \textit{global} inside the function but it's generally not recommended
\end{itemize}
\end{frame}


\begin{frame}[fragile]
\frametitle{Alternative Function Definitions}
\begin{itemize}
\item So far we have seen functions with positional parameters
\item \textit{Keyword parameters} can be used to specify default values
\item If both positional and keyword parameters are specified positional parameters must come first
\begin{lstlisting}[style=CStyle]
def ball_height(v0, t, g=9.81):    
    return v0*t - 0.5*g*t**2      

height = ball_height(5, 0.6)   
more_precise_height = ball_height(5, 0.6, 9.80665)  
\end{lstlisting}
\end{itemize}
\end{frame}

\begin{frame}[fragile]
\frametitle{Alternative Function Calls}
\begin{itemize}
\item It is possible to use function input parameter names as \textit{keyword arguments} in the function call
\item The order of the arguments can be switched and the code can be more readable
\item Keyword arguments can be used even if the function is defined with positional parameters
\end{itemize}
\begin{lstlisting}[style=CStyle]
def ball_height(v0, t, g=9.81):    
    return v0*t - 0.5*g*t**2      

# all of these function calls are correct
height = ball_height(v0=5, t=0.6)   
height = ball_height(v0=5, t=0.6, g=9.80665)  
height = ball_height(t=0.6, g=9.80665, v0=5) 
\end{lstlisting}
\end{frame}


\begin{frame}[fragile]
\frametitle{Alternative Function Calls}
\begin{itemize}
\item A mix or arguments can be used but all positional arguments must precede keyword arguments
\item The positional arguments must be in the correct order
\end{itemize}
\begin{lstlisting}[style=CStyle]
def ball_height(v0, t, g=9.81):    
    return v0*t - 0.5*g*t**2     
    
# these function calls are correct
height = ball_height(5, 0.6, g=9.80665 )   
height = ball_height(5, g=9.80665, t=0.6)  
\end{lstlisting}
\end{frame}



% try and call g outside the code
% check if changing the value of a function variable changes the variable outside
% pass g into the function instead
% use g as a keyword variable
% use a global variable --- 

\begin{frame}[fragile]
\frametitle{Functions With Multiple Return Values}
\begin{itemize}
\item In the function definition return values are separated by commas 
\item In the function call return values are also separated by commas
\item The order of the results returned from the function must be the same
\end{itemize}
\begin{lstlisting}[style=CStyle]
import numpy as np

def circle(r):
    return 2*np.pi*r, np.pi*r**2

circ, area = circle(5)
print('Circumference: {} Area: {}'.format(circ, area)) 
\end{lstlisting}
\end{frame}

\begin{frame}[fragile]
\frametitle{Writing Functions - Example 2}
\begin{itemize}
\item Lets implement the \texttt{sqrt} algorithm from the week 3 lab as a function
\item ...Then compare with \texttt{the numpy sqrt()}
\item Keep it simple: fixed iteration count \texttt{n=10}
\end{itemize}
\end{frame}
\begin{frame}[fragile]
\frametitle{Writing Functions - Example 2}
\begin{itemize}
\item In mathematics, calculate $\sqrt{k}$ by iterating:
\begin{align*}
x_n &= \frac{1}{2}\left(x_{n-1} + \frac{k}{x_{n-1}}\right)\\
x_0 &\neq 0
\end{align*}
\item In python we can encode this algorithm as:  
\begin{lstlisting}[style=CStyle]
# Calculate sqrt(k)
k = 26.0           # Test value
xn = x/2.0         # Start value x0 = x/2
for i in range(0, 10, 1):
	xn = 0.5*(xn + k/xn)
\end{lstlisting}
\end{itemize}
\end{frame}
\begin{frame}[fragile]
\frametitle{Writing Functions - Example 2}
\begin{itemize}
\item Lets make some design decisions:
	\begin{itemize}
		\item Name: \texttt{mySqrt()}
		\item Argument: \texttt{k}
		\item Return Value: \texttt{the square root of k}
	\end{itemize}
\item The function definition is therefore:
\begin{lstlisting}[style=CStyle]
def mySqrt(k):
	xn = k/2.0      # Start value x0 = x/2
    for i in range(0, 10, 1):
	   xn = 0.5*(xn + k/xn)
	return xn
\end{lstlisting}
\end{itemize}
\end{frame}

\begin{frame}
\frametitle{Function Example}
\begin{center}
... Do it live!
\end{center}
\end{frame}

\begin{frame}
\frametitle{More Information}
\begin{center}
	\begin{itemize}
		\item Further Reading: Section 4.1 of the course textbook
        \item More Practice: All the exercises in section 4.3 of the course textbook  
    \end{itemize}
\end{center}
\end{frame}
\end{document}



