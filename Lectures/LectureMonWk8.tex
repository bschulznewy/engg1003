\documentclass[english,14pt]{beamer}
\usetheme{EastLansing}
\usecolortheme{spruce}

\usepackage{xcolor}
\usepackage{listings}
\usepackage{courier}
\usepackage{graphicx}
\usepackage{amsmath}
\usepackage{algorithm2e}
\usepackage{multicol}
\usepackage{hyperref}
\usepackage{textcomp}

% http://mirrors.ibiblio.org/CTAN/macros/latex/contrib/datetime2/datetime2.pdf
\usepackage{babel}
\usepackage[useregional]{datetime2}

% https://tex.stackexchange.com/questions/42619/x-mark-to-match-checkmark
\usepackage{pifont}% http://ctan.org/pkg/pifont

%% https://stackoverflow.com/questions/1435837/how-to-remove-footers-of-latex-beamer-templates
%%gets rid of bottom navigation bars
%\setbeamertemplate{footline}[page number]
%
%gets rid of navigation symbols
\setbeamertemplate{navigation symbols}{}


\usefonttheme[onlymath]{serif}

\definecolor{mGreen}{rgb}{0,0.6,0}
\definecolor{mGray}{rgb}{0.5,0.5,0.5}
\definecolor{mPurple}{rgb}{0.8,0,0.82}
\definecolor{backgroundColour}{rgb}{0.95,0.95,0.92}
\definecolor{lightBlue}{rgb}{0.1, 0.1, 0.8}

\newcommand\red[1]{{\color{red} #1}}
\newcommand\green[1]{{\color{green} #1}}
\newcommand\blue[1]{{\color{blue} #1}}

\newcommand{\cmark}{\ding{51}}%
\newcommand{\xmark}{\ding{55}}%

\lstdefinestyle{CStyle}{
    backgroundcolor=\color{backgroundColour},   
    commentstyle=\color{mGreen},
    keywordstyle=\color{magenta},
    numberstyle=\tiny\color{mGray},
    stringstyle=\color{mPurple},
    basicstyle=\footnotesize,
    breakatwhitespace=false,         
    breaklines=true,                 
    captionpos=b,                    
    keepspaces=true,                 
    numbers=left,                    
    numbersep=5pt,                  
    showspaces=false,                
    showstringspaces=false,
    showtabs=false,                  
    tabsize=2,
    language=Python
}

\lstdefinestyle{pseudo}{
        basicstyle=\ttfamily\footnotesize,
        keywordstyle=\color{lightBlue},
        morekeywords={BEGIN,END,IF,ELSE,ENDIF,ELSEIF,PRINT,WHILE,RETURN,ENDWHILE,DO,FOR,TO,IN,ENDFOR,BREAK,INPUT,CONDITIONS},
        morecomment=[l]{//},
        commentstyle=\color{mGreen}
}

\lstset{basicstyle=\footnotesize\ttfamily,breaklines=true}
\lstset{framextopmargin=50pt,tabsize=2}

\title{ENGG1003 - Monday Week 8}
\subtitle{Solving nonlinear algebraic equations }%\\ \& computing integrals}
\author{Steve Weller}
\institute{University of Newcastle}
%\date{\today}
\date{26 April 2021}

% following is a bit of a hack, but forces page numbers (technically: frame numbers) to run 1,2,3,... 
% with titlepage counting as frame 1

\addtocounter{framenumber}{1}
\titlepage

\begin{document}

\begin{flushleft}
{\scriptsize Last compiled:~\DTMnow}
\vspace*{-5mm}
\end{flushleft}
\framebreak

%==============================================================

\begin{frame}[fragile]

\frametitle{Lecture overview}
\begin{enumerate}
	\item Solving nonlinear algebraic equations \red{pp.~175-176}
	\begin{itemize}
		\item general setting
		\item two problems: flight time, fluid level
	\end{itemize}
	
	\item[]
	
	\item Bisection method \red{\S7.4}
	
	\item[]
	
	\item Secant method \red{\S7.3}
	\begin{itemize}
		\item Newton's method
	\end{itemize}
	
	\item[]
	
	\item Extensions
	\begin{itemize}
		\item bisection \& secant methods: re-write as functions
		\item timing code in Python
		\item speed comparisons: bisection vs.~secant
%		\item initialisation \& failure to converge
	\end{itemize}
	
\end{enumerate}

\end{frame}

%==============================================================

\begin{frame}[fragile]

\frametitle{$1)$ Solving nonlinear algebraic equations}

\begin{itemize}
	\item \red{\emph{linear}} equations: $ax + b = 0$
	\begin{itemize}
		\item solution $x = -b/a$
	\end{itemize}
	\item[]
	\item \red{\emph{nonlinear}} equations
	\begin{itemize}
		\item quadratic $ax^2 + bx+c = 0$: solution $x = \frac{-b \pm\sqrt{b^2-4ac}}{2a}$
		\item cubic and quartic (orders $3$ and $4$): exact solutions exist but are \emph{very} complicated
		\item quintic (order $5$) equations: exact solutions \emph{do not exist} in general, proving that needs \emph{serious} mathematics
	\end{itemize}
	\item[]
	\item most equations in engineering applications have no exact ``pen and paper'' solutions!
	
	
\end{itemize}

\end{frame}

%==============================================================

\begin{frame}[fragile]

\frametitle{Numerical solutions to equations}

\begin{flushright}
%\small \emph{``An approximate answer to the right problem is worth a good deal more than an exact answer to an approximate problem''\\ ---John Tukey}
\small\emph{``Far better an approximate answer to the right question\ldots \\ than an exact answer to the wrong question''}\\---John Tukey
\end{flushright}
\vspace*{-3mm}
\textbf{General problem:} find $x$ satisfying
	\[
		f(x) = 0
	\]
	where $f(x)$ is a formula involving $x$
	
	\textbf{Example}
	\[
		f(x) = e^{-x}\sin(x) - \cos(x)
	\]
	has solution $x = 7.85359326$ because
	\[
	e^{-7.85359326}\sin(7.85359326) - \cos(7.85359326) = 0.000
	\]

\end{frame}

%==============================================================

\begin{frame}[fragile]

\frametitle{Flight time}

\begin{itemize}
	\item one more time!
\end{itemize}

\end{frame}

%==============================================================

\begin{frame}[fragile]

\frametitle{Fluid level}

image of measuring cup \\
Engineering applications: water in dam, coal in stockpile

\begin{figure}[ht]
	\centering
	\includegraphics[width=0.5\textwidth]{figures/cupDimensions}
\end{figure}

\end{frame}

%==============================================================

\begin{frame}[fragile]

\frametitle{Fluid level}

% https://www.sjsu.edu/me/docs/hsu-Chapter%2010%20Numerical%20solution%20methods.pdf
% Section 10.3.2

\begin{itemize}
	\item volume $V$ (in millilitres, mL) depends on depth $L$ (in cm) as follows: %, \emph{presented without proof:}
	\[
		V = 0.0268L^3 + 1.884L^2 + 44.15L
	\]
	\item plot V vs L
	\item link to proof: volumes of solids of revolution (needs calculus, MATH1110) \verb+https://www.sjsu.edu/me/docs/hsu-Chapter\%2010\%20Numerical\%20solution\%20methods.pdf+
\end{itemize}

\end{frame}

%==============================================================

\begin{frame}[fragile]

\frametitle{Fluid level}

\begin{figure}[ht]
	\centering
	\includegraphics[width=0.8\textwidth]{figures/fluidVvsL}
\end{figure}

\end{frame}

%==============================================================

\begin{frame}[fragile]

\frametitle{Fluid level}

\begin{itemize}

	\item Question: depth $L$ when cup holds $500$~mL of water?
	\item solve $f(L) = 0$ where
	\[
		F(L) = 0.0268L^3 + 1.884L^2 + 44.15L - 500
	\]
\end{itemize}

\end{frame}

%==============================================================

\begin{frame}[fragile]

\frametitle{$2)$ Bisection method}

\begin{itemize}
	\item basic idea: visualisation
\end{itemize}

\end{frame}

%==============================================================

\begin{frame}[fragile]

\frametitle{Bisection method: pseudocode}

\begin{lstlisting}[style=pseudo,basicstyle=\ttfamily\footnotesize]
INPUT: function f
       endpoint values xLO, xHI
       tolerance TOL
CONDITIONS: xLO < xHI
       f(xLO)<0 and f(xHI)>0  or  f(xLO)>0 and f(xHI)<0
       
xMID = (xLO + xHI)/2
WHILE |f(xMID)| > TOL
	IF f(xMID) is same sign as f(xLO)
		# case A
		set xLO = xMID
	ELSE
		# case B
		set xHI = xMID
	ENDIF	
	xMID = (xLO + xHI)/2
END WHILE
\end{lstlisting}

\end{frame}

%==============================================================

\begin{frame}[fragile]

\frametitle{Bisection method: Python code}
\vspace*{-4mm}
{\small \texttt{bisection.py}}
\vspace*{-2mm}
\begin{lstlisting}[style=CStyle,basicstyle=\scriptsize]
import numpy as np

def f(L):
    return L**3 + 70.3*L**2 + 1647.39*L - 18656.72

eps = 1e-6
x_LO = 6
x_HI = 10

x_MID = (x_LO + x_HI)/2
itCnt = 0
while abs(f(x_MID)) > eps:
    if f(x_MID)*f(x_LO) > 0:
        x_LO = x_MID
    else:
        x_HI = x_MID
    x_MID = (x_LO + x_HI)/2
    itCnt += 1

print('Solution: {}'.format(x_MID))
print('Number of iterations: {}'.format(itCnt))
print('Check: f({:.8f}) = {:.8f}'.format(x_MID,f(x_MID)))
\end{lstlisting}

\end{frame}

%==============================================================

\begin{frame}[fragile]

\frametitle{Bisection method: simulation results}

\begin{itemize}
	\item code commentary
	\item simulation results
	\item live demo
\end{itemize}

\end{frame}

%==============================================================

\begin{frame}[fragile]

\frametitle{$3)$ Secant method}

\begin{itemize}
	\item basic idea: visualisation
\end{itemize}

\end{frame}

%==============================================================

\begin{frame}[fragile]

\frametitle{}

\begin{itemize}
	\item secant method: key equations
\end{itemize}

\end{frame}

%%==============================================================
%
%\begin{frame}[fragile]
%
%\frametitle{}
%
%\begin{itemize}
%	\item secant method: pseudocode
%\end{itemize}
%
%\end{frame}

%==============================================================

\begin{frame}[fragile]

\frametitle{Secant method: Python code}

\texttt{secant.py}
\begin{lstlisting}[style=CStyle,basicstyle=\scriptsize]
import numpy as np

def f(L):
    return L**3 + 70.3*L**2 + 1647.39*L - 18656.72

eps = 1e-6
x0 = 6
x1 = 10
itCnt = 0       # iteration counter
while abs(f(x1)) > eps:
    # line (=secant) through (x0,f(x)) and (x1,f(x1)) intersects
    # horizontal axis at (x,0)
    x = x1 - f(x1)*((x1 - x0)/(f(x1) - f(x0)))
    x0 = x1
    x1 = x
    itCnt += 1

print('Solution: {}'.format(x))
print('Number of iterations: {}'.format(itCnt))
print('Check: f({:.8f}) = {:.8f}'.format(x,f(x)))
\end{lstlisting}

\end{frame}

%==============================================================

\begin{frame}[fragile]

\frametitle{Secant method: simulation results}

\begin{itemize}
	\item code commentary
	\item simulation results
	\item live demo
\end{itemize}

\end{frame}

%==============================================================

\begin{frame}[fragile]

\frametitle{Newton's method}

\begin{itemize}
	\item aka Newton--Raphson method
	\item discussion of derivatives, and how they're needed in Newton's method
	\item we won't consider Newton's method in this course, as can't assume knowledge of calculus
	\item secant as approximation to Newton's method
	\item Newton's method is \emph{really} popular
\end{itemize}

\end{frame}

%==============================================================

\begin{frame}[fragile]

\frametitle{Newton's method}

\begin{figure}[ht]
	\centering
	\includegraphics[width=0.8\textwidth]{figures/NewtonsMethod}
\end{figure}

\end{frame}

%==============================================================

\begin{frame}[fragile]

\frametitle{$4)$ Extensions}

\texttt{bisection\_fn.py}
\begin{lstlisting}[style=CStyle,basicstyle=\scriptsize]
def f(L):
    return L**3 + 70.3*L**2 + 1647.39*L - 18656.72

def my_bisection(f, x_LO, x_HI, tol):
    x_MID = (x_LO + x_HI) / 2
    itCnt = 0
    while abs(f(x_MID)) > tol:
        if f(x_MID) * f(x_LO) > 0:
            x_LO = x_MID
        else:
            x_HI = x_MID
        x_MID = (x_LO + x_HI) / 2
        itCnt += 1
    return x_MID, itCnt

x, numIt = my_bisection(f, 6, 10, 1e-6)

print('Solution: {}'.format(x))
print('Number of iterations: {}'.format(numIt))
print('Check: f({:.8f}) = {:.8f}'.format(x,f(x)))
\end{lstlisting}

\end{frame}

%==============================================================

\begin{frame}[fragile]

\frametitle{Bisection method as a function}

\begin{itemize}
	\item code commentary
	\item simulation results
	\item live demo
\end{itemize}

\end{frame}

%==============================================================

\begin{frame}[fragile]

\frametitle{Secant method as a function}

\texttt{secant\_fn.py}
\begin{lstlisting}[style=CStyle,basicstyle=\scriptsize]
def f(L):
    return L**3 + 70.3*L**2 + 1647.39*L - 18656.72

def my_secant(f, x0, x1, tol):
    itCnt = 0
    while abs(f(x1)) > tol:
        x = x1 - f(x1) * ((x1 - x0) / (f(x1) - f(x0)))
        x0 = x1
        x1 = x
        itCnt += 1
    return x1, itCnt

x, numIt = my_secant(f, 6, 10, 1e-6)

print('Solution: {}'.format(x))
print('Number of iterations: {}'.format(numIt))
print('Check: f({:.8f}) = {:.8f}'.format(x,f(x)))
\end{lstlisting}

\end{frame}

%==============================================================

\begin{frame}[fragile]

\frametitle{Secant method as a function}

\begin{itemize}
	\item code commentary
	\item simulation results
	\item live demo
\end{itemize}

\end{frame}

%==============================================================

\begin{frame}[fragile]

\begin{itemize}
	\item often useful to measure time taken to perform calculations; easy in Python!
	\item start by importing \texttt{time} module:
\end{itemize}

\frametitle{Timing code in Python}
\begin{lstlisting}[style=CStyle,basicstyle=\scriptsize]
import time
\end{lstlisting}

\begin{itemize}
	\item function \texttt{time.perf\_counter()} returns value of a clock
	\begin{itemize}
		\item \texttt{float} value (in seconds) 
	\end{itemize}
	\item elapsed time is \emph{difference} between two successive calls
\end{itemize}

\begin{lstlisting}[style=CStyle,basicstyle=\scriptsize]
tStart = time.perf_counter()
xB, numItB = my_bisection(f, 6, 10, 1e-6)
tStop = time.perf_counter()
tBisect = tStop - tStart
\end{lstlisting}

\end{frame}

%==============================================================

\begin{frame}[fragile]

\frametitle{Speed comparisons: bisection vs.~secant}

\begin{itemize}
	\item live demo \texttt{bisectionvssecant.py}
%	\item LL text has optimised bisection and secant to minimise function calls (typically most time-consuming operation)
\end{itemize}

\begin{figure}[ht]
	\centering
	\includegraphics[width=0.8\textwidth]{figures/BisectionSecantSpeed}
\end{figure}

\end{frame}

%%==============================================================
%
%\begin{frame}[fragile]
%
%\frametitle{}
%
%\begin{itemize}
%	\item initialisation
%	\item failure to converge
%\end{itemize}
%
%\end{frame}

%==============================================================

\begin{frame}[fragile]

\frametitle{Lecture summary}
\begin{itemize}
	\item Solving nonlinear algebraic equations

	\item[]
	
	\item Bisection method

	\item[]
	
	\item Secant method
	\begin{itemize}
		\item Newton's method
	\end{itemize}

	\item[]
	
	\item Extensions
	
\end{itemize}

\end{frame}

%==============================================================

\begin{frame}[fragile]

\frametitle{More information}
\begin{itemize}
	\item Newton's method in textbook \red{\S7.2}
	\begin{itemize}
		\item needs \emph{differentiation} from calculus (MATH1110)
		\item in particular: need expression for \emph{tangent lines} to function $f(x)$, written as $f'(x)$
	\end{itemize}

	\item[]
	
	\item ``optimised'' versions of bisection and secant methods in textbook \red{\S7.3} and \red{\S7.4}
	\begin{itemize}
		\item maximise speed of computation by minimising number of function evaluations $f(x)$
	\end{itemize}
	
	\item[]
	
\end{itemize}

\end{frame}

\end{document}