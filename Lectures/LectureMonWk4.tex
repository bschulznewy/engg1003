\documentclass[14pt]{beamer}
\usetheme{EastLansing}
\usecolortheme{spruce}

\usepackage{xcolor}
\usepackage{listings}
\usepackage{courier}
\usepackage{graphicx}
\usepackage{amsmath}
\usepackage{algorithm2e}
\usepackage{multicol}

% https://tex.stackexchange.com/questions/42619/x-mark-to-match-checkmark
\usepackage{pifont}% http://ctan.org/pkg/pifont

%% https://stackoverflow.com/questions/1435837/how-to-remove-footers-of-latex-beamer-templates
%%gets rid of bottom navigation bars
%\setbeamertemplate{footline}[page number]
%
%gets rid of navigation symbols
\setbeamertemplate{navigation symbols}{}


\usefonttheme[onlymath]{serif}

\definecolor{mGreen}{rgb}{0,0.6,0}
\definecolor{mGray}{rgb}{0.5,0.5,0.5}
\definecolor{mPurple}{rgb}{0.8,0,0.82}
\definecolor{backgroundColour}{rgb}{0.95,0.95,0.92}
\definecolor{lightBlue}{rgb}{0.1, 0.1, 0.8}

\newcommand\red[1]{{\color{red} #1}}
\newcommand\green[1]{{\color{green} #1}}
\newcommand\blue[1]{{\color{blue} #1}}

\newcommand{\cmark}{\ding{51}}%
\newcommand{\xmark}{\ding{55}}%

\lstdefinestyle{CStyle}{
    backgroundcolor=\color{backgroundColour},   
    commentstyle=\color{mGreen},
    keywordstyle=\color{magenta},
    numberstyle=\tiny\color{mGray},
    stringstyle=\color{mPurple},
    basicstyle=\footnotesize,
    breakatwhitespace=false,         
    breaklines=true,                 
    captionpos=b,                    
    keepspaces=true,                 
    numbers=left,                    
    numbersep=5pt,                  
    showspaces=false,                
    showstringspaces=false,
    showtabs=false,                  
    tabsize=2,
    language=C
}
\lstdefinestyle{pseudo}{
        basicstyle=\ttfamily\footnotesize,
        keywordstyle=\color{lightBlue},
        morekeywords={BEGIN,END,IF,ELSE,ENDIF,ELSEIF,PRINT,WHILE,RETURN,ENDWHILE,DO,FOR,TO,IN,ENDFOR,BREAK,INPUT},
        morecomment=[l]{//},
        commentstyle=\color{mGreen}
}

\lstset{basicstyle=\footnotesize\ttfamily,breaklines=true}
\lstset{framextopmargin=50pt,tabsize=2}

\title{ENGG1003 - Monday Week 4}
\subtitle{Iteration again: \texttt{for} vs.~\texttt{while} loops, \\ debugging strategies \& random numbers}
\author{Steve Weller}
\institute{University of Newcastle}
%\date{\today}
\date{15 March, 2021}

% following is a bit of a hack, but forces page numbers (technically: frame numbers) to run 1,2,3,... 
% with titlepage counting as frame 1

\addtocounter{framenumber}{1}
\titlepage

\begin{document}
\framebreak

%==============================================================

\begin{frame}[fragile]

\frametitle{Lecture overview}
\begin{enumerate}
	\item Iteration again (again): \texttt{for} vs.~\texttt{while} loops \red{\S3.3.3}

	\item[]
	
	\item Debugging strategies
	
	\item[]
	
	\item Random numbers in Python \red{\S2.4}

\end{enumerate}

\end{frame}

%==============================================================

\begin{frame}[fragile]

\frametitle{$1)$ iteration again: \texttt{for} vs.~\texttt{while} loops}

\begin{itemize}
	\item side-by-side comparison for print-1-to-10
	\item similarities and differences
	\item when to use each
\end{itemize}

\end{frame}

%==============================================================

\begin{frame}[fragile]

\frametitle{}

\begin{itemize}
	\item xxx
\end{itemize}

\end{frame}

%==============================================================

\begin{frame}[fragile]

\frametitle{Example: Finding the maximum height}

\begin{itemize}
	\item \S3.3.3
	\item new program instead finds the maximum height achieved by the ball
	\item will solve in two ways: for and while
\end{itemize}

\end{frame}

%==============================================================

\begin{frame}[fragile]

\frametitle{}

\begin{figure}[ht]
	\centering
	\includegraphics[width=\textwidth]{figures/LLp71a}
\end{figure}

\end{frame}

%==============================================================

\begin{frame}[fragile]

\frametitle{Focus:~\texttt{for} loop to find max height}

\begin{itemize}
	\item describe strategy in words
	\item live demo
\end{itemize}

% We focus our attention on the new thing here, the search performed by the for
%loop. The value in y[0] is used as a starting value for largest_height. The very
%first check then, tests whether y[1] is larger than this height. If so, y[1] is stored
%as the largest height. The for loop then updates i to 2, and continues to check
%y[2], and so on. Each time we find a larger number, we store it. When finished,
%largest_height will contain the largest number from the array y.

\begin{figure}[ht]
	\centering
	\includegraphics[width=\textwidth]{figures/LLp71b}
	\includegraphics[width=\textwidth]{figures/LLp71c}
\end{figure}

\end{frame}

%==============================================================

\begin{frame}[fragile]

\frametitle{Focus:~\texttt{while} loop to find max height}

\begin{itemize}
	\item describe strategy in words
	\item live demo
\end{itemize}

%The observant reader has already seen the similarity of finding the maximum
%height and finding the time of flight, as we addressed previously in Sect. 3.2.1.
%In fact, we could alternatively have solved the maximum height problem here by
%utilizing that y[i+1] > y[i] as the ball moves towards the top. Doing this, our
%search loop could have been written
%i = 0
%while y[i+1] > y[i]:
%i = i + 1
%When the condition y[i+1] > y[i] becomes False, we could report y[i+1] as
%our approximation of the maximum height, for example.

\begin{figure}[ht]
	\centering
	\includegraphics[width=\textwidth]{figures/LLp71d}
\end{figure}

\end{frame}

%==============================================================

\begin{frame}[fragile]

\frametitle{$2)$ Debugging strategies}

\begin{enumerate}
	\item running code by hand
	\item don't guess, print!
	\item take baby steps
\end{enumerate}

\end{frame}

%==============================================================

\begin{frame}[fragile]

\frametitle{Running code by hand}

\begin{itemize}
	\item Work through your code line-by-line, with a piece of paper and a pen
	\item Use paper/notes-app before you run code, so that you know what you want program to calculate on each line before you run code
	\item Second-best: use notes app on iPad or similar---idea is to \emph{think before computing}
	\item Check that every line of code, with every use of an array index (hint) matches what you expect
	\item Near enough isn't good enough here. Think like a computer: work systematically through each line of code. Is it doing what you want it to do?
	\item It's easy (but often misleading) to look at code and think you know what it's doing. But is it really?
	\item mindset shift: rather than hoping or thinking it's OK, try and find the bug (might be more than one)
\end{itemize}

\end{frame}

%==============================================================

\begin{frame}[fragile]

\frametitle{Don't guess, print!}

\begin{itemize}
	\item xxx
\end{itemize}

\end{frame}

%==============================================================

\begin{frame}[fragile]

\frametitle{Take baby steps}

\begin{itemize}
	\item xxx
\end{itemize}

\end{frame}

%==============================================================

\begin{frame}[fragile]

\frametitle{$3)$ Random numbers in Python}

\begin{itemize}
	\item Python provides ability to produce (apparently) random numbers
	\item[]
	\item referred to as \red{\emph{pseudo-random numbers}}
	\item[]
	\item these numbers are not \emph{truly} random
	\begin{itemize}
		\item produced in a complicated (but ``deterministic'' or predictable) way once a \red{\emph{seed}} has been set
	\end{itemize}	
	\item[]
	\item seed is a number which depends on the current time
		
\end{itemize}

\end{frame}

%==============================================================

\begin{frame}[fragile]

\frametitle{Drawing \textbf{one} random number at a time}

\begin{figure}[ht]
	\centering
	\includegraphics[width=\textwidth]{figures/LLp54}
\end{figure}

Python code:~\href{https://github.com/slgit/prog4comp_2/blob/master/py36-src/throw_2_dice.py}{\texttt{throw\_2\_dice.py}}

\begin{itemize}
	\item function \texttt{randint(a,b)}
	\begin{itemize}
		\item available from imported module \texttt{random} %, part of the standard Python library
		\item returns a pseudo-random \emph{integer} in the range $[a,b]$ where $a \leq b$
	\end{itemize}
\end{itemize}

\end{frame}

%==============================================================

\begin{frame}[fragile]

\frametitle{Fixing the seed}

\begin{itemize}
	\item when debugging programs that involve pseudo-random numbers, often helps to \red{\emph{fix the seed}}
	\item[]
	\item ensures that \emph{identical sequence of numbers will be generated} each time code is run
	\begin{itemize}
		\item hence results are \emph{repeatable}
	\end{itemize}
	\item[]
	\item tell Python what seed should be using \texttt{random.seed} function
	\item[]
	\item \textbf{Example:} \texttt{random.seed(10)} and run Python code:~\href{https://github.com/slgit/prog4comp_2/blob/master/py36-src/throw_2_dice.py}{\texttt{throw\_2\_dice.py}}
\end{itemize}

\end{frame}

%==============================================================

\begin{frame}[fragile]

\frametitle{Two functions: \texttt{random} and \texttt{uniform}}

\begin{itemize}
	\item both \texttt{random} and \texttt{uniform} return a floating point number from an interval where each number has \emph{equal probability} of being drawn
	\begin{itemize}
		\item random number drawn from \red{\emph{uniform}} probability distribution
		\item Note: \texttt{random} function in \texttt{random} module
	\end{itemize}
	\item[]
	\item \texttt{random}
	\begin{itemize}
		\item draw from interval $[0, 1)$
%		\item 0 is included, but 1 is not
	\end{itemize}
	
	\item[]
	\item \texttt{uniform}
	\begin{itemize}
		\item draw from interval $[a, b]$
	\end{itemize}

%	\item text doesn't mention Gaussian (normal) random numbers
\end{itemize}

\end{frame}

%==============================================================

\begin{frame}[fragile]

\frametitle{Live demo:~\texttt{random} and \texttt{uniform}}

\begin{figure}[ht]
	\centering
	\includegraphics[width=\textwidth]{figures/LLp55a}
\end{figure}

\end{frame}

%==============================================================

\begin{frame}[fragile]

\frametitle{Drawing \textbf{many} random numbers at a time}

\begin{itemize}
	\item three random number generators seen so far
	\begin{itemize}
		\item each generates just \emph{one} random number at a time
	\end{itemize}
	\item[]
	\item to generate an \emph{array} of random numbers\ldots
	\item[] \ldots could use a loop \& generate one random number in each iteration
	\item[]
	\item better (faster) solution: use \texttt{random} module in \texttt{numpy} library

\end{itemize}

\end{frame}

%==============================================================

\begin{frame}[fragile]

\frametitle{Live demo: random numbers from \texttt{numpy} library}

\textbf{Example:} \texttt{np.random.randint}

\begin{itemize}
	\item[]
	\begin{itemize}
		\item \texttt{numpy} library / \texttt{random} module / \texttt{randint} function
		\item \texttt{randint(a,b,n)} generates $n$ integers from $[a,b)$
	\end{itemize}
\end{itemize}

\begin{figure}[ht]
	\centering
	\includegraphics[width=\textwidth]{figures/LLp55b}
\end{figure}

\begin{itemize}
	\item live demo, also fix seed:~\texttt{np.random.seed(10)}
\end{itemize}

\end{frame}

%==============================================================

\begin{frame}[fragile]

\frametitle{Lecture summary}
\begin{itemize}
	\item Iteration again
	\begin{itemize}
		\item \texttt{for} vs.~\texttt{while}
	\end{itemize}

	\item[]
	
	\item Debugging strategies
	\begin{itemize}
		\item running code by hand
		\item don't guess, print!
		\item take baby steps
	\end{itemize}

	\item[]
	
	\item Random numbers
		\begin{itemize}
			\item random module---random numbers one at a time %: \texttt{randint}, \texttt{random} and \texttt{uniform}
			\item random module in \texttt{numpy} library---arrays of random numbers % :~\texttt{randint}, \texttt{random} and \texttt{uniform}
		\end{itemize}
		
\end{itemize}

\end{frame}

\end{document}