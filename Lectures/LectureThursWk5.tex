\documentclass[english,14pt]{beamer}
\usetheme{EastLansing}
\usecolortheme{spruce}

\usepackage{xcolor}
\usepackage{listings}
\usepackage{courier}
\usepackage{graphicx}
\usepackage{amsmath}
\usepackage{algorithm2e}
\usepackage{multicol}
\usepackage{hyperref}

% http://mirrors.ibiblio.org/CTAN/macros/latex/contrib/datetime2/datetime2.pdf
\usepackage{babel}
\usepackage[useregional]{datetime2}

% https://tex.stackexchange.com/questions/42619/x-mark-to-match-checkmark
\usepackage{pifont}% http://ctan.org/pkg/pifont

%% https://stackoverflow.com/questions/1435837/how-to-remove-footers-of-latex-beamer-templates
%%gets rid of bottom navigation bars
%\setbeamertemplate{footline}[page number]
%
%gets rid of navigation symbols
\setbeamertemplate{navigation symbols}{}


\usefonttheme[onlymath]{serif}

\definecolor{mGreen}{rgb}{0,0.6,0}
\definecolor{mGray}{rgb}{0.5,0.5,0.5}
\definecolor{mPurple}{rgb}{0.8,0,0.82}
\definecolor{backgroundColour}{rgb}{0.95,0.95,0.92}
\definecolor{lightBlue}{rgb}{0.1, 0.1, 0.8}

\newcommand\red[1]{{\color{red} #1}}
\newcommand\green[1]{{\color{green} #1}}
\newcommand\blue[1]{{\color{blue} #1}}

\newcommand{\cmark}{\ding{51}}%
\newcommand{\xmark}{\ding{55}}%

\lstdefinestyle{CStyle}{
    backgroundcolor=\color{backgroundColour},   
    commentstyle=\color{mGreen},
    keywordstyle=\color{magenta},
    numberstyle=\tiny\color{mGray},
    stringstyle=\color{mPurple},
    basicstyle=\footnotesize,
    breakatwhitespace=false,         
    breaklines=true,                 
    captionpos=b,                    
    keepspaces=true,                 
    numbers=left,                    
    numbersep=5pt,                  
    showspaces=false,                
    showstringspaces=false,
    showtabs=false,                  
    tabsize=2,
    language=Python
}

\lstdefinestyle{pseudo}{
        basicstyle=\ttfamily\footnotesize,
        keywordstyle=\color{lightBlue},
        morekeywords={BEGIN,END,IF,ELSE,ENDIF,ELSEIF,PRINT,WHILE,RETURN,ENDWHILE,DO,FOR,TO,IN,ENDFOR,BREAK,INPUT},
        morecomment=[l]{//},
        commentstyle=\color{mGreen}
}

\lstset{basicstyle=\footnotesize\ttfamily,breaklines=true}
\lstset{framextopmargin=50pt,tabsize=2}

\title{ENGG1003 - Thursday Week 5}
\subtitle{Functions and Arrays}
\author{Sarah Johnson}
\institute{University of Newcastle}
%\date{\today}
\date{25 March, 2021}

% following is a bit of a hack, but forces page numbers (technically: frame numbers) to run 1,2,3,... 
% with titlepage counting as frame 1

\addtocounter{framenumber}{1}
\titlepage

\begin{document}

%\begin{flushleft}
%{\scriptsize Last compiled:~\DTMnow}
%\vspace*{-5mm}
%\end{flushleft}
\framebreak


\begin{frame}
\frametitle{The Story So Far}
\begin{itemize}
\item Course summary:
	\begin{itemize}
	\item Variables and data types  
		\item Arrays (via numpy)
		\item Plotting (via matplotlib)
		\item Flow control
			\begin{itemize}
				\item \texttt{if}
				\item \texttt{while}
				\item \texttt{for}
			\end{itemize}
		\item Functions
	\end{itemize}
\item Today: Arrays and functions together 
	\begin{itemize}
		\item and we will add some flow control in too
	\end{itemize}
\end{itemize}
\end{frame}

\begin{frame}
\frametitle{Example 1 - Access an Array}
\begin{itemize}
    \item Write a function which takes as input an array \texttt{x} and calculates it's minimum value, maximum value and its average value
    \item First let's make some design decisions:
	    \begin{itemize}
		    \item Name: \texttt{array\_stats()}
		    \item Argument: an array \texttt{x}
		    \item Return Values: \texttt{min\_x, max\_x, avg\_x} 
	    \end{itemize}
	\item Let's do it step by step live ...    
\end{itemize}
\end{frame}


\begin{frame}
\frametitle{Functions which Modify an Array}
\begin{itemize}
\item Technically, a pointer to the array is being passed to the function not a copy of the array
\item Because arrays are passed via a pointer the function gets \textit{the actual array}
\item --  when an element of the array is accessed the memory address of the original array element is accessed
\item Consequently, modifying the array in the function modifies the original variable
\item You don't \textit{need} a return value
	\begin{itemize}
		\item In a technically incorrect way: all the array's elements are ``returned''
	\end{itemize}
\end{itemize}
\end{frame}

\begin{frame}
\frametitle{Example 2 - Modify an Array}
\begin{itemize}
\item Lets write and test this live...
\item Write a function which:
	\begin{itemize}	
		\item 1) Zeros any negative values in an array \texttt{x} (i.e. replaces any negative values in \texttt{x} with 0)
		\item 2) Creates a new array \texttt{y} which keeps the positive values in \texttt{x} and has zeros in place of the negative values in \texttt{x}
	\end{itemize}
\end{itemize}
\end{frame}

\begin{frame}[fragile]
\frametitle{Example 3 - Ball Height}
\begin{itemize}
\item Let's re-write our code from Monday so that our ball height function works with arrays:
\begin{lstlisting}[style=CStyle]
import numpy as np
            
def ball_height2(v0, t, g=9.81):    
    return v0*t - 0.5*g*t**2      
    
v0 = 5
t = np.linspace(0,1,11)
y = ball_height2(v0, t)  
\end{lstlisting}
	\end{itemize}
\end{frame}

\begin{frame}[fragile]
\frametitle{Example 3 - Ball Height}
\begin{itemize}
    \item What actually changed? Went from this:
\begin{lstlisting}[style=CStyle]
def ball_height2(v0, t, g=9.81):    
    return v0*t - 0.5*g*t**2      
    
v0 = 5
t = 0.6
y = ball_height2(v0, t)  
\end{lstlisting}
    \item to this:
\begin{lstlisting}[style=CStyle]
include numpy as np
def ball_height2(v0, t, g=9.81):    
    return v0*t - 0.5*g*t**2      
    
v0 = 5
t = np.linspace(0,1,11)
y = ball_height2(v0, t)  
\end{lstlisting}
	\end{itemize}
\end{frame}

\begin{frame}
\frametitle{Example 3 - Ball Height}
\begin{itemize}
    \item Python is hiding a lot from us here
    \item Advantage: code easy to read / easy to write
    \item Disadvantage: it's very easy to write code which is not doing what you think it is doing
    \begin{itemize}
        \item Have a look through the Discord chats for issues that arise when arrays are confused with scalars (a single integer / float etc)
 	\end{itemize}      
\end{itemize}
\end{frame}

\begin{frame}
\frametitle{When should functions be used?}
\begin{itemize}
\item When should functions be used?
 
\item Well, what do they achieve?
	\begin{itemize}
		\item \textit{Much} easier to solve problems when they're broken down into sub-tasks
		\item Reduce code line count and complexity (if they are called multiple times)
		\item Allows code re-use between projects
		\item \textit{Much} easier to perform project management between multiple programmers
		\item Bugs in a function are easier to fix than a bug in code which has been copy+pasted multiple times
		\item ...the list goes on
	\end{itemize}
\end{itemize}
\end{frame}

\begin{frame}
\frametitle{When should functions be used?}
\begin{itemize}
\item What about in an ENGG1003 context?
 
	\begin{itemize}
		\item Vague rule of thumb? More than 10-20 lines or so in your code - consider breaking up into functions
		\item Break a big problem into multiple sub-problems
			\begin{itemize}
				\item Implement each as their own function
				 
				\item Yes, even if they are only called once
				 
				\item Do what you feel is most ``readable''
				 
				\item Your opinion here will change with experience 
			\end{itemize}
	\end{itemize}
\end{itemize}
\end{frame}


\begin{frame}
\frametitle{More Information}
\begin{center}
	\begin{itemize}
		\item Further Reading: Section 4.1 of the course textbook
        \item More Practice: All the exercises in Section 4.3 of the course textbook  
    \end{itemize}
\end{center}
\end{frame}

\begin{frame}[fragile]
\frametitle{ENGG1003 Assessed Lab}
    \begin{itemize}
	    \item Results are now on Blackboard (BB)
	    \item Let Steve know via email or a Discord DM if your result is missing or doesn't match what the demonstrator told you
	    \item We're still processing all of the adverse circumstances requests
    \end{itemize}
\end{frame}

\begin{frame}[fragile]
\frametitle{ENGG1003 Quiz}
    \begin{itemize}
		\item During the scheduled Thursday lecture time 4-5pm on 1 April
        \item There will NOT be a zoom webinar running or a YouTube livestream during the quiz time
        \item The quiz paper will become available on BB, and students must submit their solutions in BB by the end of the quiz
        \item This is an OPEN-BOOK quiz covering material presented in weeks 1-4 of lectures, and therefore weeks 1-5 of labs. It counts towards 15\% of your overall course grade in ENGG1003
    \end{itemize}
\end{frame}


\begin{frame}[fragile]
\frametitle{ENGG1003 Quiz}
    \begin{itemize}
        \item As an open-book quiz, you will be asked to solve a small number of problems by reading and/or writing Python code, before pasting your answers into Blackboard for assessment

        \item In particular, students are strongly encouraged to develop \& test solutions in PyCharm during the quiz, before submitting to BB for assessment
    \end{itemize}
\end{frame}

\begin{frame}[fragile]
\frametitle{ENGG1003 Quiz}
    \begin{itemize}
\item The best exercises for students to prepare for the mid-term quiz as follows:

\item (a) all of the lab tasks, including going back and completing any you may not have completed at the time

\item (b) review the assessed lab from week 4

\item (c) students wanting additional practice can attempt questions from Chapters 1-3 of the LL textbook that were not already included in the lab tasks

    \end{itemize}
\end{frame}

\end{document}