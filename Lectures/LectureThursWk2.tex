\documentclass[14pt]{beamer}
\usetheme{EastLansing}
\usecolortheme{spruce}

\usepackage{xcolor}
\usepackage{listings}
\usepackage{courier}
\usepackage{graphicx}
\usepackage{amsmath}
\usepackage{algorithm2e}
\usepackage{multicol}

% https://tex.stackexchange.com/questions/42619/x-mark-to-match-checkmark
\usepackage{pifont}% http://ctan.org/pkg/pifont

%% https://stackoverflow.com/questions/1435837/how-to-remove-footers-of-latex-beamer-templates
%%gets rid of bottom navigation bars
%\setbeamertemplate{footline}[page number]
%
%gets rid of navigation symbols
\setbeamertemplate{navigation symbols}{}


\usefonttheme[onlymath]{serif}

\definecolor{mGreen}{rgb}{0,0.6,0}
\definecolor{mGray}{rgb}{0.5,0.5,0.5}
\definecolor{mPurple}{rgb}{0.8,0,0.82}
\definecolor{backgroundColour}{rgb}{0.95,0.95,0.92}
\definecolor{lightBlue}{rgb}{0.1, 0.1, 0.8}

\newcommand\red[1]{{\color{red} #1}}
\newcommand\green[1]{{\color{green} #1}}
\newcommand\blue[1]{{\color{blue} #1}}

\newcommand{\cmark}{\ding{51}}%
\newcommand{\xmark}{\ding{55}}%

\lstdefinestyle{CStyle}{
    backgroundcolor=\color{backgroundColour},   
    commentstyle=\color{mGreen},
    keywordstyle=\color{magenta},
    numberstyle=\tiny\color{mGray},
    stringstyle=\color{mPurple},
    basicstyle=\footnotesize,
    breakatwhitespace=false,         
    breaklines=true,                 
    captionpos=b,                    
    keepspaces=true,                 
    numbers=left,                    
    numbersep=5pt,                  
    showspaces=false,                
    showstringspaces=false,
    showtabs=false,                  
    tabsize=2,
    language=C
}
\lstdefinestyle{pseudo}{
        basicstyle=\ttfamily\footnotesize,
        keywordstyle=\color{lightBlue},
        morekeywords={BEGIN,END,IF,ELSE,ENDIF,ELSEIF,PRINT,WHILE,RETURN,ENDWHILE,DO,FOR,TO,IN,ENDFOR,BREAK,INPUT},
        morecomment=[l]{//},
        commentstyle=\color{mGreen}
}

\lstset{basicstyle=\footnotesize\ttfamily,breaklines=true}
\lstset{framextopmargin=50pt,tabsize=2}

\title{ENGG1003 - Thursday Week 2}
\subtitle{Data types, and introduction to arrays}
\author{Steve Weller}
\institute{University of Newcastle}
%\date{\today}
\date{4 March, 2021}

% following is a bit of a hack, but forces page numbers (technically: frame numbers) to run 1,2,3,... 
% with titlepage counting as frame 1

\addtocounter{framenumber}{1}
\titlepage

\begin{document}
\framebreak

%==============================================================

\begin{frame}[fragile]

\frametitle{Lecture overview}
\begin{enumerate}
	\item variables and data types \red{\S2.2}
	\begin{itemize}
		\item principles
		\item live demo
	\end{itemize}

	\item[]
	
	\item arrays in Python \red{\S2.3}
		\begin{itemize}
			\item principles
			\item live demo
		\end{itemize}

\end{enumerate}

\end{frame}

%\begin{itemize}
%	\item xxx
%\end{itemize}

%==============================================================

\begin{frame}[fragile]

\frametitle{$1)$ variables and data types}

\begin{itemize}
	\item variable names -- make them descriptive
	\item camelCase
	\item snake$\_$case
	\item matter of preference/style/taste
		\begin{itemize}
			\item experiment, find what works best for you %, won't be marked down for ``wrong'' style!			
		\end{itemize}
\end{itemize}

\end{frame}

%==============================================================

\begin{frame}[fragile]

\frametitle{Assignment}

\begin{itemize}
	\item \texttt{x = 2}
	\item \texttt{x = x + 4}
	\item \texttt{x+= 4} is short for \texttt{x = x + 4}
\end{itemize}

\end{frame}

%==============================================================

\begin{frame}[fragile]

\frametitle{The type of a variable}

\begin{itemize}
%	\item \red{\S2.2.4}
	\item types seen so far:
	\begin{itemize}
		\item \texttt{int}
		\item \texttt{float}
		\item \texttt{str}
		\item another (final?) type will be introduced next lecture % Boolean
	\end{itemize}
	\item explain ``floating point'' terminology---think of float as real number (fractional part, not an integer)
	\item mention ``objects'' only in passing
\end{itemize}

\end{frame}

%==============================================================

\begin{frame}[fragile]

\frametitle{The type of a variable (ctd.)}

\end{frame}

%==============================================================

\begin{frame}[fragile]

\frametitle{Type function}

\begin{itemize}
	\item \red{\S2.2.4} and \red{\S2.2.5}
	\item built-in function \texttt{type}
	\item type conversion
	\item automatic type conversion
\end{itemize}

\end{frame}

%==============================================================

\begin{frame}[fragile]

\frametitle{Operator precedence}

\end{frame}

%==============================================================

\begin{frame}[fragile]

\frametitle{Division---quotient and remainder}

\end{frame}

%==============================================================

\begin{frame}[fragile]
\frametitle{Live demo of variables and data types}

\end{frame}

%==============================================================

\begin{frame}[fragile]

\frametitle{2) Arrays in Python}

\end{frame}

%==============================================================

\begin{frame}[fragile]

\frametitle{Array creation and array elements}

\begin{itemize}
	\item \red{\S2.3.1}
	\item array \red{\emph{index}}, plural array \red{\emph{indices}}
	\item Python has \red{\emph{zero-based indexing}}
	\item four common ways of creating arrays:
	\begin{itemize}
		\item linspace
		\item zeros
		\item array
		\item copy
	\end{itemize}
\end{itemize}

\end{frame}

%==============================================================

\begin{frame}[fragile]

\frametitle{Zeros function}

\begin{itemize}
	\item create an array of zeros
\end{itemize}

\end{frame}

%==============================================================

\begin{frame}[fragile]

\frametitle{Len function}

\begin{itemize}
	\item length of an array
\end{itemize}

\end{frame}

%==============================================================

\begin{frame}[fragile]

\frametitle{Index out of bounds}

\begin{itemize}
	\item show error when access out of bounds
	\item contrast with C
\end{itemize}

\end{frame}

%==============================================================

\begin{frame}[fragile]

\frametitle{Copying an array}

\begin{itemize}
	\item \red{BE VERY CAREFUL} with naive/obvious copy method
	\item \texttt{copy} function creates new array and copies values
	\begin{itemize}
		\item use this method!
	\end{itemize}
\end{itemize}

\end{frame}

%==============================================================

\begin{frame}[fragile]

\frametitle{Slicing an array}

\begin{itemize}
	\item needs a figure showing boxes
\end{itemize}

\end{frame}

%==============================================================

\begin{frame}[fragile]
\frametitle{Live demo of Python arrays}

\end{frame}

%%==============================================================
%
%\begin{frame}[fragile]
%
%\frametitle{}
%
%\end{frame}
%
%%==============================================================
%
%\begin{frame}[fragile]
%
%\frametitle{}
%
%\end{frame}
%
%%==============================================================
%
%\begin{frame}[fragile]
%
%\frametitle{}
%
%\end{frame}

\end{document}
