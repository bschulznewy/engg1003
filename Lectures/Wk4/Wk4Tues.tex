\documentclass[14pt]{beamer}
\usetheme{Dresden}
\usecolortheme{orchid}

\usepackage{xcolor}
\usepackage{listings}
\usepackage{courier}
\usepackage{graphicx}
\usepackage{amsmath}
\usepackage{algorithm2e}
\usepackage{multicol}
\usepackage{amssymb}

\usefonttheme[onlymath]{serif}

\definecolor{mGreen}{rgb}{0,0.6,0}
\definecolor{mGray}{rgb}{0.5,0.5,0.5}
\definecolor{mPurple}{rgb}{0.8,0,0.82}
\definecolor{backgroundColour}{rgb}{0.95,0.95,0.92}
\definecolor{lightBlue}{rgb}{0.1, 0.1, 0.8}

\lstdefinestyle{CStyle}{
    backgroundcolor=\color{backgroundColour},   
    commentstyle=\color{mGreen},
    keywordstyle=\color{magenta},
    numberstyle=\tiny\color{mGray},
    stringstyle=\color{mPurple},
    basicstyle=\footnotesize\ttfamily,
    breakatwhitespace=false,         
    breaklines=true,                 
    captionpos=b,                    
    keepspaces=true,                 
    numbers=left,                    
    numbersep=5pt,                  
    showspaces=false,                
    showstringspaces=false,
    showtabs=false,                  
    tabsize=2,
    language=C
}

\lstdefinestyle{Ctable}{
    backgroundcolor=\color{backgroundColour},   
    commentstyle=\color{mGreen},
    keywordstyle=\color{magenta},
    numberstyle=\tiny\color{mGray},
    stringstyle=\color{mPurple},
    basicstyle=\footnotesize\ttfamily,
    breakatwhitespace=false,         
    breaklines=true,                 
    captionpos=b,                    
    keepspaces=true,                                  
    showspaces=false,                
    showstringspaces=false,
    showtabs=false,                  
    tabsize=2,
    language=C
}

\lstdefinestyle{pseudo}{
        basicstyle=\ttfamily\footnotesize,
        keywordstyle=\color{lightBlue},
        morekeywords={BEGIN,END,IF,ELSE,ENDIF,ELSEIF,PRINT,WHILE,RETURN,ENDWHILE,DO,FOR,TO,IN,ENDFOR,BREAK,INPUT,READ},
        morecomment=[l]{//},
        commentstyle=\color{mGreen}
}

\lstset{basicstyle=\footnotesize\ttfamily,breaklines=true}
\lstset{framextopmargin=50pt,tabsize=2}

\title{ENGG1003 - Tuesday Week 4}
\subtitle{Loose Ends \\ Functions}
\author{Brenton Schulz}
\institute{University of Newcastle}
\date{\today}


\begin{document}
\titlepage

\begin{frame}
\frametitle{Subscript Notation}
\begin{itemize}
\item Last chance to learn that we use:
\begin{equation}
x_1, x_2, x_3, ... , x_n
\end{equation}
and
\begin{equation}
x_n = x_{n-1} + x_{n-2}
\end{equation}
notation because it is the simplest method that gets the point across.
\end{itemize}
\end{frame}

\begin{frame}[fragile]
\frametitle{Subscript Notation}
\begin{itemize}
\item $x_n$ means that $x$ is ``some number'' and $n$ is an \textit{integer} value
\item $n$ implies \textit{uniqueness} (ie: $x_1$ and $x_2$ can differ)
\item $n$ implies an \textit{order} to the $x$'s
\item A formal mathematical statement of the above would be something like:
\begin{equation}
x_n :~x \in \mathbb{R}~\textrm{and} ~ n \in \mathbb{Z}
\end{equation}
\item $\mathbb{R}$ is the set of real numbers
\item $\mathbb{Z}$ is the set of all integers
\end{itemize}
\end{frame}

\begin{frame}[fragile]
\frametitle{Subscript Notation}
\begin{itemize}
\item Without this notation it is \textit{really} hard to write things like:
\begin{equation}
x_n = x_{n-1} + x_{n-2}
\end{equation}
\item If you instead wrote:\\
``Calculate a sequence of numbers, $a,b,c,d,...$''\\
how would you write the equation?
\end{itemize}
\end{frame}

\begin{frame}[fragile]
\frametitle{FOR Loops in C}
\begin{itemize}
\item The C FOR loop syntax is:
\begin{lstlisting}[style=CStyle]
for( initial ; condition ; increment ) {
	// Loop block
}
\end{lstlisting}
\item Where:
	\begin{itemize}
		\item \texttt{initial} is a statement executed \textit{once}
		\item \texttt{condition} is a statement executed and tested \textit{before} every loop iteration
		\item \texttt{increment} is a statement executed \textit{after} every loop iteration, but \textit{before} the \texttt{condition} is tested
	\end{itemize}
\end{itemize}
\end{frame}

\begin{frame}[fragile]
\frametitle{FOR Loops in C}
\begin{lstlisting}[style=CStyle]
for( x = 0 ; x < 10 ; x++ ) {
	printf("%d\n", x);
}
\end{lstlisting}
\begin{itemize}
\item Run this code
\item Observe that:
	\begin{itemize}
		\item 0 is printed
		\item 10 is \textbf{not} printed
		\item \texttt{x} increments automatically
	\end{itemize}

\end{itemize}
\end{frame}


\begin{frame}[fragile]
\frametitle{FOR Example 1 - Factorials}
\begin{itemize}
\item Use FOR to count from 2 to our input number
\item Keep a running product as we go
\begin{lstlisting}[style=pseudo]
BEGIN
	INPUT x
	result = 1
	FOR k = 2 TO x
		result = result * k
	ENDFOR
END
\end{lstlisting}
\item Is this algorithm robust? What happens if:
	\begin{itemize}
		\item x = -1
		\item x = 1
		\item x = 0 (\textbf{NB:} 0! = 1 because \textit{maths})
	\end{itemize}
\end{itemize}
\end{frame}

\begin{frame}
\frametitle{BREAK Statements}
\begin{itemize}
\item Sometimes you want to exit a loop \textit{before} the condition is re-tested
\item The flow-control mechanism for this is a BREAK statement
\item If executed, the loop quits
\item BREAKs typically go inside an IF
\item It adds an extra condition on loop exit placed at any point in the loop
\end{itemize}
\end{frame}

\begin{frame}[fragile]
\frametitle{FOR Example 2}
\begin{itemize}
\item Two equivalent ways to implement the $\cos()$ series from before are:
\end{itemize}
{\small\textbf{NB:} $|$\texttt{tmp}$|$ means ``absolute value of tmp''.}
\begin{multicols}{2}
\begin{lstlisting}[style=pseudo,mathescape=true,basicstyle=\ttfamily\scriptsize]
BEGIN
	INPUT x
	sum = 0
	FOR k = 0 to 10
		tmp = $\frac{(-1)^k x^{2k}}{(2k)!}$
		sum = sum + tmp
		IF |tmp| < 1e-6
			BREAK
		ENDIF
	ENDWHILE 
END
\end{lstlisting}
\columnbreak
\begin{lstlisting}[style=pseudo,mathescape=true,basicstyle=\ttfamily\scriptsize]
BEGIN
	INPUT x
	tmp = 1
	k = 0
	sum = 0
	WHILE (k<10)AND(|tmp|>1e-6)
		tmp = $\frac{(-1)^k x^{2k}}{(2k)!}$
		sum = sum + tmp
		k = k + 1
	ENDWHILE 
END
\end{lstlisting}

\end{multicols}
\end{frame}

\begin{frame}
\frametitle{\texttt{break} Statements}
\begin{itemize}
\item The example is mildly pointless
	\begin{itemize}
		\item In C, the $\left| tmp \right| < 1e-6$ condition can go in the \texttt{for()} statement. In pseudocode it \textit{sort of} can't.
	\end{itemize}
\item It is there to illustrate what \texttt{break} does, not explain how to use it
\item As the ``experienced engineer' that's up to you
\end{itemize}
\end{frame}


\begin{frame}[fragile]
\frametitle{FOR Loops in C (Advanced)}
\begin{itemize}
\item \texttt{for()} syntax allows multiple expressions in the \texttt{inital} / \texttt{condition} /\texttt{increment} sections
\item Separate expressions with commas
\item eg:
\begin{lstlisting}[style=CStyle]
int x, y=10;
for( x = 0 ; x < 10 ; x++, y++ ) {
	printf("x: %d y: %d\n", x, y);
}
\end{lstlisting}
\item This increments both \texttt{x} and \texttt{y} but only \texttt{x} is used in the condition
\end{itemize}
\end{frame}

\begin{frame}[fragile]
\frametitle{Loop \texttt{continue} Statements}
\begin{itemize}
	\item A \texttt{continue} causes execution to jump back to the loop start
	\item The \textit{condition} is tested before reentry	
	\item eg, run this in the Che debugger:
	\begin{lstlisting}[style=CStyle]
int x;
for(x = 0; x < 10; x++) {
	if(x%2 == 0)
		continue;
	printf("%d is odd\n");
}
\end{lstlisting}
\item {\small(Not the best example but gets the point across)}
\end{itemize}
\end{frame}

\begin{frame}
\frametitle{\texttt{break} and \texttt{continue}}
\begin{itemize}
\item Some programmers claim that \texttt{break} and \texttt{continue} are ``naughty''
\item Well, yes, but actually no
\item They \textit{can} make your code needlessly complicated
\item They might make it simpler
\item It is up to you to judge
\item As engineers you shouldn't follow strict rules
\item Always try to choose the best tool for the job
\end{itemize}
\end{frame}

\begin{frame}
\frametitle{GOTO}
\begin{itemize}
\item There exists a GOTO flow control mechanism
	\begin{itemize}
		\item Sometimes also called a \textit{branch}
	\end{itemize}
\item It ``jumps'' from one line to a different line
	\begin{itemize}
		\item An ability some consider to be unnatural
	\end{itemize}
\item It exists for a purpose
\item That purpose does not (typically) exist when writing C code
	\begin{itemize}
		\item C \textit{supports} a \texttt{goto} statement
		\item It results in ``spaghetti code'' which is hard to read
		\item Don't use it in ENGG1003
	\end{itemize}
\item You \textit{must} use branch instructions in ELEC1710
\end{itemize}
\end{frame}

\begin{frame}[fragile]
\frametitle{Loose End: Increment Example}
\begin{lstlisting}[style=CStyle]
#include <stdio.h>
int main() {
	int x = 0;
	int y = 0;
	int z = 0;
	y = ++x + 10;
	printf("Pre-increment: %d\n", y);
	y = z++ + 10;
	printf("Post-increment: %d\n", y);
	return 0;
}
\end{lstlisting}
Pre/post-inc/decrements have many applications, more details in coming weeks.
\end{frame}

\begin{frame}
\frametitle{Binary Nomenclature}
\begin{itemize}
\item A data type's value range is a result of the underlying binary storage mechanism
\item A single binary digit is called a \textit{bit}
\item There are 8 bits in a \textit{byte}
\item In programming we use the ``power of two'' definitions of kB, MB, etc:
	\begin{itemize}
		\item 1 kilobyte is $2^{10} = 1024$ bytes
		\item 1 Megabyte is $2^{20} = 1048576$ bytes
		\item 1 Gigabyte is $2^{30} = 1073741824$ bytes
		\item (Advanced) These numbers look better in hex: \texttt{0x3FF}, \texttt{0xFFFFF}, etc.
	\end{itemize}
\end{itemize}
\end{frame}

\begin{frame}
\frametitle{Binary Nomenclature}
\begin{itemize}
\item Observe that kilobyte, Megabyte, Gigabyte, etc use scientific prefixes
\item These \textit{normally} mean a power of 10:
	\begin{itemize}
		\item kilo- = $10^3$
		\item Mega- = $10^6$
		\item Giga- = $10^9$
		\item ...etc (see the inside cover of a physics text)
	\end{itemize}
\item Computer science stole these terms and re-defined them

\end{itemize}
\end{frame}

\begin{frame}
\frametitle{Binary Nomenclature}
\begin{itemize}
\item This has made some people \textit{illogically angry}
\item Instead, we can use a more modern standard:
	\begin{itemize}
		\item $2^{10}$ bytes = 1 kibiByte (KiB)
		\item $2^{20}$ bytes = 1 Mebibyte (MiB)
		\item $2^{30}$ bytes = 1 Gibibyte (GiB)
		\item ...etc
	\end{itemize}
\item Generally speaking, KB (etc) implies:
	\begin{itemize}
		\item powers of two to \textit{engineers}
		\item powers of ten to \textit{marketing}
			\begin{itemize}
				\item The number is smaller
				\item Hard drive manufacturers, ISPs, etc like this
			\end{itemize}
	\end{itemize}
\end{itemize}
\end{frame}

\begin{frame}
\frametitle{Unambiguous Integer Data Types}
\begin{itemize}
\item Because the standard \texttt{int} and \texttt{long} data types don't have fixed size unambiguous types exist
\item Under OnlineGDB (ie: Linux with \texttt{gcc}) these are defined in \texttt{stdint.h} (\texttt{\#include} it)
\item You will see them used commonly in embedded systems programming (eg: Arduino code)
\item The types are:
	\begin{itemize}
		\item \texttt{int8\_t}
		\item \texttt{uint8\_t}
		\item \texttt{int16\_t}
		\item ...etc
	\end{itemize}
\end{itemize}
\end{frame}

\begin{frame}[fragile]
\frametitle{Code Blocks in C}
\begin{itemize}
\item Semi-revision:
\item The curly braces \{ \} encompass a \textit{block}
\item You have used these with \texttt{if()} and \texttt{while()}
\item They define the set of lines executed inside the \texttt{if()} or \texttt{while()}

\end{itemize}
\end{frame}

\begin{frame}[fragile]
\frametitle{Code Blocks in C}
\begin{itemize}
\item You can place blocks anywhere you like
\item Nothing wrong with:
\begin{lstlisting}[style=CStyle]
int main() {
	int x;
	{
		printf("%d\n", x);
	}
	return 0;
}
\end{lstlisting}
\item This just places the \texttt{printf();} inside a block
\item It doesn't do anything useful, but...
\end{itemize}
\end{frame}

\begin{frame}[fragile]
\frametitle{Variable Scope}
\begin{itemize}
\item A variable's ``existence'' is limited to the block where it is declared
	\begin{itemize}
		\item Plus any blocks within that one
	\end{itemize}
\item Example this code won't compile:
\begin{lstlisting}[style=CStyle]
#include <stdio.h>
int main() {
	int x = 2;
	if(x == 2) {
		int k;
		k = 2*x;
	}
	printf("%d\n", k);
	return 0;
}
\end{lstlisting}
\end{itemize}
\end{frame}

\begin{frame}
\frametitle{Variable Scope}
\begin{itemize}
\item Note that \texttt{k} was declared inside the \texttt{if()}
\item That means that it no longer exists when the \texttt{if()} has finished
\item This generates a compiler error
\item It frees up some RAM
\item It also lets the variable's name be reused elsewhere
	\begin{itemize}
		\item This can be \textit{really} confusing. Be careful.
	\end{itemize}
\end{itemize}
\end{frame}

\begin{frame}
\frametitle{Functions}
\begin{itemize}
\item A \textit{function} is a block of code which can be \textit{called} multiple times, from multiple places
\item They are used when you want the same block of lines to execute in many places throughout your code
\item A function requires:
	\begin{itemize}
		\item A name
		\item (optional) A \textit{return value}
		\item (optional) One or more \textit{arguments}
	\end{itemize}
\end{itemize}
\end{frame}

\begin{frame}
\frametitle{Functions in Mathematics}
\begin{itemize}
\item In mathematics you saw functions written as:
\begin{equation*}
y = f(x)
\end{equation*}
\item Here, the function is called $f$, takes an argument of $x$ and returns a value which is given to $y$
\item C and pure mathematics have these general ideas in common
\end{itemize}
\end{frame}

\begin{frame}
\frametitle{Function Examples}
\begin{itemize}
\item So far, some of you have used \textit{library functions}
\item These are functions which are pre-existing within the compiler (and its libraries)
\item I have shown you:
	\begin{itemize}
		\item \texttt{scanf();}
		\item \texttt{printf();}
		\item \texttt{rand();}
	\end{itemize}
\end{itemize}
\end{frame}

\begin{frame}[fragile]
\frametitle{Function Syntax}
\begin{itemize}
\item Writing \texttt{rand();} in you code is \textit{calling} the function
\item The program execution ``jumps'' into the function's code, executes it, then jumps back
\item General function call syntax is:
\begin{lstlisting}[style=pseudo]
return value = name(argument1, argument2,...);
\end{lstlisting}
\item Not all functions take arguments
\item You may ignore the return value
\end{itemize}
\end{frame}

\begin{frame}[fragile]
\frametitle{Function Examples}
\begin{itemize}
\item Example 1:
\begin{lstlisting}[style=CStyle]
x = rand();
\end{lstlisting}
	\begin{itemize}
		\item \texttt{rand} is the function name
		\item It returns a ``random'' number
		\item The return value is allocated to \texttt{x}
		\item It doesn't take an argument 	
	\end{itemize}
\pause
\item Example 2:
\begin{lstlisting}[style=CStyle]
y = sqrtf(x);
\end{lstlisting}
	\begin{itemize}
		\item \texttt{sqrtf} is the function name
		\item \texttt{x} is the argument
		\item It returns the square root of \texttt{x}
		\item The return value is allocated to \texttt{y}
	\end{itemize}
\end{itemize}
\end{frame}

\begin{frame}
\frametitle{Functions}
\begin{itemize}
\item Function arguments and return values have pre-defined data types
\pause
\item Example: \texttt{int rand(void);}
	\begin{itemize}
		\item The return value is an \texttt{int}
		\item The argument is ``type'' \texttt{void}
			\begin{itemize}
				\item This just means there aren't any
			\end{itemize}
	\end{itemize}
\pause
\item Example: \texttt{float sqrtf(float x);}
	\begin{itemize}
		\item The return value is a \texttt{float}
		\item The argument is a \texttt{float}
			\begin{itemize}
				\item It is called \texttt{x} in documentation but that is irrelevant
			\end{itemize}
	\end{itemize}
\end{itemize}
\end{frame}

\begin{frame}
\frametitle{Using Functions}
\begin{itemize}
\item (semi-revision)
\item Before you use a function you must:
	\begin{itemize}
		\item Read the documentation
		\item \texttt{\#include} the correct header file
		\item Add the correct library to the compiler options
			\begin{itemize}
				\item In Che I've done this for the maths library
				\item \texttt{stdio} and \texttt{stdlib} are always there
			\end{itemize}
		\item Be aware of the return value and argument data types
			\begin{itemize}
				\item Do you need any type casting?
				\item Are you using the correct function?
			\end{itemize}
	\end{itemize}
\end{itemize}
\end{frame}

\begin{frame}
\frametitle{Maths Functions}
\begin{itemize}
\item Since some of you have already used them, lets learn about the maths library...
\item It includes functions for:
	\begin{itemize}
		\item Trigonometry
		\item Exponentials (base e) \& logarithms (base e and 10)
		\item Exponents
		\item Rounding (\texttt{floor();} \& \texttt{ceil();})
		\item Floating point modulus (\texttt{fmod();}
		\item Square roots
		\item Maybe others?
	\end{itemize}
\end{itemize}
\end{frame}

\begin{frame}
\frametitle{Maths Functions}
\begin{itemize}
\item There are typically \textit{different} functions for \texttt{float} and \texttt{double}
\item This can have a huge speed impact
\item Get in the habit of using the right ones!
\item \texttt{float} maths functions typically end in '\texttt{f}'
	\begin{itemize}
		\item \texttt{cosf();}
		\item \texttt{sqrtf();}
		\item \texttt{atanf();}
		\item ...etc
	\end{itemize}
\item \texttt{double} maths functions don't end in '\texttt{f}'
	\begin{itemize}
		\item \texttt{cos();}
		\item \texttt{log();}
	\end{itemize}
\end{itemize}
\end{frame}

\end{document}
