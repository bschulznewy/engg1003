\documentclass[14pt]{beamer}
\usetheme{Dresden}
\usecolortheme{orchid}

\usepackage{xcolor}
\usepackage{listings}
\usepackage{courier}
\usepackage{graphicx}
\usepackage{amsmath}
\usepackage{algorithm2e}
\usepackage{multicol}
\usepackage{amssymb}

\usefonttheme[onlymath]{serif}

\definecolor{mGreen}{rgb}{0,0.6,0}
\definecolor{mGray}{rgb}{0.5,0.5,0.5}
\definecolor{mPurple}{rgb}{0.8,0,0.82}
\definecolor{backgroundColour}{rgb}{0.95,0.95,0.92}
\definecolor{lightBlue}{rgb}{0.1, 0.1, 0.8}

\lstdefinestyle{CStyle}{
    backgroundcolor=\color{backgroundColour},   
    commentstyle=\color{mGreen},
    keywordstyle=\color{magenta},
    numberstyle=\tiny\color{mGray},
    stringstyle=\color{mPurple},
    basicstyle=\footnotesize\ttfamily,
    breakatwhitespace=false,         
    breaklines=true,                 
    captionpos=b,                    
    keepspaces=true,                 
    numbers=left,                    
    numbersep=5pt,                  
    showspaces=false,                
    showstringspaces=false,
    showtabs=false,                  
    tabsize=2,
    language=C
}

\lstdefinestyle{Ctable}{
    backgroundcolor=\color{backgroundColour},   
    commentstyle=\color{mGreen},
    keywordstyle=\color{magenta},
    numberstyle=\tiny\color{mGray},
    stringstyle=\color{mPurple},
    basicstyle=\footnotesize\ttfamily,
    breakatwhitespace=false,         
    breaklines=true,                 
    captionpos=b,                    
    keepspaces=true,                                  
    showspaces=false,                
    showstringspaces=false,
    showtabs=false,                  
    tabsize=2,
    language=C
}

\lstdefinestyle{pseudo}{
        basicstyle=\ttfamily\footnotesize,
        keywordstyle=\color{lightBlue},
        morekeywords={BEGIN,END,IF,ELSE,ENDIF,ELSEIF,PRINT,WHILE,RETURN,ENDWHILE,DO,FOR,TO,IN,ENDFOR,BREAK,INPUT,READ},
        morecomment=[l]{//},
        commentstyle=\color{mGreen}
}

\lstset{basicstyle=\footnotesize\ttfamily,breaklines=true}
\lstset{framextopmargin=50pt,tabsize=2}

\title{ENGG1003 - Tuesday Week 4}
\subtitle{Loose Ends \\ Functions}
\author{Brenton Schulz}
\institute{University of Newcastle}
\date{\today}


\begin{document}
\titlepage

\begin{frame}
\frametitle{Subscript Notation}
\begin{itemize}
\item Last chance to learn that we use:
\begin{equation}
x_1, x_2, x_3, ... , x_n
\end{equation}
and
\begin{equation}
x_n = x_{n-1} + x_{n-2}
\end{equation}
notation because it is the simplest method that gets the point across.
\end{itemize}
\end{frame}

\begin{frame}[fragile]
\frametitle{Subscript Notation}
\begin{itemize}
\item $x_n$ means that $x$ is ``some number'' and $n$ is an \textit{integer} value
\item $n$ implies \textit{uniqueness} (ie: $x_1$ and $x_2$ can differ)
\item $n$ implies an \textit{order} to the $x$'s
\item A formal mathematical statement of the above would be something like:
\begin{equation}
x_n :~x \in \mathbb{R}~\textrm{and} ~ n \in \mathbb{Z}
\end{equation}
\item $\mathbb{R}$ is the set of real numbers
\item $\mathbb{Z}$ is the set of all integers
\end{itemize}
\end{frame}

\begin{frame}[fragile]
\frametitle{Subscript Notation}
\begin{itemize}
\item Without this notation it is \textit{really} hard to write things like:
\begin{equation}
x_n = x_{n-1} + x_{n-2}
\end{equation}
\pause
\item If you instead wrote:\\
``Calculate a sequence of numbers, $a,b,c,d,...$''\\
how would you write the equation?
\end{itemize}
\end{frame}

\begin{frame}
\frametitle{Considering Dropping?}
\begin{itemize}
\item HECS census is Fri 22nd
\item Before you drop:
	\begin{itemize}
		\item Talk to me
		\item Are you \textit{legitimately} unprepared or experiencing ``imposter syndrome''?
			\begin{itemize}
				\item It is surprisingly common
			\end{itemize}
		\item Most of you have to pass eventually
		\item There are some legitimate reasons
	\end{itemize}
\item Ignore unsolicited advice from demonstrators
	\begin{itemize}
		\item Seriously, this isn't their job
	\end{itemize}
\end{itemize}
\end{frame}

\begin{frame}[fragile]
\frametitle{FOR Loops in C}
\begin{itemize}
\item The C FOR loop syntax is:
\begin{lstlisting}[style=CStyle]
for( initial ; condition ; increment ) {
	// Loop block
}
\end{lstlisting}
\item Where:
	\begin{itemize}
		\item \texttt{initial} is a statement executed \textit{once}
		\item \texttt{condition} is a statement executed and tested \textit{before} every loop iteration
		\item \texttt{increment} is a statement executed \textit{after} every loop iteration, but \textit{before} the \texttt{condition} is tested
	\end{itemize}
\end{itemize}
\end{frame}

\begin{frame}[fragile]
\frametitle{FOR Loops in C}
\begin{lstlisting}[style=CStyle]
for( x = 0 ; x < 10 ; x++ ) {
	printf("%d\n", x);
}
\end{lstlisting}
\begin{itemize}
\item Run this code
\item Observe that:
	\begin{itemize}
		\item 0 is printed
		\item 10 is \textbf{not} printed
		\item \texttt{x} increments automatically
	\end{itemize}

\end{itemize}
\end{frame}


\begin{frame}[fragile]
\frametitle{FOR Example 1 - Factorials}
\begin{itemize}
\item Use FOR to count from 2 to our input number
\item Keep a running product as we go
\begin{lstlisting}[style=pseudo]
BEGIN
	INPUT x
	result = 1
	FOR k = 2 TO x
		result = result * k
	ENDFOR
END
\end{lstlisting}
\item Is this algorithm robust? What happens if:
	\begin{itemize}
		\item x = -1
		\item x = 1
		\item x = 0 (\textbf{NB:} 0! = 1 because \textit{maths})
	\end{itemize}
\end{itemize}
\end{frame}

\begin{frame}
\frametitle{BREAK Statements}
\begin{itemize}
\item Sometimes you want to exit flow control early
\item The flow-control mechanism for this is a BREAK statement
\item If executed, execution jumps outside the current \texttt{if()}, \texttt{while()}, \texttt{for()}, etc
\item BREAKs \textit{typically} go inside an IF
\end{itemize}
\end{frame}

\begin{frame}[fragile]
\frametitle{FOR Example 2}
\begin{itemize}
\item Two equivalent ways to implement the $\cos()$ series from before are:
\end{itemize}
{\small\textbf{NB:} $|$\texttt{tmp}$|$ means ``absolute value of tmp''.}
\begin{multicols}{2}
\begin{lstlisting}[style=pseudo,mathescape=true,basicstyle=\ttfamily\scriptsize]
BEGIN
	INPUT x
	sum = 0
	FOR k = 0 to 10
		tmp = $\frac{(-1)^k x^{2k}}{(2k)!}$
		sum = sum + tmp
		IF |tmp| < 1e-6
			BREAK
		ENDIF
	ENDWHILE 
END
\end{lstlisting}
\columnbreak
\begin{lstlisting}[style=pseudo,mathescape=true,basicstyle=\ttfamily\scriptsize]
BEGIN
	INPUT x
	tmp = 1
	k = 0
	sum = 0
	WHILE (k<10)AND(|tmp|>1e-6)
		tmp = $\frac{(-1)^k x^{2k}}{(2k)!}$
		sum = sum + tmp
		k = k + 1
	ENDWHILE 
END
\end{lstlisting}

\end{multicols}
\end{frame}

\begin{frame}
\frametitle{\texttt{break} Statements}
\begin{itemize}
\item The example is mildly pointless
	\begin{itemize}
		\item In C, the $\left| tmp \right| < 1e-6$ condition can go in the \texttt{for()} statement. In pseudocode it \textit{sort of} can't.
	\end{itemize}
\item It is there to illustrate what \texttt{break} does, not explain how to use it
\item As the ``experienced engineer' you choose when to use it
\end{itemize}
\end{frame}


\begin{frame}[fragile]
\frametitle{FOR Loops in C (Advanced)}
\begin{itemize}
\item \texttt{for()} syntax allows multiple expressions in the \texttt{inital} / \texttt{condition} /\texttt{increment} sections
\item Separate expressions with commas
\item eg:
\begin{lstlisting}[style=CStyle]
int x, y=10;
for( x = 0 ; x < 10 ; x++, y++ ) {
	printf("x: %d y: %d\n", x, y);
}
\end{lstlisting}
\item This increments both \texttt{x} and \texttt{y} but only \texttt{x} is used in the condition
\end{itemize}
\end{frame}

\begin{frame}[fragile]
\frametitle{Loop \texttt{continue} Statements}
\begin{itemize}
	\item A \texttt{continue} causes execution to jump back to the loop start
	\item The \textit{condition} is tested before reentry	
	\item eg, run this in the Che debugger:
	\begin{lstlisting}[style=CStyle]
int x;
for(x = 0; x < 10; x++) {
	if(x%2 == 0)
		continue;
	printf("%d is odd\n");
}
\end{lstlisting}
\item {\small(Not the best example but gets the point across)}
\end{itemize}
\end{frame}

\begin{frame}
\frametitle{\texttt{break} and \texttt{continue}}
\begin{itemize}
\item Some programmers claim that \texttt{break} and \texttt{continue} are ``naughty''
\item Well, yes, but actually no
\item They \textit{can} make your code needlessly complicated
\item They might make it simpler
\item It is up to you to judge
\item As engineers you shouldn't follow strict rules
\item Always try to choose the best tool for the job
\end{itemize}
\end{frame}

\begin{frame}
\frametitle{GOTO}
\begin{itemize}
\item There exists a GOTO flow control mechanism
	\begin{itemize}
		\item Sometimes also called a \textit{branch}
	\end{itemize}
\item It ``jumps'' from one line to a different line
	\begin{itemize}
		\item An ability some consider to be unnatural
	\end{itemize}
\item It exists for a purpose
\item That purpose does not (typically) exist when writing C code
	\begin{itemize}
		\item C \textit{supports} a \texttt{goto} statement
		\item It results in ``spaghetti code'' which is hard to read
		\item Don't use it in ENGG1003
	\end{itemize}
\item You \textit{must} use branch instructions in ELEC1710
\end{itemize}
\end{frame}

\begin{frame}[fragile]
\frametitle{\texttt{switch()} - \texttt{case:}}
\begin{itemize}
\item Sometimes you want to code something like:
\begin{lstlisting}[style=CStyle]
if(x == 0) {
	// stuff
} else if(x == 1) {
	// stuff
} else if(x == 2) {
	// stuff
} ...etc
\end{lstlisting}
\item This is difficult to read and gets unwieldy. Fast.
\end{itemize}
\end{frame}

\begin{frame}[fragile]
\frametitle{\texttt{switch()} - \texttt{case:}}
\begin{itemize}
\item Instead, C has:
\begin{lstlisting}[style=CStyle]
switch(expression) {
	case constant:
			break;
	case constant:
			break;
	default:
}
\end{lstlisting}
\item The \textit{expression} is anything which evaluates to a number
\item The \textit{constant}s are either literals or variables declared as \texttt{const} (covered later)
\end{itemize}
\end{frame}

\begin{frame}[fragile]
\frametitle{\texttt{switch()} - \texttt{case:} Example}
\begin{lstlisting}[style=CStyle]
int x=1, y=2;

switch(x==y) { // Evaluates to 0 or 1
	case 0:
			printf("x and y differ\n");
			break;
	case 1:
			printf("x and y are equal\n");
			break;
	default:
			printf("Something went very wrong\n");
}
\end{lstlisting}
\begin{itemize}
\item The \texttt{default:} case happens if the expression doesn't match any other option
\end{itemize}
\end{frame}

\begin{frame}[fragile]
\frametitle{\texttt{switch()} - \texttt{case:} Example}
\begin{itemize}
\item If the \texttt{break;} is omitted execution continues line by line - run this in Che
\end{itemize}
\begin{lstlisting}[style=CStyle]
#include<stdio.h>
int main() {
	int x = 2;
	switch(x) {
		case 1: printf("x is 1\n");
		case 2: printf("x is 2\n");
		case 3: printf("x is 3\n");
		default: printf("x is not 1, 2, or 3\n");
	}
	return 0;
}
\end{lstlisting}
\end{frame}

\begin{frame}[fragile]
\frametitle{\texttt{switch()} - \texttt{case:} Limits}
\begin{itemize}
\item Because the \texttt{case} statements only accept \textit{constants} there are some limitations
\item Example, this doesn't translate well:
\begin{lstlisting}[style=CStyle]
if(x < 0) {
	// stuff
} else if (x == 0) {
	// stuff
} else if (x > 0) {
	// stuff
}
\end{lstlisting}
\item \texttt{(x<0)}, \texttt{(x==0)}, and \texttt{(x>0)} are all 0 or 1
\item Can't \textit{easily} translate this into three unique constants
\end{itemize}
\end{frame}

\begin{frame}[fragile]
\frametitle{Loose End: Increment Example}
\begin{lstlisting}[style=CStyle]
#include <stdio.h>
int main() {
	int x = 0;
	int y = 0;
	int z = 0;
	y = ++x + 10;
	printf("Pre-increment: %d\n", y);
	y = z++ + 10;
	printf("Post-increment: %d\n", y);
	return 0;
}
\end{lstlisting}
Pre/post-inc/decrements have many applications, more details in coming weeks.
\end{frame}

\begin{frame}
\frametitle{Binary Nomenclature}
\begin{itemize}
\item A data type's value range is a result of the underlying binary storage mechanism
\item A single binary digit is called a \textit{bit}
\item There are 8 bits in a \textit{byte}
\item In programming we use the ``power of two'' definitions of kB, MB, etc:
	\begin{itemize}
		\item 1 kilobyte is $2^{10} = 1024$ bytes
		\item 1 Megabyte is $2^{20} = 1048576$ bytes
		\item 1 Gigabyte is $2^{30} = 1073741824$ bytes
		\item (Advanced) These numbers look better in hex: \texttt{0x3FF}, \texttt{0xFFFFF}, etc.
	\end{itemize}
\end{itemize}
\end{frame}

\begin{frame}
\frametitle{Binary Nomenclature}
\begin{itemize}
\item Observe that kilobyte, Megabyte, Gigabyte, etc use scientific prefixes
\item These \textit{normally} mean a power of 10:
	\begin{itemize}
		\item kilo- = $10^3$
		\item Mega- = $10^6$
		\item Giga- = $10^9$
		\item ...etc (see the inside cover of a physics text)
	\end{itemize}
\item Computer science adopted these terms and re-defined them

\end{itemize}
\end{frame}

\begin{frame}
\frametitle{Binary Nomenclature}
\begin{itemize}
\item This has made some people \textit{illogically angry}
\item Instead, we can use a more modern standard:
	\begin{itemize}
		\item $2^{10}$ bytes = 1 kibiByte (KiB)
		\item $2^{20}$ bytes = 1 Mebibyte (MiB)
		\item $2^{30}$ bytes = 1 Gibibyte (GiB)
		\item ...etc
	\end{itemize}
\item Generally speaking, KB (etc) implies:
	\begin{itemize}
		\item powers of two to \textit{engineers}
		\item powers of ten to \textit{marketing}
			\begin{itemize}
				\item The number is smaller
				\item Hard drive manufacturers, ISPs, etc like this
			\end{itemize}
	\end{itemize}
\end{itemize}
\end{frame}

\begin{frame}
\frametitle{Unambiguous Integer Data Types}
\begin{itemize}
\item Because the standard \texttt{int} and \texttt{long} data types don't have fixed size unambiguous types exist
\item Under OnlineGDB (ie: Linux with \texttt{gcc}) these are defined in \texttt{stdint.h} (\texttt{\#include} it)
\item You will see them used commonly in embedded systems programming (eg: Arduino code)
\item The types are:
	\begin{itemize}
		\item \texttt{int8\_t}
		\item \texttt{uint8\_t}
		\item \texttt{int16\_t}
		\item ...etc
	\end{itemize}
\end{itemize}
\end{frame}

\begin{frame}[fragile]
\frametitle{Code Blocks in C}
\begin{itemize}
\item Semi-revision:
\item The curly braces \{ \} encompass a \textit{block}
\item You have used these with \texttt{if()} and \texttt{while()}
\item They define the set of lines executed inside the \texttt{if()} or \texttt{while()}

\end{itemize}
\end{frame}

\begin{frame}[fragile]
\frametitle{Code Blocks in C}
\begin{itemize}
\item You can place blocks anywhere you like
\item Nothing wrong with:
\begin{lstlisting}[style=CStyle]
int main() {
	int x;
	{
		printf("%d\n", x);
	}
	return 0;
}
\end{lstlisting}
\item This just places the \texttt{printf();} inside a block
\item It doesn't do anything useful, but...
\end{itemize}
\end{frame}

\begin{frame}[fragile]
\frametitle{Variable Scope}
\begin{itemize}
\item A variable's ``existence'' is limited to the block where it is declared
	\begin{itemize}
		\item Plus any blocks within that one
	\end{itemize}
\item Example this code won't compile:
\begin{lstlisting}[style=CStyle]
#include <stdio.h>
int main() {
	int x = 2;
	if(x == 2) {
		int k;
		k = 2*x;
	}
	printf("%d\n", k);
	return 0;
}
\end{lstlisting}
\end{itemize}
\end{frame}

\begin{frame}
\frametitle{Variable Scope}
\begin{itemize}
\item Note that \texttt{k} was declared inside the \texttt{if()}
\item That means that it no longer exists when the \texttt{if()} has finished
\item This generates a compiler error
\item It frees up some RAM
\item It also lets the variable's name be reused elsewhere
	\begin{itemize}
		\item This can be \textit{really} confusing. Be careful.
	\end{itemize}
\end{itemize}
\end{frame}

\begin{frame}
\frametitle{Functions}
\begin{itemize}
\item A \textit{function} is a block of code which can be \textit{called} multiple times, from multiple places
\item They are used when you want the same block of code to execute in many places throughout your code
\item A function requires:
	\begin{itemize}
		\item A name
		\item (optional) A \textit{return value}
		\item (optional) One or more \textit{arguments}
	\end{itemize}
\end{itemize}
\end{frame}

\begin{frame}
\frametitle{Functions in Mathematics}
\begin{itemize}
\item In mathematics you saw functions written as:
\begin{equation*}
y = f(x)
\end{equation*}
\item Here, the function is called $f$, takes an argument of $x$ and returns a value which is given to $y$
\item C and pure mathematics have these general ideas in common
\pause
\item The similarities stop there
\end{itemize}
\end{frame}

\begin{frame}
\frametitle{Function Examples}
\begin{itemize}
\item So far, some of you have used \textit{library functions}
\item These are functions which are pre-existing within the compiler (and its libraries)
\item I have shown you:
	\begin{itemize}
		\item \texttt{scanf();}
		\item \texttt{printf();}
		\item \texttt{rand();}
	\end{itemize}
\end{itemize}
\end{frame}

\begin{frame}[fragile]
\frametitle{Function Syntax}
\begin{itemize}
\item Writing \texttt{rand();} in you code is \textit{calling} the function
\item The program execution ``jumps'' into the function's code, executes it, then jumps back
\item Function call syntax is:\\
{\small \texttt{name([arguments])} }
\item Not all functions take arguments
\item The function can ``turn into'' its \textit{return value}
\end{itemize}
\end{frame}

\begin{frame}[fragile]
\frametitle{Function Examples}
\begin{itemize}
\item Example 1:
\begin{lstlisting}[style=CStyle]
x = rand();
\end{lstlisting}
	\begin{itemize}
		\item \texttt{rand} is the function name
		\item It returns a ``random'' integer
		\item The return value is allocated to \texttt{x}
		\item It doesn't take an argument 	
	\end{itemize}
\pause
\item Example 2:
\begin{lstlisting}[style=CStyle]
y = sqrtf(x);
\end{lstlisting}
	\begin{itemize}
		\item \texttt{sqrtf} is the function name
		\item \texttt{x} is the argument
		\item It returns the square root of \texttt{x}
		\item The return value is allocated to \texttt{y}
	\end{itemize}
\end{itemize}
\end{frame}

\begin{frame}
\frametitle{Functions}
\begin{itemize}
\item Function arguments and return values have pre-defined data types
\pause
\item Example from documentation
\begin{itemize}
\item \texttt{int~rand(void);}
	\begin{itemize}
		\item The return value is an \texttt{int}
		\item The argument is type \texttt{void}
		\item This just means there aren't any
	\end{itemize}
\pause
\item \texttt{float sqrtf(float x);}
	\begin{itemize}
		\item The return value is a \texttt{float}
		\item The argument is a \texttt{float}
		\item Argument is called \texttt{x} in documentation but you can pass it any \texttt{float}
	\end{itemize}
\end{itemize}
\end{itemize}
\end{frame}

\begin{frame}
\frametitle{Return Values (an Engineer's View)}
\begin{itemize}
\item The function's \textit{return value} is the number a function gets ``replaced with'' in a line of code
\pause
\item Function return values, variables, and literals can all be used in the same places:
	\begin{itemize}
		\item In arithmetic
		\item In conditions
		\item As arguments to other functions
	\end{itemize}	 
\pause
\item The C standard is \textit{very} specific about what return values are but I will be informal for now
	\begin{itemize}
		\item Technically, for example, an expression like \texttt{x=y+5.0;} also has a ``return value`` equal to the value allocated to \texttt{x}
	\end{itemize}
\end{itemize}
\end{frame}

\begin{frame}
\frametitle{Return Values}
\begin{itemize}
\item I use functions from \texttt{math.h} in these examples, we'll cover them in a few mins
\pause
\item The following are all valid:
	\begin{itemize}
		\item \texttt{x = rand();}
		\item \texttt{printf("\%f\textbackslash n", sin(y));}
		\item \texttt{if( (rand()\%6) < 2)}
		\item \texttt{while( sin(x) < 0 )}
		\pause
		\item This next one is complicated...
		\pause
		\item \texttt{x = sin((double)rand());}
			\begin{itemize}
			\pause
				\item Generates a random integer, casts to \texttt{double}, uses that number as an argument to the \texttt{sin()} function
			\end{itemize}
	\end{itemize}
\end{itemize}
\end{frame}

\begin{frame}
\frametitle{Using Functions}
\begin{itemize}
\item (semi-revision)
\item Before you use a function you must:
	\begin{itemize}
		\item Read the documentation
		\item \texttt{\#include} the correct header file
		\item Add the correct library to the compiler options
			\begin{itemize}
				\item In Che I've done this for the maths library
				\item \texttt{stdio} and \texttt{stdlib} are always there
			\end{itemize}
		\item Be aware of the data types
			\begin{itemize}
				\item Do you need any type casting?
				\item Are you using the correct function?
			\end{itemize}
	\end{itemize}
\end{itemize}
\end{frame}

\begin{frame}
\frametitle{Maths Functions}
\begin{itemize}
\item Since some of you have already used them, lets learn about the maths library...
\item It includes functions for:
	\begin{itemize}
		\item Trigonometry
		\item Exponentials (base e) \& logarithms (base e, 10, 2)
		\item Exponents (\texttt{pow();})
		\item Rounding (\texttt{floor();} \& \texttt{ceil();})
		\item Floating point modulus (\texttt{fmod();})
		\item Square roots
		\item ...etc
	\end{itemize}
\end{itemize}
\end{frame}

\begin{frame}
\frametitle{Maths Functions}
\begin{itemize}
\item There are typically \textit{different} functions for \texttt{float} and \texttt{double}
\item This can have a huge speed impact
\item Use the right ones!
\item \texttt{float} maths functions typically end in '\texttt{f}'
	\begin{itemize}
		\item \texttt{cosf();}
		\item \texttt{sqrtf();}
		\item \texttt{atanf();}
		\item ...etc
	\end{itemize}
\item \texttt{double} maths functions \textbf{don't}
	\begin{itemize}
		\item \texttt{cos();}
		\item \texttt{log();}
	\end{itemize}
\end{itemize}
\end{frame}

\begin{frame}
\frametitle{Maths Functions}
\begin{itemize}
\item Math functions are written by mathematicians
	\begin{itemize}
		\item All angles are in radians
		\pause
		\item \texttt{log();} is $\log_e$
			\begin{itemize}
				\item \texttt{log10();} is $\log_{10}$
				\item \texttt{log2();} is $\log_2$
			\end{itemize}
		\pause
		\item Inverse trig functions are called ``arcus functions''
			\begin{itemize}
				\item $\sin^{-1}$ is \texttt{asin();}
				\item $\cos^{-1}$ is \texttt{acos();}
				\item $\tan^{-1}$ is \texttt{atan();}
			\end{itemize}
		\pause
		\item The ``4 quadrant'' arctan function is \texttt{atan2();}
			\begin{itemize}
				\item \texttt{atan(x);} returns $[-\pi/2,\pi/2]$
				\item \texttt{atan2(x,y);} returns $[-\pi, \pi]$ depending on the quadrant of the point \texttt{x,y}
				\item Very useful for polar to Cartesian coordinate transforms (probably beyond 1st semester 1st year)
			\end{itemize}
	\end{itemize}
\end{itemize}
\end{frame}

\begin{frame}
\frametitle{Example - Quadratic Equation}
Write a C program which uses the standard library function \texttt{sqrtf();} as part of the calculations required to produce solutions to a quadratic equation:
\begin{equation}
ax^2 + bx + c = 0
\end{equation}
using
\begin{equation}
x = \frac{-b \pm \sqrt{b^2 - 4ac}}{2a}
\end{equation}
\end{frame}

\begin{frame}
\begin{center}
...do it live in Che
\end{center}
\end{frame}

\begin{frame}
\frametitle{Example - \texttt{scanf();}'s Return Value}
Read the \texttt{scanf();} \underline{\href{http://man7.org/linux/man-pages/man3/scanf.3.html}{documentation}} and observe that it returns an \texttt{int}. What does that \texttt{int} represent? Write some test code and experiment with its behaviour.
\\~\\
Demonstrate it live in Che
.\end{frame}

\begin{frame}
\frametitle{Writing Functions}
\begin{itemize}
\item What about writing your own functions?
\pause
\item Do the following:
	\begin{enumerate}
		\item Choose a name
		\pause
		\item Decide on the function arguments
		\pause
		\item Decide on the return value
		\pause
		\item Write a \textit{function prototype}
			\begin{itemize}
				\item Write it at the top of your code [or in a header file]
			\end{itemize}
		\pause
		\item Somewhere below \texttt{main()} (or in another \texttt{.c} file) write the function \textit{definition}
	\end{enumerate}
\pause
\item For now just keep everything in one file
	\begin{itemize}
		\item Unless you study ahead. I won't stop you.
	\end{itemize}
\end{itemize}
\end{frame}

\begin{frame}
\frametitle{Function Prototypes}
\begin{itemize}
\item Huh? What's a function prototype?
\pause
\item Before a function is \textit{called} the compiler needs to know:
	\begin{itemize}
		\item Its name
		\item Its argument's data type(s)
		\item Its return data type 
	\end{itemize}
\pause
\item A function prototype documents these things for the compiler
\end{itemize}
\end{frame}

\begin{frame}[fragile]
\frametitle{Function Prototypes}
\begin{itemize}
\item The function prototype syntax is:
\begin{lstlisting}[style=CStyle]
[return data type] function_name(arguments);
\end{lstlisting}
\pause
\item The \texttt{arguments} section is a comma separated list with the following syntax:
\begin{lstlisting}[style=CStyle]
(datatype name, datatype name, ...)
\end{lstlisting}
\item Examples:
	\begin{itemize}
		\item \texttt{float sqrtf(float x);}
		\item \texttt{int rand(void);}
		\item \texttt{double log(double x);}
		\item \texttt{double atan2(double x, double y);}
	\end{itemize}
\end{itemize}
\end{frame}

\begin{frame}
\frametitle{Void}
\begin{itemize}
\item If either the arguments or return value aren't required declare them as \texttt{void}
\item This is an explicit way of saying ``this item doesn't exist'' 
\end{itemize}
\end{frame}

\begin{frame}
\frametitle{Function Prototypes}
\begin{itemize}
\item The function prototype must be \textit{before} the function's first use
\item For ``small'' projects: above \texttt{main()}
\item For ``big'' projects: in their own \textit{header file}
	\begin{itemize}
		\item We'll cover this later
	\end{itemize}
\item Don't leave the prototype's arguments blank
	\begin{itemize}
		\item The compiler won't complain but it is a deprecated language feature
	\end{itemize}
\end{itemize}
\end{frame}

\begin{frame}[fragile]
\frametitle{Function Definitions}
\begin{itemize}
\item The function prototype tells the compiler how the function interacts with other code
\item The function definition is the \textit{actual code} that gets executed when the function is called
\begin{lstlisting}[style=CStyle]
int add(int a, int b); // Prototype

main() {
	// do stuff
}

int add(int a, int b) { // Definition
	return a + b;
}
\end{lstlisting}
\end{itemize}
\end{frame}

\begin{frame}
\frametitle{Function Prototypes Vs Definitions}
\begin{itemize}
\item For the time being:
\begin{itemize}
	\item The prototype goes \textit{above} \texttt{main()}
		\begin{itemize}
			\item It is 1 line and ends with a semicolon ;
		\end{itemize}
	\item The definition goes \textit{below} \texttt{main()}
		\begin{itemize}
			\item It is the prototype repeated followed by a \texttt{\{ \}} block
		\end{itemize}
\end{itemize}
\end{itemize}
\end{frame}

\begin{frame}[fragile]
\frametitle{Writing Functions - Example}
\begin{itemize}
\item Lets implement the Week 2 \texttt{sqrt} algorithm as a function
\item ...Then compare with \texttt{sqrtf();}
\item Keep it simple: fixed iteration count \texttt{n=10}

\end{itemize}
\end{frame}

\begin{frame}[fragile]
\frametitle{Writing Functions - Example}
\begin{itemize}
\item In mathematics, calculate $\sqrt{k}$ by iterating:
\begin{align*}
x_n &= \frac{1}{2}\left(x_{n-1} + \frac{k}{x_{n-1}}\right)\\
x_0 &\neq 0
\end{align*}
\item In a code snippet:
\begin{lstlisting}[style=CStyle]
// Calculate sqrt(k)
float k = 26; // Test value, sqrt(26)=5.0990
float xn = x/2.0; // x0 = x/2 because why not?
int n;
for(n = 0; n < 10; n++) {
	xn = 0.5*(xn + k/xn);
}
\end{lstlisting}
\end{itemize}
\end{frame}

\begin{frame}[fragile]
\frametitle{Writing Functions - Example}
\begin{itemize}
\item Lets make some design decisions:
	\begin{itemize}
		\item Name: \texttt{mySqrt();}
		\item Argument: \texttt{float k}
		\item Return Value: \texttt{float}
	\end{itemize}
\item The function prototype is therefore:
\begin{lstlisting}[style=CStyle]
float mySqrt(float k);
\end{lstlisting}
\end{itemize}
\end{frame}

\begin{frame}[fragile]
\frametitle{Writing Functions - Example}
\begin{itemize}
\item Place the function prototype before \texttt{main()}:
\begin{lstlisting}[style=CStyle]
#include <stdio.h>

float mySqrt(float k);

int main() {
	// Do stuff
}
\end{lstlisting}
\end{itemize}
\end{frame}

\begin{frame}[fragile]
\frametitle{Writing Functions - Example}
\begin{itemize}
\item Write the function definition below \texttt{main()}
\begin{lstlisting}[style=CStyle]
#include <stdio.h>
float mySqrt(float k);
int main() {
	printf("sqrt(26) = %f\n", mySqrt(26.0));
}

float mySqrt(float k) {
	int n;
	float xn = k/2.0;
	for(n = 0; n < 10; n++)
		xn = 0.5*(xn + k/xn);
	return xn;
}
\end{lstlisting}
\end{itemize}
\end{frame}

\begin{frame}
End of Tuesday lecture marker
\end{frame}

\begin{frame}[fragile]
\frametitle{Writing Functions - Example}

\begin{itemize}
\item Lets view a few common errors
\begin{lstlisting}[style=CStyle]
#include <stdio.h>
float mySqrt(float k);
int main() {
	printf("%f\n", mySqrt(26));
}
\end{lstlisting}
\item Results in:
\begin{lstlisting}[style=pseudo]
/tmp/ccT6mLDi.o: In function `main':
/projects/voidTest/hello.c:4: undefined reference to `mySqrt'
collect2: error: ld returned 1 exit status
\end{lstlisting}
\end{itemize}
\end{frame}

\begin{frame}[fragile]
\frametitle{Writing Functions - Example}
\begin{itemize}
\item Likewise, forgetting the prototype:
\begin{lstlisting}[style=CStyle]
#include <stdio.h>
int main() {
	printf("%f\n", mySqrt(26));
}
\end{lstlisting}
\item Results in (cut down):
\begin{lstlisting}[style=pseudo]
hello.c: In function `main':
hello.c:4:17: warning: implicit declaration of function 'mySqrt'
  printf("%f\n", mySqrt(26));
/projects/voidTest/hello.c:4: undefined reference to `mySqrt'

\end{lstlisting}
\end{itemize}
\end{frame}

\begin{frame}
\frametitle{Function Compiler Errors}
\begin{itemize}
\item ``implicit declaration of...''
	\begin{itemize}
		\item The function prototype is missing
	\end{itemize}
\item ``undefined reference to...''
	\begin{itemize}
		\item The function definition is missing
	\end{itemize}
\end{itemize}
\end{frame}

\begin{frame}[fragile]
\frametitle{Function Definition Placement}
\begin{itemize}
\item The following \textit{works} but isn't recommended:
\begin{lstlisting}[style=CStyle,basicstyle=\ttfamily\scriptsize]
#include <stdio.h> 
#include <math.h> 

float mySqrt(float k) { 
	int n; 
	float xn = k/2.0; 
	for(n = 0; n < 10; n++) 
		xn = 0.5*(xn + k/xn); 
	return xn; 
}

int main() { 
	printf("sqrt(26) = %.8f\n", mySqrt(26.0)); 
	printf("Libarsy sqrtf(26): %.8f\n", sqrtf(26.0)); 
}
\end{lstlisting}
\item Only useful in very small projects but common
\end{itemize}
\end{frame}

\begin{frame}[fragile]
\frametitle{Function Arguments}
\begin{itemize}
\item Function arguments become variables inside the function
\begin{lstlisting}[style=CStyle]
float mySqrt(float k) {  // k is an argument
	int n; 
	float xn = k/2.0;  //k used here
	for(n = 0; n < 10; n++) 
		xn = 0.5*(xn + k/xn); // and here
	return xn; 
}
\end{lstlisting}
\item Don't declare them as variables!
\end{itemize}
\end{frame}

\begin{frame}
\frametitle{Function Arguments}
\begin{itemize}
\item By default, arguments are ``passed by value''
\item The function gets \textit{copies}
\item Modifying them in a function doesn't change the original variable
	\begin{itemize}
		\item No, not even if they have the same name
	\end{itemize}
\item The argument variables are discarded on function return
\item The return value is the \textit{only thing} that goes back
\end{itemize}
\end{frame}

\begin{frame}
\frametitle{Function Return Values}
\begin{itemize}
\item Return values can only be one number
\item How can we write a function which modifies (or returns) multiple things?
\pause
\item Trigger warning....
\pause
\item Pointers!
\pause
\item We'll learn how to use pointers in Week 6(ish)
\item For now, just learn to live with the single return value
\end{itemize}
\end{frame}

\begin{frame}
\frametitle{Function Example}
Write a C function, \texttt{isPrime()}, which takes an \texttt{int} as an argument and returns 1 if it is prime and zero otherwise
\begin{itemize}
\item Name: \texttt{isPrime}
\item Argument(s): \texttt{(int x)}
\item Return Value: \texttt{int}
\pause
\item Function prototype: \texttt{int~isPrime(int~x);}
\end{itemize}
\end{frame}

\begin{frame}
\frametitle{Static Vs Auto Variables}
\begin{itemize}
\item Any ``normal'' variable declared within the function (including arguments) is lost on function exit
	\begin{itemize}
		\item These are called \textit{auto} variables
	\end{itemize}
\item By default, any declared variable is an auto variable
	\begin{itemize}
		\item Their value is lost outside the block where they are declared
	\end{itemize}
\pause
\item The other type is a \texttt{static} variable
 \begin{itemize}
 	\pause
 	\item Their value is retained
 	\pause
 	\item Their scope is still limited
 \end{itemize}
\end{itemize}
\end{frame}

\begin{frame}
\frametitle{Static Variables}
\begin{itemize}
\item Example: the \texttt{rand()} function returns different random numbers each time it is called
	\begin{itemize}
		\item How? Shouldn't everything be lost when the function returns?
		\item Not always! The \texttt{rand()} function's ``state'' is kept by a \texttt{static} variable.
	\end{itemize}
\pause
\item Variables are static if declared with the \texttt{static} keyword
\item Declaration examples:
\pause
{\small
\item \texttt{static int k = 0;}
\item \texttt{static float z = 0, y = 0;}
\item \texttt{static long bigNum = 2345235234432;}
}
\end{itemize}
\end{frame}

\begin{frame}[fragile]
\frametitle{Static Variable Example}
\begin{itemize}
\item Example: Write a function, \texttt{counter()} which returns an integer equal to the number of times it has been called.
\pause
\item Function prototype: \texttt{int counter(void);}
\pause
\item Function definition:
\begin{lstlisting}[style=CStyle]
int counter() {
	static int count = 0;
	return count++;
}
\end{lstlisting}
\end{itemize}
\end{frame}

\begin{frame}[fragile]
\frametitle{Static Variable Example}
\begin{itemize}
\item The variable \texttt{count} is declared \texttt{static}
\item The initialisation, \texttt{count = 0}, happens \textit{once}
\item The value of \texttt{count} is retained between function calls 
\begin{lstlisting}[style=CStyle]
int counter() {
	static int count = 0;
	return count++;
}
\end{lstlisting}
\end{itemize}
\end{frame}

\begin{frame}[fragile]
\frametitle{Static Variable Example}
Wrapping the function in some test code:
\begin{lstlisting}[style=CStyle]
#include <stdio.h>

int counter(void);

int main() {
	for(int k = 0; k < 10; k++)
		printf("counter(): %d\n", counter() );
	return 0;
}

int counter(void) {
	static int count = 0;
	return count++;
}
\end{lstlisting}
\end{frame}

\begin{frame}
\frametitle{Test Code?}
\begin{itemize}
\item ``Test code'' is a term I made up
\item It means the minimum amount of code required to verify a function's behaviour
\item Always test your functions \textit{in isolation}!
\pause
\item If you write ``too much'' code before testing it will make debugging \textbf{much} harder
\end{itemize}
\end{frame}

\begin{frame}
\frametitle{Test Code}
\begin{itemize}
\item How much is ``too much''?
\pause
\item Personally?
\pause
\item After 20 years of experience?
\pause
	\begin{itemize}
		\item 1-5 lines
	\end{itemize}
\pause
\item Never underestimate:
	\begin{itemize}
		\item How hard programming is
		\item How easy it is to make mistakes
		\item How \textit{brutally catastrophic} bugs can be
	\end{itemize}
\pause
\end{itemize}
\end{frame}

\begin{frame}
\frametitle{Bug Case Study}
Paraphrased from \underline{\href{https://en.wikipedia.org/wiki/Therac-25}{Wikipedia}}:
\\~\\
``The Therac-25 was a computer-controlled radiation therapy machine ... It was involved in at least six accidents ...  in which patients were given massive overdoses of radiation. Because of concurrent programming errors, it sometimes gave its patients radiation doses that were hundreds of times greater than normal, resulting in death or serious injury.''

\end{frame}

\end{document}
