\documentclass{lab}

\usepackage{graphicx}
\usepackage{float}
\usepackage{soul}
\usepackage{multicol}

\addtolength{\oddsidemargin}{-.4in}
\addtolength{\evensidemargin}{-.4in}

\title{ENGG1003 - Lab Week 10}
\author{Brenton Schulz}
\date{\today}

\begin{document}
\maketitle

\section{Introduction}

Task types

\begin{itemize}
	\item Scalar equations
		\begin{itemize}
			\item $s=ut+\frac{1}{2}at^2$
			\item Projectile motion, plots?
		\end{itemize}
\end{itemize}

\begin{task}{Projectile Motion}{}
Write a MATLAB script which plots the path of a particle undergoing projectile motion given its initial velocity. The velocity is specified as a speed, $v_0$, and angle from the horizon, $\theta$.
\\~\\
As the particle moves the horizontal, $x$, and vertical, $y$, displacements as a function of time, $t$, can be calculated as:
\begin{align*}
x &= v_0 t \cos(\theta) \\
y &= v_0 t \sin(\theta) - \frac{1}{2}g t^2
\end{align*}
Where $g$ is acceleration due to gravity. If we choose positive $x$ to be ``upwards'' then $g$, in SI units, is $-9.8~{m/s^2}$
Your code should declare a time vector which is long enough to plot the particle's path until it returns to $y=0$. This is achieved by declaring $t$ from 0 to:
\begin{equation*}
t = \frac{2 v_0 \sin(\theta)}{g}
\end{equation*}
To keep the output plot reasonably ``smooth'' declare $t$ with a few hundred to a thousand points. You may use the \texttt{linspace()} function or \texttt{start:interval:end} syntax to declare $t$.
\\~\\
\end{task}

\end{document}
