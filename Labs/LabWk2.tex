\UseRawInputEncoding
\documentclass{lab}

\usepackage{graphicx}
\usepackage{float}
\usepackage{soul}
\usepackage{multicol}

\addtolength{\oddsidemargin}{-.4in}
\addtolength{\evensidemargin}{-.4in}

\title{ENGG1003 - Lab 2}
\author{Brenton Schulz}
\date{\today}

\begin{document}
\maketitle

\section{Library Basics}

This lab will be your first official introduction to using Python \textit{libraries}.

A library is a collection of \textit{functions} (blocks of code referenced by name with specific input, functionality, and output) which perform tasks beyond those ``built in'' to the Python language.

For example, Python has no built in way of calculating square roots so this mathematical operation is implemented in the \texttt{sqrt()} function from the \texttt{math} or \texttt{numpy} libraries.

Libraries can also contain data in addition to functions. For example, the \texttt{math} library (used in the task below) contains the numerical constant $\pi$, accessed by using the text \texttt{pi}.

\begin{task}{Math Library Example}{}
Read through Section 1.3 of the textbook (\url{https://link.springer.com/chapter/10.1007/978-3-030-16877-3_1#Sec9}).
\\~\\
Be sure to observe the (intentional!) error which results from running the first code example
\end{task}

The full documentation for all \texttt{math} library functions can be read here: \url{https://docs.python.org/3/library/math.html}

\section{Importing Library Functions}

Python allows for several different ways to \textit{import} libraries, each with their own \textit{syntax} and behaviour.

\begin{task}{Importing Libraries}{}
Read through Section 1.4 of the textbook \url{https://link.springer.com/chapter/10.1007/978-3-030-16877-3_1#Sec10}
\\~\\
Note the difference between:

\begin{itemize}
\item \texttt{from library import things}
\item \texttt{from library import *}
\item \texttt{import library}
\item \texttt{import library as name}
\end{itemize}

\end{task}

Section 1.4.5 lists the libraries used by the textbook. Some of them (like \texttt{math}) are installed by default while others need to be explicitly installed.

\pagebreak
\section{Installing Libraries}

Normally you would only install the libraries needed for a particular project. It is also best practice to limit the installation of a library to a \textit{virtual environment}. The reasons can be complex, but you should be aware that libraries are often under constant development and their behaviour can change as updates are released. Containing each Python project inside a virtual environment means that each project can use different versions of each library. This is done so that code written in the past will continue to work as new updates are released because that project's virtual environment only contains libraries known to work with that code.

\begin{task}{Installing Libraries}{}
Install the \texttt{numpy} and \texttt{matplotlib} libraries in your PyCharm project.
\\~\\
To do this, click on the ``Terminal'' tab (to the left of the ``Python Console'' tab) and execute the following commands:
\\~
\begin{itemize}
\item \texttt{pip install numpy}
\item \texttt{pip install matplotlib}
\end{itemize}
~\\
Test that the installation was successful by running \texttt{import numpy} and \texttt{import matplotlib} in the Python console.
\end{task}

\section{An Introduction to Plotting and Vectorisation}

% Urgh, can't /ref the task blocks?
This section requires \texttt{numpy} and \texttt{matplotlib} so if you skipped the previous Task please go back and finish it.

Plotting requires a new programming concept: the \textit{array} (also known as a \textit{vector} or, in Python a \textit{list}). So far a single variable has been a single value (eg: \texttt{g = 9.8}). An array is an ordered collection of numbers all referred to by the same variable name.

In the task below you will see a line of code which reads:

\begin{lstlisting}
t = np.linspace(0, 1, 1001)
\end{lstlisting}

This line of code uses the \texttt{linspace} library function to create an array of 1001 numbers linearly spaced between \texttt{0}, and \texttt{1}.\footnote{Complete \texttt{linspace()} documentation is here: \url{https://numpy.org/doc/stable/reference/generated/numpy.linspace.html}}.

\begin{task}{Your First Array}{}
Run the following code in the Python Console and observe the resulting array of values:

\begin{lstlisting}
import numpy as np
x = np.linspace(0,10,11)
print(x)
\end{lstlisting}

Note how the numbers \texttt{0} to \texttt{10} are printed. All of these values are stored as a ``group'' inside the variable \texttt{x}. The trailing decimal points indicate that the values are \textit{floating point} numbers - more on these later, but a \texttt{float} \textit{datatype} is required to store numbers with both integer and fractional components.
\end{task}

In Python (and most programming languages) you can only plot \textit{numbers}. This is a subtle but crucial point: Python (specifically the \texttt{matplotlib} library) doesn't understand how to draw an ``equation'' or ``formula''. It is only able to join lines between an ordered set of points on a 2D plane.

To plot the graph of an equation you must first write other Python code which generates a set of points on that graph then pass the resulting arrays to \texttt{matplotlib}.

To do this ``simply'' (in few lines of code) a new idea is required: \textit{vectorisation}. Vectorisation is a large topic, but for now know that using an array in a mathematical expression repeats the calculation for \textit{every} value in the array.

Note the output of the previous task: the variable \texttt{x} contained 11 values. The next example will demonstrate \textit{vectorising} some arithmetic using that result.

\begin{task}{Vectorisation Example}{}
Building on the code in the previous task, copy the code below into a script and run it. Observe that the result of the $y = 2*x$ expression is also an array. This is a \textit{vectorised} expression because it operates on an entire array in one line of code. All 11 values in \texttt{x} have been multiplied by 2 and the results stored in the variable \texttt{y}.

\begin{lstlisting}
import numpy as np
x = np.linspace(0,10,11)
print(x)
y = 2*x
print(y)
\end{lstlisting}

\end{task}

With those fundamental principles explored, now is a good time to complete reading and executing textbook Sections 1.5 and 1.6.

\begin{task}{Sections 1.5 and 1.6}{}
Read through and execute all code in textbook Sections 1.5 and 1.6.
\\~\\
Link: \url{https://link.springer.com/chapter/10.1007/978-3-030-16877-3_1#Sec16}
\end{task}

\pagebreak
\section{Further Exercises}

\begin{task}{Plotting Polynomials}{}
With what you have learned so far, write a Python script which initialises three variables: \texttt{a}, \texttt{b}, and \texttt{c}, and plots the resulting parabola from the equation:
\begin{equation}
y = ax^2 + bx + c
\end{equation}

Experiment with the range of \texttt{x} values used so that the graph produced for \texttt{a = 2}, \texttt{b = 5}, and \texttt{c = -2} clearly shows both roots (x-axis crossings). Estimate their values from the plot and compare your numbers to the exact values of:

\begin{align*}
x &= - \frac{5}{4} - \frac{\sqrt{41}}{4} = -2.8508\\
x &= - \frac{5}{4} + \frac{\sqrt{41}}{4} = 0.35078
\end{align*}
\end{task}

\begin{task}{Simultaneous Equations}{}
Write a Python script which allows a user to solve the following simultaneous equations:

\begin{align}
2x + 4y &= 3\\
x - 2y &= 1
\end{align}
Hint: Re-arrange both equations to make $y$ the subject and plot them both on the same graph. The ``solution'' is the point at which the lines intersect.
\end{task}

\begin{task}{Plotting Trigonometric Functions}{}
The \texttt{numpy} library includes a vast array of mathematical functions from trigonometry to logarithms. Write a Python script which plots $\sin(x)$, $\cos(x)$, and $\tan(x)$ on the same graph from $0$ to $4\pi$.
\\~\\
Note that:
\\~
\begin{itemize}
\item These functions assume \textit{radian} input
\item Since they are from the \texttt{numpy} library you need to reference them appropriatly. Review Section 1.4 for details, or use \texttt{import numpy as np} followed by \texttt{np.sin()} (etc).
\end{itemize}
~\\
You will encounter a problem: $tan(x)$ approaches $\pm \infty$ and, by default, totally obscures $\sin()$ and $\cos()$!
\\~\\
To fix this, read about the \texttt{ylim()} library function (from \texttt{pyplot}) and restrict the y-axis to the range $[-1.5, 1.5]$. The documentation is here: \url{https://stackabuse.com/how-to-set-axis-range-xlim-ylim-in-matplotlib/}
\end{task}
\pagebreak
\begin{task}{Linear Interpolation}{}
There are many instances in science and engineering when data needs to be \textit{interpolated}. Informally, this is the process of predicting a quantity's value somewhere between two (or more) data points.
\\ \\
In this task you will be using the \textit{two-point formula} to interpolate between two data points. You can imagine these data points to be measurements of temperature / brightness / sound intensity / pressure / etc as a function of time.\\ \\
The two-point formula states that the equation of a line passing through two points $(x_1, y_1)$ and $(x_2, y_2)$ is:
\begin{equation}
y = \frac{y_2 - y_1}{x_2 - x_1}(x - x_1) + y_1.
\end{equation}
The value of $y$ at some given $x$ can then be estimated. For the purposes of interpolation $x$ will always be in between the given data points. ie, assuming $x_2 > x_1$:
\begin{equation}
x_2 > x > x_1.
\end{equation}
Note that $y$ is only an \textit{estimate} of the true value. You will often see this notated as $\hat{y}$, especially in statistics.
\\ \\
And just to really hammer the point home: $x$ is any point; $(x_1,y_1)$ and $(x_2, y_2)$ are fixed points, and $y$ is the value to be calculated given $(x_1,y_1)$, $(x_2, y_2)$, and $x$.
\\ \\
Given the above, write a Python script which implements the following pseudocode:
\begin{lstlisting}[style=pseudo]
BEGIN
	Create variables for x1, y1, x2, y2, and x
	Estimate the value of y at x using the linear interpolation formula
	PRINT y
	Give the user confidence in the answer by plotting a graph
		showing the three points.
END
\end{lstlisting}

See Fig. 1.2 of the text (and surrounding context) for how to plot individual points.
\\ \\
\textbf{Extension:} Given any value of $(x_1,y_1)$ and $(x_2, y_2)$ plot the line which passes through these two points between $x=-10$ and $x=10$. You will need to create your own $x$ array with \texttt{linspace()}. Plot both the resulting line and the two input points with an \texttt{*} marker.

\end{task}
\end{document}
