\documentclass{lab}

\usepackage{graphicx}
\usepackage{float}
\usepackage{soul}

\title{ENGG1003 - Lab 1}
\author{Brenton Schulz}
\date{\today}

\begin{document}
\maketitle

\section{Introduction}
This laboratory exposes you to the fundamental tools required to write computer programs in C. No prior programming experience is assumed but if you lack basic computer literacy you're gonna have a bad time. Advanced PC gaming skills such as:

\begin{itemize}
\item Editing .cfg or .ini files
\item Configuring command line arguments (eg: via a shortcut or Steam)
\item Debugging complicated driver problems
\item Using a game engine's debug console (typically accessed via the tilde \textasciitilde\ key)
\item Use of keyboard shortcuts
\item Hosting LAN parties and knowing how to configure the network
\end{itemize}

are also excellent preparation for your programming adventure. If you lack this experience then hopefully today's lab gets you up to speed.

The ENGG1003 labs involve sitting at a PC with 20 to 40 other students and working through lab exercises. Most weeks the lab exercises are there for you to learn, others there will be an assessment task which must be marked by a lab demonstrator before you leave.

The lab demonstrator's job is to answer questions you have about the material and help you work through unexpected problems. There will be limits to how much help they can provide during assessment tasks (default assumption is zero) but during non-assessed labs you can ask them anything you want.

Their job is not \textit{necessarily} to directly answer your question but to provide information which best supports your learning. This means that, instead of giving you an answer, they may guide you through the process of using Google/reference documentation/etc to help you become an independent learner.

While you are going through this course it is a near certainty that you will encounter difficult problems which take hours (or even days) to solve. This is normal. Engineering is \textit{hard} and watching other students solve problems faster than you is demoralising. Don't hesitate to ask for help. Talk to people and \st{copy from them}\footnote{Seriously, don't do this.} solve problems together. Don't suffer in silence.

\begin{figure}[H]
\begin{center}
\includegraphics[width=0.6\textwidth]{Wk1Images/google.png}
\end{center}
\caption{This image isn't really relevant, I just thought it was funny. Yes, it is real.}
\end{figure}

\pagebreak
\tableofcontents

\pagebreak
\section{C Programming Basics}
In order to write programs in C (and most other languages) the following software tools are required:

\begin{itemize}
\item An \textit{editor}, to create and edit raw text files.
\item A \textit{compiler}, to convert your text files into an \textit{executable} file.
\end{itemize}

A programming editor is very different to a \textit{word processor} (eg: Microsoft Word) in that it displays and stores raw ASCII text only. What you see printed to the screen represents the \textit{actual data} stored in the file. By contrast, Word will store a combination of text and display formatting and, as such, is not suitable for writing code.

Programming editors will generally have features optimised for coding, such as:

\begin{itemize}
\item Syntax highlighting
\item Line numbering
\item Auto completion
\item Pre-emptive error notifications
\item Communication with the compiler to highlight errors
\item Automatic indenting
\item Highlighting of matching blocks 
	\begin{itemize}
		\item ie: an easy method to find matching pairs of ( ), \{ \}, " ", etc.
	\end{itemize}
\end{itemize}

It is hoped that you will discover these features and learn to work with them. In time you will learn which features work well with your style and which simply get in the way.

For the time being the ``compiler'' noun will be used to colloquially reference a highly complex set of software tools which turn your source code into an executable binary file. You will be shielded from the details until otherwise necessary.

\subsection{Introduction to OnlineGDB}

OnlineGDB is a basic (\textit{very} basic) browser-based development environment for a variety of programming languages. It gives you access to an editor, a small amount of cloud storage, compiler, and standard input~/~output. It also contains a \textit{debugging} feature however for technical reasons\footnote{It only allows one debug session per IP address. The entire campus uses the same public IP so we can't use it in labs.} we won't be using it.

All compilation and execution is performed on the OnlineGDB server. As such, the service has an incredibly low barrier to entry: there is (almost) zero installation/configuration required to get started running code.

\begin{task}{}{}
\begin{enumerate}

\item Open a web browser and navigate to \url{http://www.onlinegdb.com}.

\item Configure OnlineGDB to run C code by selecting ``C'' from the ``Language'' drop-down box in the upper-right.
\end{enumerate}

This website supports many languages\footnote{MATLAB is not one of them because it is a \textit{very} expensive commercial package.} so feel free to come back here later if you're interested in learning any of the others. Python, although not taught in an Engineering degree, is a common choice for Engineering PhD students as a free MATLAB alternative and is probably worth playing around with.

\textbf{NB:} If a demonstrator sees you using Edge or IE they may instinctively think you need more help than students using Chrome or Firefox.
\end{task}

\begin{figure}[H]
\begin{center}
\includegraphics[width=\textwidth]{Wk1Images/onlinegdb.png}
\end{center}
\caption{The view when OnlineGDB is first opened.}\label{fig:onlinegdb}
\end{figure}

After OnlineGDB has loaded you will be greeted with the screen seen in Figure \ref{fig:onlinegdb}. The large area in the middle is the editor screen, this is where you will type C code. Immediately you can observe that this editor supports line numbering and syntax highlighting.

Above the editor is a toolbar which, from left to right, performs the following functions:

\begin{enumerate}
\item Create a blank new file
\item Run the project
\item Debug the project (not used in ENGG1003)
\item Stop execution of a running program
\item Share - Generates a link to your current source code
\item Save - When logged in this saves the project files to your personal cloud storage
\item \{ \} Beautify - Modifies your code's whitespace to adhere to the OnlineGDB indenting style (NB: I tried this at time of writing and it didn't work on my personal computer. Go figure.)
\item Download - Downloads the currently viewed file.
\end{enumerate}

The area below the editor is where standard output is written to and standard input read from. When the code is run its appearance changes to that of a basic console (ie: the GUI elements disappear and it becomes just text).

\begin{task}{}{}
\begin{enumerate}
\item Observe that there is already a template C listing in the editor.
\item Click the green Run button.
\end{enumerate}
The box at the bottom of the screen will produce a ``Compiling'' animation and, after execution of the template code, will produce the output seen in Figure \ref{fig:onlinegdb_hello}.
\end{task}

\begin{figure}[H]
\begin{center}
\includegraphics[width=0.7\textwidth]{Wk1Images/onlinegdb_hello.png}
\end{center}
\caption{A cropped screenshot showing the "Hello World" program output.}\label{fig:onlinegdb_hello}
\end{figure}

\textbf{Off-topic note:} Remember the ``returns zero to the operating system'' comment in lecture 1? Well that's what the text ``...Program finished with exit code 0'' is referencing. The 0 is the number that \texttt{main()} \textit{returned}. We will learn about \textit{function return values} later in the semester.

\begin{task}{}{}

\begin{enumerate}

\item Running the default template demonstrates \textit{standard output}. Modify the code to match that in Listing~\ref{lst:stdio}.
\\ \\
\textbf{NB:} You can use either tabs or spaces for indenting. Tabs is fewer keystrokes, spaces are more well defined. Some editors change the width of a tab and this causes illogical anger in some programmers.
\\ \\
While making the changes you will observe OnlineGDB's auto-complete features. When, for example, you type a double quote " character it \textit{automatically} types two and places the cursor between them. The editor will also automatically indent new lines and provide auto-complete suggestions, although many of them will be inappropriate (it is, after all, just a computer program; not a science fiction grade artificial intelligence).

What other ``helpful'' editor behaviour did you notice? Some of it will be annoying at first (some of it will be annoying \textit{forever}) but learning to work with the editor's features will improve your coding speed in the long term.

\begin{lstlisting}[style=CStyle,caption=A basic C program which demonstrates input and output.,label=lst:stdio]
#include <stdio.h>

int main() {
	int k;
	scanf("%d", &k);
	printf("You entered: %d\n", k);
	return 0;
}
\end{lstlisting}

\item After editing the code press Run. After it is compiled you will notice that the console is just displaying a cursor. This is because \texttt{scanf()} waits for data to be typed (specifically, it will wait until a new line character, ASCII value 10, is sent).

\item Type an integer and press \texttt{enter / return}. There will be some ``lag'' because the data is being sent to OnlineGDB's server before being displayed.

\item After pressing enter the console should display the text ``\texttt{You typed: 123}''.

\item Run the program again except this time don't type just an integer, try typing a word, or a word containing a number, or a number followed by letters (with and without a space). What is the behaviour each time? Are you getting annoyed by the slow compile time and lag yet? OnlineGDB may be simple but it is, at times, a compromise.

\end{enumerate}
\end{task}\label{tsk:stdio}

\pagebreak
\section{Compiler Errors and Warnings}

This section will demonstrate several common compiler \textit{errors} and \textit{warnings}.

An error occurs when the code does not meet the rigorous and unambiguous syntax rules specified in the ANSI C standard and, as such, the compiler does not know how to interpret the code. For example, if you miss a " symbol inside a \texttt{printf()} where does the text to be printed end? This can't be assumed because there are not enough rules about what you can and can't write in this context. As such, missing a " will produce a \textit{syntax error}.

By contrast, a warning occurs when the code is in some way ``mildly'' problematic but the compiler is still able to make assumptions about what the code should do and produce a binary executable. For example, if the \texttt{return 0;} at the end of \texttt{main()} is missing the compiler will throw it in for you because, well, what else is it going to do at the end of \texttt{main()}?\footnote{The programmer could want \texttt{main()} to return a value other than zero but it is overwhelmingly common to just return zero here. Historically, a program returning zero means ``the program finished without error'' and non-zero indicates some kind of error code.}

Unfortunately compiler errors can be \textit{highly} technical and difficult to interpret. Furthermore, they often report an error on a line \textit{after} the actual mistake! The exercises below will give you some experience with interpreting compiler errors but this is a topic in which you will likely engage in ``life-long learning''.

\subsection{Missing a semicolon}

\begin{task}{}{}
\begin{enumerate}
	\item If you have not done so already, get the code shown in Listing \ref{lst:stdio} working correctly.
	\item Remove the semicolon (the ; character) from the end of line 4.
	\item Compile the code, what error did the compiler produce? Which line is the error is on?
\end{enumerate}
\end{task}
	
	The compiler errors will look something like Listing \ref{lst:errs} (line numbers include the comments at the top of the OnlineGDB template code; Listing \ref{lst:errs} omitted these).
	
	\begin{lstlisting}[caption=The multiple errors produced by removing a \textit{single} semicolon.,label=lst:errs,basicstyle=\ttfamily,frame=single]
main.c: In function ‘main’:
main.c:14:5: error: expected `=`, `,`, `;`, `asm` or `__attribute__`
	before `scanf`
     scanf("%d", &k);
     ^
main.c:14:18: error: ‘k’ undeclared (first use in this function)
     scanf("%d", &k);
                  ^
main.c:14:18: note: each undeclared identifier is reported only once for
	each function it appears in
	\end{lstlisting}

	
Lets break this down a bit. Each error starts with a location, the syntax is: \texttt{{file}:{line}:{column}}. So \texttt{main.c:14:5} means the file \texttt{main.c} at line 14, column\footnote{\textit{Column} means the number of characters since the start of the line} 5.

The first error (\texttt{error: expected '=' , ‘,’, ‘;’, ‘asm'}... etc.) is telling you that something was missing and the compiler noticed the omission at line 14. In this case we removed a ; so that's what is missing. Note, however, that while the error was reported to be on line 14 the omission was actually on line 13! Also observe the extra technical jargon that you don't understand yet (what is \textit{\_\_attribute\_\_}??). Some of these are, in fact, beyond the scope of ENGG1003. You have to get used to reading text you don't understand and extracting the small pieces of information you actually need.

Let's look at the second error: \texttt{'k' undeclared}. Wait, wasn't \texttt{int k} still in our source code? Why is the compiler complaining that it is undeclared when it is \textit{right there}? The problem is that by removing the semicolon the declaration syntax was incorrect, so the compiler did not interpret that line as a declaration. It didn't find the expected ; threw its hand in the air and gave up.

The final line (\texttt{each undeclared identifier...} etc.) is just telling you that if you forget to declare a variable the compiler will only tell you about it \textit{once}. This is because it could appear multiple times and producing the same error over and over is redundant and confusing.

Lets try a different syntax error.

\subsection{Missing a Quote Symbol}

\begin{task}{}{}
\begin{enumerate}
\item Fix the previous error by re-inserting the semicolon.
\item Remove the closing " from the \texttt{printf} line so it reads:
\begin{lstlisting}[style=CStyle]
printf("You entered: %d\n, k);
\end{lstlisting}
\end{enumerate}
\end{task}

You will see an immediate change to the syntax highlighting; all characters between where the " was and the end of the line are now green, as if they were still inside the double quotes. Take-home message: \textit{pay attention to syntax highlighting!}

Again, removing a \textit{single character} generated a slew of errors:

\begin{lstlisting}[basicstyle=\ttfamily,caption=Errors produced by removing a \texttt{"} character.,frame=single]
main.c: In function ‘main’:
main.c:15:12: warning: missing terminating " character
     printf("You entered: %d\n, k);
            ^
main.c:15:5: error: missing terminating " character
     printf("You entered: %d\n, k);
     ^
main.c:16:5: error: expected expression before ‘return’
     return 0;
     ^
main.c:17:1: error: expected ‘;’ before ‘}’ token
 }
 ^
\end{lstlisting}

Observe how missing the closing " generates both an error and a warning at different locations. Why does it do this? To be honest, I don't know; I've only been using gcc for 20 years, there's always \textit{something} that you haven't learned yet. In the end it doesn't matter, an error exists that needs to be fixed before the compiler can output an executable.

Missing the " generated two other errors as without the " the \texttt{printf} \textit{expression} was not completed. In C an expression can be thought of as ``a syntactically complete line of code that does something unambiguous''. This is a \textit{very} informal definition but it will do for now. In order to complete the \texttt{printf} expression the syntax rules require the closing ", a closing right parenthesis ) [to match the opening left parenthesis (] and a semicolon ;. Removing the " character makes the compiler think that the existing ``\texttt{);}'' string at the end of the line is actually part of the data to be printed, not part of the C expression syntax (as such, the expression was never finished, leaving the compiler with an unresolved tension that will stay with it \textit{all day}\footnote{\url{https://xkcd.com/859/}}.

Remember how missing a ; caused errors on lines \textit{below} where the error actually was? Well the same thing has happened here. Because the \texttt{printf()} line was malformed an error occurred on the line below it (\texttt{expected expression before return}) \textit{and} the one below that (\texttt{expected ; before \} token}); and you thought your \textit{ex} was highly strung!

\pagebreak

\subsection{Failing to Define \texttt{main()}'s Return Type}

Enough about errors, the final example in this section demonstrates a compiler \textit{warning}. Unfortunately, OnlineGDB doesn't display the compiler output if only warnings are generated. If you are running \texttt{gcc} in Linux (as described at the end of this lab document) you won't have this problem\footnote{Yeah, I'm very biased towards Linux. It gets a bad rap because games, Microsoft Office, Adobe, and Autodesk don't support it but it is otherwise \textit{overwhelmingly} dominant. It runs: mobile phones (Android is a Linux derivative, iOS is a Unix variant), servers, supercomputers, ``internet of things'' devices, etc. As engineers you will probably see it in many places throughout your career, just hidden from the end user.}.

\begin{task}{}{}
\begin{enumerate}
\item Fix the previous error by re-inserting the appropriate quote symbol
\item Remove \texttt{int} before \texttt{main} \textbf{and} the semicolon in \texttt{return 0}, as per Listing \ref{lst:warn}

\begin{lstlisting}[style=CStyle,caption=Example code which generates a warning.,label=lst:warn]
main()
{
    int k;
    scanf("%d", &k);
    printf("You entered: %d\n", k);
    return 0
}
\end{lstlisting}
\end{enumerate}
\end{task}

Attempting to compile this code generates the following compiler output:

\begin{lstlisting}[basicstyle=\ttfamily,frame=single,caption=Compiler output produced when \texttt{main()} return type is omitted.,captionpos=b]
main.c:11:1: warning: return type defaults to ‘int’ [-Wimplicit-int]
 main()
 ^
main.c: In function ‘main’:
main.c:17:1: error: expected ‘;’ before ‘}’ token
 }
 ^
\end{lstlisting}

The error can be ignored here, lets focus on \texttt{warning: return type defaults to `int`}. The keyword which goes before \texttt{main()} specifies its \textit{return type}. This is the type of data which it sends back to the operating system on program exit. We will study function return types in later weeks. To my knowledge, operating systems only support the \texttt{int} return type (ie: an integer) so if you leave it out the compiler can quite safely \textit{assume} that's what should be there and only issue a warning instead of an error.

Generally speaking, warnings should be fixed when you see them. With experience you will know when they \textit{need} to be fixed (because the compiler is assuming something incorrectly) and when they can be ignored (because you have more important stuff to fix first).

No doubt you will see \textit{many} other errors and warnings during your foray into C programming. I still occasionally see new ones! If in doubt, throw the compiler output into Google, chances are someone on Stack Overflow\footnote{Why are you even taking this course? Stack Overflow is all you \textit{really} need, right?} has written a good explanation about it.

\pagebreak
\section{Comments}

So far all the code examples have been very basic and (hopefully) easy enough to read. In ``real'' projects, however, this is rarely the case and the code needs some kind of explanation for the reader to quickly, and accurately, understand what it does\footnote{Despite what you might believe experienced engineers and programmers are not wizards, our understanding is not \textit{magic}, it is based on experience and frequently needs supplementing with code comments.}.

You will hopefully gain experience with code comments as you progress through this course (and the rest of you career!) but for now we will just see the basics. Any text in the source files\footnote{Did I forget to define what source code, or source files, are? It is just another word for any programming code. It can also be called a source \textit{listing}.} which:

\begin{itemize}
\item Is between \texttt{/*} and \texttt{*/}, or
\item Is between \texttt{//} and the end of the line
\end{itemize}

is \textit{totally ignored} by the compiler and called a ``comment''.

Code comments are a place for you to explain how your code works, or what it does, to future people who work with it. It is also a great place to leave little memos to yourself, typically in the form of:

\texttt{// TODO: Fix this because <reasons>}.

In fact some code editors (like the Linux editor \texttt{vim}) will automatically highlight the text \texttt{TODO} so it is easy to find.

\begin{task}{}{} Take whatever source code is currently shown in OnlineGDB and add some comments in various places. Compile the code and observe that they have no effect.
\end{task}

Students frequently ask what should and should not be commented. In this course your comments should be written such that a student who is on track to achieve 50\% can understand what your code does without having to consult external reference material. This will seem overly verbose but it will be used in lieu of a better standard. As a general rule: if you needed to look up something in a reference manual when writing the code write a comment explaining it. For example:

\begin{lstlisting}[style=CStyle]
printf("%d\n", x); // %d formats an integer
\end{lstlisting}

Anything which, at first glance, appears to be at least a little bit cryptic should probably get a comment.

\pagebreak
\subsection{Intrinsic Documentation}

As a supplement to comments, \textit{intrinsic documentation} is the idea behind choosing informative names for variables and functions\footnote{No, you don't need to know what a function is yet. It is in week 5 or so.}. Compare, for example, the following two code listings:

\begin{lstlisting}[style=CStyle]
int main()
{
	float x, y;
	scanf("%f", &x);
	y = x*9.0/5.0 + 32.0;
	printf("%f\n", y);
}
\end{lstlisting}

\begin{lstlisting}[style=CStyle]
int main()
{
	float tempFahrenheit, tempCelcius;
	scanf("%f", &tempCelcius);
	tempFahrenheit = tempCelcius*9.0/5.0 + 32.0;
	printf("%f\n", tempFahrenheit);
}
\end{lstlisting}

The first one, especially without comments, is cryptic and not possible to understand without more information. The second, even if you don't understand all the code, quite intuitively converts Celcius to Fahrenheit.

You will observe that my notes break intrinsic documentation rules all the time. Sorry about that (the examples in notes will also tend to not do anything particularly useful, so it gets difficult when you try to document ``nothing'').

\pagebreak
\section{Basic Arithmetic in C}

So, it is Page \thepage{} of these notes and we haven't done anything \textit{useful} yet! Ok \textit{fine}, lets do some data processing.

In this section we will do some basic arithmetic on numbers which are read from the console (ie: read from \texttt{stdin}). Since we haven't learnt much C (or even much programming in general) these examples are either going to be a bit boring (because they don't do much) or look like black magic (because you haven't learned how they work yet). I have to ask you just to go through the motions at this stage. Hopefully exposure to the code below now will make it easier to understand how it works in the coming weeks.

Lets start with the basic C arithmetic operators:

\begin{table}[H]
\centering
\begin{tabular}{|l|c|}
\hline
Operation      & C Symbol \\
\hline
Addition       & +        \\
Subtraction    & -        \\
Multiplication & *        \\
Division       & /       \\
\hline
\end{tabular}
\caption{Basic arithmetic operators in C}
\end{table}

These are all \textit{binary} operators, meaning they operate on two \textit{operands}\footnote{An operand is one of the ``things'' that a mathematical operator operates on. Eg: In a + b the variables a and b are operands.}. This may feel obvious, but C includes several \textit{unary} operators and even a \textit{ternary} one (that nobody uses because it's confusing). Each operand could be a variable (eg: \texttt{a + b}), a constant (eg: \texttt{a + 5}) or some complicated expression (eg: \texttt{(2*a + 6) / (12 + b - x)} )\footnote{I contemplated leaving this closing parenthesis out to give you unresolved tension but I'm not quite \textit{that} evil.}.

We begin with a basic example:

\begin{task}{}{}
Modify your code to match that of Listing \ref{lst:arithmetic1}.

\begin{lstlisting}[style=CStyle,caption=A basic arithmetic example,label=lst:arithmetic1]
#include <stdio.h>

int main() {
	int k;
	printf("Enter an integer: ");
	scanf("%d", &k);
	k = 2*k;
	printf("That integer doubled is: %d\n", k);
	return 0;
}
\end{lstlisting}
\end{task}

Notice that this code has a \textit{slightly} improved user experience to previous examples; it produces a prompt (\texttt{Enter an integer: }) which tells you what to do. Also notice that the first \texttt{printf()} does \textit{not} end with a \texttt{\textbackslash n} (newline), so the number you type appears on the same line as the prompt.

The line \texttt{k = 2*k} takes the value of \texttt{k}, multiplies it by 2, then assigns that result back into the variable \texttt{k}. This is a \textit{crucial} concept in programming: the \texttt{=} symbol is \textbf{not} equality, \texttt{k = 2*k} is \textbf{not} an equation, it is \textit{assignment}. Assignment takes what's on the right, evaluates it, then stores it into the variable on the left.

\subsection{Operator Precedence Basics}

Just like in ``normal'' algebra different operators take precedence over others (PEMDAS / BODMAS anyone? How about those Facebook posts where \textit{everybody} gets this wrong?). You can view the full C operator precedence here: \url{https://en.cppreference.com/w/c/language/operator_precedence} but you don't need to learn all of it now, it will be covered more in future lectures.

For now we will observe some basic examples and look at how engineers can deal with the complexity of full C operator precedence.

The predominant engineer's approach to this topic is to vaguely learn how the language behaves and then throw parentheses everywhere just to be \textit{absolutely sure} that the intention is not ambiguous. In the end it is \textit{other people} who will have trouble reading your code, not the compiler, so you might as well make their job easy.

\begin{task}{}{}
Using the template in Listing \ref{lst:arithmetic2} implement the following equation in C:

\begin{equation}\label{eq:fraction}
y = 2x+3\times5
\end{equation}



\begin{lstlisting}[style=CStyle,caption=A basic arithmetic example,label=lst:arithmetic2]
#include <stdio.h>

int main() {
	float x;
	float y;
	printf("Enter a number: ");
	scanf("%f", &x); // Note change of %d to %f
	// y = ??? uncomment this line and write your answer instead
	printf("y: %f\n", y);
	return 0;
}
\end{lstlisting}
\end{task}

You will notice a few new things about Listing \ref{lst:arithmetic2}. Firstly, it uses the \texttt{float} datatype. This type will store any real number with a magnitude of $1.2 \times 10^{-38}$ to $3.4 \times 10^{38}$ with a precision of approximately 6 decimal digits. Its bigger brother, the \texttt{double} will be seen later.

The other major change is that inside \texttt{scanf} and \texttt{printf} the \texttt{\%d} has changed to \texttt{\%f}. The \texttt{f} stands for floating point, which is a standard method\footnote{The other major standard is known as \textit{fixed point}. Details are beyond ENGG1003 but you'll probably see it if you study enough digital signal processing (DSP).} for storing fractional numbers using only binary integers. The details are beyond this course, but you can read the details on Wikipedia: \url{https://en.wikipedia.org/wiki/IEEE_754}.

Back to the task at hand, have a go at implementing Equation \ref{eq:fraction} and see if the result works. Observe that you don't need to force precedence with parentheses because, in C, the multiplication operations are performed before addition.

\begin{task}{}{}
Implement the following equations on line 8. What happens when you choose x to force a division by zero?

\begin{enumerate}
\item $y = \frac{9}{5} x + 32$
\item $y = \frac{x}{1 - x}$.  This one will require parentheses.
\item $y = x^2 + 2x$. \textbf{NB:} C does \textit{not} include an exponent operator. Implement $x^2$ as \texttt{x*x}:
\item $y = \frac{x + 2}{x - 1}$
\end{enumerate}
\end{task}
\pagebreak
\section{A Very Brief Introduction to Git - Optional}

\textit{NB: Git is not in the course outline so it isn't strictly required nor assessed. It is, however, incredibly useful and included here because several 3rd and 4th year course coordinators (and industry contacts) asked me to expose you to it. I agreed.}

Programming projects typically involve two potentially problematic issues:

\begin{enumerate}
	\item Multiple people contribute to the same project. How should their changes be ``merged'' into some kind of ``master'' source listing?
	\item Programmers frequently want to ``roll-back'' to an earlier code version.
\end{enumerate}

The second item arises from the fact that after going down a problematic design pathway it is typically much faster to scrap the idea and start again than it is to try and fix all the problems you just created. The fastest way to do this is to load up known-working code from the past and go from there.

As such, programmers designed so-called \textit{versioning} subsystems. These are ways of storing data which allows someone to access either a current or past version of a set of data.

In this course it will be \textit{recommended} that you take advantage of the modern versioning system known as Git. It was developed for tracking code in the Linux kernel and has since expanded to be an industry standard; even Microsoft purchased the website GitHub because it was \textit{just better} than anything else they had developed internally. I am using Git for tracking changes while writing course content and the Git host GitHub is also doubling as cloud storage as I move writing between three different computers.

The Git package can be installed on any computer, it doesn't even require a server / client architecture spread across two or more computers. GitHub is simply a website which hosts Git servers.

\begin{task}{}{} Navigate to \url{https://github.com/bschulznewy/engg1003/blob/master/Lectures/Wk1/Friday/LectureFriWk1.pdf}.
\\ \\ 
This is the GitHub ``repository'' that I have been using when writing ENGG1003 content. In particular, it is the lecture notes for the Friday Week 1 Lecture (I'm subtly trying to get you to read them ahead of time, apologies for improving your study efficiency).
\end{task}

The link is for a PDF document and you will see that GitHub automatically renders it inside the website. It will attempt to display any file directly linked to.

But that's not what we're here for. I want to show you the real power of Git: version tracking. Click on the ``History'' button to the top right of the PDF preview. You will be greeted with a view similar to that shown in Figure \ref{fig:githistory}. It lists the git \textit{commits} made which affect that file. A commit is a snapshot of what the entire repository looked like at a particular point in time. Each commit happens manually (ie: I type a command to cause a commit to be created) and is attached to a ``commit message'' which (hopefully) describes the changes which were made.

\begin{figure}[H]
\begin{center}
\includegraphics[width=0.8\textwidth]{Wk1Images/github_lecture.png}
\end{center}
\caption{The git repo history for the Friday lecture notes.}\label{fig:githistory}
\end{figure}

Click on the top one (specifically the text ``Changed lecture notes to blue.'') and you will see what changed in this particular git commit.

\begin{figure}[H]
\begin{center}
\includegraphics[width=0.9\textwidth]{Wk1Images/github_lecture_changes.png}
\end{center}
\caption{The changes which occured in a particular commit.}\label{fig:gitchanges}
\end{figure}

The box at the bottom of Figure \ref{fig:gitchanges} shows the actual changes which occurred. In this case I changed the colour theme for the lecture notes to roughly match that of the first lecture. The changes to the PDF are not shown because they would be unintelligible garbage (it is a binary file, not text).

Click the ``View file'' button next to LectureFriWk1.pdf to view what the PDF looked like in this commit.

Browse through the other commits and observe the writing process for the Lecture 1 PDF. Feel free to also browse the other files in the repository.

\begin{task}{}{} Go to \url{www.github.com} and create a GitHub account. You will use it in later weeks when we swap to a development environment which supports Git. You can also use a GitHub account to login to OnlineGDB, allowing you to have your source files stored on their server.
\end{task}

\begin{task}{}{} Read more about how Git works and what it does here: \url{https://developer.ibm.com/tutorials/d-learn-workings-git/}. You probably won't understand everything, but that's not important right now. Just get some general exposure to the ideas and we'll fill in the practicalities later.
\end{task}

\pagebreak
\section{Installing a C Compiler on Your Personal Machine - Optional}

Although ENGG1003 officially uses cross-platform web-based services like OnlineGDB (later, hopefully, Eclipse Che) you are welcome to install a suitable editor and C compiler on your personal machine.

My personal recommendation for Windows and Mac users is to either run the Ubuntu virtual machine (see Section \ref{sec:virtualbox}) or, if you know what you're doing, install \texttt{gcc} natively (instructions below). Feel free to also take advice from your demonstrator, opinions on this will vary wildly.

\subsection{Getting started with C in Windows}

My support for Windows is lukewarm. Given you \textit{probably} grew up with Windows hopefully you can work out how to install and use the software below without much help. I don't have access to a Windows machine to test and debug installation of these recommendations.

I recommend \texttt{tcc} for a simple C compiler on Windows, download from: \url{https://bellard.org/tcc/}. You will need to pair it with a editor and the Windows command line \texttt{cmd}.

For more advanced usage you can install \texttt{gcc} with the MinGW package: \url{http://www.mingw.org/}. It installs all the development software you would find on in Linux development environment. You may experience difficulties, this one is pretty complicated.

The MinGW getting started documentation is here: \url{http://www.mingw.org/wiki/Getting_Started}.

For editing I recommend notepad++: \url{https://notepad-plus-plus.org/download/v7.6.3.html}.

If you're really keen you can try to setup MinGW with Eclips. Instructions here: \url{http://www.multigesture.net/articles/how-to-install-mingw-msys-and-eclipse-on-windows/}.

If you would like to use the Ubuntu virtual machine follow the instructions in Section \ref{sec:virtualbox}. Instructions for installation of VirtualBox can be found through Google: \url{https://bit.ly/2T4sjAW}.

\subsection{Getting started with C in MacOS}

I've never used a Mac but under the hood they are very similar to Linux (mostly the same command line, both Unix variants) so \texttt{gcc} should run fine...right??

Since I don't know what I'm doing here instructions are outsourced to this StackOverflow thread: \url{https://stackoverflow.com/questions/9353444/how-to-use-install-gcc-on-mac-os-x-10-8-xcode-4-4}

If in doubt, run the Ubuntu virtual machine from Section \ref{sec:virtualbox}.

Instructions which appear to explain how to install VirtualBox on a Mac can be found here: \url{https://matthewpalmer.net/blog/2017/12/10/install-virtualbox-mac-high-sierra/index.html}.

\pagebreak
\subsection{Getting started with C in Ubuntu Linux}

Desktop Linux is not supported by university IT but, personally, I find it to be a fantastic development platform. The following instructions are completely optional, only do this if you are keen to learn.

Installing Ubuntu is beyond the scope of this document (and course). If you think this is a daunting task I would recommend only using the officially supported tools (OnlineGDB, a better one later) to complete ENGG1003. There are many thousands of websites and YouTube videos which will guide you through the Ubuntu (or Mint, Arch, etc.) installation process. \textbf{NB:} Installing a new operating system can \textit{very easily} destroy all existing software on a machine. Don't do this if you don't know what you're doing.

If you don't want to risk installing Linux ``natively'' you can use the software package ``VirtualBox'' to run a virtual Ubuntu machine under Windows or macOS; details in Section \ref{sec:virtualbox}.

That out of the way, here's how you can get started. These steps assume a fresh Ubuntu 18.04 installation, in other distributions YMMV\footnote{Your mileage may vary. ie: this step might not work.}:

\begin{enumerate}
\item \textit{If you are using the virtual machine image described in Section \ref{sec:virtualbox} this step is optional, I've already done it for you. Running it again will just bring up ``xxx is already at the latest version'' messages, so by all means try it anyway.}

The C compiler in Linux (and OnlineGDB, and \textit{many} other platforms) is gcc (the GNU C Compiler). To install it, open a terminal (\texttt{ctrl + alt + t}) and type:

\texttt{\$ sudo apt install gcc libc6-dev gedit}\\ (the \$ indicates the terminal prompt, don't type that character)

When prompted, enter your password (the virtual machine password is ``engg1003''), wait a few seconds, press \texttt{enter} if it wants installation confirmation, and wait a few more seconds. An Internet connection is required for apt to download the required software.

The \texttt{libc6-dev} package provides all the basic C libraries (\texttt{printf} etc.) and \texttt{gedit} is a basic text editor.

\item Lets make a new directory for writing C files, type:
\begin{enumerate}
	\item \texttt{\$ mkdir c}
	\item \texttt{\$ cd c}
\end{enumerate}
The first command creates a directory called ``c'' and the second ``changes into'' that directory.

\item Create a new .c file. We will do this in \texttt{gedit} (because it is easy and simple) but there are many others (the more nerdy among you may want to learn \texttt{vim} or \texttt{emacs}. They are \textit{very} powerful editors).

Type: \texttt{\$ gedit test.c \&} \\ (The \& symbol at the end of a command runs the command ``in the background''. This gives you the command line back straight away, instead of having to quit gedit first. \textbf{NB:} gedit will quit if you close the terminal.)

\item Type out the code seen in Figure \ref{fig:gedit}.

\item Click the Save button (or type \texttt{ctrl + s}).

\begin{figure}[H]
\begin{center}
\includegraphics[width=0.5\textwidth]{Wk1Images/gedit.png}
\end{center}
\caption{The gedit window with some C code typed out.}\label{fig:gedit}
\end{figure}

\item Move the \textit{keyboard focus} back to the terminal (ie: click the terminal window).

\item If you can't see the command prompt yet, press \texttt{enter}

\item You can type \texttt{ls} to see a list of all files in the current directory. \texttt{test.c} should be there.

\item To compile the \texttt{.c} file run: \texttt{\$ gcc test.c -o test}

This will create a binary executable called \texttt{test}. If the \texttt{-o} \textit{command line argument} is not given to \texttt{gcc} the binary file defaults to the name \texttt{a.out}.

\item Run \texttt{test} by typing: \texttt{\$ ./test}

The \texttt{./} is a special character string meaning ``relative to the current directory''. If you try to run \texttt{test} from any other directory nothing will happen because \texttt{test} is a built-in command. With most other names you will get a ``Command not found...'' error.

\item When the program runs you should see something similar to Figure \ref{fig:c_ubuntu}.

\item Go back to \texttt{gedit} and keep coding as you desire. Return to the command line to run \texttt{gcc} to re-compile your code. \textbf{NB:} The command line has a \textit{history} feature, pressing the up arrow will scroll through past commands, \textit{you don't need to type them out from scratch}.

\item If you enjoyed this you're a bit weird, welcome to the club! Recommended further reading would be a tutorial on \texttt{make} followed by investigations into a more powerful editor like \texttt{vim} or \texttt{emacs}. If nothing else you will then be able to appreciate ``exit vim'' memes.
\end{enumerate}

\begin{figure}[H]
\begin{center}
\includegraphics[width=0.8\textwidth]{Wk1Images/c_ubuntu.png}
\end{center}
\caption{The complete command line sequence performed in the Ubuntu virtual machine. Note that gcc produced a \textit{warning} because I didn't explicitly state that \texttt{main} returned \texttt{int}.}\label{fig:c_ubuntu}
\end{figure}

\pagebreak
\subsection{Running Ubuntu With VirtualBox}\label{sec:virtualbox}

If you are interested in investigating Linux but don't want to risk destroying data on an existing personal computer this section is for you.

\textbf{\textit{These steps will not work on university lab machines.}}

A \textit{virtual machine} is an instance of an operating system (could be Windows, macOS, Linux, or something else) which \textit{thinks} it is running directly on real hardware (ie: a motherboard, CPU, RAM, graphics card, etc) but, in reality, it is being run within another operating system (OS). The host OS allocates part of its physical CPU/RAM/HDD space to the virtual machine, along with access to optional devices such as the sound card, network interface, USB, etc.

As such, a virtual machine allows you to run an installation of Linux \textit{as if it was physically installed} but without the risk and hassle of actually installing it. The only compromise is a potential decrease in performance (the virtual machine can't use \textit{all} the host machine's RAM, for example).

I have installed Ubuntu 18.04 onto a virtual machine image and installed the necessary packages for compiling basic C programs. To run it:

\begin{enumerate}
\item Download VirtualBox from: \url{https://www.virtualbox.org/wiki/Downloads}
\item Install it. I don't have access to Windows or Mac machine to test and document this step, hopefully you can work it out.

\textbf{NB:} Windows 10 comes with its own virtualization software called Hyper-V. If enabled it conflicts with VirtualBox and causes:

\texttt{ERROR "Raw-mode is unavailable courtesy of Hyper-V.\\	 (VERR\_SUPDRV\_NO\_RAW\_MODE\_HYPER\_V\_ROOT)."}

To get around this you can build your own Ubuntu image with Hyper-V (try \url{https://www.windowscentral.com/how-run-linux-distros-windows-10-using-hyper-v} or just Google it) or disable Hyper-V with the following steps:

	\begin{enumerate}
		\item Open the command prompt (\texttt{cmd}) as Administrator.
		\item Disable Hyper-V by running: \texttt{bcdedit /set hypervisorlaunchtype off}
		\item Restart your PC.
	\end{enumerate}

\item To get Ubuntu running you have a few options:
	\begin{enumerate}
		\item Download the pre-installed virtual machine image I created (2.2~GB) from: \url{http://chetesting.vk2dds.net/Ubuntu.ova}

The file is hosted on virtual machine I pay \$4 per month for. If you know what you're doing (ie: have Linux webserver administration experience) web hosting is \textit{really} cheap.

The default username / password for this image is:

\textbf{Username:} engg1003\\
\textbf{Password:} engg1003

The password will be needed if you do anything with \texttt{sudo}, like install software. The username won't be needed, it is configured to login automatically.

		\item Download the Ubuntu installation \texttt{.iso} from \url{http://releases.ubuntu.com/18.04/} and create your own virtual machine
		\item Download the installation \texttt{.iso} for another Linux distribution of your choice. I don't want to start a religious war about which distribution is ``best'', there are \textit{plenty} of threads debating the merits of various distributions all over the Internet. Ubuntu is one of the more popular choices but you could also try Mint (entry level difficulty), Debian (semi-advanced), Arch (\textit{very} advanced), Gentoo (your beard is taller than you advanced), etc.
	\end{enumerate}
	\pagebreak
	The following instructions assume you downloaded the pre-installed image.
\item Open VirtualBox
\item Click File -$>$ Import Appliance
\item Click the little folder icon to the right of the text box and find \texttt{Ubuntu.ova}
\item Click Next
\item Review the ``Appliance settings'' (Figure \ref{fig:virtualbox1}). You shouldn't need to change anything

\begin{figure}[H]
\begin{center}
\includegraphics[width=0.6\textwidth]{Wk1Images/ubuntu_import.png}
\end{center}
\caption{You will see this if \texttt{Ubuntu.ova} was selected correctly.}\label{fig:virtualbox1}
\end{figure}

\item Click Import. A progress bar will appear as the image is unpacked and copied into VirtualBox's virtual machine folder.

\item Select the Ubuntu image from the list and click the green ``Start'' button (Figure \ref{fig:virtualbox2}).
\begin{figure}[H]
\begin{center}
\includegraphics[width=0.6\textwidth]{Wk1Images/ubuntu_start.png}
\end{center}
\caption{The ``start'' button which fires up a virtual machine image.}\label{fig:virtualbox2}
\end{figure}

\item A window will appear with the title ``Ubuntu [Running] - Oracle VM VirtualBox''. It will be black, flash up some random text about /dev/sda1, bring up the Ubuntu start image, then finally present you with the Ubuntu desktop seen in Figure \ref{fig:virtualbox3}.

\begin{figure}[H]
\begin{center}
\includegraphics[width=0.6\textwidth]{Wk1Images/ubuntu_desktop.png}
\end{center}
\caption{The default Ubuntu desktop. The resolution will be higher when you open my image.}\label{fig:virtualbox3}
\end{figure}

\item If Ubuntu can access the Internet it will probably bring up a ``Software Updater'' window. Installing updates is optional (don't get me started on the way Windows handles this...) but quite fast (30sec or less if you have an SSD and 100/40 NBN). If you click ``Install now'' it will give a preview of what will be downloaded before you confirm the upgrade.

\item You are now ready to start the \texttt{gcc} exercise from the previous section. Enjoy! (Or not, I know this isn't for everyone).

\item When you want to turn off the virtual machine you can either tell Ubuntu to turn off or just click the window's X button and tell VirtualBox to send the shutdown signal. It will close after a second or two. If you select ``power off the machine'' that is the equivalent of unplugging the power lead, probably not recommended although Linux typically handles this without issue.

\item Next time you boot the machine all changes (eg: files created, settings changed) will be present.

\end{enumerate}

\end{document}
