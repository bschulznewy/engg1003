%\UseRawInputEncoding
\documentclass{lab}

\usepackage{graphicx}
\usepackage{float}
\usepackage{soul}

\title{ENGG1003 - Lab 1}
\author{Brenton Schulz}
\date{\today}

\begin{document}
\maketitle

Welcome to the ENGG1003 Lab!

Being the first week of semester we're mostly here to find our feet so this lab is a mix of admin and programming.

\section{Admin Tasks}

\begin{enumerate}
\item Learn how to perform the attendance check-in
	\begin{itemize}
		\item Only for labs \textbf{on campus}.
		\item If in doubt, ask the demonstrator! They have the ability to force a check-in and confirm that a check-in was successful.
	\end{itemize}
\item Join the ENGG1003 Discord server: \url{https://discord.gg/sfgpR4kMbN}
	\begin{itemize}
		\item If you haven't used Discord before your demonstrator can walk you through the process of signing up and installing a Discord client.
		\item Set your server nickname to your name as it appears in Blackboard.
		\item Send your demonstrator a direct message with your student number (or a photo of your student card) for verification purposes. They will then add you to the \texttt{@students} role so you can see all the student channels.
			\begin{itemize}
				\item Please be patient, there are a lot of you!
			\end{itemize}
	\end{itemize}
\item Install a Zoom client and make sure you can log in
	\begin{itemize}
		\item Hopefully you've done this already for the lecture, otherwise now is the time to catch up!
	\end{itemize}
\item Subscribe to the ENGG1003 YouTube channel
	\begin{itemize}
		\item Lectures will be streamed on both Zoom and YouTube, but subscribing on YouTube will give you a notification when lectures start: \url{https://www.youtube.com/channel/UCU0BR2_STrZjttnYVdI-r6Q}
	\end{itemize}
\item Access (or download) the textbook
	\begin{itemize}
		\item Available for FREE to read online or download as PDF or EPUB: \url{https://link.springer.com/book/10.1007%2F978-3-030-16877-3}
	\end{itemize}
\end{enumerate}

\section{Programming Tasks}

\begin{enumerate}
\item Install PyCharm
	\begin{itemize}
		\item Download from: \url{https://www.jetbrains.com/pycharm/download/}
		\item Watch the installation video, \texttt{pycharm introduction.mp4}, on Blackboard under \texttt{Course Materials > Week 1}.
		\item If you can't get this working on your laptop please get help from a demonstrator \textbf{this week}.
	\end{itemize}
\item Create a new PyCharm project as-per the video. Ensure that you can run the template code without error
\item Delete the template code (select all and delete in the editor window, don't delete the file)
\item Read through Section 1.2 of the textbook, executing the lines of code as you go
	\begin{enumerate}
		\item Code in the textbook is in image format but there are ``GitHub'' links to code files. \texttt{ball.py} is here: \url{https://github.com/slgit/prog4comp_2/blob/master/py36-src/ball.py}
		\item Run the script, observing the output
		\item Continue reading from 1.2.2, executing each line of code (eg: \texttt{v0 = 5}) into the \texttt{Python Console} and observe the behaviour of the console. Note how you can run code as a script or run individual lines in the console.
	\end{enumerate}
\item From Section 1.9 complete Exercise 1.1: Error Messages
\item Complete Exercises 1.2
	\begin{itemize}
		\item There is no template code for this one. Copy the \textit{style} of \texttt{ball.py} as:
			\begin{enumerate}
				\item Variable \textit{initialisations} (in this case \texttt{L=}\textit{something})
				\item Implementation of the equation (\texttt{V = L*L*L} or \texttt{V=L**3} (\texttt{**} means ``to the power of''))
				\item A \texttt{print()} statement to print the result to the Python Console
			\end{enumerate}
	\end{itemize}
\item Complete Exercise 1.3
\end{enumerate}

\end{document}
