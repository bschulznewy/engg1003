\documentclass{lab}

\usepackage{graphicx}
\usepackage{float}
\usepackage{soul}
\usepackage{multicol}

\addtolength{\oddsidemargin}{-.4in}
\addtolength{\evensidemargin}{-.4in}

\title{ENGG1003 - Lab Week 8}
\author{Brenton Schulz}
\date{\today}

\begin{document}
\maketitle

\begin{task}{}{}

\end{task}

\begin{task}{Creating 2D and 3D Arrays}{}

\end{task}

\begin{task}{Indexing 2D and 3D Arrays}{}

\end{task}

\begin{task}{Slicing 2D and 3D Arrays}{}

\end{task}

\begin{task}{Mapping Cartesian Coordinates to 2D Array Indices}{}

\end{task}

\begin{task}{Barnsley Fern}{}

In this task you will modify an existing Python program to generate an image file of the \textit{Barnsley fern} \textit{fractal}.
\\~\\
A fractal is a mathematically generated image which exhibits ``self-similar'' geometry. As the image is zoomed in the the same patterns are seen repeated and, in theory, the image can be zoomed in forever and still show the same level of detail as it did when zoomed out.
\\~\\
The Barnsley fern is from a class of fractals known as iterated function systems (IFS). The general pattern for generating fractals of this type is to:
\begin{enumerate}
\item Pick (or be given) a point $x_0,y_0$
\item Generate a new point, $x_1,y_1$, by applying some mathematical rules
\item Draw a dot on an $x$-$y$ plane where the new point lies
\item Repeat millions (or billions) of times until an image is drawn
\end{enumerate}

The rules for the Barnsley fern are as follows:
\begin{itemize}
\item There are four functions which generate a new point:
	\begin{itemize}
		\item 
		\item 
		\item
		\item
	\end{itemize}
\item Each iteration, \textit{one} of the four functions is chosen at random with a probability, $p$, of:
	\begin{itemize}
		\item $f_1$: $p=0.01$
		\item $f_2$: $p=0.85$
		\item $f_3$: $p=0.07$
		\item $f_4$: $p=0.07$
	\end{itemize}
\end{itemize}

\end{task}

\end{document}
