\documentclass{lab}

\usepackage{graphicx}
\usepackage{float}
\usepackage{soul}
\usepackage{multicol}

\addtolength{\oddsidemargin}{-.4in}
\addtolength{\evensidemargin}{-.4in}

\title{ENGG1003 - Lab 3}
\author{Brenton Schulz}
\date{\today}

\begin{document}
\maketitle

\begin{itemize}
\item arrays, indexing, sliding
\item  for loops
\item boolean expressions / relational operators
\item if statements
\item while loops
\end{itemize}

\begin{task}{Array Reading}{}
Read Section 2.3 of the textbook, stopping at 2.3.6. Execute examples as you go.
\\~\\
Direct link: \url{https://link.springer.com/chapter/10.1007/978-3-030-16877-3_2#Sec16}
\\~\\
You are welcome to read 2.3.6 (regarding 2D arrays) but that content will be covered later.
\end{task}

\begin{task}{Fibonacci  Sequence - Naive Implementation}{}
The Fibonacci sequence is a sequence of numbers, $x_0$, $x_1$, $x_2$, ... etc, with the following equation used to calculate $x_n$ given $x_{n-1}$ and $x_{n-2}$:

\begin{equation}
x_n = x_{n-1} + x_{n-2}
\end{equation}

Write a Python script which, given $x_0=1$ and $x_1=1$, calculates and prints the next 8 values of the Fibonacci sequence.
\\~\\
To do this, create a NumPy array, \texttt{fib[]} containing 10 zeros, manually assign the above 1's to \texttt{fib[0]} and \texttt{fib[1]}, then write out the equation manually for the next 8 values, eg:
\begin{lstlisting}
fib[2] = fib[1] + fib[0]
print(fib[2])
fib[3] = fig[2] + fib[1]
print(fib[3])
... etc
\end{lstlisting}

Note that there is a far more efficient method using \textit{loops}. This will be explored later.
\end{task}


\end{document}
