\documentclass{lab}

\usepackage{graphicx}
\usepackage{float}
\usepackage{soul}
\usepackage{multicol}

\addtolength{\oddsidemargin}{-.4in}
\addtolength{\evensidemargin}{-.4in}

\title{ENGG1003 - Lab 3}
\author{Brenton Schulz}
\date{\today}

\begin{document}
\maketitle

\begin{itemize}
\item arrays, indexing, sliding
\item for loops
\item boolean expressions / relational operators
\item if statements
\item while loops
\end{itemize}

\begin{task}{Pre-Lab Reading}{}
Read Sections 2.1 and 2.2 of the textbook: \url{https://link.springer.com/chapter/10.1007/978-3-030-16877-3_2}
\\~\\
These sections provide general background information which will help you write Python scripts with confidence. The content is best learned ``by immersion''. All the details covered in these sections will be constantly used throughout your programming career.
\end{task}

\begin{task}{Array Background Reading}{}
Read Section 2.3 of the textbook, stopping at 2.3.6. Execute examples as you go.
\\~\\
Direct link: \url{https://link.springer.com/chapter/10.1007/978-3-030-16877-3_2#Sec16}
\\~\\
You are welcome to read 2.3.6 (regarding 2D arrays) but that content will be covered later.
\end{task}

\begin{task}{Fibonacci  Sequence - Naive Implementation}{}
The Fibonacci sequence is a sequence of numbers, $x_0$, $x_1$, $x_2$, ... etc, with the following equation used to calculate $x_n$ given $x_{n-1}$ and $x_{n-2}$:

\begin{equation}\label{eq:fib}
x_n = x_{n-1} + x_{n-2}
\end{equation}

Write a Python script which, given $x_0=1$ and $x_1=1$, calculates and prints the next 8 values of the Fibonacci sequence.
\\~\\
To do this, create a NumPy array, \texttt{fib[]} containing 10 zeros, manually assign the above 1's to \texttt{fib[0]} and \texttt{fib[1]}, then write out the equation as follows for the next 8 values:
\begin{lstlisting}
fib[2] = fib[1] + fib[0]
fib[3] = fig[2] + fib[1]
... etc
print(fib)
\end{lstlisting}

Note that there is a far more efficient method using \textit{loops}. This will be explored later.
\end{task}

\begin{task}{\texttt{for} Loops - Reading}{}
Read Section 3.1 of the textbook: \url{https://link.springer.com/chapter/10.1007/978-3-030-16877-3_3#Sec1}
\end{task}

\begin{task}{Fibonacci Sequence with a \texttt{for} Loop}{}
Modify your Fibonacci sequence script to utilise a \texttt{for} loop and the \texttt{range()} function. Note that by utilising a \texttt{for} loop you now only need to write out Equation \ref{eq:fib} \textit{once}, irrelevant of how many values you wish to calculate.
\\~\\
A few notes \& tips:
\begin{itemize}
\item Try to use a single variable \texttt{N} which specifies how many values to calculate
\item Since the first calculation is giving the 3rd value the \texttt{range()} function needs to be called as \texttt{range(2,N)}.
\item If \texttt{N} is large (more than about 90) care must be taken with the choice of data type. \texttt{np.zeros()} will, by default, create \texttt{np.float64}s but the Fibonacci sequence is intrinsically an \textit{integer} sequence. Experiment with different datatypes specified in the call to \texttt{np.zeros()}. eg: \texttt{fib = np.zeros(N,dtype=np.uint64)}\\~\\The full list of NumPy datatypes is here: \url{https://numpy.org/devdocs/user/basics.types.html}. How many terms can you calculate before an ``overflow'' error with \texttt{uint8}, \texttt{uint32}, and \texttt{uint64}?
\item \texttt{print(fib)} should print the entire array but you can call \texttt{print()} from within the loop so that only a single value is printed on each line.
\end{itemize}
\end{task}

\begin{task}{\texttt{while} Loops - Reading}{}
Read Section 3.2 of the textbook: \url{https://link.springer.com/chapter/10.1007/978-3-030-16877-3_3#Sec7}
\end{task}

\begin{task}{Fibonacci Sequence with \texttt{while} Loops}{}
Fork your Fibonacci sequence code (ie: save a copy of it so it can be loaded later).
\\~\\
Using a \texttt{while} loop, implement a Fibonacci sequence generator which prints the Fibonacci sequence until the printed value exceeds 1 million.
\end{task}

\begin{task}{Fibonacci Sequence Without Arrays}{}
Modify your code so that instead of using an array it calculates the sequence using only 3 variables:
\\~
\begin{itemize}
\item \texttt{xn} - The current value
\item \texttt{xnm1} - The previous value, $x_{n-1}$
\item \texttt{xnm2} - The value of $x_{n-2}$
\end{itemize}
~\\
The calculation will have to happen in two steps:
\\
\begin{enumerate}
\item Calculate the current value, $x_n$
\item ``Move forward in time'' by executing \texttt{xnm2 = xnm1} and \texttt{xnm1 = xn}.
\end{enumerate}
~\\
You may use \texttt{for} or \texttt{while} loops to complete this task.
\\~\\
Note that this implementation has the advantage of using \textit{significantly} less RAM than the array-based version. The disadvantages are that you must print each value as it is calculated and the code is somewhat less ``readable'' - it looks less like Equation \ref{eq:fib} than the array based versions.
\end{task}

\end{document}
